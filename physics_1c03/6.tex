%%%%%%%%%%%%%%%%%%%%%%%%%%%%% Define Article %%%%%%%%%%%%%%%%%%%%%%%%%%%%%%%%%%
\documentclass{article}
%%%%%%%%%%%%%%%%%%%%%%%%%%%%%%%%%%%%%%%%%%%%%%%%%%%%%%%%%%%%%%%%%%%%%%%%%%%%%%%

%%%%%%%%%%%%%%%%%%%%%%%%%%%%% Using Packages %%%%%%%%%%%%%%%%%%%%%%%%%%%%%%%%%%
\usepackage{geometry}
\usepackage{graphicx}
\usepackage{amssymb}
\usepackage{amsmath}
\usepackage{amsthm}
\usepackage{empheq}
\usepackage{mdframed}
\usepackage{booktabs}
\usepackage{lipsum}
\usepackage{graphicx}
\usepackage{color}
\usepackage{psfrag}
\usepackage{pgfplots}
\usepackage{bm}
%%%%%%%%%%%%%%%%%%%%%%%%%%%%%%%%%%%%%%%%%%%%%%%%%%%%%%%%%%%%%%%%%%%%%%%%%%%%%%%

% Other Settings

%%%%%%%%%%%%%%%%%%%%%%%%%% Page Setting %%%%%%%%%%%%%%%%%%%%%%%%%%%%%%%%%%%%%%%
\geometry{a4paper}

%%%%%%%%%%%%%%%%%%%%%%%%%% Define some useful colors %%%%%%%%%%%%%%%%%%%%%%%%%%
\definecolor{ocre}{RGB}{243,102,25}
\definecolor{mygray}{RGB}{243,243,244}
\definecolor{deepGreen}{RGB}{26,111,0}
\definecolor{shallowGreen}{RGB}{235,255,255}
\definecolor{deepBlue}{RGB}{61,124,222}
\definecolor{shallowBlue}{RGB}{235,249,255}
%%%%%%%%%%%%%%%%%%%%%%%%%%%%%%%%%%%%%%%%%%%%%%%%%%%%%%%%%%%%%%%%%%%%%%%%%%%%%%%

%%%%%%%%%%%%%%%%%%%%%%%%%% Define an orangebox command %%%%%%%%%%%%%%%%%%%%%%%%
\newcommand\orangebox[1]{\fcolorbox{ocre}{mygray}{\hspace{1em}#1\hspace{1em}}}
%%%%%%%%%%%%%%%%%%%%%%%%%%%%%%%%%%%%%%%%%%%%%%%%%%%%%%%%%%%%%%%%%%%%%%%%%%%%%%%

%%%%%%%%%%%%%%%%%%%%%%%%%%%% English Environments %%%%%%%%%%%%%%%%%%%%%%%%%%%%%
\newtheoremstyle{mytheoremstyle}{3pt}{3pt}{\normalfont}{0cm}{\rmfamily\bfseries}{}{1em}{{\color{black}\thmname{#1}~\thmnumber{#2}}\thmnote{\,--\,#3}}
\newtheoremstyle{myproblemstyle}{3pt}{3pt}{\normalfont}{0cm}{\rmfamily\bfseries}{}{1em}{{\color{black}\thmname{#1}~\thmnumber{#2}}\thmnote{\,--\,#3}}
\theoremstyle{mytheoremstyle}
\newmdtheoremenv[linewidth=1pt,backgroundcolor=shallowGreen,linecolor=deepGreen,leftmargin=0pt,innerleftmargin=20pt,innerrightmargin=20pt,]{theorem}{Theorem}[section]
\theoremstyle{mytheoremstyle}
\newmdtheoremenv[linewidth=1pt,backgroundcolor=shallowBlue,linecolor=deepBlue,leftmargin=0pt,innerleftmargin=20pt,innerrightmargin=20pt,]{definition}{Definition}[section]
\theoremstyle{myproblemstyle}
\newmdtheoremenv[linecolor=black,leftmargin=0pt,innerleftmargin=10pt,innerrightmargin=10pt,]{problem}{Problem}[section]
%%%%%%%%%%%%%%%%%%%%%%%%%%%%%%%%%%%%%%%%%%%%%%%%%%%%%%%%%%%%%%%%%%%%%%%%%%%%%%%

%%%%%%%%%%%%%%%%%%%%%%%%%%%%%%% Plotting Settings %%%%%%%%%%%%%%%%%%%%%%%%%%%%%
\usepgfplotslibrary{colorbrewer}
\pgfplotsset{width=8cm,compat=1.9}
%%%%%%%%%%%%%%%%%%%%%%%%%%%%%%%%%%%%%%%%%%%%%%%%%%%%%%%%%%%%%%%%%%%%%%%%%%%%%%%

%%%%%%%%%%%%%%%%%%%%%%%%%%%%%%% Title & Author %%%%%%%%%%%%%%%%%%%%%%%%%%%%%%%%
\title{Relative Motion}
\author{Patrick Chen}
\date{Sept 26, 2024}
%%%%%%%%%%%%%%%%%%%%%%%%%%%%%%%%%%%%%%%%%%%%%%%%%%%%%%%%%%%%%%%%%%%%%%%%%%%%%%%

\begin{document}
    \maketitle
    \section*{Center of mass}
    The center of mass is a point at the average of the masses. A multi-object
    system may be replaced with a point mass at this point to simplify the
    model. Object with any symmetry will lie on the axis of symmetry. For a
    uniform object, the center of geometry and mass is distributed evenly around
    the center of mass.

    \begin{align*}
        x_{CM} &= \frac{m_1x_1+m_2x_2}{m_1+m_2} \\
        x_{CM} &= \frac{\sum_i m_i \vec{r}_i}{\sum_i m_i}
    \end{align*}

    The derivative of center of mass with respect to time is the total momentum
    of the collection. Since the total momentum is constant, the velocity of the
    center of mass will also remain constant.
    \begin{align*}
        Mr_{CM} &= \sum m_ir_i \\
                &\Downarrow \frac{d}{dt} \\
        Mv_{CM} &= \sum m_iv_i = p_{total} \\
    \end{align*}

    \section*{Relative motion}
    Since velocity is relative, it needs to be measured with respect to a frame
    of reference.
    \begin{align*}
        V_{AC} &= V_{AB} + V_{BC} \\
        V_{AB} &= -V_{BA}
    \end{align*}
    $V_{AC}$ is the velocity of $C$ relative to $A$. When changing reference
    frames, the graph of velocities are shifted up or down.

    When a collision is observed from the center of mass reference frame, the
    collision looks like a head on collision. If the two masses are the same,
    the two masses will be the same speed.

    \section*{Kinetic Energy}
    \begin{align*}
        v_{cm}      &= 0 \\
        K_E         &= K_Z + \frac{1}{2} Mv_{cm}^2 \\
        \Delta K_E  &= \Delta K_Z
    \end{align*}

    $K_Z$ is the kinetic energy in the center of mass frame. $\frac{1}{2} MV_{cm}^2$
    is the kinetic energy of the center of mass in the earth frame. The
    difference of the kinetic energy in the center of mass frame is called the
    convertible kinetic energy. During a collision, the kinetic energy cannot be
    lower than the kinetic energy of the center of mass because the kinetic
    energy of the center of mass is the lowest kinetic energy possible to
    conserve momentum. This means that the convertible kinetic energy is the
    only kinetic energy that can be converted into potential energy during the
    collisions, hence the name convertible.

    \begin{align*}
        K_E    = \frac{1}{2} m_1v_1^2 + \frac{1}{2} m_2v_2^2 \\
        K_{CM} = \frac{1}{2} (m_1+m_2)v_{cm}^2 \\
        K_Z    = \frac{1}{2} m_1 v_{z1}^2 + \frac{1}{2} m_2 v_{z2}^2
    \end{align*}

    \section*{Galilean Relativity}
    \begin{align*}
        r=r'+r_0 && x_{AC} = x_{AB} + x_{BC} \\
        v=v'+v_0 && v_{AC} = v_{AB} + v_{BC} \\
        a=a'+a_0 && a_{AC} = a_{AB} + a_{BC} \\
    \end{align*}
    Since it is an inertial reference frame, $a=0$.

\end{document}
