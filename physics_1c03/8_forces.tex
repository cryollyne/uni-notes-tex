%%%%%%%%%%%%%%%%%%%%%%%%%%%%% Define Article %%%%%%%%%%%%%%%%%%%%%%%%%%%%%%%%%%
\documentclass{article}
%%%%%%%%%%%%%%%%%%%%%%%%%%%%%%%%%%%%%%%%%%%%%%%%%%%%%%%%%%%%%%%%%%%%%%%%%%%%%%%

%%%%%%%%%%%%%%%%%%%%%%%%%%%%% Using Packages %%%%%%%%%%%%%%%%%%%%%%%%%%%%%%%%%%
\usepackage{geometry}
\usepackage{graphicx}
\usepackage{amssymb}
\usepackage{amsmath}
\usepackage{amsthm}
\usepackage{empheq}
\usepackage{mdframed}
\usepackage{booktabs}
\usepackage{lipsum}
\usepackage{graphicx}
\usepackage{color}
\usepackage{psfrag}
\usepackage{pgfplots}
\usepackage{bm}
%%%%%%%%%%%%%%%%%%%%%%%%%%%%%%%%%%%%%%%%%%%%%%%%%%%%%%%%%%%%%%%%%%%%%%%%%%%%%%%

% Other Settings

%%%%%%%%%%%%%%%%%%%%%%%%%% Page Setting %%%%%%%%%%%%%%%%%%%%%%%%%%%%%%%%%%%%%%%
\geometry{a4paper}

%%%%%%%%%%%%%%%%%%%%%%%%%% Define some useful colors %%%%%%%%%%%%%%%%%%%%%%%%%%
\definecolor{ocre}{RGB}{243,102,25}
\definecolor{mygray}{RGB}{243,243,244}
\definecolor{deepGreen}{RGB}{26,111,0}
\definecolor{shallowGreen}{RGB}{235,255,255}
\definecolor{deepBlue}{RGB}{61,124,222}
\definecolor{shallowBlue}{RGB}{235,249,255}
%%%%%%%%%%%%%%%%%%%%%%%%%%%%%%%%%%%%%%%%%%%%%%%%%%%%%%%%%%%%%%%%%%%%%%%%%%%%%%%

%%%%%%%%%%%%%%%%%%%%%%%%%% Define an orangebox command %%%%%%%%%%%%%%%%%%%%%%%%
\newcommand\orangebox[1]{\fcolorbox{ocre}{mygray}{\hspace{1em}#1\hspace{1em}}}
%%%%%%%%%%%%%%%%%%%%%%%%%%%%%%%%%%%%%%%%%%%%%%%%%%%%%%%%%%%%%%%%%%%%%%%%%%%%%%%

%%%%%%%%%%%%%%%%%%%%%%%%%%%% English Environments %%%%%%%%%%%%%%%%%%%%%%%%%%%%%
\newtheoremstyle{mytheoremstyle}{3pt}{3pt}{\normalfont}{0cm}{\rmfamily\bfseries}{}{1em}{{\color{black}\thmname{#1}~\thmnumber{#2}}\thmnote{\,--\,#3}}
\newtheoremstyle{myproblemstyle}{3pt}{3pt}{\normalfont}{0cm}{\rmfamily\bfseries}{}{1em}{{\color{black}\thmname{#1}~\thmnumber{#2}}\thmnote{\,--\,#3}}
\theoremstyle{mytheoremstyle}
\newmdtheoremenv[linewidth=1pt,backgroundcolor=shallowGreen,linecolor=deepGreen,leftmargin=0pt,innerleftmargin=20pt,innerrightmargin=20pt,]{theorem}{Theorem}[section]
\theoremstyle{mytheoremstyle}
\newmdtheoremenv[linewidth=1pt,backgroundcolor=shallowBlue,linecolor=deepBlue,leftmargin=0pt,innerleftmargin=20pt,innerrightmargin=20pt,]{definition}{Definition}[section]
\theoremstyle{myproblemstyle}
\newmdtheoremenv[linecolor=black,leftmargin=0pt,innerleftmargin=10pt,innerrightmargin=10pt,]{problem}{Problem}[section]
%%%%%%%%%%%%%%%%%%%%%%%%%%%%%%%%%%%%%%%%%%%%%%%%%%%%%%%%%%%%%%%%%%%%%%%%%%%%%%%

%%%%%%%%%%%%%%%%%%%%%%%%%%%%%%% Plotting Settings %%%%%%%%%%%%%%%%%%%%%%%%%%%%%
\usepgfplotslibrary{colorbrewer}
\pgfplotsset{width=8cm,compat=1.9}
%%%%%%%%%%%%%%%%%%%%%%%%%%%%%%%%%%%%%%%%%%%%%%%%%%%%%%%%%%%%%%%%%%%%%%%%%%%%%%%

%%%%%%%%%%%%%%%%%%%%%%%%%%%%%%% Title & Author %%%%%%%%%%%%%%%%%%%%%%%%%%%%%%%%
\title{Forces}
\author{Patrick Chen}
\date{Oct 24, 2024}
%%%%%%%%%%%%%%%%%%%%%%%%%%%%%%%%%%%%%%%%%%%%%%%%%%%%%%%%%%%%%%%%%%%%%%%%%%%%%%%

\begin{document}
    \maketitle
    Changes in acceleration require a force. All influences on a particle from
    its surroundings are forces exerted on the particle. There are four
    fundamental forces.

    \section*{Four fundamental forces}
    \begin{center}
        \begin{tabular}[c]{l|l|l|l|l|l}
            \hline
            Type & Attribute & Strength & Range & Gauge
            Particle & Propagation Speed \\
            \hline
            gravity & mass & 1 & $\infty$ & graviton & $c$ \\
            weak & weak charge & $10^{25}$ & $10^{-18}$ & vector bosons & varies \\
            electromagnetic & charge & $10^{36}$ & $\infty$ & photon & $c$ \\
            strong color & color charge & $10^{38}$ & $10^{-15}$ & gluon & $c$ \\
            \hline
        \end{tabular}
    \end{center}

    \subsection*{Gravity}

    Gravity is a attractive long range force between any two objects with mass.
    Weight is the force of gravity exerted on a particle. Note that weight is
    the same as mass.
    \begin{align*}
        \vec{F_g} &= - \frac{GMm}{r^2} \\
        |\vec{F_g}| &= mg
    \end{align*}

    \subsection*{Electromagnetism}
    Electromagnetic forces are responsible for the structure of atoms and
    molecules, chemical and biological processes, repulsion between objects
    in contact, and light. The repulsion is what causes objects to not intersect
    when they collide.
    \begin{align*}
        \vec{F_E} &= \frac{kQq}{r^2} \\
        \vec{F_B} &= q \vec{v}\times \vec{B}
    \end{align*}

    \section*{Types of Forces}
    \subsection*{Normal}
    Normal forces are always perpendicular to the contact surface. The normal
    force is a passive force which means that it gets as big as it needs to be
    to hold a object in place. We have to calculate it based off the situation a
    object is in. A normal force is a type of electromagnetic force because it
    is caused by the interaction between atom.

    \subsection*{Friction}
    Friction acts when a object is in contact and in motion. It opposes the
    direction of motion and is parallel to the contact surface. There exists two
    kinds: kinetic friction and static friction

    Static motion is friction that happens when there is no relative motion.
    Static friction prevents sliding. There is a maximum static friction and the
    static friction force will be as big as it needs to be while under the
    maximum static friction to prevent the object from moving relative to the
    surface.
    Kinetic friction happens when an object is moving.

    \begin{align*}
        f_{s,max} &= \mu_s N \\
        f_s       &\le \mu_s N \\
        f_k       &= \mu_k N
    \end{align*}

    Coefficient of friction is a approximation of friction depending on
    smoothness and temperature. Usually static friction is greater than kinetic
    friction.

    \subsection*{Spring}
    The spring force is a restoring force that depends on how much a spring is
    stretched or compressed from its natural length.
    \begin{align*}
        F_s = -kd
    \end{align*}

    \subsection*{Tension}
    Tension is a pulling force that points away from the object. An ideal rope
    has the same tension across the entire rope and ideal pulleys change the
    direction of the tension but not the magnitude. In the real world, none of
    these exists so the tension on a rope will be non-uniform and a pulley will
    change the magnitude of the tension.

    When a pulley is stationary, the tension will only change directions, but if
    the pulley can be moved, there will be two tension forces acting on the
    pulley.

    \subsection*{Drag}
    Drag or air resistance of a object in a fluid is the friction between an
    object and the fluid.

    \subsection*{Buoyancy}
    Buoyant forces is a upward force equal to the mass of the displaced fluid.

    \section*{Free Body Diagrams}
    For free body diagrams, we pick one single objects and identify all external
    forces which act directly on that body. Then we set up a coordinate system
    and draw the object as either a rough sketch or a point at the center of
    mass. Every force is drawn as vector starting at the object and in the
    direction of the force. The direction of acceleration should be indicated
    but not as a vector on the body because it is not a force.


    \section*{Passive Forces}
    A passive force can only be calculated as a response to other forces. They
    will depend on the load it has to support and the acceleration they have to
    provide. They are often limited.

    \section*{Equilibrium}
    A system whose motion or state is not changing is said to be in equilibrium.
    A object at rest or moving at a constant velocity is said to be in
    translational equilibrium. When objects are in equilibrium, the sum of all
    forces is zero.

    \section*{Tips for forces problems}
    \begin{itemize}
        \item identify known and unknowns
        \item draw free body diagrams for each objects
        \item relate forces in action reaction pairs
        \item analyze constraints of motion.
        \item use Newton's second law
        \item determine if there are any free variables
    \end{itemize}
\end{document}
