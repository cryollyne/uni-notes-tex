%%%%%%%%%%%%%%%%%%%%%%%%%%%%% Define Article %%%%%%%%%%%%%%%%%%%%%%%%%%%%%%%%%%
\documentclass{article}
%%%%%%%%%%%%%%%%%%%%%%%%%%%%%%%%%%%%%%%%%%%%%%%%%%%%%%%%%%%%%%%%%%%%%%%%%%%%%%%

%%%%%%%%%%%%%%%%%%%%%%%%%%%%% Using Packages %%%%%%%%%%%%%%%%%%%%%%%%%%%%%%%%%%
\usepackage{geometry}
\usepackage{graphicx}
\usepackage{amssymb}
\usepackage{amsmath}
\usepackage{amsthm}
\usepackage{empheq}
\usepackage{mdframed}
\usepackage{booktabs}
\usepackage{lipsum}
\usepackage{graphicx}
\usepackage{color}
\usepackage{psfrag}
\usepackage{pgfplots}
\usepackage{bm}
%%%%%%%%%%%%%%%%%%%%%%%%%%%%%%%%%%%%%%%%%%%%%%%%%%%%%%%%%%%%%%%%%%%%%%%%%%%%%%%

% Other Settings

%%%%%%%%%%%%%%%%%%%%%%%%%% Page Setting %%%%%%%%%%%%%%%%%%%%%%%%%%%%%%%%%%%%%%%
\geometry{a4paper}

%%%%%%%%%%%%%%%%%%%%%%%%%% Define some useful colors %%%%%%%%%%%%%%%%%%%%%%%%%%
\definecolor{ocre}{RGB}{243,102,25}
\definecolor{mygray}{RGB}{243,243,244}
\definecolor{deepGreen}{RGB}{26,111,0}
\definecolor{shallowGreen}{RGB}{235,255,255}
\definecolor{deepBlue}{RGB}{61,124,222}
\definecolor{shallowBlue}{RGB}{235,249,255}
%%%%%%%%%%%%%%%%%%%%%%%%%%%%%%%%%%%%%%%%%%%%%%%%%%%%%%%%%%%%%%%%%%%%%%%%%%%%%%%

%%%%%%%%%%%%%%%%%%%%%%%%%% Define an orangebox command %%%%%%%%%%%%%%%%%%%%%%%%
\newcommand\orangebox[1]{\fcolorbox{ocre}{mygray}{\hspace{1em}#1\hspace{1em}}}
%%%%%%%%%%%%%%%%%%%%%%%%%%%%%%%%%%%%%%%%%%%%%%%%%%%%%%%%%%%%%%%%%%%%%%%%%%%%%%%

%%%%%%%%%%%%%%%%%%%%%%%%%%%% English Environments %%%%%%%%%%%%%%%%%%%%%%%%%%%%%
\newtheoremstyle{mytheoremstyle}{3pt}{3pt}{\normalfont}{0cm}{\rmfamily\bfseries}{}{1em}{{\color{black}\thmname{#1}~\thmnumber{#2}}\thmnote{\,--\,#3}}
\newtheoremstyle{myproblemstyle}{3pt}{3pt}{\normalfont}{0cm}{\rmfamily\bfseries}{}{1em}{{\color{black}\thmname{#1}~\thmnumber{#2}}\thmnote{\,--\,#3}}
\theoremstyle{mytheoremstyle}
\newmdtheoremenv[linewidth=1pt,backgroundcolor=shallowGreen,linecolor=deepGreen,leftmargin=0pt,innerleftmargin=20pt,innerrightmargin=20pt,]{theorem}{Theorem}[section]
\theoremstyle{mytheoremstyle}
\newmdtheoremenv[linewidth=1pt,backgroundcolor=shallowBlue,linecolor=deepBlue,leftmargin=0pt,innerleftmargin=20pt,innerrightmargin=20pt,]{definition}{Definition}[section]
\theoremstyle{myproblemstyle}
\newmdtheoremenv[linecolor=black,leftmargin=0pt,innerleftmargin=10pt,innerrightmargin=10pt,]{problem}{Problem}[section]
%%%%%%%%%%%%%%%%%%%%%%%%%%%%%%%%%%%%%%%%%%%%%%%%%%%%%%%%%%%%%%%%%%%%%%%%%%%%%%%

%%%%%%%%%%%%%%%%%%%%%%%%%%%%%%% Plotting Settings %%%%%%%%%%%%%%%%%%%%%%%%%%%%%
\usepgfplotslibrary{colorbrewer}
\pgfplotsset{width=8cm,compat=1.9}
%%%%%%%%%%%%%%%%%%%%%%%%%%%%%%%%%%%%%%%%%%%%%%%%%%%%%%%%%%%%%%%%%%%%%%%%%%%%%%%

%%%%%%%%%%%%%%%%%%%%%%%%%%%%%%% Title & Author %%%%%%%%%%%%%%%%%%%%%%%%%%%%%%%%
\title{Momentum}
\author{Patrick Chen}
\date{Sept 16, 2024}
%%%%%%%%%%%%%%%%%%%%%%%%%%%%%%%%%%%%%%%%%%%%%%%%%%%%%%%%%%%%%%%%%%%%%%%%%%%%%%%

\begin{document}
    \maketitle
    \section*{Inertia}
    Inertia or mass is a measure of how much an object resists acceleration. It
    is represented by the symbol $m$ and the SI unit is inertia is the kilogram
    (kg).

    \section*{Systems}
    A system is a object of group of objects that we mentally separate from the
    rest of the environment. The objects we care about will guide our choice of
    systems. Extensive quantities are quantities that are proportional to the
    size of the system. Intensive quantities do not depend on the extent of the
    system. Isolated systems are system without inputs or outputs. Conserved
    quantities can only be changed by input or output. \\
    In a system of trees, cutting down trees or planting new trees are external
    because they require an external effect. Trees naturally dying and
    reproducing are internal because they do not require external input.
    \begin{align*}
        change = input-output+creation-destruction
    \end{align*}

    \section*{Collisions}
    When objects collide, they could bounce off each other, stick together,
    stop, deform, transfer energy, etc. 

    \begin{align*}
        \Delta v &\propto \frac{1}{m} \\
        \frac{\Delta v_a}{\Delta v_b} &= -\frac{m_b}{m_a} \\
        m_a \Delta v_a &= - m_b \Delta v_b
    \end{align*}

    Linear momentum $p$ of a particle is its mass times is velocity. Momentum is
    a vector, since velocity is a vector. It has units of $kg \frac{m}{s}$.

    \begin{align*}
        \vec{p} \equiv m \vec{v}
    \end{align*}

    Momentum is transferred though interactions between objects, but cannot be
    created or destroyed. For an isolated system, the total momentum change is
    zero.

    \begin{align*}
        p_{1i} = p_{2i} &= p_{1f} + p_{2f} \\
        p_{1i} - p_{1f} &= p_{2f} - p_{2i} \\
        -\Delta p_1     &= \Delta p_2 \\
    \end{align*}

    \begin{align*}
        \vec{p}_{f,sys} = \vec{p}_{i,sys} \\
        \vec{p}_{total} = \Sigma \vec{p}_i = constant \\
        \Delta \vec{p}_{sys} = 0
    \end{align*}

    \subsection*{Collisions with Friction}
    We need to separate the effect of friction from the effect of the collision
    and extrapolate the velocity of the cart.

    \subsection*{Impulse}
    impulse is the change in momentum of an object.
    \begin{align*}
        \Delta \vec{p} = \vec{J}
    \end{align*}

    \subsection*{Type of collisions}
    \begin{enumerate}
        \item Perfectly inelastic collisions are collisions where the two
            objects come out stuck to each other and their final velocity is the
            same.

        \item Inelastic collisions are collisions where kinetic energy is lost.

        \item Perfectly elastic collisions are collisions where the total
            kinetic energy is conserved.
    \end{enumerate}

    In all collisions, momentum is conserved.

    \subsection*{Elastic Collisions}
    momentum and kinetic energy is conserved
    \begin{align*}
        m_1v_{1i} + m_2v_{2i} &= m_1v_{1f} + m_2v_{2f} & \text{(p)} \\
        \frac{1}{2} m_{1i}v_{1i}^2 + \frac{1}{2} m_{2i}v_{2i}^2 &= \frac{1}{2}
        m_{1f}v_{1f}^2 + \frac{1}{2} m_{2f}v_{2f}^2 & \text{(k)}
    \end{align*}

    \begin{align}
        m_1(v_{1i}-v_{1f}) &= m_2(v_{2f}-v_{2i}) & \text{rearrange variables in (p)}\\
        m_{1i}v_{1i}^2 + m_{2i}v_{2i}^2 &= m_{1f}v_{1f}^2 + m_{2f}v_{2f}^2 & \text{multiply (k) by two}\\
        m_1(v_{1i}^2 - v_{1f}^2) &= m_2(v_{2f}^2-v_{2i}^2) & \text{rearrange (2)}\\
        m_1(v_{1i}-v_{1f})(v_{1i}+v_{1f}) &= m_2(v_{2f}-v_{2i})(v_{2f}+v_{2i}) &\text{factor (3)}\\
        \frac{m_1(v_{1i}-v_{1f})(v_{1i}+v_{1f})}{
            m_1(v_{1i} - v_{1f})
        } &= \frac{m_2(v_{2f}-v_{2i})(v_{2f}+v_{2i})} {
            m_2(v_{2f}-v_{2i})
        } & \text{divide (4) by (1)} \\
        v_{1i} + v_{1f}  &= v_{2f} + v_{2i}
    \end{align}

    \begin{align*}
        v_{1f} &= v_{2f} + v_{2i} -v_{1i}  \\
        m_1(v_{1i}-v_{1f}) &= m_2(v_{2f}-v_{2i}) \\
        m_1(v_{1i}-(v_{2f} + v_{2i} -v_{1i})) &= m_2(v_{2f}-v_{2i}) \\
        m_1(v_{1i}-v_{2f} - v_{2i} +v_{1i}) &= m_2(v_{2f}-v_{2i}) \\
        2m_1v_{1i} - m_1v_{2i} - m_1 v_{2f} &= m_2v_{2f}-m_2v_{2i} \\
        2m_1v_{1i} - m_1v_{2i} + m_2v_{2i} &= m_2v_{2f} + m_1 v_{2f} \\
        2m_1v_{1i} + v_{2i} (m_2-m_1) &= v_{2f} (m_2 + m_1) \\
        \frac{2m_1}{m_1 + m_2} v_{1i} + \frac{(m_1-m_2)}{m_2 + m_1} v_{2i} &= v_{2f} \\
    \end{align*}
    \begin{align*}
        v_{2f} &= v_{1i} + v_{1f} -v_{2i} \\
        m_1(v_{1i}-v_{1f}) &= m_2(v_{2f}-v_{2i}) \\
        m_1(v_{1i}-v_{1f}) &= m_2(v_{1i} + v_{1f} -v_{2i}-v_{2i}) \\
        m_1v_{1i}-m_1v_{1f} &= m_2v_{1i} + m_2v_{1f} -2m_2v_{2i} \\
        m_1v_{1i} - m_2v_{1i} + 2m_2v_{2i} &=  m_2v_{1f}  + m_1v_{1f} \\
        (m_1 - m_2)v_{1i} + 2m_2v_{2i} &=  v_{1f}(m_1  + m_2) \\
        \frac{m_1 - m_2}{m_1  + m_2} v_{1i} + \frac{2m_2}{m_1  + m_2}v_{2i} &=  v_{1f} \\
    \end{align*}

    \begin{align*}
        v_{1f}= \frac{m_1-m_2}{m_1+m_2} v_{1i} + \frac{2m_2}{m_1+m_2} v_{2i} \\
        v_{2f}= \frac{2m_1}{m_1+m_2} v_{1i} + \frac{m_2-m_1}{m_1+m_2} v_{2i}
    \end{align*}

    \begin{itemize}
        \item When two carts have the same mass and the second cart is
            stationary, the velocity is interchanges.

        \item When two carts have the same mass but different velocities, the
            velocities still interchanges
    \end{itemize}

    \subsection*{Inelastic Collisions}
    Momentum is still conserved In Perfectly inelastic collisions, the final
    velocity is the same:
    \begin{align*}
        m_1 v_{1i} + m_2 v_{2i} = (m_1 + m_2) v_f \\
        v_f = \frac{m_1 v_{1i} + m_2 v_{2i}}{m_1 + m_2}
    \end{align*}


\end{document}
