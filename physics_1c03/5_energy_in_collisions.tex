%%%%%%%%%%%%%%%%%%%%%%%%%%%%% Define Article %%%%%%%%%%%%%%%%%%%%%%%%%%%%%%%%%%
\documentclass{article}
%%%%%%%%%%%%%%%%%%%%%%%%%%%%%%%%%%%%%%%%%%%%%%%%%%%%%%%%%%%%%%%%%%%%%%%%%%%%%%%

%%%%%%%%%%%%%%%%%%%%%%%%%%%%% Using Packages %%%%%%%%%%%%%%%%%%%%%%%%%%%%%%%%%%
\usepackage{geometry}
\usepackage{graphicx}
\usepackage{amssymb}
\usepackage{amsmath}
\usepackage{amsthm}
\usepackage{empheq}
\usepackage{mdframed}
\usepackage{booktabs}
\usepackage{lipsum}
\usepackage{graphicx}
\usepackage{color}
\usepackage{psfrag}
\usepackage{pgfplots}
\usepackage{bm}
%%%%%%%%%%%%%%%%%%%%%%%%%%%%%%%%%%%%%%%%%%%%%%%%%%%%%%%%%%%%%%%%%%%%%%%%%%%%%%%

% Other Settings

%%%%%%%%%%%%%%%%%%%%%%%%%% Page Setting %%%%%%%%%%%%%%%%%%%%%%%%%%%%%%%%%%%%%%%
\geometry{a4paper}

%%%%%%%%%%%%%%%%%%%%%%%%%% Define some useful colors %%%%%%%%%%%%%%%%%%%%%%%%%%
\definecolor{ocre}{RGB}{243,102,25}
\definecolor{mygray}{RGB}{243,243,244}
\definecolor{deepGreen}{RGB}{26,111,0}
\definecolor{shallowGreen}{RGB}{235,255,255}
\definecolor{deepBlue}{RGB}{61,124,222}
\definecolor{shallowBlue}{RGB}{235,249,255}
%%%%%%%%%%%%%%%%%%%%%%%%%%%%%%%%%%%%%%%%%%%%%%%%%%%%%%%%%%%%%%%%%%%%%%%%%%%%%%%

%%%%%%%%%%%%%%%%%%%%%%%%%% Define an orangebox command %%%%%%%%%%%%%%%%%%%%%%%%
\newcommand\orangebox[1]{\fcolorbox{ocre}{mygray}{\hspace{1em}#1\hspace{1em}}}
%%%%%%%%%%%%%%%%%%%%%%%%%%%%%%%%%%%%%%%%%%%%%%%%%%%%%%%%%%%%%%%%%%%%%%%%%%%%%%%

%%%%%%%%%%%%%%%%%%%%%%%%%%%% English Environments %%%%%%%%%%%%%%%%%%%%%%%%%%%%%
\newtheoremstyle{mytheoremstyle}{3pt}{3pt}{\normalfont}{0cm}{\rmfamily\bfseries}{}{1em}{{\color{black}\thmname{#1}~\thmnumber{#2}}\thmnote{\,--\,#3}}
\newtheoremstyle{myproblemstyle}{3pt}{3pt}{\normalfont}{0cm}{\rmfamily\bfseries}{}{1em}{{\color{black}\thmname{#1}~\thmnumber{#2}}\thmnote{\,--\,#3}}
\theoremstyle{mytheoremstyle}
\newmdtheoremenv[linewidth=1pt,backgroundcolor=shallowGreen,linecolor=deepGreen,leftmargin=0pt,innerleftmargin=20pt,innerrightmargin=20pt,]{theorem}{Theorem}[section]
\theoremstyle{mytheoremstyle}
\newmdtheoremenv[linewidth=1pt,backgroundcolor=shallowBlue,linecolor=deepBlue,leftmargin=0pt,innerleftmargin=20pt,innerrightmargin=20pt,]{definition}{Definition}[section]
\theoremstyle{myproblemstyle}
\newmdtheoremenv[linecolor=black,leftmargin=0pt,innerleftmargin=10pt,innerrightmargin=10pt,]{problem}{Problem}[section]
%%%%%%%%%%%%%%%%%%%%%%%%%%%%%%%%%%%%%%%%%%%%%%%%%%%%%%%%%%%%%%%%%%%%%%%%%%%%%%%

%%%%%%%%%%%%%%%%%%%%%%%%%%%%%%% Plotting Settings %%%%%%%%%%%%%%%%%%%%%%%%%%%%%
\usepgfplotslibrary{colorbrewer}
\pgfplotsset{width=8cm,compat=1.9}
%%%%%%%%%%%%%%%%%%%%%%%%%%%%%%%%%%%%%%%%%%%%%%%%%%%%%%%%%%%%%%%%%%%%%%%%%%%%%%%

%%%%%%%%%%%%%%%%%%%%%%%%%%%%%%% Title & Author %%%%%%%%%%%%%%%%%%%%%%%%%%%%%%%%
\title{Energy}
\author{Patrick Chen}
\date{Sept 23, 2024}
%%%%%%%%%%%%%%%%%%%%%%%%%%%%%%%%%%%%%%%%%%%%%%%%%%%%%%%%%%%%%%%%%%%%%%%%%%%%%%%

\begin{document}
    \maketitle
    In a Newton's cradle, if one ball hits, then one ball will be come out the
    other side. If two balls hits, two balls come out the other end. This effect
    is due to conservation of energy. If one ball hits and two balls come out
    the other end, the velocity would have to be halved to conserve the
    momentum, but that would mean that kinetic energy is lost.

    In a collision, momentum is conserved throughout the entire collision.
    Kinetic energy in a elastic collision is constant before and after the
    collision but not during. This is because the kinetic energy is converted
    into elastic potential energy during the collision and is then released back
    into kinetic energy.

    The state of a system is the condition of an object completely specified by
    a set of parameters such as shape and temperature. Transformations of a
    system from an initial state to a final state is called a process. A closed
    system is a system where no energy is transferred into or out of it. The
    chosen system should include all the objects undergoing these changes in
    state or motion. 

    \section*{Properties of Collisions}
    Inelastic collisions are irreversible processes: the changes that occur in
    the state of the colliding objects cannot spontaneously undo themselves. In
    inelastic collisions, the relative speed changes and therefore the total
    kinetic energy of the system changes. In inelastic collisions, the kinetic
    energy is transformed into internal energy. The sum of the kinetic energy
    and internal energy remains constant. 

    Elastic collisions are reversible processes: there are no permanent changes
    in the state of the colliding object. In elastic collisions the magnitude of
    the relative velocity remains the same before and after the collision. If a
    small object hits a much more massive object, the small object will bounce
    back. If a massive object hits a small object, the massive object will
    continue at approximately the same velocity.

    In explosive separations, there is some stored internal energy that gets
    converted into kinetic energy. In explosive separations, momentum and total
    energy is conserved, but kinetic energy is increased. The internal energy
    could be elastic potential, chemical, etc.

    \section*{Coefficient of Restitution}
    The coefficient of restitution is the ratio between the final relative
    velocity and the initial relative velocity.
    \begin{align*}
        e = \bigg|\frac{ v_{2f} - v_{1f} }{v_{2i} - v_{1i} } \bigg| \\
        e = - \frac{ v_{2f} - v_{1f} }{ v_{1i} - v_{2i} }
    \end{align*}
    For a perfectly inelastic collision, the final velocity is zero, thus the
    coefficient of restitution is zero. When the collision is inelastic, the
    coefficient of restitution is between zero and one. In a perfectly elastic
    collision, the coefficient of restitution is one. In explosive separation,
    the coefficient of restitution is greater than one.

\end{document}
