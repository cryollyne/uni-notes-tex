%%%%%%%%%%%%%%%%%%%%%%%%%%%%% Define Article %%%%%%%%%%%%%%%%%%%%%%%%%%%%%%%%%%
\documentclass{article}
%%%%%%%%%%%%%%%%%%%%%%%%%%%%%%%%%%%%%%%%%%%%%%%%%%%%%%%%%%%%%%%%%%%%%%%%%%%%%%%

%%%%%%%%%%%%%%%%%%%%%%%%%%%%% Using Packages %%%%%%%%%%%%%%%%%%%%%%%%%%%%%%%%%%
\usepackage{geometry}
\usepackage{graphicx}
\usepackage{amssymb}
\usepackage{amsmath}
\usepackage{amsthm}
\usepackage{empheq}
\usepackage{mdframed}
\usepackage{booktabs}
\usepackage{lipsum}
\usepackage{graphicx}
\usepackage{color}
\usepackage{psfrag}
\usepackage{pgfplots}
\usepackage{bm}
%%%%%%%%%%%%%%%%%%%%%%%%%%%%%%%%%%%%%%%%%%%%%%%%%%%%%%%%%%%%%%%%%%%%%%%%%%%%%%%

% Other Settings

%%%%%%%%%%%%%%%%%%%%%%%%%% Page Setting %%%%%%%%%%%%%%%%%%%%%%%%%%%%%%%%%%%%%%%
\geometry{a4paper}

%%%%%%%%%%%%%%%%%%%%%%%%%% Define some useful colors %%%%%%%%%%%%%%%%%%%%%%%%%%
\definecolor{ocre}{RGB}{243,102,25}
\definecolor{mygray}{RGB}{243,243,244}
\definecolor{deepGreen}{RGB}{26,111,0}
\definecolor{shallowGreen}{RGB}{235,255,255}
\definecolor{deepBlue}{RGB}{61,124,222}
\definecolor{shallowBlue}{RGB}{235,249,255}
%%%%%%%%%%%%%%%%%%%%%%%%%%%%%%%%%%%%%%%%%%%%%%%%%%%%%%%%%%%%%%%%%%%%%%%%%%%%%%%

%%%%%%%%%%%%%%%%%%%%%%%%%% Define an orangebox command %%%%%%%%%%%%%%%%%%%%%%%%
\newcommand\orangebox[1]{\fcolorbox{ocre}{mygray}{\hspace{1em}#1\hspace{1em}}}
%%%%%%%%%%%%%%%%%%%%%%%%%%%%%%%%%%%%%%%%%%%%%%%%%%%%%%%%%%%%%%%%%%%%%%%%%%%%%%%

%%%%%%%%%%%%%%%%%%%%%%%%%%%% English Environments %%%%%%%%%%%%%%%%%%%%%%%%%%%%%
\newtheoremstyle{mytheoremstyle}{3pt}{3pt}{\normalfont}{0cm}{\rmfamily\bfseries}{}{1em}{{\color{black}\thmname{#1}~\thmnumber{#2}}\thmnote{\,--\,#3}}
\newtheoremstyle{myproblemstyle}{3pt}{3pt}{\normalfont}{0cm}{\rmfamily\bfseries}{}{1em}{{\color{black}\thmname{#1}~\thmnumber{#2}}\thmnote{\,--\,#3}}
\theoremstyle{mytheoremstyle}
\newmdtheoremenv[linewidth=1pt,backgroundcolor=shallowGreen,linecolor=deepGreen,leftmargin=0pt,innerleftmargin=20pt,innerrightmargin=20pt,]{theorem}{Theorem}[section]
\theoremstyle{mytheoremstyle}
\newmdtheoremenv[linewidth=1pt,backgroundcolor=shallowBlue,linecolor=deepBlue,leftmargin=0pt,innerleftmargin=20pt,innerrightmargin=20pt,]{definition}{Definition}[section]
\theoremstyle{myproblemstyle}
\newmdtheoremenv[linecolor=black,leftmargin=0pt,innerleftmargin=10pt,innerrightmargin=10pt,]{problem}{Problem}[section]
%%%%%%%%%%%%%%%%%%%%%%%%%%%%%%%%%%%%%%%%%%%%%%%%%%%%%%%%%%%%%%%%%%%%%%%%%%%%%%%

%%%%%%%%%%%%%%%%%%%%%%%%%%%%%%% Plotting Settings %%%%%%%%%%%%%%%%%%%%%%%%%%%%%
\usepgfplotslibrary{colorbrewer}
\pgfplotsset{width=8cm,compat=1.9}
%%%%%%%%%%%%%%%%%%%%%%%%%%%%%%%%%%%%%%%%%%%%%%%%%%%%%%%%%%%%%%%%%%%%%%%%%%%%%%%

%%%%%%%%%%%%%%%%%%%%%%%%%%%%%%% Title & Author %%%%%%%%%%%%%%%%%%%%%%%%%%%%%%%%
\title{Algorithms}
\author{Patrick Chen}
\date{Feb 10, 2025}
%%%%%%%%%%%%%%%%%%%%%%%%%%%%%%%%%%%%%%%%%%%%%%%%%%%%%%%%%%%%%%%%%%%%%%%%%%%%%%%

\begin{document}
    \maketitle
    An algorithm is a finite sequence of precise instructions for solving a
    problem. Pseudocode is a high level description of the steps of an
    algorithm. The trace of an algorithm is following the steps of the algorithm
    and writing the state of the algorithm along the steps.
    \subsection*{Example: Maximum Element in List}
    \begin{verbatim}
        function find_max(L,n):
            max_value = L[1]
            for i = 2 to n do
                if L[i]> max_value then
                max_value = L[i]
            return max_value
    \end{verbatim}
    trace:
    \begin{verbatim}
         list: -1, 0, 5, 15, -3, 10
         max:  -1, 0, 5, 15, 15, 15
    \end{verbatim}

    \subsection*{Example: Linear Search Element in List}
    \begin{verbatim}
         function LinearSearch(L, n, target)
            for i = 1 to n do
                if L[i] = target then
                    return i
            return -1 #target not found
    \end{verbatim}

    \section*{Complexity}
    The complexity of an algorithm is roughly speaking, the amount of operations
    in relation to the size of the inputs.
    Let $f: \mathbb{R} \mapsto \mathbb{R}$ and $g:
            \mathbb{R} \mapsto \mathbb{R}$ be two functions.
    \begin{itemize}
        \item Big-$O$ notation: A function $f(x)$ is $O(g(x))$ if there exists a
            positive real constant $C$ and $k$ such that $|f(x)| \le C|g(x)|$
            for all $x\ge k$. Big $O$ notation concerns itself with the
            asymptotic growth rate of a function.
        \item Big-$\Omega$ notation: A function $f(x)$ is $\Omega(g(x))$ if
            there exists a positive real constant $C$ and $k$ such that $|f(x)|
            \ge C|g(x)|$ for all $x\ge k$.
        \item Big-$\Theta$ notation: We say that $f(x)$ is $\Theta(g(x))$ if
            $f(x)$ is both $O(g(x))$ and $\Omega(g(x))$. Equivalently, $f(x)$ is
            $\Theta(g(x))$ if there exists positive real constants $C_1$, $C_2$,
            and $k$ such for all $x>k$, $C_1|g(x)| \le |f(x)| \le C_2|g(x)|$
            Polynomials of degree $n$ is $\Theta(x^n)$.
    \end{itemize}

    \subsection*{Proving Complexity With Limits}
    By the definition of the limit to infinity, for all $\varepsilon>0$, there
    exists a $k>0$ such that for all $x>k$, the absolute difference of the
    function at $x$ and the limit $L$ is less than $\varepsilon$.
    \[
         x>k \quad\Rightarrow\quad |f(x)-L|<\varepsilon
    \]
    Thus
    \[
        x>k \quad\Rightarrow\quad L-\varepsilon<f(x)<L+\varepsilon
    \]
    Suppose the limit of the ratio of two functions $f(x)$ and $g(x)$ is defined
    and approaches a finite value.
    \[
        \lim_{x\to \infty} \Big|\frac{f(x)}{g(x)}\Big| = L
    \]
    Applying the definition of the limit

    \begin{align*}
        x>k \quad&\Rightarrow\quad L-\varepsilon< \Big|\frac{f(x)}{g(x)}\Big| <L+\varepsilon \\
        x>k \quad&\Rightarrow\quad (L-\varepsilon)|g(x)| < |f(x)| < (L+\varepsilon)|g(x)| \\
    \end{align*}
    Taking $\varepsilon= \frac{L}{2}$
    \[
        x>k \quad\Rightarrow\quad \Big(\frac{1}{2} L\Big)|g(x)| < |f(x)| < \Big(\frac{3}{2} L\Big)|g(x)| \\
    \]
    Since $L$ comes from the limit of the absolute value of some function,
    $L\ge0$. If the limit exists and is a finite positive value, then $f(x)$ is
    by definition $\Omega(g(x))$ and vise versa.


    \subsection*{Special functions}
    \begin{itemize}
        \item $n!$ is $O(n^n)$
            \[
                n! = n (n-1) \dots (2)(1) \le
                \underbrace{(n)(n)\dots(n)}_n = n^n
            \]
        \item $n^b$ is $O(c^n)$
        \item $c^n$ is $O(n!)$
        \item Big-$O$ of common functions in order
            \[
                1,\quad \log n,\quad n,\quad n\log n,\quad n^2,\quad 2^n,\quad n!
            \]
    \end{itemize}

    \subsection*{Types of Complexity for Algorithms}
    Space complexity is the amount of storage capacity required for an algorithm
    to run. Time complexity is the amount of time required for a algorithm to
    run. The worst case complexity of an algorithm is the largest amount of
    operations used to the problem over all possible inputs of a given size. The
    average case complexity is the average amount of operations for all inputs
    of a given size.
\end{document}
