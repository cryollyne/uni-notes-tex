%%%%%%%%%%%%%%%%%%%%%%%%%%%%% Define Article %%%%%%%%%%%%%%%%%%%%%%%%%%%%%%%%%%
\documentclass{article}
%%%%%%%%%%%%%%%%%%%%%%%%%%%%%%%%%%%%%%%%%%%%%%%%%%%%%%%%%%%%%%%%%%%%%%%%%%%%%%%

%%%%%%%%%%%%%%%%%%%%%%%%%%%%% Using Packages %%%%%%%%%%%%%%%%%%%%%%%%%%%%%%%%%%
\usepackage{ebproof}
\usepackage{geometry}
\usepackage{graphicx}
\usepackage{amssymb}
\usepackage{amsmath}
\usepackage{amsthm}
\usepackage{empheq}
\usepackage{mdframed}
\usepackage{booktabs}
\usepackage{lipsum}
\usepackage{graphicx}
\usepackage{color}
\usepackage{psfrag}
\usepackage{pgfplots}
\usepackage{bm}
%%%%%%%%%%%%%%%%%%%%%%%%%%%%%%%%%%%%%%%%%%%%%%%%%%%%%%%%%%%%%%%%%%%%%%%%%%%%%%%

% Other Settings

%%%%%%%%%%%%%%%%%%%%%%%%%% Page Setting %%%%%%%%%%%%%%%%%%%%%%%%%%%%%%%%%%%%%%%
\geometry{a4paper}

%%%%%%%%%%%%%%%%%%%%%%%%%% Define some useful colors %%%%%%%%%%%%%%%%%%%%%%%%%%
\definecolor{ocre}{RGB}{243,102,25}
\definecolor{mygray}{RGB}{243,243,244}
\definecolor{deepGreen}{RGB}{26,111,0}
\definecolor{shallowGreen}{RGB}{235,255,255}
\definecolor{deepBlue}{RGB}{61,124,222}
\definecolor{shallowBlue}{RGB}{235,249,255}
%%%%%%%%%%%%%%%%%%%%%%%%%%%%%%%%%%%%%%%%%%%%%%%%%%%%%%%%%%%%%%%%%%%%%%%%%%%%%%%

%%%%%%%%%%%%%%%%%%%%%%%%%% Define an orangebox command %%%%%%%%%%%%%%%%%%%%%%%%
\newcommand\orangebox[1]{\fcolorbox{ocre}{mygray}{\hspace{1em}#1\hspace{1em}}}
%%%%%%%%%%%%%%%%%%%%%%%%%%%%%%%%%%%%%%%%%%%%%%%%%%%%%%%%%%%%%%%%%%%%%%%%%%%%%%%

%%%%%%%%%%%%%%%%%%%%%%%%%%%% English Environments %%%%%%%%%%%%%%%%%%%%%%%%%%%%%
\newtheoremstyle{mytheoremstyle}{3pt}{3pt}{\normalfont}{0cm}{\rmfamily\bfseries}{}{1em}{{\color{black}\thmname{#1}~\thmnumber{#2}}\thmnote{\,--\,#3}}
\newtheoremstyle{myproblemstyle}{3pt}{3pt}{\normalfont}{0cm}{\rmfamily\bfseries}{}{1em}{{\color{black}\thmname{#1}~\thmnumber{#2}}\thmnote{\,--\,#3}}
\theoremstyle{mytheoremstyle}
\newmdtheoremenv[linewidth=1pt,backgroundcolor=shallowGreen,linecolor=deepGreen,leftmargin=0pt,innerleftmargin=20pt,innerrightmargin=20pt,]{theorem}{Theorem}[section]
\theoremstyle{mytheoremstyle}
\newmdtheoremenv[linewidth=1pt,backgroundcolor=shallowBlue,linecolor=deepBlue,leftmargin=0pt,innerleftmargin=20pt,innerrightmargin=20pt,]{definition}{Definition}[section]
\theoremstyle{myproblemstyle}
\newmdtheoremenv[linecolor=black,leftmargin=0pt,innerleftmargin=10pt,innerrightmargin=10pt,]{problem}{Problem}[section]
%%%%%%%%%%%%%%%%%%%%%%%%%%%%%%%%%%%%%%%%%%%%%%%%%%%%%%%%%%%%%%%%%%%%%%%%%%%%%%%

%%%%%%%%%%%%%%%%%%%%%%%%%%%%%%% Plotting Settings %%%%%%%%%%%%%%%%%%%%%%%%%%%%%
\usepgfplotslibrary{colorbrewer}
\pgfplotsset{width=8cm,compat=1.9}
%%%%%%%%%%%%%%%%%%%%%%%%%%%%%%%%%%%%%%%%%%%%%%%%%%%%%%%%%%%%%%%%%%%%%%%%%%%%%%%

%%%%%%%%%%%%%%%%%%%%%%%%%%%%%%% Title & Author %%%%%%%%%%%%%%%%%%%%%%%%%%%%%%%%
\title{Inference}
\author{Patrick Chen}
\date{Jan 13, 2025}
%%%%%%%%%%%%%%%%%%%%%%%%%%%%%%%%%%%%%%%%%%%%%%%%%%%%%%%%%%%%%%%%%%%%%%%%%%%%%%%

\begin{document}
    \maketitle
    \section*{Inference}
    We denote a inference as one (or more) premises leading to a conclusion. If
    $p_1,\dots,p_n$ are premises and $q$ is a conclusion, then an inference is
    $(p_1 \wedge \dots \wedge p_n) \rightarrow q$.

    \[
        \begin{prooftree}
            \hypo{p_1}
            \hypo{\dots}
            \hypo{p_n}
            \infer3{q}
        \end{prooftree}
    \]

    If there are propositional functions, then 
    \[
        \begin{prooftree}
            \hypo{\forall x P(x)}
            \infer1{P(c) \text{ for all $c$ in the domain}}
        \end{prooftree}
        \quad
        \begin{prooftree}
            \hypo{\exists x P(x)}
            \infer1{P(c) \text{ for some $c$ in the domain}}
        \end{prooftree}
    \]

    \begin{itemize}
        \item Modus Ponens
            \[
                \begin{prooftree}
                    \hypo{p \rightarrow q}
                    \hypo {p}
                    \infer2{q}
                \end{prooftree}
            \]
        \item Modus Tollens
            \[
                \begin{prooftree}
                    \hypo{p \rightarrow q}
                    \hypo{\neg q}
                    \infer2{\neg p}
                \end{prooftree}
            \]
        \item Hypothetical Syllogism
            \[
                \begin{prooftree}
                    \hypo{p \rightarrow q}
                    \hypo{q \rightarrow r}
                    \infer2{p \rightarrow r}
                \end{prooftree}
            \]
        \item Disjunctive Syllogism
            \[
                \begin{prooftree}
                    \hypo{p \vee q}
                    \hypo{\neg p}
                    \infer2{q}
                \end{prooftree}
            \]
        \item Addition
            \[
                \begin{prooftree}
                    \hypo{p}
                    \infer1{p \vee q}
                \end{prooftree}
            \]
        \item Simplification
            \[
                \begin{prooftree}
                    \hypo{p \wedge q}
                    \infer1{q}
                \end{prooftree}
            \]
        \item Conjunction
            \[
                \begin{prooftree}
                    \hypo{p}
                    \hypo{q}
                    \infer2{p \wedge q}
                \end{prooftree}
            \]
        \item Resolution
            \[
                \begin{prooftree}
                    \hypo{p \vee q}
                    \hypo{\neg p \vee r}
                    \infer2{q \vee r}
                \end{prooftree}
            \]
    \end{itemize}

    \subsection*{Example}
    \[
        \begin{prooftree}
            \hypo{\neg p \wedge q}
            \hypo{r \rightarrow p}
            \hypo{\neg r \rightarrow s}
            \hypo{s \rightarrow t}
            \infer4{t}
        \end{prooftree}
    \]
    Since $\neg p$ is true from $r \rightarrow p$ we see that $\neg r$ must be
    true.

    \[
        \begin{prooftree}
            \hypo{\neg p \wedge q}
            \hypo{r \rightarrow p}
            \infer2{\neg r}
        \end{prooftree}
    \]
    From $\neg r \rightarrow s$ and $\neg r$, we see that $s$ must be true
    (modus ponens).

    \[
        \begin{prooftree}
            \hypo{\neg r}
            \hypo{\neg r \rightarrow s}
            \infer2{s}
        \end{prooftree}
    \]
    From $s \rightarrow t$ and $s$ we conclude that $t$ must be true.

    \[
        \begin{prooftree}
            \hypo{s}
            \hypo{s \rightarrow t}
            \infer2{t}
        \end{prooftree}
    \]

    \subsection*{Example 2}
    \[
        \begin{prooftree}
            \hypo{(p \wedge q) \vee r}
            \hypo{r \rightarrow s}
            \infer2{p \vee s}
        \end{prooftree}
    \]
    If $r$ is false, then $(p \wedge q) \vee r$, $p \wedge q$ must be T.
    Therefore, $p$ is true and hence $p \vee s$ is true.
    \[
        \begin{prooftree}
            \hypo{\neg r}
            \hypo{(p \wedge q) \vee r}
            \infer2{p \wedge q}
            \infer1{p}
            \infer1{p \vee s}
        \end{prooftree}
    \]
    If $r$ is true, then $s$ is true from $r \rightarrow s$, therefore $p \vee
    s$ is true.
    \[
        \begin{prooftree}
            \hypo{r}
            \hypo{r \rightarrow s}
            \infer2{s}
            \infer1{p \vee s}
        \end{prooftree}
    \]

    \subsection*{Example 3}
    Determine whether the following argument is valid.
    \[
        \begin{prooftree}
            \hypo{p \rightarrow r}
            \hypo{q \rightarrow r}
            \hypo{\neg r}
            \infer3{p \vee q}
        \end{prooftree}
    \]
    This argument is not valid.
    \[
        \begin{prooftree}
            \hypo{p \rightarrow r}
            \hypo{\neg r}
            \infer2{\neg p}
            \hypo{q \rightarrow r}
            \hypo{\neg r}
            \infer2{\neg q}
            \infer2{\neg p \wedge \neg q}
            \infer1{\neg (p \vee q)}
        \end{prooftree}
    \]

    \section*{Proofs}
    \subsection*{Nomenclature}
    \begin{itemize}
        \item Theorem: and important mathematical result
        \item Proposition: less important mathematical result
        \item Lemma a result that is needed to prove a theorem
        \item Corollary: a result that directly follows from a theorem
    \end{itemize}

    \subsection*{Methods of Proofs}
    \begin{itemize}
        \item Direct proof: use all lines of reasoning. In a direct proof, we
            show that $P(c) \rightarrow Q(c)$ for any arbitrary $c$ in the
            domain. We start with a hypothesis $P(c)$ and work to show that
            $Q(c)$ is true.
        \item Proof by contraposition: proving the contraposition. Since $P(c)
            \rightarrow Q(c) \equiv \neg Q(c) \rightarrow \neg P(c)$, we can
            prove that $\neg Q(c) \rightarrow \neg P(c)$.
        \item Proof by contradiction: assume that the theorem is false, then use
            lines of reasoning until there is a contradiction. If we wish to
            prove that $P(c) \rightarrow Q(c)$, for some $c$ in the domain, we
            want to show that $P(c) \wedge \neg Q(c)$ is false.
        \item Proof by cases: proving all cases of a theorem. If we have as
            statement that can be expressed as multiple cases $P(c)\equiv P_1(c)
            \vee P_2(c) \vee \dots P_n(c)$, then we need to show that all
            possible cases is true, or equivalently $(P_1(c) \rightarrow Q(c))
            \wedge \dots \wedge (P_n(c) \rightarrow Q(c))$.
    \end{itemize}

    \subsection*{Example}
    $P(n) = $ $n$ is odd\\
    $Q(n) = $ $n^2$ is odd\\
    Prove that $P(n) \rightarrow Q(n)$ \\
    If $n$ is odd, then $n=2k+1$ for some $k\in \mathbb{Z}$. \\
    $n^2 = (2k+1)^2 = 4k^2+4k+1 = 2(2k^2+2k) + 1 = 2k'+1$

    \subsection*{Example 2}
    For integers $m$ and $n$, show that if $nm$ is even, then either $m$ or $n$
    is even. \\
    Using contraposition, if $m$ and $n$ is odd, then $mn$ is odd.
    \begin{align*}
        m &= 2k_1 + 1 \\
        n &= 2k_2 + 1 \\
        mn &= (2k_1 + 1) (2k_2 + 1) \\
        &= 4k_1k_2 + 2k_1 + 2k_2 + 1 \\
        &= 2(2k_1k_2 + k_1 + k_2) + 1 \\
        &= 2k' + 1 \\
        \text{where } k' &= 2k_1k_2 + k_1 + k_2
    \end{align*}


    \subsection*{Example 3}
    For all real number $x$, prove that $x \le |x|$. \\
    $P(x)\equiv\forall x (x\le|x|)\equiv (\forall x<0\ (x\le |x|)) \wedge
    (\forall x \ge 0\ (x\le|x|))$ \\
    If $x<0$ then negative $<$ positive. \\
    If $x\ge 0$, then $|x| = x \le x$
    therefore $x\le |x|$

    \subsection*{Example 4}
    Prove that $\sqrt{2}$ is irrational.
    Suppose that $\sqrt{2}$ is rational. Then:
    \begin{align*}
        \sqrt{2} = \frac{a}{b}
    \end{align*}
    where $a$ and $b$ are coprime (no common factors).
    \begin{align*}
        b\sqrt{2} &= a \\
        2b^2 &= a^2
    \end{align*}
    Since $2b^2$ is even, $a^2$ is even, thus $a=2k$.
    \begin{align*}
        2b^2 &= (2k)^2 \\
             &= 4k^2 \\
        b^2  &= 2k^2
    \end{align*}
    Thus, $b$ is also even and have a common factor of 2. This is a
    contradiction, therefore the assumption is false. Thus, $\sqrt{2}$ must be
    irrational.

\end{document}
