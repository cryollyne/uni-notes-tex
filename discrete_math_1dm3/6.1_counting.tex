%%%%%%%%%%%%%%%%%%%%%%%%%%%%% Define Article %%%%%%%%%%%%%%%%%%%%%%%%%%%%%%%%%%
\documentclass{article}
%%%%%%%%%%%%%%%%%%%%%%%%%%%%%%%%%%%%%%%%%%%%%%%%%%%%%%%%%%%%%%%%%%%%%%%%%%%%%%%

%%%%%%%%%%%%%%%%%%%%%%%%%%%%% Using Packages %%%%%%%%%%%%%%%%%%%%%%%%%%%%%%%%%%
\usepackage{geometry}
\usepackage{graphicx}
\usepackage{amssymb}
\usepackage{amsmath}
\usepackage{amsthm}
\usepackage{empheq}
\usepackage{mdframed}
\usepackage{booktabs}
\usepackage{lipsum}
\usepackage{graphicx}
\usepackage{color}
\usepackage{psfrag}
\usepackage{pgfplots}
\usepackage{bm}
%%%%%%%%%%%%%%%%%%%%%%%%%%%%%%%%%%%%%%%%%%%%%%%%%%%%%%%%%%%%%%%%%%%%%%%%%%%%%%%

% Other Settings

%%%%%%%%%%%%%%%%%%%%%%%%%% Page Setting %%%%%%%%%%%%%%%%%%%%%%%%%%%%%%%%%%%%%%%
\geometry{a4paper}

%%%%%%%%%%%%%%%%%%%%%%%%%% Define some useful colors %%%%%%%%%%%%%%%%%%%%%%%%%%
\definecolor{ocre}{RGB}{243,102,25}
\definecolor{mygray}{RGB}{243,243,244}
\definecolor{deepGreen}{RGB}{26,111,0}
\definecolor{shallowGreen}{RGB}{235,255,255}
\definecolor{deepBlue}{RGB}{61,124,222}
\definecolor{shallowBlue}{RGB}{235,249,255}
%%%%%%%%%%%%%%%%%%%%%%%%%%%%%%%%%%%%%%%%%%%%%%%%%%%%%%%%%%%%%%%%%%%%%%%%%%%%%%%

%%%%%%%%%%%%%%%%%%%%%%%%%% Define an orangebox command %%%%%%%%%%%%%%%%%%%%%%%%
\newcommand\orangebox[1]{\fcolorbox{ocre}{mygray}{\hspace{1em}#1\hspace{1em}}}
%%%%%%%%%%%%%%%%%%%%%%%%%%%%%%%%%%%%%%%%%%%%%%%%%%%%%%%%%%%%%%%%%%%%%%%%%%%%%%%

%%%%%%%%%%%%%%%%%%%%%%%%%%%% English Environments %%%%%%%%%%%%%%%%%%%%%%%%%%%%%
\newtheoremstyle{mytheoremstyle}{3pt}{3pt}{\normalfont}{0cm}{\rmfamily\bfseries}{}{1em}{{\color{black}\thmname{#1}~\thmnumber{#2}}\thmnote{\,--\,#3}}
\newtheoremstyle{myproblemstyle}{3pt}{3pt}{\normalfont}{0cm}{\rmfamily\bfseries}{}{1em}{{\color{black}\thmname{#1}~\thmnumber{#2}}\thmnote{\,--\,#3}}
\theoremstyle{mytheoremstyle}
\newmdtheoremenv[linewidth=1pt,backgroundcolor=shallowGreen,linecolor=deepGreen,leftmargin=0pt,innerleftmargin=20pt,innerrightmargin=20pt,]{theorem}{Theorem}[section]
\theoremstyle{mytheoremstyle}
\newmdtheoremenv[linewidth=1pt,backgroundcolor=shallowBlue,linecolor=deepBlue,leftmargin=0pt,innerleftmargin=20pt,innerrightmargin=20pt,]{definition}{Definition}[section]
\theoremstyle{myproblemstyle}
\newmdtheoremenv[linecolor=black,leftmargin=0pt,innerleftmargin=10pt,innerrightmargin=10pt,]{problem}{Problem}[section]
%%%%%%%%%%%%%%%%%%%%%%%%%%%%%%%%%%%%%%%%%%%%%%%%%%%%%%%%%%%%%%%%%%%%%%%%%%%%%%%

%%%%%%%%%%%%%%%%%%%%%%%%%%%%%%% Plotting Settings %%%%%%%%%%%%%%%%%%%%%%%%%%%%%
\usepgfplotslibrary{colorbrewer}
\pgfplotsset{width=8cm,compat=1.9}
%%%%%%%%%%%%%%%%%%%%%%%%%%%%%%%%%%%%%%%%%%%%%%%%%%%%%%%%%%%%%%%%%%%%%%%%%%%%%%%

%%%%%%%%%%%%%%%%%%%%%%%%%%%%%%% Title & Author %%%%%%%%%%%%%%%%%%%%%%%%%%%%%%%%
\title{Counting}
\author{Patrick Chen}
\date{March 17, 2025}
%%%%%%%%%%%%%%%%%%%%%%%%%%%%%%%%%%%%%%%%%%%%%%%%%%%%%%%%%%%%%%%%%%%%%%%%%%%%%%%

\begin{document}
    \maketitle
    \section*{Product Rule}
    Suppose a procedure $A$ can be divided into $k$ tasks $A_k$. If there are
    $n_k$ ways to do task $k$, then the amount of ways to complete the procedure
    is the product of the ways to do each task.
    \[
        |A_1 \times A_2 \times \dots \times A_k| = |A_1|\ |A_2|\dots|A_k|
    \]

    \subsection*{Counting functions}
    If set $A$ has $n$ elements and set $B$ has $m$ elements, then each elements
    in $A$ can be mapped to an element in $B$, leading to $m$ choices. With $n$
    total elements, the amount of functions that map from $A$ to $B$ is the
    \[
        | f: A \mapsto B | = \underbrace{m\times m\times \dots\times m}_{n} = m^n
    \]
    For the set of injective functions, the first choice is $m$ and the second
    is $(m-1)$.
    \[
        | f: A \hookrightarrow B | = \underbrace{m(m-1)(m-2)\dots(m-n+1)}_n
        = \frac{m!}{(m-n)!}
    \]

    \section*{Sum Rule}
    Suppose a procedure can be done in $k$ ways. If the kth way requires $n_k$
    without any overlap, then the amount of ways to do the procedure is the sum
    of all the ways.
    \[
        | A_1 \cup A_2 \cup \dots \cup A_k | = |A_1|+|A_2|+\dots+|A_k|
    \]
    If the intersection between $A_i$ and $A_j$ is not empty, then the
    intersection of the two sets is double counted. Thus to get an accurate
    count, the intersection must be subtracted.
    \[
        |A \cup B| = |A| + |B| - |A\cap B|
    \]
    When there are more than two sets, all permutations of the sets have to be
    added or subtracted. If there is an even amount of sets in the intersection,
    the size will have to be subtracted, otherwise it will have to be added.
    \begin{align*}
        |A\cup B\cup C| = &\quad (|A| + |B| + |C|) \\
                        &- (|A\cap B| + |B\cap C| + |C\cap A|) \\
                        &+ (|A\cap B\cap C|)
    \end{align*}

    \subsection*{Example}
    How many hexadecimal strings are there of length 2 or 3.
    \begin{align*}
        |S_{[2,3]}| &= |S_2| + |S_3| \\
                    &= 16^2 + 16^3
    \end{align*}

    \subsection*{Example 2}
    How many bit strings of length 8 either start with $1$ of end with $00$? \\
    Let $A$ be the set of strings that start with $1$ \\
    Let $B$ be the set of  strings that end with $00$ \\
    The set $A$ will be in the form 1\_\_\_\_\_\_\_ and will have 7 binary choices \\
    The set $B$ will be in the form \_\_\_\_\_\_00 and will have 6 binary choices \\
    The set $A\cup B$ will be in the form 1\_\_\_\_\_00 and will have 5 binary choices
    \begin{align*}
        |A\cup B| &= |A| + |B| - |A \cup B| \\
                  &= 2^7 + 2^6 - 2^5
    \end{align*}

    \section*{Division rule}
    Suppose a task can be done in $n$ ways. For each way, there are $d$
    equivalent ways. There are $\frac{n}{d}$ distinct ways to do the task.

    \subsection*{Example}
    How many different ways to arrange $1,1,2,3,4$ in a sequence? \\
    There are $5!$ ways to arrange five distinct values, but there are $2!$
    equivalent ways to arrange them because there are 2 ones in the sequence.
    Thus there are $\frac{5!}{2!}=60$ ways to arrange them.

    \subsection*{Example 2}
    How many ways are there to seat 5 people around a circular table? Rotations
    around to circular table are not considered distinct. \\
    There are $5!$ ways to seat 5 people in a circular table and there are 5
    possible rotations, thus there are $\frac{5!}{5}=4!=24$ distinct ways to seat 5
    people around a circular table.

    \section*{Pigeonhole Principle}
    If $k$ is a positive integer and $k+1$ more objects are place into $k$
    boxes, then there is at least one box containing two or more of the objects.
    Formally, if $A$ and $B$ be two countable sets such that  $|A| > |B|$, then
    for all $f: A \mapsto B$, there are $a,b\in A$ such that $f(a)=f(b)$.

    % \subsection*{Theorem}
    % Every sequence of $n^2+1$ distinct real numbers containing a subsequence of
    % length $n+1$, that is either strictly increasing or strictly decreasing
    % order.

    \subsection*{Generalized Pigeonhole Principle}
    If $n$ objects are placed into $k$ boxes, then there is at least one box
    containing at least $\lceil \frac{n}{k} \rceil$ objects.
    \[
        \text{argmin}\Big[n\ |\ \Big\lceil \frac{n}{k} \Big\rceil = c\Big]
    \]

    \subsection*{Example}
    How many cards must be selected from a standard deck of 52 cards to
    guarantee that at least three cards of the same suit are selected
    \[
        \text{argmin}\Big[n\ |\ \Big\lceil \frac{n}{4} \Big\rceil = 3\Big] = 9
    \]

    \subsection*{Example 2}
    Show that among any $(n+1)$ positive integers (set $A$) not exceeding $2n$,
    there must be an integer that divides another integer.

    \begin{align*}
        \text{let } a_i &= 2^{u_i} q_i \text{ for some } u_i,q_i \\
        \text{where } q_i &\text{ is odd}, \\
        j&\in\{1,2,\dots,n+1\} \\
        u_i&\in \mathbb{N}_0
    \end{align*}
    Since $a_j \le 2n$, then $q_j\le 2n$ thus the set $(q_1,\dots,q_{n+1})$ is
    $n+1$ odd numbers. \\
    Since there are $n$ odd numbers less than $2n$, \\
    by the pigeonhole principle $q_r=q_s$ for some $1\le r\le n+1$ and $1\le s\le n+1$. \\
    Thus $a_r=2^{u_r}q_r, a_s=2^{u_s}q_s$. \\
    If $u_r\le u_s$ then $a_r\ |\ a_s$ otherwise $a_s\ |\ a_r$
    \section*{Permutations}
    A permutation is an ordered arrangement of a set of objects. In general, any
    set with size $n$ has $n!$ permutations. The r-permutation of a set $A$ is a
    permutation of a subset of $A$ of size $r$. The total amount of
    r-permutations of size $n$ is $\frac{n!}{(n-r)!}$
    \[
        P(n,r) = \frac{n!}{(n-r)!}
    \]

    \subsection*{Example}
    For the set $A$, how many permutations contain the string
    $abc$?
    \[
        A=\{a,b,c,d,e,f,g\}
    \]
    This is equivalent to asking how many permutations are there when $abc$ is
    considered one element.
    \[
        A' = \{abc,d,e,f,g\}
    \]
    Thus there are $5!$ permutations.

    \section*{Combinations}
    An r-combination of elements of a set $A$ is an unordered selection of r
    elements from $A$.
    \[
        C(n,r) = \binom{n}{r} = \frac{n!}{r!(n-r)!}
    \]

    \subsection*{Example 1}
    7 women and 9 men are on the faculty in cs department, how many ways are
    there to select a committee of 5 members such that there is at least one
    woman on the committee \\
    committee with no women: $\binom{9}{5}$ \\
    arbitrary committee: $\binom{16}{5}$
    \[
        \binom{16}{5} - \binom{9}{5}
    \]

    \subsection*{Example 2}
    7 women and 9 men are on the faculty in cs department, how many ways are
    there to select a committee of 5 members such that there is at least one
    woman and one man on the committee \\
    committee with no women: $\binom{9}{5}$ \\
    committee with no men: $\binom{9}{5}$ \\
    arbitrary committee: $\binom{16}{5}$
    \[
        {\binom{16}{5}} - {\binom{9}{5}} - {\binom{7}{5}}
    \]

    \subsection*{Example 3}
    Prove that for $n\ge 0$ and $0\le r\le n$ that $C(n,r)=C(n,n-r)$
    \begin{align*}
        C(n,n-r) &= \frac{n!}{(n-r)!(n-(n-r))!} \\
                 &= \frac{n!}{(n-r)!r!} \\
                 &= C(n,r)
    \end{align*}

    \section*{Binomial Theorem}
    \begin{align*}
        (x+y)^n = (x+y)(x+y)\dots(x+y)
    \end{align*}
    All terms in the product are of the form $x^jy^{n-j}$ where $0 \le j \le n$.
    There are $\binom{n }{n-j}$ ways to choose $(n-j)$ y terms from $n$ factors.
    Therefore, the coefficient of $x^jy^{j-n}$ is $\binom{n }{n-j}$
    \begin{align*}
        (x+y)^n = 
        \binom{n}{0}x^n+
        \binom{n}{1}x^{n-1}y+
        \dots+
        \binom{n}{n-1}xy^{n-1}+
        {\binom{n}{n}}y^n =
        \sum_{j=0}^{n} {\binom{n}{j}}x^{n-j}y^j
    \end{align*}

    \subsection*{Interesting Conclusions}
    \begin{itemize}
        \item $x=y=1$ leads to
            \begin{align*}
                (x+y)^n &= \sum_{j=0}^{n} \binom{n}{j}1^{n-j}1^j \\
                2^n &= \sum_{j=0}^{n} \binom{n}{j}
            \end{align*}
            The sum of chooses is equal to a base two exponential.
        \item $x=1$, $y=-1$ leads to
            \begin{align*}
                (x+y)^n &= \sum_{j=0}^{n} \binom{n}{j}1^{n-j}(-1)^j \\
                0 &= \sum_{j=0}^{n} \binom{n}{j}(-1)^j \\
            \end{align*}
            The alternating sum of chooses is equal to zero.
    \end{itemize}

    \subsection*{Pascel's identity}
    \begin{align*}
        \binom{n}{k} = \binom{n-1}{k-1} + \binom{n-1}{k}
    \end{align*}

    \begin{align*}
        \binom{n-1}{k-1} + \binom{n-1}{k}
        &= \frac{(n-1)!}{(k-1)!(n-k)!} + \frac{(n-1)!}{k!(n-k-1)!} \\
        &= \frac{k(n-1)! + (n-k)(n-1)!}{k!(n-k)!} \\
        &= \frac{(k + (n-k))(n-1)!}{k!(n-k)!} \\
        &= \frac{n(n-1)!}{k!(n-k)!} \\
        &= \frac{n!}{k!(n-k)!} \\
        &= \binom{n}{k}
    \end{align*}

    \subsection*{Pascals Triangle}
    Pascal's triangle is defined as follows
    \begin{gather*}
        \binom{0}{0} \\
        \binom{1}{0}
        \binom{1}{1} \\
        \binom{2}{0}
        \binom{2}{1}
        \binom{2}{2} \\
        \binom{3}{0}
        \binom{3}{1}
        \binom{3}{2}
        \binom{3}{3} \\
        \vdots
    \end{gather*}
    Using pascal's identity, its possible to express the next term as a sum of
    the previous terms.
    \begin{gather*}
        \binom{n-1}{k-1} \binom{n}{k-1} \\
        \binom{n}{k}
    \end{gather*}

    \subsection*{Vandermonde's Identity}
    For $r\le \min(m,n)$
    \[
        \binom{m+n}{r} = \sum_{k=0}^{r} \binom{m}{r-k}\binom{n}{k}
    \]
    The left hand side is choosing $r$ from $m+n$ \\
    The right hand side is choosing $k$ objects from the second bin and the
    remaining objects from the right bin

    \subsection*{Identity 2}
    For $r \le n$
    \[
        \binom{n+1}{r+1} = \sum_{k=r}^{n} \binom{k}{r}
    \]
    \begin{align*}
        \binom{k+1}{r+1} &= \binom{k}{r} + \binom{k}{r+1} & \text{Pascal's Identity} \\
        \binom{k}{r} &= \binom{k+1}{r+1} - \binom{k}{r+1} \\
        \sum_{k=r}^{n} \binom{k}{r} &= \sum_{k=r}^{n} \Bigg( \binom{k+1}{r+1} - \binom{k}{r+1}\Bigg) \\
        \sum_{k=r}^{n} \binom{k}{r} &= \binom{n+1}{r+1} - \binom{r}{r+1} & \text{Alternating Series} \\
        \sum_{k=r}^{n} \binom{k}{r} &= \binom{n+1}{r+1}
    \end{align*}


\end{document}
