%%%%%%%%%%%%%%%%%%%%%%%%%%%%% Define Article %%%%%%%%%%%%%%%%%%%%%%%%%%%%%%%%%%
\documentclass{article}
%%%%%%%%%%%%%%%%%%%%%%%%%%%%%%%%%%%%%%%%%%%%%%%%%%%%%%%%%%%%%%%%%%%%%%%%%%%%%%%

%%%%%%%%%%%%%%%%%%%%%%%%%%%%% Using Packages %%%%%%%%%%%%%%%%%%%%%%%%%%%%%%%%%%
\usepackage{geometry}
\usepackage{graphicx}
\usepackage{amssymb}
\usepackage{amsmath}
\usepackage{amsthm}
\usepackage{empheq}
\usepackage{mdframed}
\usepackage{booktabs}
\usepackage{lipsum}
\usepackage{graphicx}
\usepackage{color}
\usepackage{psfrag}
\usepackage{pgfplots}
\usepackage{bm}
%%%%%%%%%%%%%%%%%%%%%%%%%%%%%%%%%%%%%%%%%%%%%%%%%%%%%%%%%%%%%%%%%%%%%%%%%%%%%%%

% Other Settings

%%%%%%%%%%%%%%%%%%%%%%%%%% Page Setting %%%%%%%%%%%%%%%%%%%%%%%%%%%%%%%%%%%%%%%
\geometry{a4paper}

%%%%%%%%%%%%%%%%%%%%%%%%%% Define some useful colors %%%%%%%%%%%%%%%%%%%%%%%%%%
\definecolor{ocre}{RGB}{243,102,25}
\definecolor{mygray}{RGB}{243,243,244}
\definecolor{deepGreen}{RGB}{26,111,0}
\definecolor{shallowGreen}{RGB}{235,255,255}
\definecolor{deepBlue}{RGB}{61,124,222}
\definecolor{shallowBlue}{RGB}{235,249,255}
%%%%%%%%%%%%%%%%%%%%%%%%%%%%%%%%%%%%%%%%%%%%%%%%%%%%%%%%%%%%%%%%%%%%%%%%%%%%%%%

%%%%%%%%%%%%%%%%%%%%%%%%%% Define an orangebox command %%%%%%%%%%%%%%%%%%%%%%%%
\newcommand\orangebox[1]{\fcolorbox{ocre}{mygray}{\hspace{1em}#1\hspace{1em}}}
%%%%%%%%%%%%%%%%%%%%%%%%%%%%%%%%%%%%%%%%%%%%%%%%%%%%%%%%%%%%%%%%%%%%%%%%%%%%%%%

%%%%%%%%%%%%%%%%%%%%%%%%%%%% English Environments %%%%%%%%%%%%%%%%%%%%%%%%%%%%%
\newtheoremstyle{mytheoremstyle}{3pt}{3pt}{\normalfont}{0cm}{\rmfamily\bfseries}{}{1em}{{\color{black}\thmname{#1}~\thmnumber{#2}}\thmnote{\,--\,#3}}
\newtheoremstyle{myproblemstyle}{3pt}{3pt}{\normalfont}{0cm}{\rmfamily\bfseries}{}{1em}{{\color{black}\thmname{#1}~\thmnumber{#2}}\thmnote{\,--\,#3}}
\theoremstyle{mytheoremstyle}
\newmdtheoremenv[linewidth=1pt,backgroundcolor=shallowGreen,linecolor=deepGreen,leftmargin=0pt,innerleftmargin=20pt,innerrightmargin=20pt,]{theorem}{Theorem}[section]
\theoremstyle{mytheoremstyle}
\newmdtheoremenv[linewidth=1pt,backgroundcolor=shallowBlue,linecolor=deepBlue,leftmargin=0pt,innerleftmargin=20pt,innerrightmargin=20pt,]{definition}{Definition}[section]
\theoremstyle{myproblemstyle}
\newmdtheoremenv[linecolor=black,leftmargin=0pt,innerleftmargin=10pt,innerrightmargin=10pt,]{problem}{Problem}[section]
%%%%%%%%%%%%%%%%%%%%%%%%%%%%%%%%%%%%%%%%%%%%%%%%%%%%%%%%%%%%%%%%%%%%%%%%%%%%%%%

%%%%%%%%%%%%%%%%%%%%%%%%%%%%%%% Plotting Settings %%%%%%%%%%%%%%%%%%%%%%%%%%%%%
\usepgfplotslibrary{colorbrewer}
\pgfplotsset{width=8cm,compat=1.9}
%%%%%%%%%%%%%%%%%%%%%%%%%%%%%%%%%%%%%%%%%%%%%%%%%%%%%%%%%%%%%%%%%%%%%%%%%%%%%%%

%%%%%%%%%%%%%%%%%%%%%%%%%%%%%%% Title & Author %%%%%%%%%%%%%%%%%%%%%%%%%%%%%%%%
\title{Induction and Recursion}
\author{Patrick Chen}
\date{March 10, 2025}
%%%%%%%%%%%%%%%%%%%%%%%%%%%%%%%%%%%%%%%%%%%%%%%%%%%%%%%%%%%%%%%%%%%%%%%%%%%%%%%

\begin{document}
    \maketitle
    \section*{Induction}
    Suppose we with to prove a proposition $P(n)$. If it is true for a base case
    $P(1)$ and for all $k$, $P(k)$ implies $P(k+1)$ is true, then it is true for
    all $n$.
    \begin{itemize}
        \item Basis step: $P(1)$
        \item Inductive step: $P(k) \rightarrow P(k+1)$
    \end{itemize}
    Strong induction is a form of induction where in the inductive step, we
    assume that all $P(i)$ where $i\le k$ is true. Strong induction is
    mathematically equivalent to regular induction.
    \begin{itemize}
        \item Basis step: $P(m)$
        \item Inductive step: $\Big(\forall i : m\le i\le k,\ P(i)\Big) \rightarrow P(k+1)$
    \end{itemize}

    \subsection*{Well Ordering Principle}
    The well ordering principle is an axiom says that every non-empty set of
    non-negative integers has a smallest element. The well ordering principle is
    equivalent to assuming that induction is true.

    \subsection*{Proof of Induction}
    Assume the well ordering principle.
    \[
        P(1) \quad P(k) \rightarrow P(k+1)
    \]
    Let $\mathcal{C} = \{n\in \mathbb{N}^+ | P(n) \text{ is not true}\}$ \\
    Assume $\mathcal{C}$ is non-empty \\
    By the well ordering principle, pick the smallest element $k$ of $\mathcal{C}$ \\
    Since $P(1)$ is true, $k>1$, thus $k-1$ is a positive integer not in $\mathcal{C}$. \\
    Thus $P(k-1)$ is true. \\
    By the inductive hypothesis $P(k)$ is true. \\
    This contradicts that $P(k)\in\mathcal{C}$ \\
    Thus $\mathcal{C}$ is empty. \\
    Thus proving the base case and the inductive hypothesis is equivalent to
    proving for all elements.

    \section*{Structural Induction}
    \begin{itemize}
        \item Basis step: Prove the result for the basis element
        \item Prove that if the result holds for each element in the
            construction of the new element, it holds for the new construction.
    \end{itemize}


    \subsection*{Example}
    Prove that $\sum_{j=1}^{n} j = \frac{n(n+1)}{2}$
    \[
        P(n) = \Big(\sum_{j=1}^{n} j = \frac{n(n+1)}{2}\Big) \\
    \]
    Base case:
    \begin{align*}
        P(1) &= \Big(\sum_{j=1}^{1} j = \frac{1(1+1)}{2}\Big) \\
             &= (1=1) \\
             &= T \\
    \end{align*}
    Inductive step: \\
    Assume $P(k)$ is true
    \begin{align*}
        p(k+1) &= \Big(\sum_{j=1}^{k+1} j = \frac{(k+1)((k+1)+1)}{2}\Big) \\
               &= \Big((k+1) + \sum_{j=1}^{k} j = \frac{(k+1)(k+2)}{2}\Big) \\
               &= \Big((k+1) + \frac{k(k+1)}{2} = \frac{(k+1)(k+2)}{2}\Big) \\
               &= \Big(\frac{2(k+1) + k(k+1)}{2} = \frac{(k+1)(k+2)}{2}\Big) \\
               &= \Big(\frac{(k+1)(k+2)}{2} = \frac{(k+1)(k+2)}{2}\Big) \\
               &= T
    \end{align*}
    Therefore, $\sum_{j=1}^{n} j = \frac{n(n+1)}{2}$ for all $n\ge 1$

    \subsection*{Example 2}
    Prove that $21\ |\ 4^{n+1} + 5^{2n-1}$ \\
    $21\ |\ n$ if both  $3\ |\ n$ and $7\ |\ n$
    \[
        P(n) = (3\ |\ 4^{n+1} + 5^{2n-1})
    \]
    Base case:
    \begin{align*}
        P(1) &= (3\ |\ 4^{1+1} + 5^{2(1)-1}) \\
             &= (3\ |\ 21) \\
             &= T
    \end{align*}
    Inductive Step
    \begin{align*}
        P(k+1) &= (3\ |\ 4^{k+1+1} + 5^{2(k+1) - 1}) \\
               &= (3\ |\ 4^{k+2} + 5^{2k + 1}) \\
               &= (3\ |\ 4 \cdot 4^{k+1} + 25 \cdot 5^{2k - 1}) \\
               &= (3\ |\ 
               \underbrace{3\cdot 4^{k+1} + 24\cdot 5^{2k+1}}_{e_1}
             + \underbrace{4^{k+1} + 5^{2k-1}}_{e_2})
    \end{align*}
    3 divides $e_1$ trivially and 3 divides $e_1$ by inductive hypothesis.
    Therefore $3\ |\ 4^{n+1} + 5^{2n-1}$
    \[
        Q(n) = (7\ |\ 4^{n+1} + 5^{2n-1})
    \]
    Base case:
    \begin{align*}
        Q(1) &= (7\ |\ 21) \\
             &= 1
    \end{align*}
    Inductive Step
    \begin{align*}
        Q(k+1) &= (7\ |\ 4^{(k+1)+1} + 5^{2(k+1)-1}) \\
             &= (7\ |\ 4^{k + 2} + 5^{2k + 1}) \\
             &= (7\ |\ 4\cdot 4^{k + 1} + 25\cdot 5^{2k - 1}) \\
             &= (7\ |\ 4\cdot 4^{k + 1} + 4\cdot 5^{2k - 1} + 21\cdot 5^{2k - 1}) \\
             &= (7\ |\ 4\underbrace{(\cdot 4^{k + 1} + \cdot 5^{2k - 1})}_{e_3} + \underbrace{21\cdot 5^{2k - 1}}_{e_4})
    \end{align*}
    7 divides $e_3$ by inductive hypothesis and 7 divides $e_4$ trivially \\
    Since $3\ |\ 4^{n+1} + 5^{2n-1}$ and $(7\ |\ 4^{n+1} + 5^{2n-1})$, then
    $21\ |\ 4^{n+1} + 5^{2n-1}$

    \subsection*{Example 3}
    Prove that every integer $n\ge 2$ is a product of primes.
    \[
        p(n) = (\text{every $n\ge 2$ is a product of primes})
    \]
    $P(2)$ is trivially true. \\
    Suppose that $P(i)$ is true for all $2\le i\le k$ \\
    If $k+1$ is prime, then it is trivially a product of itself\\
    If $k+1$ is composite, $k+1=ab$ for $2\le a,b < k+1$ \\
    By the inductive hypothesis, $a$ and $b$ are a product of prime numbers \\
    Therefore $k+1$ is also a product of primes

    \subsection*{Example 4}
    Using the well ordering principle, prove that $\sum_{i=1}^{n} n =
    \frac{n(n+1)}{2}$ for all $n \ge 1$\\
    Let $C = \{x \ge 1\ |\ \neg P(n)\}$ \\
    Let $k\in C$ be the smallest element \\
    $P(k)$ is not true but $P(k-1)$ is true.
    \begin{align*}
        \sum_{i=1}^{k-1} i &= \frac{(k-1)k}{2} \\
        k + \sum_{i=1}^{k-1} i &= k + \frac{(k-1)k}{2} \\
        \sum_{i=1}^{k} i &= \frac{k(k+1)}{2} \\
    \end{align*}
    This contradicts that $P(k)$ is false. \\
    Thus $C$ is empty.

    \subsection*{Example 5}
    Every well-formed formula for compound propositions contains an equal number
    of left and right parentheses.
    \begin{itemize}
        \item T, F have no parentheses
        \item Given well formed compound propositions $p,q$,
            let $l_p$ and $r_p$ be the number of left and right parentheses in
            $p$ and similarly $l_q$ and $r_q$ be for $q$. \\
            Assume $l_p=r_p$ and $l_q=r_q$.
            \begin{itemize}
                \item $(\neg p)$ has parentheses $l = l_p+1$, $r=r_p+1$
                \item $(p \vee q)$ has parentheses $l = l_p+l_q+1$, $r=r_p+r_q+1$
                \item $(p \wedge q)$ has parentheses $l = l_p+l_q+1$, $r=r_p+r_q+1$
                \item $(p \rightarrow q)$ has parentheses $l = l_p+l_q+1$, $r=r_p+r_q+1$
                \item $(p \leftrightarrow q)$ has parentheses $l = l_p+l_q+1$, $r=r_p+r_q+1$
            \end{itemize}
            In all cases, $l=r$
    \end{itemize}
    Thus for all compound propositions, there is an equal number of left and
    right parentheses.

\end{document}
