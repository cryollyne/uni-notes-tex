%%%%%%%%%%%%%%%%%%%%%%%%%%%%% Define Article %%%%%%%%%%%%%%%%%%%%%%%%%%%%%%%%%%
\documentclass{article}
%%%%%%%%%%%%%%%%%%%%%%%%%%%%%%%%%%%%%%%%%%%%%%%%%%%%%%%%%%%%%%%%%%%%%%%%%%%%%%%

%%%%%%%%%%%%%%%%%%%%%%%%%%%%% Using Packages %%%%%%%%%%%%%%%%%%%%%%%%%%%%%%%%%%
\usepackage{geometry}
\usepackage{graphicx}
\usepackage{amssymb}
\usepackage{amsmath}
\usepackage{amsthm}
\usepackage{empheq}
\usepackage{mdframed}
\usepackage{booktabs}
\usepackage{lipsum}
\usepackage{graphicx}
\usepackage{color}
\usepackage{psfrag}
\usepackage{pgfplots}
\usepackage{bm}
%%%%%%%%%%%%%%%%%%%%%%%%%%%%%%%%%%%%%%%%%%%%%%%%%%%%%%%%%%%%%%%%%%%%%%%%%%%%%%%

% Other Settings

%%%%%%%%%%%%%%%%%%%%%%%%%% Page Setting %%%%%%%%%%%%%%%%%%%%%%%%%%%%%%%%%%%%%%%
\geometry{a4paper}

%%%%%%%%%%%%%%%%%%%%%%%%%% Define some useful colors %%%%%%%%%%%%%%%%%%%%%%%%%%
\definecolor{ocre}{RGB}{243,102,25}
\definecolor{mygray}{RGB}{243,243,244}
\definecolor{deepGreen}{RGB}{26,111,0}
\definecolor{shallowGreen}{RGB}{235,255,255}
\definecolor{deepBlue}{RGB}{61,124,222}
\definecolor{shallowBlue}{RGB}{235,249,255}
%%%%%%%%%%%%%%%%%%%%%%%%%%%%%%%%%%%%%%%%%%%%%%%%%%%%%%%%%%%%%%%%%%%%%%%%%%%%%%%

%%%%%%%%%%%%%%%%%%%%%%%%%% Define an orangebox command %%%%%%%%%%%%%%%%%%%%%%%%
\newcommand\orangebox[1]{\fcolorbox{ocre}{mygray}{\hspace{1em}#1\hspace{1em}}}
%%%%%%%%%%%%%%%%%%%%%%%%%%%%%%%%%%%%%%%%%%%%%%%%%%%%%%%%%%%%%%%%%%%%%%%%%%%%%%%

%%%%%%%%%%%%%%%%%%%%%%%%%%%% English Environments %%%%%%%%%%%%%%%%%%%%%%%%%%%%%
\newtheoremstyle{mytheoremstyle}{3pt}{3pt}{\normalfont}{0cm}{\rmfamily\bfseries}{}{1em}{{\color{black}\thmname{#1}~\thmnumber{#2}}\thmnote{\,--\,#3}}
\newtheoremstyle{myproblemstyle}{3pt}{3pt}{\normalfont}{0cm}{\rmfamily\bfseries}{}{1em}{{\color{black}\thmname{#1}~\thmnumber{#2}}\thmnote{\,--\,#3}}
\theoremstyle{mytheoremstyle}
\newmdtheoremenv[linewidth=1pt,backgroundcolor=shallowGreen,linecolor=deepGreen,leftmargin=0pt,innerleftmargin=20pt,innerrightmargin=20pt,]{theorem}{Theorem}[section]
\theoremstyle{mytheoremstyle}
\newmdtheoremenv[linewidth=1pt,backgroundcolor=shallowBlue,linecolor=deepBlue,leftmargin=0pt,innerleftmargin=20pt,innerrightmargin=20pt,]{definition}{Definition}[section]
\theoremstyle{myproblemstyle}
\newmdtheoremenv[linecolor=black,leftmargin=0pt,innerleftmargin=10pt,innerrightmargin=10pt,]{problem}{Problem}[section]
%%%%%%%%%%%%%%%%%%%%%%%%%%%%%%%%%%%%%%%%%%%%%%%%%%%%%%%%%%%%%%%%%%%%%%%%%%%%%%%

%%%%%%%%%%%%%%%%%%%%%%%%%%%%%%% Plotting Settings %%%%%%%%%%%%%%%%%%%%%%%%%%%%%
\usepgfplotslibrary{colorbrewer}
\pgfplotsset{width=8cm,compat=1.9}
%%%%%%%%%%%%%%%%%%%%%%%%%%%%%%%%%%%%%%%%%%%%%%%%%%%%%%%%%%%%%%%%%%%%%%%%%%%%%%%

%%%%%%%%%%%%%%%%%%%%%%%%%%%%%%% Title & Author %%%%%%%%%%%%%%%%%%%%%%%%%%%%%%%%
\title{Sets}
\author{Patrick Chen}
\date{Jan 20, 2025}
%%%%%%%%%%%%%%%%%%%%%%%%%%%%%%%%%%%%%%%%%%%%%%%%%%%%%%%%%%%%%%%%%%%%%%%%%%%%%%%

\begin{document}
    \maketitle
    \section*{Sets}
    A set is an unordered collections of distinct objects. They are represented
    by curly brackets. An empty set is denoted by $\emptyset$. If $a$ is a
    member of the set $A$, then it is denoted by $a\in A$. If a set is small,
    then the roster method is usually used. The roster method is just writing
    all of the members. If there is a pattern, then ellipses can be used. The
    members of a set can be described by set builder notation.
    \[
        A = \{a_1,\ a_2,\ \dots\}
    \]
    \subsection*{Tuples}
    A n-tuple $(a_1,a_2,\dots,a_n)$ is an ordered collection of $n$ elements.
    \subsection*{Common Sets}
    \begin{itemize}
        \item Natural Numbers $\mathbb{N} = \{0,1,2,3,4,\dots\}$
        \item Integers $\mathbb{Z} = \{\dots,-4,-3,-2,-1,0,1,2,3,4,\dots\}$
        \item Positive Integers $\mathbb{Z}^+ = \{1,2,3,4,\dots\}$
        \item Rational $\mathbb{Q}^+ = \{ \frac{p}{q}\ |\ p\in\mathbb{Z},q\in\mathbb{Z},q\ne 0\}$
        \item Real $\mathbb{R}$
        \item Positive Real $\mathbb{R}^+$
        \item Universal Set $U$
    \end{itemize}
    The Universal set is the set of all elements in a bounding set.

    \subsection*{Set Builder Notation}
    Let $\mathbb{K}$ be a set and $P$ be a proposition.
    \[
        A = \{x \in \mathbb{K}\ |\ P\}
    \]
    A represents a set that contain all the elements of $\mathbb{K}$ such that
    the proposition $P$ holds. $\mathbb{K}$ is considered the bounding set.

    \subsection*{Interval Notation}
    \begin{align*}
        [a,b] = \{x\in \mathbb{R}\ |\ a \le x \le b\} \\
        [a,b) = \{x\in \mathbb{R}\ |\ a \le x < b\} \\
        (a,b] = \{x\in \mathbb{R}\ |\ a < x \le b\} \\
        (a,b) = \{x\in \mathbb{R}\ |\ a < x < b\}
    \end{align*}

    \subsection*{Set operations}
    \begin{itemize}
        \item Equal: $A=B$ if and only if $\forall x (x\in A \leftrightarrow x
            \in B)$
            \begin{align*}
                \{1,3,5,7\} &= \{1,1,1,5,5,7,7,4\} \\
                \{1,2,3\} &\ne \{\{1,2\},3\} \\
                \emptyset &\not\in \{1,2,3,4,5\}
            \end{align*}

        \item Subset: $A \subseteq B$ if and only if
            $\forall x (x \in A \rightarrow x \in B)$
            \begin{itemize}
                \item $A \subseteq A$
                \item If $A=B$ then $A\subseteq B$
            \end{itemize}
            \begin{align*}
                \{1,3,5,7\} &\subseteq \{1,2,3,4,5,6,7,8,9\} \\
                \{1\} &\subseteq \{1,2,3,4,5,6,7,8,9\} \\
                \{\{1,2\}\} &\subseteq \{\{1,2\},3\} \\
                \{1,3\} &\not\subseteq \{\{1,2\},3\} \\
                \emptyset &\subseteq \{1,2,3,4,5\} \\
            \end{align*}

        \item Proper subset: $A \subset B$ if and only if
            $(A\subseteq B) \wedge (A\ne B)$

        \item Cardinality: $|A|$. The cardinality is the size of the set.
            \begin{align*}
                |\emptyset| &= 0 \\
                |\{1,2,2,4\}| &= 3 \\
                |\{\{1,2\},3\}| &= 2
            \end{align*}

        \item Power set: $\mathcal{P}(A)$ The set of all possible unique subsets.
            $\mathcal{P}(A) = \{B\ |\ B \subseteq A\}$
            \begin{align*}
                \mathcal{P}(\{1,2,3\}) = \{\emptyset,\{1\}, \{2\}, \{3\},
                \{1,2\}, \{1,3\}, \{2,3\}, \{1,2,3\}\}
            \end{align*}

        \item Cartesian product: The Cartesian product $A \times B$ is the
            collection of all pairs $(a,b)$ with $a\in A$ and $b\in B$. Elements
            of the Cartesian product are tuples.
            \begin{align*}
                A \times B = \{(a,b)\ |\ a\in A, b\in B\}
            \end{align*}
            The Cartesian product can be extended to work on multiple sets
            \[
                A_1\times A_2\times \dots\times A_n = \{(a_1,a_2,\dots,a_n)\
                |\ a_1\in A_1,\ a_2\in A_2,\ \dots\ ,\ a_n\in A_n\}
            \]

        \item Union: The union of $A$ and $B$ is the elements that are in either
            or both sets.
            \[
                A\cup B = \{x\ |\ x\in A \vee x\in B\}
            \]

        \item Intersection: The intersection of $A$ and $B$ is the elements that
            both sets.
            \[
                A\cap B = \{x\ |\ x\in A \wedge x\in B\}
            \]

        \item Two sets are disjoint if they have no shared elements.
            \[
                A \cap B = \emptyset
            \]

        \item The difference between $A$ and $B$ is all the elements in $A$ that
            are not in $B$.
            \[
                A-B = \{x\ |\ x\in A \wedge x\not\in B\}
            \]

        \item The complement of a set $A$ is everything that is not in $A$.
            \[
                \overline{A} = U-A
            \]

        \item Symmetric Difference: the symmetric difference of $A$ and $B$ are
            all the elements in exactly one set but not both.
            \[
                A \oplus B = \{x\ |\ x\in A\cup B \wedge x\not\in A\cap B\}
            \]
    \end{itemize}

    \subsection*{Example}
    Show that $A-B = A\cap \overline{B}$
    \begin{align*}
        x\in (A-B) &= x \in A \wedge x \not\in B \\
        &= x\in A \wedge x\in\overline{B} \\
        &= x\in A\cap\overline{B}
    \end{align*}

    \section*{Set Identities}
    \begin{itemize}
        \item Identity
            \begin{align*}
                A\cup \emptyset &= A \\
                A\cap U &= A
            \end{align*}
        \item Domination:
            \begin{align*}
                A\cup U &= U \\
                A\cap \emptyset &= \emptyset
            \end{align*}
        \item Commutative:
            \begin{align*}
                A \cup B = B \cup A \\
                A \cap B = B \cap A
            \end{align*}
        \item Associative
            \begin{align*}
                A \cup (B \cup C) = (A \cup B) \cup C \\
                A \cap (B \cap C) = (A \cap B) \cap C
            \end{align*}
        \item Complementation:
            \[
                \overline{\overline{A}} = A
            \]
        \item DeMorgan's:
            \begin{align*}
                \overline{A\cap B} &= \overline{A} \cup \overline{B} \\
                \overline{A\cup B} &= \overline{A} \cap \overline{B}
            \end{align*}
        \item Distributive
            \begin{align*}
                A\cup (B\cap C) &= (A\cup B) \cap (A\cup C) \\
                A\cap (B\cup C) &= (A\cap B) \cup (A\cap C)
            \end{align*}
        \item Negation:
            \begin{align*}
                A\cup\overline{A}&=U \\
                A\cap\overline{A}&=\emptyset
            \end{align*}
        \item Absorption
            \begin{align*}
                A\cup(A\cap B) &= A \\
                A\cap(A\cup B) &= A \\
            \end{align*}
    \end{itemize}
    \subsection*{Example}
    Show that $A\oplus B=(A-B)\cup(B-A)$
    \begin{align*}
        (A-B)\cup (B-A)
        &= (A\cap\overline{B})\cup(B\cap\overline{A}) & \text{Definition} \\
        &= ((A\cap\overline{B})\cup B)\cap((A\cap\overline{B})\cup \overline{A}) & \text{Distributive}\\
        &= ((A\cup B) \cap (\overline{B}\cup B)) \cap ((A\cup \overline{A}) \cap(\overline{B}\cup\overline{A})) & \text{Distributive} \\
        &= ((A\cup B) \cap U) \cap (U \cap(\overline{B}\cup\overline{A})) \\
        &= (A\cup B) \cap (\overline{B}\cup \overline{A}) \\
        &= (A\cup B) \cap \overline{A\cap B} & \text{DeMorgan}\\
        &= (A\cup B) - (A\cap B) \\
        &= A\oplus B & \text{Definition}
    \end{align*}

    \subsection*{Union Notations}
    Let $A_1,\dots,A_n$ be sets. The union and intersections of many sets can be
    written with the big union and big intersection notation.
    \begin{align*}
        \bigcup_{i=1}^n A_i = A_1 \cup \dots \cup A_n \\
        \bigcap_{i=1}^n A_i = A_1 \cap \dots \cap A_n
    \end{align*}

    \subsection*{Multiset}
    A multiset is a generalization of set which allows repetition. The notation
    we use for showing notation of an element is $n.a$ where $a$ is the element
    and $n$ is the amount of notation.
    \[
        A = \{a_1.n_1,\ a_2.n_2\ \dots\}
    \]
\end{document}
