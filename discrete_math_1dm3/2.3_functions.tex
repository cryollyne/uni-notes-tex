%%%%%%%%%%%%%%%%%%%%%%%%%%%%% Define Article %%%%%%%%%%%%%%%%%%%%%%%%%%%%%%%%%%
\documentclass{article}
%%%%%%%%%%%%%%%%%%%%%%%%%%%%%%%%%%%%%%%%%%%%%%%%%%%%%%%%%%%%%%%%%%%%%%%%%%%%%%%

%%%%%%%%%%%%%%%%%%%%%%%%%%%%% Using Packages %%%%%%%%%%%%%%%%%%%%%%%%%%%%%%%%%%
\usepackage{geometry}
\usepackage{graphicx}
\usepackage{amssymb}
\usepackage{amsmath}
\usepackage{amsthm}
\usepackage{empheq}
\usepackage{mdframed}
\usepackage{booktabs}
\usepackage{lipsum}
\usepackage{graphicx}
\usepackage{color}
\usepackage{psfrag}
\usepackage{pgfplots}
\usepackage{bm}
%%%%%%%%%%%%%%%%%%%%%%%%%%%%%%%%%%%%%%%%%%%%%%%%%%%%%%%%%%%%%%%%%%%%%%%%%%%%%%%

% Other Settings

%%%%%%%%%%%%%%%%%%%%%%%%%% Page Setting %%%%%%%%%%%%%%%%%%%%%%%%%%%%%%%%%%%%%%%
\geometry{a4paper}

%%%%%%%%%%%%%%%%%%%%%%%%%% Define some useful colors %%%%%%%%%%%%%%%%%%%%%%%%%%
\definecolor{ocre}{RGB}{243,102,25}
\definecolor{mygray}{RGB}{243,243,244}
\definecolor{deepGreen}{RGB}{26,111,0}
\definecolor{shallowGreen}{RGB}{235,255,255}
\definecolor{deepBlue}{RGB}{61,124,222}
\definecolor{shallowBlue}{RGB}{235,249,255}
%%%%%%%%%%%%%%%%%%%%%%%%%%%%%%%%%%%%%%%%%%%%%%%%%%%%%%%%%%%%%%%%%%%%%%%%%%%%%%%

%%%%%%%%%%%%%%%%%%%%%%%%%% Define an orangebox command %%%%%%%%%%%%%%%%%%%%%%%%
\newcommand\orangebox[1]{\fcolorbox{ocre}{mygray}{\hspace{1em}#1\hspace{1em}}}
%%%%%%%%%%%%%%%%%%%%%%%%%%%%%%%%%%%%%%%%%%%%%%%%%%%%%%%%%%%%%%%%%%%%%%%%%%%%%%%

%%%%%%%%%%%%%%%%%%%%%%%%%%%% English Environments %%%%%%%%%%%%%%%%%%%%%%%%%%%%%
\newtheoremstyle{mytheoremstyle}{3pt}{3pt}{\normalfont}{0cm}{\rmfamily\bfseries}{}{1em}{{\color{black}\thmname{#1}~\thmnumber{#2}}\thmnote{\,--\,#3}}
\newtheoremstyle{myproblemstyle}{3pt}{3pt}{\normalfont}{0cm}{\rmfamily\bfseries}{}{1em}{{\color{black}\thmname{#1}~\thmnumber{#2}}\thmnote{\,--\,#3}}
\theoremstyle{mytheoremstyle}
\newmdtheoremenv[linewidth=1pt,backgroundcolor=shallowGreen,linecolor=deepGreen,leftmargin=0pt,innerleftmargin=20pt,innerrightmargin=20pt,]{theorem}{Theorem}[section]
\theoremstyle{mytheoremstyle}
\newmdtheoremenv[linewidth=1pt,backgroundcolor=shallowBlue,linecolor=deepBlue,leftmargin=0pt,innerleftmargin=20pt,innerrightmargin=20pt,]{definition}{Definition}[section]
\theoremstyle{myproblemstyle}
\newmdtheoremenv[linecolor=black,leftmargin=0pt,innerleftmargin=10pt,innerrightmargin=10pt,]{problem}{Problem}[section]
%%%%%%%%%%%%%%%%%%%%%%%%%%%%%%%%%%%%%%%%%%%%%%%%%%%%%%%%%%%%%%%%%%%%%%%%%%%%%%%

%%%%%%%%%%%%%%%%%%%%%%%%%%%%%%% Plotting Settings %%%%%%%%%%%%%%%%%%%%%%%%%%%%%
\usepgfplotslibrary{colorbrewer}
\pgfplotsset{width=8cm,compat=1.9}
%%%%%%%%%%%%%%%%%%%%%%%%%%%%%%%%%%%%%%%%%%%%%%%%%%%%%%%%%%%%%%%%%%%%%%%%%%%%%%%

%%%%%%%%%%%%%%%%%%%%%%%%%%%%%%% Title & Author %%%%%%%%%%%%%%%%%%%%%%%%%%%%%%%%
\title{Functions}
\author{Patrick Chen}
\date{Jan 23, 2025}
%%%%%%%%%%%%%%%%%%%%%%%%%%%%%%%%%%%%%%%%%%%%%%%%%%%%%%%%%%%%%%%%%%%%%%%%%%%%%%%

\newcommand{\ceil}[1]{\lceil#1\rceil}
\newcommand{\floor}[1]{\lfloor#1\rfloor}

\begin{document}
    \maketitle
    A function is a mapping between two sets that maps an element from the input
    set to exactly one element of the output set. If $f$ is a function that
    takes an element of the set $A$ as the input and outputs a element of the
    set $B$, then it is written as $f: A\mapsto B$

    \subsection*{Representations}
    \begin{itemize}
        \item Implicit representation
            \begin{align*}
                f&: \mathbb{R} \mapsto \mathbb{R} \\
                y &= f(x) \\
                f(x) &= x^2 \\
                y-x^2 &= 0
            \end{align*}
        \item Set representation
            \begin{align*}
                \{(a,3), (b,1), (c, 4),(d,3)\}
            \end{align*}
    \end{itemize}
    \subsection*{Terminology}
    For a function $f: A \mapsto B$:
    \begin{itemize}
        \item Domain: The domain is $A$
        \item Codomain: The codomain is $B$
        \item Image: For any $x\in A$, $f(x)$ is called the image of $A$.
        \item Preimage: If for a $y\in B$, there is an $x\in A$ such that
            $f(x)=y$, then $x$ is called the preimage of $y$.
        \item Range: The range is the set of all possible outputs of $f$.
        \item Image of a set: The image of a set $X\subseteq A$ is a set with
            all the elements mapped through the function.
            \[
                f(X) = \{ f(x) \ |\ x\in X\}
            \]
    \end{itemize}

    a function is real-valued if codomain is $\mathbb{R}$
    a function is integer-valued if codomain is $\mathbb{Z}$

    \subsection*{Operations}
    \begin{itemize}
        \item Two functions $f$ and $g$ are equal if:
            \begin{itemize}
                \item $f$ and $g$ have the same domain.
                \item $f$ and $g$ have the same codomain.
                \item $f(x) = g(x)$ for all values $x$ in the domain.
            \end{itemize}
        \item Addition: $(f+g)(x) = f(x) + g(x)$
        \item Multiplication: $(fg)(x) = f(x)g(x)$
    \end{itemize}

    \subsection*{Types of Functions}
    \begin{itemize}
        \item A function $f$ is increasing if
            \[
                \forall x,y\in\mathbb{R}: (x\le y) \rightarrow (f(x)\le f(y))
            \]
        \item A function $f$ is strictly increasing if
            \[
                \forall x,y\in\mathbb{R}: (x<y) \rightarrow (f(x)<f(y))
            \]
        \item A function $f$ is decreasing if
            \[
                \forall x,y\in\mathbb{R}: (x\le y) \rightarrow (f(x)\ge f(y))
            \]
        \item A function $f$ is strictly decreasing if
            \[
                \forall x,y\in\mathbb{R}: (x<y) \rightarrow (f(x)>f(y))
            \]
    \end{itemize}
    \begin{itemize}
        \item A function is injective (one-to-one) if for all values in the
            range, there is only one value in the domain that maps to it.
            \[
                f(a) = f(b) \Rightarrow a=b
            \]
            Every strictly increasing function $f$ is injective. \\
            Proof: \\
            Suppose $x\ne y$. Since $x\ne y$, $x<y$ or $x>y$ \\
            If $x<y$, then $f(x)<f(y)$ since $f$ is strictly increasing.
            Thus $f(x)\ne f(y)$. \\
            If $x>y$, then $f(x)>f(y)$ since $f$ is strictly increasing.
            Thus $f(x)\ne f(y)$.
        \item A function $f: A \mapsto B$ is surjective (onto) if the range is the entire
            codomain
            \[
                \forall b\in B\ \exists a\in A\ (f(a)=b)
            \]
        \item A function is bijective if it is both injective and surjective.
    \end{itemize}

    \subsection*{Bijective Functions}
    For a function $f: A \mapsto B$, the following properties hold.
    \begin{itemize}
        \item range of $f$ is codomain $B$. This is because bijective functions
            are surjective and by definition, onto.
        \item Cardinality of domain and codomain is same size $|A|=|B|$. This is
            because $f$ is a one-to-one correspondence.
        \item the cardinality of preimage of each $b\in B$ is one. Since $f$ is
            injective, the preimage is unique.
    \end{itemize}
    \subsection*{Function Composition}
    The composition of functions $f: A \mapsto B$ and $g: B \mapsto C$ is
    defined as
    \begin{align*}
        f \circ g: A&\mapsto C \\
        x &\mapsto g(f(x))
    \end{align*}

    \subsection*{Inverse Functions}
    A function $g: B \mapsto A$ is said to be the inverse if a function $f: A
    \mapsto B$ if their composition is the identity. An inverse function of $f$
    only exists if $F$ is bijective.
    \begin{align*}
        (g \circ f) (x) = id(x) = x
    \end{align*}

    \subsection*{Graph of a function}
    The graph of a function is a set or ordered pair
    \begin{align*}
        \{ (a, f(a))\ |\ a \in A\}
    \end{align*}

    \subsection*{Important Functions}
    \begin{itemize}
        \item Floor function $f: \mathbb{R} \mapsto \mathbb{Z}$ is the largest
            integer that is less than or equal to the input.
            \[
                f(x) = \floor{x}
            \]
        \item Ceiling function $f: \mathbb{R} \mapsto \mathbb{Z}$ is the
            smallest integer that is greater than or equal to the input.
            \[
                f(x) = \ceil{x}
            \]
        \item Factorial $f: \mathbb{N} \mapsto \mathbb{Z}^+$
            \[
                f(x) = \begin{cases}
                    1,              &\text{ if }x = 0\\
                    x \cdot f(x-1)  &\text{ otherwise})
                \end{cases}
            \]
    \end{itemize}

    \subsection*{Example}
    Prove that $\floor{2x} = \floor{x} + \floor{x+ \frac{1}{2}}$ for all $x\in
    \mathbb{R}$ \\
    let $x = n + \epsilon$ \\
    If $0 \le \epsilon < \frac{1}{2}$
    \begin{align*}
        \floor{2x} &= \floor{2n+2\epsilon} \\
        &= 2n + \floor{2\epsilon} \\
        &= 2n
    \end{align*}
    \begin{align*}
        \floor{x} + \floor{x + \frac{1}{2}}
        &= \floor{n+\epsilon} + \floor{n + \epsilon + \frac{1}{2}} \\
        &= n + \floor{\epsilon} + n + \floor{\epsilon + \frac{1}{2}} \\
        &= 2n
    \end{align*}
    If $\frac{1}{2} \le \epsilon < 1$
    \begin{align*}
        \floor{2x} &= \floor{2n+2\epsilon} \\
        &= 2n + \floor{2\epsilon} \\
        &= 2n + 1
    \end{align*}
    \begin{align*}
        \floor{x} + \floor{x + \frac{1}{2}}
        &= n + \floor{\epsilon} + n + \floor{\epsilon + \frac{1}{2}} \\
        &= 2n + 1\\
    \end{align*}

\end{document}
