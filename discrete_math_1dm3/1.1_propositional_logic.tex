%%%%%%%%%%%%%%%%%%%%%%%%%%%%% Define Article %%%%%%%%%%%%%%%%%%%%%%%%%%%%%%%%%%
\documentclass{article}
%%%%%%%%%%%%%%%%%%%%%%%%%%%%%%%%%%%%%%%%%%%%%%%%%%%%%%%%%%%%%%%%%%%%%%%%%%%%%%%

%%%%%%%%%%%%%%%%%%%%%%%%%%%%% Using Packages %%%%%%%%%%%%%%%%%%%%%%%%%%%%%%%%%%
\usepackage{geometry}
\usepackage{graphicx}
\usepackage{amssymb}
\usepackage{amsmath}
\usepackage{amsthm}
\usepackage{empheq}
\usepackage{mdframed}
\usepackage{booktabs}
\usepackage{lipsum}
\usepackage{graphicx}
\usepackage{color}
\usepackage{psfrag}
\usepackage{pgfplots}
\usepackage{bm}
%%%%%%%%%%%%%%%%%%%%%%%%%%%%%%%%%%%%%%%%%%%%%%%%%%%%%%%%%%%%%%%%%%%%%%%%%%%%%%%

% Other Settings

%%%%%%%%%%%%%%%%%%%%%%%%%% Page Setting %%%%%%%%%%%%%%%%%%%%%%%%%%%%%%%%%%%%%%%
\geometry{a4paper}

%%%%%%%%%%%%%%%%%%%%%%%%%% Define some useful colors %%%%%%%%%%%%%%%%%%%%%%%%%%
\definecolor{ocre}{RGB}{243,102,25}
\definecolor{mygray}{RGB}{243,243,244}
\definecolor{deepGreen}{RGB}{26,111,0}
\definecolor{shallowGreen}{RGB}{235,255,255}
\definecolor{deepBlue}{RGB}{61,124,222}
\definecolor{shallowBlue}{RGB}{235,249,255}
%%%%%%%%%%%%%%%%%%%%%%%%%%%%%%%%%%%%%%%%%%%%%%%%%%%%%%%%%%%%%%%%%%%%%%%%%%%%%%%

%%%%%%%%%%%%%%%%%%%%%%%%%% Define an orangebox command %%%%%%%%%%%%%%%%%%%%%%%%
\newcommand\orangebox[1]{\fcolorbox{ocre}{mygray}{\hspace{1em}#1\hspace{1em}}}
%%%%%%%%%%%%%%%%%%%%%%%%%%%%%%%%%%%%%%%%%%%%%%%%%%%%%%%%%%%%%%%%%%%%%%%%%%%%%%%

%%%%%%%%%%%%%%%%%%%%%%%%%%%% English Environments %%%%%%%%%%%%%%%%%%%%%%%%%%%%%
\newtheoremstyle{mytheoremstyle}{3pt}{3pt}{\normalfont}{0cm}{\rmfamily\bfseries}{}{1em}{{\color{black}\thmname{#1}~\thmnumber{#2}}\thmnote{\,--\,#3}}
\newtheoremstyle{myproblemstyle}{3pt}{3pt}{\normalfont}{0cm}{\rmfamily\bfseries}{}{1em}{{\color{black}\thmname{#1}~\thmnumber{#2}}\thmnote{\,--\,#3}}
\theoremstyle{mytheoremstyle}
\newmdtheoremenv[linewidth=1pt,backgroundcolor=shallowGreen,linecolor=deepGreen,leftmargin=0pt,innerleftmargin=20pt,innerrightmargin=20pt,]{theorem}{Theorem}[section]
\theoremstyle{mytheoremstyle}
\newmdtheoremenv[linewidth=1pt,backgroundcolor=shallowBlue,linecolor=deepBlue,leftmargin=0pt,innerleftmargin=20pt,innerrightmargin=20pt,]{definition}{Definition}[section]
\theoremstyle{myproblemstyle}
\newmdtheoremenv[linecolor=black,leftmargin=0pt,innerleftmargin=10pt,innerrightmargin=10pt,]{problem}{Problem}[section]
%%%%%%%%%%%%%%%%%%%%%%%%%%%%%%%%%%%%%%%%%%%%%%%%%%%%%%%%%%%%%%%%%%%%%%%%%%%%%%%

%%%%%%%%%%%%%%%%%%%%%%%%%%%%%%% Plotting Settings %%%%%%%%%%%%%%%%%%%%%%%%%%%%%
\usepgfplotslibrary{colorbrewer}
\pgfplotsset{width=8cm,compat=1.9}
%%%%%%%%%%%%%%%%%%%%%%%%%%%%%%%%%%%%%%%%%%%%%%%%%%%%%%%%%%%%%%%%%%%%%%%%%%%%%%%

%%%%%%%%%%%%%%%%%%%%%%%%%%%%%%% Title & Author %%%%%%%%%%%%%%%%%%%%%%%%%%%%%%%%
\title{Propositional Logic}
\author{Patrick Chen}
\date{Jan 6, 2025}
%%%%%%%%%%%%%%%%%%%%%%%%%%%%%%%%%%%%%%%%%%%%%%%%%%%%%%%%%%%%%%%%%%%%%%%%%%%%%%%

\begin{document}
    \maketitle
    A proposition is a declarative sentence with either a true or false value.
    Usually, propositions are denoted by lowercase letters: $p,q,r$
    \subsection*{Operators}
    \begin{itemize}
        \item Negation $\neg$ (NOT): $\neg p$ is the opposite of $p$.
        \item conjunction $\wedge$ (AND): $p\wedge q$ is true when both $p$ and
            $q$ are true.
        \item disjunction $\vee$ (OR): $p\vee q$ is true when either $p$ or
            $q$ is true.
        \item Exclusive disjunction $\oplus$ (XOR): $p\oplus q$ is true when exactly one of $p$
            and $q$ is true but not both.
        \item Implication $\rightarrow$: conditional statement $p\rightarrow q$
            is false when $p$ is true and $q$ is false, and true otherwise.
            \begin{itemize}
                \item if $p$, then $q$
                \item $q$ if $p$
                \item $p$ is sufficient for $q$
                \item $q$ unless $\neg p$
            \end{itemize}
        \item Biconditional $\leftrightarrow$ (EQ): biconditional statement
            $p\leftrightarrow q$ is the formalism for $(p \rightarrow q)\wedge(q
            \rightarrow p)$.
            \begin{itemize}
                \item $p$ if and only if $q$
            \end{itemize}
    \end{itemize}

    \section*{Manipulations of operators}
    Two statements are equivalent if both have the same truth value. Equivalence
    of statements is shown with the "$\equiv$" symbol. 

    \subsection*{Implication}
    If $p \rightarrow q$ then:
    \begin{itemize}
        \item converse: $q \rightarrow q$
        \item contrapositive $\neg q \rightarrow \neg p$
        \item inverse: $\neg p \rightarrow \neg q$
    \end{itemize}
    $p \rightarrow q \equiv \neg q \rightarrow \neg p$

    \subsection*{Laws}
    \begin{itemize}
        \item Identity: $p \wedge T \equiv p$ and $p \vee F \equiv p$
        \item Domination: $p \wedge F \equiv F$ and $p \vee T \equiv T$
        \item Idempotent: $p \wedge p \equiv p$ and $p \vee p \equiv p$
        \item Double Negation $\neg(\neg p) \equiv p$
        \item Commutative: $p \wedge q \equiv q \wedge p$
            and $p \vee q \equiv q \vee p$
        \item Negation: $p \vee \neg p \equiv T$ and $p \wedge \neg p \equiv F$
        \item Associative: $(p \vee q) \vee r \equiv p \vee (q \vee r)$
            and $(p \wedge q) \wedge r \equiv p \wedge (q \wedge r)$
        \item Distribution: $p \vee (q \wedge r) \equiv (p \vee q) \wedge (p \vee r)$
            and $p \wedge (q \vee r) \equiv (p \wedge q) \vee (p \wedge r)$
        \item De Morgan's: $\neg (p \vee q) \equiv \neg p \wedge \neg q$
            and $\neg (p and q) \equiv \neg p \vee \neg q$
        \item Absorption: $p \vee (p \wedge q) \equiv p$
            and $p \wedge (p \vee q) \equiv p$
        \item Implication: $p \rightarrow q \equiv \neg p \vee q$
    \end{itemize}

    \subsection*{Tautology}
    A tautology is a statement that is always true. If the statement
    $p \leftrightarrow q$ is a tautology, then $p \equiv q$

    \subsection*{Example 1}
    Show that $p \rightarrow q = \neg q \rightarrow \neg p$
    \begin{align*}
        p \rightarrow q &= \neg p \vee q  & \text{Implication} \\
                        &= q \vee \neg p & \text{Commutative} \\
                        &= \neg (\neg q) \vee (\neg p) & \text{Double Negative}\\
                        &= \neg q \rightarrow \neg p & \text{implies}
    \end{align*}

    \subsection*{Example 2}
    Show that $(p \rightarrow r) \wedge (q \rightarrow r) = (p \vee q) \rightarrow r$
    \begin{align*}
        (p \rightarrow r) \wedge (q \rightarrow r)
        &= (\neg p \vee r) \wedge (\neg q \vee r) & \text{Implication} \\
        &= (\neg p \wedge \neg q) \vee r & \text{Distribution} \\
        &= \neg (p \vee q) \vee r & \text{De Morgan's} \\
        &= (p \vee q) \rightarrow r & \text{Implication}
    \end{align*}

    \section*{Propositional Functions}
    In zero order logic, there are no quantifiers (for all, there exists, etc.).
    In order to extent Propositional logic, propositional functions are needed.
    \subsection*{Quantifiers}
    \begin{itemize}
        \item Universal quantification of $P(x)$ is the proposition "$P(x)$ for
            all values of x in the domain". This is denoted by $\forall x P(x)$.
            If the proposition holds for all values in the domain, then
            $\forall x P(x)$ evaluates to true, but if even one value in the
            domain does not satisfy the proposition, then $\forall x P(x)$
            evaluates to false.
        \item Existential quantifiers of $P(x)$ is the proposition "There exists
            an x in the domain such that $P(x)$". This is denoted by
            $\exists x P(x)$. If the proposition $P(x)$ is true for even a
            single value $x$ in the domain, then $\exists x P(x)$ is true. If
            there are no values $x$ in the domain that satisfies the
            proposition, then $\exists x P(x)$ is false.
    \end{itemize}

    \subsection*{Manipulations of quantifiers}
    A negation can be brought inside a quantifier by inverting the quantifier.
    \begin{align*}
        \neg \forall x P(x) &= \exists x \neg P(x) \\
        \neg \exists x P(x) &= \forall x \neg P(x)
    \end{align*}
    A negation can also be brought inside a propositional function.
    \begin{align*}
        \neg \forall x (x^2>x) = \exists x \neg (x^2>x) = \exists x (x^2 \le x)
    \end{align*}

    \subsection*{Example 3}
    Show that
    $\neg \forall x (P(x) \rightarrow Q(x)) = \exists x (P(x) \wedge \neg Q(x))$
    \begin{align*}
        \neg \forall x (P(x) \rightarrow Q(x))
        &= \neg \forall x (\neg P(x) \vee Q(x)) & \text{ Implication}\\
        &= \exists x \neg (\neg P(x) \vee Q(x)) & \text{ Quantifier}\\
        &= \exists x (P(x) \wedge \neg Q(x)) & \text{ De Morgan's}\\
    \end{align*}

    \subsection*{Nested Quantifiers}
    When quantifiers are nested, they are read left to right. Quantifiers of the
    same type commute, but quantifiers of different types do not commute. For
    example, let $P(x)$ = student $x$ falls asleep during lecture $y$. $\exists
    x \forall y P(x)$ means that there is a student that falls asleep during all
    lectures. $\forall y \exists x P(x)$ means that in every lecture, there is a
    student that falls asleep.
    \begin{align*}
        \forall x \forall y P(x,y) = \forall y \forall x P(x,y) \\
        \exists x \exists y P(x,y) = \exists y \exists x P(x,y) \\
        \forall x \exists y P(x,y) \ne \exists y \forall x P(x,y) \\
    \end{align*}
    Although the quantifiers do not commute, then can imply other propositions
    with nested quantifiers are true.
    \begin{align*}
        \exists x \forall y P(x,y) \rightarrow \forall y \exists x P(x,y)
    \end{align*}

    \subsection*{Example 4}
    Is $\forall x \exists y (x^2-y^2 = 1)$ true for the domain $x,y\in \mathbb{R}$
    \begin{align*}
        \text{If } x &= \frac{1}{2} \\
        \text{Then } y^2 &= -\frac{3}{4}
    \end{align*}
    Since no real number squared to a negative, this proposition is false.

\end{document}
