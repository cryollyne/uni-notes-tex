%%%%%%%%%%%%%%%%%%%%%%%%%%%%% Define Article %%%%%%%%%%%%%%%%%%%%%%%%%%%%%%%%%%
\documentclass{article}
%%%%%%%%%%%%%%%%%%%%%%%%%%%%%%%%%%%%%%%%%%%%%%%%%%%%%%%%%%%%%%%%%%%%%%%%%%%%%%%

%%%%%%%%%%%%%%%%%%%%%%%%%%%%% Using Packages %%%%%%%%%%%%%%%%%%%%%%%%%%%%%%%%%%
\usepackage{geometry}
\usepackage{graphicx}
\usepackage{amssymb}
\usepackage{amsmath}
\usepackage{amsthm}
\usepackage{empheq}
\usepackage{mdframed}
\usepackage{booktabs}
\usepackage{lipsum}
\usepackage{graphicx}
\usepackage{color}
\usepackage{psfrag}
\usepackage{pgfplots}
\usepackage{bm}
%%%%%%%%%%%%%%%%%%%%%%%%%%%%%%%%%%%%%%%%%%%%%%%%%%%%%%%%%%%%%%%%%%%%%%%%%%%%%%%

% Other Settings

%%%%%%%%%%%%%%%%%%%%%%%%%% Page Setting %%%%%%%%%%%%%%%%%%%%%%%%%%%%%%%%%%%%%%%
\geometry{a4paper}

%%%%%%%%%%%%%%%%%%%%%%%%%% Define some useful colors %%%%%%%%%%%%%%%%%%%%%%%%%%
\definecolor{ocre}{RGB}{243,102,25}
\definecolor{mygray}{RGB}{243,243,244}
\definecolor{deepGreen}{RGB}{26,111,0}
\definecolor{shallowGreen}{RGB}{235,255,255}
\definecolor{deepBlue}{RGB}{61,124,222}
\definecolor{shallowBlue}{RGB}{235,249,255}
%%%%%%%%%%%%%%%%%%%%%%%%%%%%%%%%%%%%%%%%%%%%%%%%%%%%%%%%%%%%%%%%%%%%%%%%%%%%%%%

%%%%%%%%%%%%%%%%%%%%%%%%%% Define an orangebox command %%%%%%%%%%%%%%%%%%%%%%%%
\newcommand\orangebox[1]{\fcolorbox{ocre}{mygray}{\hspace{1em}#1\hspace{1em}}}
%%%%%%%%%%%%%%%%%%%%%%%%%%%%%%%%%%%%%%%%%%%%%%%%%%%%%%%%%%%%%%%%%%%%%%%%%%%%%%%

%%%%%%%%%%%%%%%%%%%%%%%%%%%% English Environments %%%%%%%%%%%%%%%%%%%%%%%%%%%%%
\newtheoremstyle{mytheoremstyle}{3pt}{3pt}{\normalfont}{0cm}{\rmfamily\bfseries}{}{1em}{{\color{black}\thmname{#1}~\thmnumber{#2}}\thmnote{\,--\,#3}}
\newtheoremstyle{myproblemstyle}{3pt}{3pt}{\normalfont}{0cm}{\rmfamily\bfseries}{}{1em}{{\color{black}\thmname{#1}~\thmnumber{#2}}\thmnote{\,--\,#3}}
\theoremstyle{mytheoremstyle}
\newmdtheoremenv[linewidth=1pt,backgroundcolor=shallowGreen,linecolor=deepGreen,leftmargin=0pt,innerleftmargin=20pt,innerrightmargin=20pt,]{theorem}{Theorem}[section]
\theoremstyle{mytheoremstyle}
\newmdtheoremenv[linewidth=1pt,backgroundcolor=shallowBlue,linecolor=deepBlue,leftmargin=0pt,innerleftmargin=20pt,innerrightmargin=20pt,]{definition}{Definition}[section]
\theoremstyle{myproblemstyle}
\newmdtheoremenv[linecolor=black,leftmargin=0pt,innerleftmargin=10pt,innerrightmargin=10pt,]{problem}{Problem}[section]
%%%%%%%%%%%%%%%%%%%%%%%%%%%%%%%%%%%%%%%%%%%%%%%%%%%%%%%%%%%%%%%%%%%%%%%%%%%%%%%

%%%%%%%%%%%%%%%%%%%%%%%%%%%%%%% Plotting Settings %%%%%%%%%%%%%%%%%%%%%%%%%%%%%
\usepgfplotslibrary{colorbrewer}
\pgfplotsset{width=8cm,compat=1.9}
%%%%%%%%%%%%%%%%%%%%%%%%%%%%%%%%%%%%%%%%%%%%%%%%%%%%%%%%%%%%%%%%%%%%%%%%%%%%%%%

%%%%%%%%%%%%%%%%%%%%%%%%%%%%%%% Title & Author %%%%%%%%%%%%%%%%%%%%%%%%%%%%%%%%
\title{Matrices}
\author{Patrick Chen}
\date{Feb 6}
%%%%%%%%%%%%%%%%%%%%%%%%%%%%%%%%%%%%%%%%%%%%%%%%%%%%%%%%%%%%%%%%%%%%%%%%%%%%%%%

\begin{document}
    \maketitle
    A matrix with $m$ rows and $n$ columns is called an $m\times n$ matrix and
    can be considered a 2d array. If $m=n$, then the matrix is square.
    \begin{align*}
        \begin{bmatrix}
            a_{11} & a_{12} & \\
            a_{21} & a_{22} &\dots \\
                   & \vdots &
        \end{bmatrix}
    \end{align*}
    Two matrices are equal if they have the same dimension and all their
    elements are equal. The sum and difference is done entry-wise.
    \[
        C = A + B \Leftrightarrow c_{ij} = a_{ij} + b_{ij}
    \]
    For a $m\times k$ matrix $A$ and a $k \times n$ matrix $B$ then the product
    is a $m\times n$ matrix defined as follows. Matrix multiplication is not
    commutative.
    \begin{align*}
        C = AB \Leftrightarrow c_{ij} = \sum_{r=1}^k a_{ir}b_{rj}
    \end{align*}
    The identity matrix (denoted by $I$) is a matrix with ones along the diagonal and zeros in
    all other elements. It has the special property that any matrix multiplied
    with it will result in the same matrix.
    \[
        I \Leftrightarrow i_{ij} = \begin{cases}
            1, &\text{ if } i = j\\
            0, &\text{ otherwise}
        \end{cases}
    \]
    The power of a matrix is the repeated multiplication of that matrix with
    itself. A matrix to the power of zero is the identity matrix.
    \[
        A^n = \underbrace{A A \dots A}_{n}
    \]
    The inverse of a $n\times n$ matrix $A$ is a matrix $B$ such that the
    product of the two is the identity matrix.
    \[
        A^{-1} A = I
    \]
    The transpose of a $n\times m$ matrix $A$ is a $m\times n$ matrix $A^T$. The
    transpose of a matrix if another matrix that is reflected about the main
    diagonal.
    \[
        B = A^T \Leftrightarrow b_{ij} = a_{ji}
    \]
    A symmetric matrix $A$ is a matrix such that $A=A^T$

    \subsection*{Zero-One Matrix}
    A zero-one matrix is a matrix where all the elements are either zero or one.
    There are some operations that work on them
    \begin{itemize}
        \item The join of $A$ and $B$ is
            \[
                C=A \vee B \Leftrightarrow c_{ij} = a_{ij} \vee b_{ij}
            \]
        \item The meet of $A$ and $B$ is
            \[
                C=A \wedge B \Leftrightarrow c_{ij} = a_{ij} \wedge b_{ij}
            \]
        \item The boolean product of two matrices
            \[
                C=A\odot B \Leftrightarrow (a_{i1} \wedge b_{1j}) \vee (a_{i2} \wedge b_{2j})\dots
            \]
    \end{itemize}

\end{document}
