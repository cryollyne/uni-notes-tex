%%%%%%%%%%%%%%%%%%%%%%%%%%%%% Define Article %%%%%%%%%%%%%%%%%%%%%%%%%%%%%%%%%%
\documentclass{article}
%%%%%%%%%%%%%%%%%%%%%%%%%%%%%%%%%%%%%%%%%%%%%%%%%%%%%%%%%%%%%%%%%%%%%%%%%%%%%%%

%%%%%%%%%%%%%%%%%%%%%%%%%%%%% Using Packages %%%%%%%%%%%%%%%%%%%%%%%%%%%%%%%%%%
\usepackage{geometry}
\usepackage{graphicx}
\usepackage{amssymb}
\usepackage{amsmath}
\usepackage{amsthm}
\usepackage{empheq}
\usepackage{mdframed}
\usepackage{booktabs}
\usepackage{lipsum}
\usepackage{graphicx}
\usepackage{color}
\usepackage{psfrag}
\usepackage{pgfplots}
\usepackage{bm}
%%%%%%%%%%%%%%%%%%%%%%%%%%%%%%%%%%%%%%%%%%%%%%%%%%%%%%%%%%%%%%%%%%%%%%%%%%%%%%%

% Other Settings

%%%%%%%%%%%%%%%%%%%%%%%%%% Page Setting %%%%%%%%%%%%%%%%%%%%%%%%%%%%%%%%%%%%%%%
\geometry{a4paper}

%%%%%%%%%%%%%%%%%%%%%%%%%% Define some useful colors %%%%%%%%%%%%%%%%%%%%%%%%%%
\definecolor{ocre}{RGB}{243,102,25}
\definecolor{mygray}{RGB}{243,243,244}
\definecolor{deepGreen}{RGB}{26,111,0}
\definecolor{shallowGreen}{RGB}{235,255,255}
\definecolor{deepBlue}{RGB}{61,124,222}
\definecolor{shallowBlue}{RGB}{235,249,255}
%%%%%%%%%%%%%%%%%%%%%%%%%%%%%%%%%%%%%%%%%%%%%%%%%%%%%%%%%%%%%%%%%%%%%%%%%%%%%%%

%%%%%%%%%%%%%%%%%%%%%%%%%% Define an orangebox command %%%%%%%%%%%%%%%%%%%%%%%%
\newcommand\orangebox[1]{\fcolorbox{ocre}{mygray}{\hspace{1em}#1\hspace{1em}}}
%%%%%%%%%%%%%%%%%%%%%%%%%%%%%%%%%%%%%%%%%%%%%%%%%%%%%%%%%%%%%%%%%%%%%%%%%%%%%%%

%%%%%%%%%%%%%%%%%%%%%%%%%%%% English Environments %%%%%%%%%%%%%%%%%%%%%%%%%%%%%
\newtheoremstyle{mytheoremstyle}{3pt}{3pt}{\normalfont}{0cm}{\rmfamily\bfseries}{}{1em}{{\color{black}\thmname{#1}~\thmnumber{#2}}\thmnote{\,--\,#3}}
\newtheoremstyle{myproblemstyle}{3pt}{3pt}{\normalfont}{0cm}{\rmfamily\bfseries}{}{1em}{{\color{black}\thmname{#1}~\thmnumber{#2}}\thmnote{\,--\,#3}}
\theoremstyle{mytheoremstyle}
\newmdtheoremenv[linewidth=1pt,backgroundcolor=shallowGreen,linecolor=deepGreen,leftmargin=0pt,innerleftmargin=20pt,innerrightmargin=20pt,]{theorem}{Theorem}[section]
\theoremstyle{mytheoremstyle}
\newmdtheoremenv[linewidth=1pt,backgroundcolor=shallowBlue,linecolor=deepBlue,leftmargin=0pt,innerleftmargin=20pt,innerrightmargin=20pt,]{definition}{Definition}[section]
\theoremstyle{myproblemstyle}
\newmdtheoremenv[linecolor=black,leftmargin=0pt,innerleftmargin=10pt,innerrightmargin=10pt,]{problem}{Problem}[section]
%%%%%%%%%%%%%%%%%%%%%%%%%%%%%%%%%%%%%%%%%%%%%%%%%%%%%%%%%%%%%%%%%%%%%%%%%%%%%%%

%%%%%%%%%%%%%%%%%%%%%%%%%%%%%%% Plotting Settings %%%%%%%%%%%%%%%%%%%%%%%%%%%%%
\usepgfplotslibrary{colorbrewer}
\pgfplotsset{width=8cm,compat=1.9}
%%%%%%%%%%%%%%%%%%%%%%%%%%%%%%%%%%%%%%%%%%%%%%%%%%%%%%%%%%%%%%%%%%%%%%%%%%%%%%%

%%%%%%%%%%%%%%%%%%%%%%%%%%%%%%% Title & Author %%%%%%%%%%%%%%%%%%%%%%%%%%%%%%%%
\title{Sequences}
\author{Patrick Chen}
\date{Jan 30, 2025}
%%%%%%%%%%%%%%%%%%%%%%%%%%%%%%%%%%%%%%%%%%%%%%%%%%%%%%%%%%%%%%%%%%%%%%%%%%%%%%%

\begin{document}
    \maketitle
    \section*{Sequences}
    A sequence is a function from a subset of integers to a set $S$. Usually,
    the subset is the set of natural numbers $\mathbb{N}$ or positive integers
    $\mathbb{Z}^+$. A finite sequence is called a string. Below, the expression
    on the left denotes a infinite sequence and the expression on the right
    denotes a finite sequence.
    \[
        \{a_n\} \quad \{a_n\}_{n=A}^{B}
    \]
    \begin{itemize}
        \item $a_n$ is a term in the sequence
        \item $A$ is where the sequence begins
        \item $B$ is where the sequence ends.
    \end{itemize}
    \subsection*{Progression}
    A geometric progression is
    \[
        \{a\cdot r^k\}_{k=0}^n = \{a, a\cdot r, a\cdot r^2, \dots, a\cdot r^n\}
    \]
    A arithmetic progression is:
    \[
        \{a+kd\}_{k=0}^n = \{a, a+d, a+2d, \dots, a+nd\}
    \]
    \subsection*{Recurrences}
    A recurrence relation is a sequence where the next term depends on the
    previous values.
    \[
        a_{n+1} = f(a_n)
    \]
    For a recursive sequence to have a unique closed form formula, there must be
    initial conditions for the sequence.

    \section*{Summations}
    Given a sequence $\{a_k\}_{k=0}^n$, the partial sum of the sequence is
    denoted by $S_n$.
    \begin{align*}
        S_n = \sum_{k=0}^n a_k
    \end{align*}

    The index of the sum can be changes.
    \begin{align*}
        \sum_{i=m}^{n} a_i = \sum_{i=m+k}^{n+k} a_{i-k} &&
        \sum_{i=m}^{n} a_i = \sum_{i=m-k}^{n-k} a_{i+k}
    \end{align*}

    \subsection*{Summations of Geometric Progression}
    The sum of a geometric progression is:
    \begin{align*}
        \sum_{k=0}^n ar^k
    \end{align*}
    If $r=1$, then
    \begin{align*}
        S_n &= \sum_{k=0}^n a \\
        &= (n+1)a
    \end{align*}
    If $r \ne 1$, then
    \begin{align*}
        S_n &= \sum_{k=0}^n ar^k \\
        rS_n &=\sum_{k=0}^n ar^{k+1} \\
        rS_n &=\sum_{k=1}^{n+1} ar^{k} \\
             &= \Big(\sum_{k=1}^{n} ar^{k}\Big) + ar^{n+1} \\
             &= \Big(\sum_{k=1}^{n} ar^{k}\Big) + ar^{n+1} \\
             &= \Big(\sum_{k=0}^{n} ar^{k}\Big) + a(r^{n+1} - 1) \\
        rS_n &= S_n + a(r^{n+1} - 1) \\
        S_n (r - 1) &= a(r^{n+1} - 1) \\
        S_n  &= \frac{a(r^{n+1} - 1)}{r - 1} \\
    \end{align*}
    Thus,
    \[
        \sum_{k=0}^n ar^k = \begin{cases}
            \frac{a(r^{n+1} - 1)}{r - 1}, &\text{ if } r\ne 1\\
            (n+1)a &\text{ if } r = 1
        \end{cases}
    \]
    When the sum is to infinity:
    \begin{align*}
        \sum_{k=0}^\infty ar^k
        &= \lim_{n\to \infty} \sum_{k=0}^n ar^k \\
        &= \lim_{n\to \infty} \begin{cases}
            \frac{a(r^{n+1} - 1)}{r - 1}, &\text{ if } r\ne 1\\
            (n+1)a &\text{ if } r = 1
        \end{cases} \\
        &= \begin{cases}
            \frac{-a}{r - 1}, &\text{ if } |r| < 1 \\
            DNE &\text{ if } |r| \ge 1
        \end{cases} \\
        &= \begin{cases}
            \frac{a}{1 - r}, &\text{ if } |r| < 1 \\
            DNE &\text{ if } |r| \ge 1
        \end{cases}
    \end{align*}

    \subsection*{Common Summations}
    There are a closed form solutions for many summations
    \begin{align*}
        \sum_{i=1}^{n} c &= cn \\
        \sum_{i=1}^{n} i &= \frac{n(n+1)}{2} \\
        \sum_{i=1}^{n} i^2 &= \frac{n(n+1)(2n+1)}{6} \\
        \sum_{i=1}^{n} i^3 &= \bigg(\frac{n(n+1)}{2}\bigg)^2 \\
        \sum_{k=0}^n ar^k &= \frac{a(r^{n+1} - 1)}{r - 1} \\
        \sum_{k=0}^\infty x^k &= \frac{1}{1-x} \\
        \sum_{k=1}^\infty kx^{k-1} &= \frac{1}{(1-x)^2} \\
    \end{align*}

    \subsection*{Double Sum}
    For a sequence with two indices $\{a_{i,j}\}_{i,j\in \mathbb{Z}}$, the sum
    is written as a double sum.
    \[
        \sum_i\sum_j a_{i,j}
    \]
    For finite sums, the order of evaluation does not matter and can be
    exchanged.
    \[
        \sum_i\sum_j a_{i,j} = \sum_j\sum_i a_{i,j}
    \]

    \section*{Cardinality}
    For finite sets, the cardinality is the amount of elements in the set. For
    infinite set, if there is a bijection from one set to another, then they
    have the same cardinality. If there is an injection from a set $A$ to a set
    $B$, then the cardinality of $A$ is less than or equal to $B$
    \[
        A \hookrightarrow B \Rightarrow |A| \le |B|
    \]

    \subsection*{Countable Sets}
    A set $A$ is countable if either $A$ is finite or there is a bijection
    $f: A \mapsto \mathbb{Z}^+$. This is equivalent to being able to list the
    elements of the set, indexed by $\mathbb{Z}^+$.
    \begin{itemize}
        \item Any subset of countable sets is countable.
        \item The union of two countable sets is countable.
    \end{itemize}
    If a set is not countable, then it is called uncountable. The cardinality of
    a countable set is denoted by $\aleph_0$.

    \subsection*{Example 1}
    The set of odd positive integers $A$ is a countable set.
    \[
        f: \mathbb{Z}^+ \mapsto A,\ f(n) = 2n+1
    \]
    Since $f$ is bijective, the cardinality of $A$ and $\mathbb{Z}^+$ is the
    same and thus, it is countable.

    \subsection*{Example 2}
    Is $|\mathbb{Z}| = \mathbb{Z}^+$?
    \[
        \mathbb{Z} = \{0, 1, -1, 2, -2, 3, -3, \dots, n, -n, \dots\}
    \]
    Since there is an integer indexed list of elements for the set $\mathbb{Z}$,
    it is a countable set.

    \subsection*{Example 3}
    Prove that the set of real numbers in between zero and one
    $\mathbb{R}_{(0,1)}$ is an uncountable set.

    Assume that (0,1) is countable and $d_{i,j}$ is an digit in base ten
    \begin{align*}
        r_{1} = 0.d_{11}d_{12}d_{13}d_{14}\dots \\
        r_{2} = 0.d_{21}d_{22}d_{23}d_{24}\dots \\
        r_{3} = 0.d_{31}d_{32}d_{33}d_{34}\dots \\
        r_{4} = 0.d_{41}d_{42}d_{43}d_{44}\dots \\
        \vdots
    \end{align*}
    Let $r=0.d_1d_2d_3d_4\dots$ be a new number such that
    \[
        d_i = \begin{cases}
            4, &\text{ if } d_{ii} \ne 4 \\
            5, &\text{ if } d_{ii} = 4
        \end{cases}
    \]
    $r$ cannot be equal to any number in the list since $d_i \ne d_{ii}$. Thus
    it is not in the list of numbers, therefore, the set of real numbers from
    zero to one is not countable.
\end{document}
