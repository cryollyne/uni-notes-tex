%%%%%%%%%%%%%%%%%%%%%%%%%%%%% Define Article %%%%%%%%%%%%%%%%%%%%%%%%%%%%%%%%%%
\documentclass{article}
%%%%%%%%%%%%%%%%%%%%%%%%%%%%%%%%%%%%%%%%%%%%%%%%%%%%%%%%%%%%%%%%%%%%%%%%%%%%%%%

%%%%%%%%%%%%%%%%%%%%%%%%%%%%% Using Packages %%%%%%%%%%%%%%%%%%%%%%%%%%%%%%%%%%
\usepackage{geometry}
\usepackage{graphicx}
\usepackage{amssymb}
\usepackage{amsmath}
\usepackage{amsthm}
\usepackage{empheq}
\usepackage{mdframed}
\usepackage{booktabs}
\usepackage{lipsum}
\usepackage{graphicx}
\usepackage{color}
\usepackage{psfrag}
\usepackage{pgfplots}
\usepackage{bm}
%%%%%%%%%%%%%%%%%%%%%%%%%%%%%%%%%%%%%%%%%%%%%%%%%%%%%%%%%%%%%%%%%%%%%%%%%%%%%%%

% Other Settings

%%%%%%%%%%%%%%%%%%%%%%%%%% Page Setting %%%%%%%%%%%%%%%%%%%%%%%%%%%%%%%%%%%%%%%
\geometry{a4paper}

%%%%%%%%%%%%%%%%%%%%%%%%%% Define some useful colors %%%%%%%%%%%%%%%%%%%%%%%%%%
\definecolor{ocre}{RGB}{243,102,25}
\definecolor{mygray}{RGB}{243,243,244}
\definecolor{deepGreen}{RGB}{26,111,0}
\definecolor{shallowGreen}{RGB}{235,255,255}
\definecolor{deepBlue}{RGB}{61,124,222}
\definecolor{shallowBlue}{RGB}{235,249,255}
%%%%%%%%%%%%%%%%%%%%%%%%%%%%%%%%%%%%%%%%%%%%%%%%%%%%%%%%%%%%%%%%%%%%%%%%%%%%%%%

%%%%%%%%%%%%%%%%%%%%%%%%%% Define an orangebox command %%%%%%%%%%%%%%%%%%%%%%%%
\newcommand\orangebox[1]{\fcolorbox{ocre}{mygray}{\hspace{1em}#1\hspace{1em}}}
%%%%%%%%%%%%%%%%%%%%%%%%%%%%%%%%%%%%%%%%%%%%%%%%%%%%%%%%%%%%%%%%%%%%%%%%%%%%%%%

%%%%%%%%%%%%%%%%%%%%%%%%%%%% English Environments %%%%%%%%%%%%%%%%%%%%%%%%%%%%%
\newtheoremstyle{mytheoremstyle}{3pt}{3pt}{\normalfont}{0cm}{\rmfamily\bfseries}{}{1em}{{\color{black}\thmname{#1}~\thmnumber{#2}}\thmnote{\,--\,#3}}
\newtheoremstyle{myproblemstyle}{3pt}{3pt}{\normalfont}{0cm}{\rmfamily\bfseries}{}{1em}{{\color{black}\thmname{#1}~\thmnumber{#2}}\thmnote{\,--\,#3}}
\theoremstyle{mytheoremstyle}
\newmdtheoremenv[linewidth=1pt,backgroundcolor=shallowGreen,linecolor=deepGreen,leftmargin=0pt,innerleftmargin=20pt,innerrightmargin=20pt,]{theorem}{Theorem}[section]
\theoremstyle{mytheoremstyle}
\newmdtheoremenv[linewidth=1pt,backgroundcolor=shallowBlue,linecolor=deepBlue,leftmargin=0pt,innerleftmargin=20pt,innerrightmargin=20pt,]{definition}{Definition}[section]
\theoremstyle{myproblemstyle}
\newmdtheoremenv[linecolor=black,leftmargin=0pt,innerleftmargin=10pt,innerrightmargin=10pt,]{problem}{Problem}[section]
%%%%%%%%%%%%%%%%%%%%%%%%%%%%%%%%%%%%%%%%%%%%%%%%%%%%%%%%%%%%%%%%%%%%%%%%%%%%%%%

%%%%%%%%%%%%%%%%%%%%%%%%%%%%%%% Plotting Settings %%%%%%%%%%%%%%%%%%%%%%%%%%%%%
\usepgfplotslibrary{colorbrewer}
\pgfplotsset{width=8cm,compat=1.9}
%%%%%%%%%%%%%%%%%%%%%%%%%%%%%%%%%%%%%%%%%%%%%%%%%%%%%%%%%%%%%%%%%%%%%%%%%%%%%%%

%%%%%%%%%%%%%%%%%%%%%%%%%%%%%%% Title & Author %%%%%%%%%%%%%%%%%%%%%%%%%%%%%%%%
\title{Surface Area of a Revolution}
\author{Patrick Chen}
\date{Feb 10, 2025}
%%%%%%%%%%%%%%%%%%%%%%%%%%%%%%%%%%%%%%%%%%%%%%%%%%%%%%%%%%%%%%%%%%%%%%%%%%%%%%%

\begin{document}
    \maketitle
    The surface area of a rotated shape can be approximated by the frustum of a
    cone.
    \[
        A_{frustum} = \frac{2\pi(R+r)s}{2}
    \]
    where
    \begin{itemize}
        \item $R$ is the larger radius
        \item $r$ is the smaller radius
        \item $s$ is the side length
    \end{itemize}
    When integrating, $R=r+dr\Rightarrow R\approx r$, thus $\frac{r+R}{2} \approx r$.
    \[
        SA = \int_I 2\pi r \ ds
    \]
    \begin{itemize}
        \item When Revolving around the $x$ axis:
            \[
                \int_I 2\pi r \ ds = \int_{a}^{b} 2\pi f(x) \sqrt{1+(f'(x))^2} \ dx
            \]
        \item When Revolving around the $y$ axis:
            \[
                \int_I 2\pi r \ ds = \int_{a}^{b} 2\pi x \sqrt{1+(f'(x))^2} \ dx
            \]
    \end{itemize}

    \subsection*{Example}
    Find the surface area of the revolution of the curve $y=x^3$ on the
    interval $[1,2]$ about the x-axis.
    \[
        f(x) = x^3
    \]
    \begin{align*}
        \int_{1}^{2} 2\pi f(x) \sqrt{1+(f'(x))^2} \ dx
        &= 2\pi \int_{1}^{2} x^3 \sqrt{1+(3x^2)^2} \ dx \\
        &= 2\pi \int_{1}^{2} x^3 \sqrt{1 + 9x^4} \ dx \\
        u &= 1 + 9x^4 \\
        du &= 36x^3 \\
        I &= \frac{1}{18} \pi \int_{10}^{145} u^{\frac{1}{2}} \ dx \\
        I &= \frac{1}{18} \pi \Big[ \frac{2}{3} u^{\frac{3}{2}}\Big]_{10}^{145} \\
        I &= \frac{1}{27} \pi (145^{\frac{3}{2}} - 10^{\frac{3}{2}})
    \end{align*}

    \subsection*{Example}
    Find the surface area of a revolution of the curve $y=x^2$ on the interval
    $x\in [0,2]$ about the y-axis.
    \[
        f(x) = x^2
    \]
    \begin{align*}
        \int_{0}^{2} 2\pi x \sqrt{1+(f'(x))^2} \ dx
        &= \int_{0}^{2} 2\pi x \sqrt{1+(2x)^2} \ dx \\
        &= \int_{0}^{2} 2\pi x \sqrt{1 + 4x^2} \ dx \\
        u &= 1 + 4x^2 \\
        du &= 8x \\
        I &= \frac{\pi}{4} \int_{1}^{17} u^{\frac{1}{2}} \ dx \\
          &= \frac{\pi}{4} \Big[ \frac{2}{3} u^{\frac{3}{2}}\Big]_1^{17} \\
        &= \frac{\pi}{6} (17^{\frac{3}{2}} - 1)
    \end{align*}
    Now in terms of $y$
    \[
        f(y) = \sqrt{y}, \quad y\in[0, 4]
    \]
    \begin{align*}
        \int_{0}^{4} 2 \pi f(y) \sqrt{1+(f'(y))^2} \ dy
        &= \int_{0}^{4} 2 \pi \sqrt{y} \sqrt{1+(\frac{1}{2\sqrt{y}})^2} \ dy \\
        &= \int_{0}^{4} 2 \pi \sqrt{y} \sqrt{1+\frac{1}{4y}} \ dy \\
        &= \int_{0}^{4} 2 \pi \sqrt{y+\frac{1}{4}} \ dy \\
        &= \frac{4\pi}{3} [(y+ \frac{1}{4} )^3/2]_0^4 \\
        &= \frac{\pi}{6} (17^{\frac{3}{2}} - 1)
    \end{align*}
\end{document}
