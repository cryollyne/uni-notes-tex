%%%%%%%%%%%%%%%%%%%%%%%%%%%%% Define Article %%%%%%%%%%%%%%%%%%%%%%%%%%%%%%%%%%
\documentclass{article}
%%%%%%%%%%%%%%%%%%%%%%%%%%%%%%%%%%%%%%%%%%%%%%%%%%%%%%%%%%%%%%%%%%%%%%%%%%%%%%%

%%%%%%%%%%%%%%%%%%%%%%%%%%%%% Using Packages %%%%%%%%%%%%%%%%%%%%%%%%%%%%%%%%%%
\usepackage{geometry}
\usepackage{graphicx}
\usepackage{amssymb}
\usepackage{amsmath}
\usepackage{amsthm}
\usepackage{empheq}
\usepackage{mdframed}
\usepackage{booktabs}
\usepackage{lipsum}
\usepackage{graphicx}
\usepackage{color}
\usepackage{psfrag}
\usepackage{pgfplots}
\usepackage{bm}
%%%%%%%%%%%%%%%%%%%%%%%%%%%%%%%%%%%%%%%%%%%%%%%%%%%%%%%%%%%%%%%%%%%%%%%%%%%%%%%

% Other Settings

%%%%%%%%%%%%%%%%%%%%%%%%%% Page Setting %%%%%%%%%%%%%%%%%%%%%%%%%%%%%%%%%%%%%%%
\geometry{a4paper}

%%%%%%%%%%%%%%%%%%%%%%%%%% Define some useful colors %%%%%%%%%%%%%%%%%%%%%%%%%%
\definecolor{ocre}{RGB}{243,102,25}
\definecolor{mygray}{RGB}{243,243,244}
\definecolor{deepGreen}{RGB}{26,111,0}
\definecolor{shallowGreen}{RGB}{235,255,255}
\definecolor{deepBlue}{RGB}{61,124,222}
\definecolor{shallowBlue}{RGB}{235,249,255}
%%%%%%%%%%%%%%%%%%%%%%%%%%%%%%%%%%%%%%%%%%%%%%%%%%%%%%%%%%%%%%%%%%%%%%%%%%%%%%%

%%%%%%%%%%%%%%%%%%%%%%%%%% Define an orangebox command %%%%%%%%%%%%%%%%%%%%%%%%
\newcommand\orangebox[1]{\fcolorbox{ocre}{mygray}{\hspace{1em}#1\hspace{1em}}}
%%%%%%%%%%%%%%%%%%%%%%%%%%%%%%%%%%%%%%%%%%%%%%%%%%%%%%%%%%%%%%%%%%%%%%%%%%%%%%%

%%%%%%%%%%%%%%%%%%%%%%%%%%%% English Environments %%%%%%%%%%%%%%%%%%%%%%%%%%%%%
\newtheoremstyle{mytheoremstyle}{3pt}{3pt}{\normalfont}{0cm}{\rmfamily\bfseries}{}{1em}{{\color{black}\thmname{#1}~\thmnumber{#2}}\thmnote{\,--\,#3}}
\newtheoremstyle{myproblemstyle}{3pt}{3pt}{\normalfont}{0cm}{\rmfamily\bfseries}{}{1em}{{\color{black}\thmname{#1}~\thmnumber{#2}}\thmnote{\,--\,#3}}
\theoremstyle{mytheoremstyle}
\newmdtheoremenv[linewidth=1pt,backgroundcolor=shallowGreen,linecolor=deepGreen,leftmargin=0pt,innerleftmargin=20pt,innerrightmargin=20pt,]{theorem}{Theorem}[section]
\theoremstyle{mytheoremstyle}
\newmdtheoremenv[linewidth=1pt,backgroundcolor=shallowBlue,linecolor=deepBlue,leftmargin=0pt,innerleftmargin=20pt,innerrightmargin=20pt,]{definition}{Definition}[section]
\theoremstyle{myproblemstyle}
\newmdtheoremenv[linecolor=black,leftmargin=0pt,innerleftmargin=10pt,innerrightmargin=10pt,]{problem}{Problem}[section]
%%%%%%%%%%%%%%%%%%%%%%%%%%%%%%%%%%%%%%%%%%%%%%%%%%%%%%%%%%%%%%%%%%%%%%%%%%%%%%%

%%%%%%%%%%%%%%%%%%%%%%%%%%%%%%% Plotting Settings %%%%%%%%%%%%%%%%%%%%%%%%%%%%%
\usepgfplotslibrary{colorbrewer}
\pgfplotsset{width=8cm,compat=1.9}
%%%%%%%%%%%%%%%%%%%%%%%%%%%%%%%%%%%%%%%%%%%%%%%%%%%%%%%%%%%%%%%%%%%%%%%%%%%%%%%

%%%%%%%%%%%%%%%%%%%%%%%%%%%%%%% Title & Author %%%%%%%%%%%%%%%%%%%%%%%%%%%%%%%%
\title{Polar Coordinates}
\author{Patrick Chen}
\date{March 6, 2025}
%%%%%%%%%%%%%%%%%%%%%%%%%%%%%%%%%%%%%%%%%%%%%%%%%%%%%%%%%%%%%%%%%%%%%%%%%%%%%%%

\begin{document}
    \maketitle
    Polar coordinated describe a point by the angle and distance.
    \begin{align*}
        r^2 = x^2 + y^2, &\quad \tan\theta = \frac{y}{x} \\
        x=r\cos\theta, &\quad y=r\sin\theta
    \end{align*}
    By some convention, the principal angle is $-\pi< \theta \le \pi$ although,
    a principal angle of $0 \le \theta < 2\pi$ is also common.

    \subsection*{Negative Radius}
    A point with a $\theta \in (-\pi, \pi]$ and $r > 0$ can also be expressed
    with a negative radius.
    \begin{align*}
        \theta &\mapsto \theta + \pi \\
        r      &\mapsto -r
    \end{align*}

    \subsection*{Example 1}
    \[
        r = \frac{1}{cos(\theta) + sin(\theta)}
    \]
    \begin{align*}
        r &= \frac{1}{cos(\theta) + sin(\theta)} \\
        rcos(\theta) + rsin(\theta) &= 1 \\
        x + y &= 1 \\
        y &= 1-x
    \end{align*}
    This is a straight line with slope $-1$ and y-intercept $1$.

    \subsection*{Example 2}
    \begin{align*}
        r &= \sin \theta \\
        r^2 &= r \sin \theta \\
        x^2 + y^2 &= y \\
        x^2 + y^2 - 2(\frac{1}{2} y) + (\frac{1}{2})^2 &= (\frac{1}{2})^2 \\
        x^2 + (y-\frac{1}{2})^2 &= (\frac{1}{2})^2
    \end{align*}
    This is a circle with radius $r=\frac{1}{2}$ and origin $O=(0, \frac{1}{2})$.

    \section*{Shapes in polar Coordinates}
    \subsection*{Graphing}
    A graph of a polar equation $r=g(\theta)$ is the set of all points in the
    plane what satisfy the equation. $\theta$ does not need to be a principal
    angle and $r$ does not need to be positive. Sometimes it is easier to
    convert a polar function into a Cartesian function.

    \subsection*{Circle}
    A circle is described with $r(\theta) = c$ where $c$ is a constant. The
    radius of this circle is $|c|$.

    \section*{Polar Tangents}
    The derivative of a polar function can be calculated the same way a
    parametric function's derivative is calculated with $x=r\cos \theta$, and
    $y=r\sin \theta$.
    \begin{align*}
        \frac{dy}{dx} = \frac{\Big(\frac{dy}{d\theta}\Big)}{\Big(\frac{dx}{d\theta}\Big)}
        = \frac{\frac{dr}{d\theta} \sin(\theta) + r\cos(\theta)}{\frac{dr}{d\theta} \cos(\theta) - r\sin(\theta)}
    \end{align*}
\end{document}
