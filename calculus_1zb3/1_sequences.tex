%%%%%%%%%%%%%%%%%%%%%%%%%%%%% Define Article %%%%%%%%%%%%%%%%%%%%%%%%%%%%%%%%%%
\documentclass{article}
%%%%%%%%%%%%%%%%%%%%%%%%%%%%%%%%%%%%%%%%%%%%%%%%%%%%%%%%%%%%%%%%%%%%%%%%%%%%%%%

%%%%%%%%%%%%%%%%%%%%%%%%%%%%% Using Packages %%%%%%%%%%%%%%%%%%%%%%%%%%%%%%%%%%
\usepackage{geometry}
\usepackage{graphicx}
\usepackage{amssymb}
\usepackage{amsmath}
\usepackage{amsthm}
\usepackage{empheq}
\usepackage{mdframed}
\usepackage{booktabs}
\usepackage{lipsum}
\usepackage{graphicx}
\usepackage{color}
\usepackage{psfrag}
\usepackage{pgfplots}
\usepackage{bm}
%%%%%%%%%%%%%%%%%%%%%%%%%%%%%%%%%%%%%%%%%%%%%%%%%%%%%%%%%%%%%%%%%%%%%%%%%%%%%%%

% Other Settings

%%%%%%%%%%%%%%%%%%%%%%%%%% Page Setting %%%%%%%%%%%%%%%%%%%%%%%%%%%%%%%%%%%%%%%
\geometry{a4paper}

%%%%%%%%%%%%%%%%%%%%%%%%%% Define some useful colors %%%%%%%%%%%%%%%%%%%%%%%%%%
\definecolor{ocre}{RGB}{243,102,25}
\definecolor{mygray}{RGB}{243,243,244}
\definecolor{deepGreen}{RGB}{26,111,0}
\definecolor{shallowGreen}{RGB}{235,255,255}
\definecolor{deepBlue}{RGB}{61,124,222}
\definecolor{shallowBlue}{RGB}{235,249,255}
%%%%%%%%%%%%%%%%%%%%%%%%%%%%%%%%%%%%%%%%%%%%%%%%%%%%%%%%%%%%%%%%%%%%%%%%%%%%%%%

%%%%%%%%%%%%%%%%%%%%%%%%%% Define an orangebox command %%%%%%%%%%%%%%%%%%%%%%%%
\newcommand\orangebox[1]{\fcolorbox{ocre}{mygray}{\hspace{1em}#1\hspace{1em}}}
%%%%%%%%%%%%%%%%%%%%%%%%%%%%%%%%%%%%%%%%%%%%%%%%%%%%%%%%%%%%%%%%%%%%%%%%%%%%%%%

%%%%%%%%%%%%%%%%%%%%%%%%%%%% English Environments %%%%%%%%%%%%%%%%%%%%%%%%%%%%%
\newtheoremstyle{mytheoremstyle}{3pt}{3pt}{\normalfont}{0cm}{\rmfamily\bfseries}{}{1em}{{\color{black}\thmname{#1}~\thmnumber{#2}}\thmnote{\,--\,#3}}
\newtheoremstyle{myproblemstyle}{3pt}{3pt}{\normalfont}{0cm}{\rmfamily\bfseries}{}{1em}{{\color{black}\thmname{#1}~\thmnumber{#2}}\thmnote{\,--\,#3}}
\theoremstyle{mytheoremstyle}
\newmdtheoremenv[linewidth=1pt,backgroundcolor=shallowGreen,linecolor=deepGreen,leftmargin=0pt,innerleftmargin=20pt,innerrightmargin=20pt,]{theorem}{Theorem}[section]
\theoremstyle{mytheoremstyle}
\newmdtheoremenv[linewidth=1pt,backgroundcolor=shallowBlue,linecolor=deepBlue,leftmargin=0pt,innerleftmargin=20pt,innerrightmargin=20pt,]{definition}{Definition}[section]
\theoremstyle{myproblemstyle}
\newmdtheoremenv[linecolor=black,leftmargin=0pt,innerleftmargin=10pt,innerrightmargin=10pt,]{problem}{Problem}[section]
%%%%%%%%%%%%%%%%%%%%%%%%%%%%%%%%%%%%%%%%%%%%%%%%%%%%%%%%%%%%%%%%%%%%%%%%%%%%%%%

%%%%%%%%%%%%%%%%%%%%%%%%%%%%%%% Plotting Settings %%%%%%%%%%%%%%%%%%%%%%%%%%%%%
\usepgfplotslibrary{colorbrewer}
\pgfplotsset{width=8cm,compat=1.9}
%%%%%%%%%%%%%%%%%%%%%%%%%%%%%%%%%%%%%%%%%%%%%%%%%%%%%%%%%%%%%%%%%%%%%%%%%%%%%%%

%%%%%%%%%%%%%%%%%%%%%%%%%%%%%%% Title & Author %%%%%%%%%%%%%%%%%%%%%%%%%%%%%%%%
\title{Sequences}
\author{Patrick Chen}
\date{Jan 13, 2025}
%%%%%%%%%%%%%%%%%%%%%%%%%%%%%%%%%%%%%%%%%%%%%%%%%%%%%%%%%%%%%%%%%%%%%%%%%%%%%%%

\begin{document}
    \maketitle
    A sequence is a ordered list of real values. The list may or may not be
    infinite length. A sequence can be expressed either explicitly where there
    is a formula for the nth term or recursively where each term relies on the
    previous terms. A sequence is convergent if it approaches a value when the
    limit is taken, otherwise it is
    divergent.
    \begin{align*}
        \Big(\lim_{n\rightarrow\infty} a_n = L\Big) &\Rightarrow \text{ convergent} \\
        \Big(\lim_{n\rightarrow\infty} a_n \ne L\Big) &\Rightarrow \text{ divergent}
    \end{align*}
    For a recursive sequence, if the limit exists, it must be a fixed point of
    the recursion relation.
    \[
        \Big(\lim_{n\to \infty} a_{n+1}=f(a_n)\Big) \Rightarrow \Big(f(L)=L\Big)
    \]

    \subsection*{Example}
    For the following series, find the limit assuming that it exists.
    \begin{align*}
        a_{n+1} = \sqrt{a_n + 2} \\
        a_1 = 0 \\
    \end{align*}
    \begin{align*}
        L &= \sqrt{L+2} \\
        L^2 &= L+2 \\
        L^2 - L - 2 &= 0 \\
        (L-2)(L+1) &= 0 \\
        L &= 2, -1
    \end{align*}
    Since $-1$ can not be the result of a square root in the domain of real
    numbers, $L=2$

    \section*{Series}
    A series is an infinite sum of a sequence.
    \begin{align*}
        \sum_{n=1}^{\infty} a_n = \lim_{m\to \infty} \sum_{n=1}^{m} a_n
    \end{align*}
    If the limit exists, then the series is convergent. Otherwise the series is
    divergent. A m-th partial sum $S_m$ is the sum containing the first m
    vales.
    \begin{gather*}
        S_m = \sum_{n=1}^{m} a_n \\
        S = S_\infty = \lim_{m\to \infty} S_m
    \end{gather*}

    \subsection*{Geometric Series}
    \begin{align*}
        \sum_{i=1}^{n} ar^{i-1} = \frac{a(1-r^n)}{1-r}
    \end{align*}
    Convergence:
    \begin{align*}
        \lim_{n\to \infty} S_n &= \lim_{n\to \infty} \frac{a}{1-r} (1-r^n) \\
        &= \frac{a}{1-r} (1-\lim_{n\to \infty} r^n)
    \end{align*}
    If $|r|<1$, then the limit will be zero and the result is $\frac{a}{1-r}$.
    If $|r|>1$, then the limit will diverge to infinity. If $r=1$, then the sum
    will add a constant an infinite amount of times and will diverge to
    infinity. If $r=-1$, the sum will osculate and not converge.

    \subsection*{P-Series}
    \begin{align*}
        \sum_{n=1}^{\infty} \frac{1}{n^p}
    \end{align*}
    The p-series converge for all $p>1$ and diverge for all $p\le 1$

    \subsection*{Alternating Series}
    An alternating series is a series that alternates between adding and
    subtracting a term. If $b_n>0$ then:
    \[
        \sum_{n=1}^{\infty} (-1)^n b_n
    \]

    \section*{Convergence}
    A sequence $\{a_n\}_{n=1}^{\infty}$ is bounded if for all $n$, $a < a_n < b$
    for some values $a$, $b$. A sequence is monotonic increasing if $a_n \le
    a_{n+1}$ and is monotonic decreasing if $a_n \ge a_{n+1}$. A sequence that
    is both bonded and monotonic will converge.

    \subsection*{Example}
    Use induction to show that the sequence $a_n = \sqrt{a_n + 2}$, $a_1=0$ is
    monotonic increasing. \\
    Base Case:
    \begin{align*}
        a_1 &= 0 \\
        a_2 &= \sqrt{a_1 + 2} \\
        &= \sqrt{0+2} \\
        &= \sqrt{2} \\
        0 &\le \sqrt{2} \\
        \therefore a_1 &\le a_2
    \end{align*}
    Assume $0\le a_{k} \le a_{k+1}$ \\
    $2 \le  a_k + 2 \le a_{k+1}$ \\
    Since square roots is monotonic increasing, $\sqrt{2} \le  \sqrt{a_k + 2} \le \sqrt{a_{k+1}+2}$ \\
    Thus, $0 \le a_{k+1} \le a_{k+2}$ for all $k$

    \subsection*{Example 2}
    \begin{align*}
        \sum_{n=1}^{\infty} \ln(\frac{n}{n+1})
    \end{align*}
    \begin{align*}
        \sum_{n=1}^{\infty} \ln(\frac{n}{n+1})
        &= \sum_{n=1}^{\infty} \ln(n) - \ln(n+1) \\
        &= \lim_{m\to \infty} \sum_{n=1}^{m} \ln(n) - \ln(n+1) \\
        &= \lim_{m\to \infty} \Big(\ln(1) - \ln(2)\Big) + \Big(\ln(2) - \ln(3)\Big) + \dots + \ln(m+1) \\
        &= \lim_{m\to \infty} \ln(1) - \ln(m + 1) \\
        &= -\infty
    \end{align*}
    Therefore, this series diverges.

    \subsection*{Absolute and Conditional Convergence}
    If a positive series converges, then a series with any sign change will also
    converge.
    \[
        \Big(\sum_{n=1}^{\infty} |a_n| \text{ converges}\Big) \Rightarrow
        \Big(\sum_{n=1}^{\infty} a_n \text{ converges}\Big)
    \]
    Assume $\sum |a_n|$ converges.
    \[
        0 \le a_n + |a_n| \le 2|a_n|
    \]
    By comparison test, $\sum (a_n + |a_n|)$ will converge if $\sum |a_n|$
    converges. Since $\sum(a_n+|a_n|)$ and $\sum |a_n|$ both converges, the
    difference between them will also converge.
    \[
        \sum (a_n + |a_n|) - \sum |a_n| = \sum a_n
    \]
    If $\sum |a_n|$ diverges but $\sum a_n$ converges, it is said to be
    conditionally convergent.

    \subsection*{Divergence Test}
    If the Limit of a sequence converges to a non-zero constant, then the series
    will diverge.
    If the term goes to zero, this doesn't imply it will converge.
    \begin{align*}
        \Big(\lim_{m\to \infty} a_m \ne 0\Big)
        \Rightarrow \Big(\sum_{n=1}^{\infty} a_n \text{ diverges}\Big)
    \end{align*}
    Proof:
    \begin{align*}
        a_n &= S_m - S_{m-1} \\
        L &= \lim_{m\to \infty} S_m \\
    \end{align*}
    \begin{align*}
        \lim_{m\to \infty} S_m - S_{m-1} &= \lim_{m\to \infty} a_m \\
        L-L &= \lim_{m\to \infty} a_m \\
        0 &= \lim_{m\to \infty} a_m
    \end{align*}

    \subsection*{Integral Test}
    The integral test works by comparing the area under the graph of a function
    $f(n)=a_n$ with the area of a summation. If a $f(n)$ is positive,
    continuous, and monotonic decreasing, then the convergence of the integral
    of $f(x)$ will imply the convergence of the series. The $f(x)$ only needs to
    be monotonic decreasing after a certain point $a$ for this test to work.
    \[
        \Big(\int_{a}^{\infty} f(x) \ dx \text{ converges}\Big)
        \Rightarrow \Big(\sum_{n=a}^{\infty} a_n \text{ converges}\Big)
    \]

    \subsection*{Example}
    \begin{align*}
        \sum_{n=2}^{\infty} \frac{1}{n\ln(n)}
    \end{align*}
    \begin{align*}
        \text{Let } f(x) = \frac{1}{x\ln(x)}
    \end{align*}
    $f(x)$ is positive for values of $n\ge 2$ and continuous for $n\ge 2$. Since
    the denominator is always growing, then $f(x)$ is monotonic decreasing.
    \begin{align*}
        \int_{2}^{\infty} \frac{1}{x\ln(x)} \ dx \\
        u = \ln(x) \\
        du = \frac{1}{x} dx \\
        \int_{\ln(2)}^{\infty} \frac{1}{u} \ du
    \end{align*}
    Since this integral diverges, the series will also diverge.

    \subsection*{Comparison Test}
    If there are to series $A$ and $B$ with $0\le a_n \le b_n$, then if $A$ diverges
    to infinity, then $B$ will diverge to infinity. If $B$ converges to a finite
    value and $a_n$ is monotonic, then $A$ will converge because $a_n$ is
    monotonic and bounded.

    \subsection*{Limit Comparison Test}
    For two series $A$ and $B$, if the limit of the ratio between terms $L$ is a
    non-zero finite value, then $A$ will converge if and only if $B$ converges
    and $A$ will diverge if and only if $B$ diverges.
    \[
        \lim_{n\to \infty} \frac{a_n}{b_n} = L
    \]

    \subsection*{Alternating Series Test}
    An alternating series converges if $b_n$ is monotonic decreasing and the
    limit goes to zero.
    \[
        b_n \ge b_{n+1}, \quad \lim_{n\to \infty} b_n = 0
    \]
    % when split S_even and S_odd is monotonic and bounded
    If the limit does not go to zero, the divergence test shows that it
    diverges. If the terms are not monotonic decreasing, it may or may not
    converge.

    \subsection*{Ratio Test}
    In a geometric series, the next term is a ratio of the previous term. Since
    the geometric series is absolutely convergent, any sequence that behaves
    like geometric series will be absolutely convergent.
    \[
        \sum_{n=1}^{\infty} ar^{n-1} = a + ar + ar^2 + \dots
    \]
    Given a series, if the limit of the absolute value of the ratio between
    consecutive terms is less than 1, then it will converge. If it is greater
    than 1, it will diverge. The ratio test does not provide any useful
    information if the ratio is equal to 1.
    \[
        \lim_{n\to \infty} \Big|\frac{a_n}{a_{n+1}}\Big| = L
    \]
    \begin{align*}
        \big(L = 1\big) &\Rightarrow \text{ No useful information} \\
        \big(L < 1\big) &\Rightarrow \text{ Convergence} \\
        \big(L > 1\big) &\Rightarrow \text{ Divergence} \\
    \end{align*}

    \subsection*{Root Test}
    The root test is more powerful than the ratio test but is much more painful
    to work with.
    \begin{align*}
        \sqrt[n]{|a_n|}
        &= |a_n|^{\frac{1}{n}} \\
        &= |a|^{\frac{1}{n}}|r|^{\frac{n-1}{n}} \\
        \lim_{n\to \infty} |a_n|^{\frac{1}{n}} &= |r|
    \end{align*}
    \[
        L = \lim_{n\to \infty} \sqrt[n]{|a_n|}
    \]
    \begin{align*}
        \big(L = 1\big) &\Rightarrow \text{ No useful information} \\
        \big(L < 1\big) &\Rightarrow \text{ Convergence} \\
        \big(L > 1\big) &\Rightarrow \text{ Divergence} \\
    \end{align*}

    \subsection*{Example}
    Does the following series converge? Use the ratio test.
    \begin{align*}
        \sum_{n=1}^{\infty} \frac{3\cdot 5\cdot 7\cdots (2n+1)}{2n!}
    \end{align*}
    \begin{align*}
        | \frac{a_{n+1}}{a_n} |
        &= 
        \frac{\frac{3\cdot 5\cdot 7\cdots (2(n+1)+1)}{(2(n+1))!}}
        {\frac{3\cdot 5\cdot 7\cdots (2n+1)}{2n!}} \\
        &= 
        \frac{\frac{3\cdot 5\cdot 7\cdots (2n+1)\cdot(2n+3)}{(n+1)(n+2)\cdot(2n)!}}
        {\frac{3\cdot 5\cdot 7\cdots (2n+1)}{2n!}} \\
        &= 
        \frac{2n+3} {(n+1)(n+2)} \\
        \lim_{n\to \infty} | \frac{a_{n+1}}{a_n} |
        &= \lim_{n\to \infty} \frac{2n+3} {(n+1)(n+2)} \\
        &= 0
    \end{align*}
    Since the limit goes to zero, this sequence is absolutely convergent.

    \subsection*{Example 2}
    Does the following series converge? Use the root test.
    \begin{align*}
        \sum_{n=1}^{\infty} \Big(\frac{n^2+2n+1}{3n^2+n}\Big)^{2n}
    \end{align*}
    \begin{align*}
        a_n           &= \Big(\frac{n^2+2n+1}{3n^2+n}\Big)^{2n} \\
        \sqrt[n]{a_n} &= \bigg(\Big(\frac{n^2+2n+1}{3n^2+n}\Big)^{2n}\bigg)^{\frac{1}{n}} \\
                      &= \Big(\frac{n^2+2n+1}{3n^2+n}\Big)^{2} \\
        \lim_{n\to \infty} \sqrt[n]{a_n} 
                      &= \lim_{n\to \infty} \Big(\frac{n^2+2n+1}{3n^2+n}\Big)^{2} \\
                      &= \Big(\lim_{n\to \infty} \frac{n^2+2n+1}{3n^2+n}\Big)^{2} \\
                      &= \Big(\frac{1}{3}\Big)^{2} \\
                      &= \frac{1}{9}
    \end{align*}
    Thus, this series is absolutely convergent.

\end{document}
