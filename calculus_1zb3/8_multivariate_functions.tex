%%%%%%%%%%%%%%%%%%%%%%%%%%%%% Define Article %%%%%%%%%%%%%%%%%%%%%%%%%%%%%%%%%%
\documentclass{article}
%%%%%%%%%%%%%%%%%%%%%%%%%%%%%%%%%%%%%%%%%%%%%%%%%%%%%%%%%%%%%%%%%%%%%%%%%%%%%%%

%%%%%%%%%%%%%%%%%%%%%%%%%%%%% Using Packages %%%%%%%%%%%%%%%%%%%%%%%%%%%%%%%%%%
\usepackage{geometry}
\usepackage{graphicx}
\usepackage{amssymb}
\usepackage{amsmath}
\usepackage{amsthm}
\usepackage{empheq}
\usepackage{mdframed}
\usepackage{booktabs}
\usepackage{lipsum}
\usepackage{graphicx}
\usepackage{color}
\usepackage{psfrag}
\usepackage{pgfplots}
\usepackage{bm}
%%%%%%%%%%%%%%%%%%%%%%%%%%%%%%%%%%%%%%%%%%%%%%%%%%%%%%%%%%%%%%%%%%%%%%%%%%%%%%%

% Other Settings

%%%%%%%%%%%%%%%%%%%%%%%%%% Page Setting %%%%%%%%%%%%%%%%%%%%%%%%%%%%%%%%%%%%%%%
\geometry{a4paper}

%%%%%%%%%%%%%%%%%%%%%%%%%% Define some useful colors %%%%%%%%%%%%%%%%%%%%%%%%%%
\definecolor{ocre}{RGB}{243,102,25}
\definecolor{mygray}{RGB}{243,243,244}
\definecolor{deepGreen}{RGB}{26,111,0}
\definecolor{shallowGreen}{RGB}{235,255,255}
\definecolor{deepBlue}{RGB}{61,124,222}
\definecolor{shallowBlue}{RGB}{235,249,255}
%%%%%%%%%%%%%%%%%%%%%%%%%%%%%%%%%%%%%%%%%%%%%%%%%%%%%%%%%%%%%%%%%%%%%%%%%%%%%%%

%%%%%%%%%%%%%%%%%%%%%%%%%% Define an orangebox command %%%%%%%%%%%%%%%%%%%%%%%%
\newcommand\orangebox[1]{\fcolorbox{ocre}{mygray}{\hspace{1em}#1\hspace{1em}}}
%%%%%%%%%%%%%%%%%%%%%%%%%%%%%%%%%%%%%%%%%%%%%%%%%%%%%%%%%%%%%%%%%%%%%%%%%%%%%%%

%%%%%%%%%%%%%%%%%%%%%%%%%%%% English Environments %%%%%%%%%%%%%%%%%%%%%%%%%%%%%
\newtheoremstyle{mytheoremstyle}{3pt}{3pt}{\normalfont}{0cm}{\rmfamily\bfseries}{}{1em}{{\color{black}\thmname{#1}~\thmnumber{#2}}\thmnote{\,--\,#3}}
\newtheoremstyle{myproblemstyle}{3pt}{3pt}{\normalfont}{0cm}{\rmfamily\bfseries}{}{1em}{{\color{black}\thmname{#1}~\thmnumber{#2}}\thmnote{\,--\,#3}}
\theoremstyle{mytheoremstyle}
\newmdtheoremenv[linewidth=1pt,backgroundcolor=shallowGreen,linecolor=deepGreen,leftmargin=0pt,innerleftmargin=20pt,innerrightmargin=20pt,]{theorem}{Theorem}[section]
\theoremstyle{mytheoremstyle}
\newmdtheoremenv[linewidth=1pt,backgroundcolor=shallowBlue,linecolor=deepBlue,leftmargin=0pt,innerleftmargin=20pt,innerrightmargin=20pt,]{definition}{Definition}[section]
\theoremstyle{myproblemstyle}
\newmdtheoremenv[linecolor=black,leftmargin=0pt,innerleftmargin=10pt,innerrightmargin=10pt,]{problem}{Problem}[section]
%%%%%%%%%%%%%%%%%%%%%%%%%%%%%%%%%%%%%%%%%%%%%%%%%%%%%%%%%%%%%%%%%%%%%%%%%%%%%%%

%%%%%%%%%%%%%%%%%%%%%%%%%%%%%%% Plotting Settings %%%%%%%%%%%%%%%%%%%%%%%%%%%%%
\usepgfplotslibrary{colorbrewer}
\pgfplotsset{width=8cm,compat=1.9}
%%%%%%%%%%%%%%%%%%%%%%%%%%%%%%%%%%%%%%%%%%%%%%%%%%%%%%%%%%%%%%%%%%%%%%%%%%%%%%%

%%%%%%%%%%%%%%%%%%%%%%%%%%%%%%% Title & Author %%%%%%%%%%%%%%%%%%%%%%%%%%%%%%%%
\title{Multivariate Functions}
\author{Patrick Chen}
\date{March 12, 2025}
%%%%%%%%%%%%%%%%%%%%%%%%%%%%%%%%%%%%%%%%%%%%%%%%%%%%%%%%%%%%%%%%%%%%%%%%%%%%%%%

\begin{document}
    \maketitle
    \section*{Single Variables Functions}
    Single variable functions have one independent variable and one dependent
    variable. They can be graphed in $\mathbb{R}^2$.

    \section*{Multi Variable Functions}
    Functions with $n$ independent variables and $1$ dependent variables can be
    graphed in $\mathbb{R}^{n+1}$. In three dimensions with the axis $(x,y,z)$,
    the axis is usually ordered by the right hand rule. For $z=f(x,y)$, the
    domain is a set of $(x,y)$ pairs. The range is some interval on $z$ axis.
    \[
        f: \mathbb{R}^2 \mapsto \mathbb{R}
    \]

    \subsection*{Example}
    Find the domain of the following functions
    \begin{itemize}
        \item $z = \sqrt{4-x^2-y^2}$
            \begin{align*}
                4-x^2-y^2 &> 0 \\
                2^2 &> x^2+y^2
            \end{align*}
            The domain is the inside and border of a circle on the xy-plane
            with a radius of 2.
            \[
                D = \{(x,y)\in \mathbb{R}^2\ |\ x^2+y^2 \le 4\}
            \]
        \item $z=\ln(x^2-y^2-1)$
            \begin{align*}
                x^2-y^2-1 &> 0 \\
                x^2 - y ^2 &> 1
            \end{align*}
            This is a graph of a hyperbola.
    \end{itemize}

    \subsection*{Level Sets}
    A level set (also called contours) is a set of points in the domain where
    the function has the same value. Since a function can only result in one
    output, level sets cannot cross each other.
    \[
        L_k = \{(x,y)\in \mathbb{R}^2\ |\ f(x,y) = k\}
    \]

    \section*{Limits and Continuity in 3D}
    For a function of two variables $f(x,y)$, a limit exists if for all
    directions, the limit approaches the same value $L$. A function is
    continuous at a point $(a,b)$ if when the limit as $(x,y)$ approaches
    $(a,b)$, $f(a,b)=f(x,y)$.
    \begin{align*}
        \lim_{(x,y)\to (a,b)} f(x,y) = L
    \end{align*}

    Most of the single variable rules apply to the multi variable limits except
    for H'Lopital's rule. H'Lopital's rule only applies to single variables.
    \begin{align*}
        \lim_{(x,y)\rightarrow(a,b)} f(x,y) = L
        \lim_{(x,y)\rightarrow(a,b)} g(x,y) = G
    \end{align*}
    \begin{align*}
        \lim_{(x,y)\rightarrow(a,b)} kf(x,y) &= kL \\
        \lim_{(x,y)\rightarrow(a,b)} f(x,y) + g(x,y) &= L + G \\
        \lim_{(x,y)\rightarrow(a,b)} f(x,y)g(x,y) &= LG \\
        \lim_{(x,y)\rightarrow(a,b)} \frac{f(x,y)}{g(x,y)} &= \frac{L}{G}
    \end{align*}

    \subsection*{Example}
    \begin{align*}
        \lim_{(x,y)\to (0,0)} \frac{x^2-y^2}{x^2+y^2}
    \end{align*}
    When $x$ is fixed to $0$
    \begin{align*}
        \lim_{y\to 0} \frac{-y^2}{y^2} = -1
    \end{align*}
    When $y$ is fixed to $0$
    \begin{align*}
        \lim_{x\to 0} \frac{x^2}{x^2} = 1
    \end{align*}
    Since the directional limits do not agree with each other, the limit does
    not exist.

    \subsection*{Example 2}
    \begin{align*}
        \lim_{(x,y)\to (0,0)} \frac{xy}{x^2+y^2}
    \end{align*}
    Fix $x=0$
    \begin{align*}
        \lim_{y\to 0} \frac{(0)y}{(0)^2+y^2} = \lim_{y\to 0} \frac{0}{y^2} =  0
    \end{align*}
    Fix $y=0$
    \begin{align*}
        \lim_{x\to 0} \frac{x(0)}{x^2+(0)^2} = 
        \lim_{x\to 0} \frac{0}{x^2} = 0
    \end{align*}
    Fix $x=y$
    \begin{align*}
        \lim_{x\to 0} \frac{xx}{x^2+x^2} = \lim_{x\to 0} \frac{x^2}{2x^2} = \frac{1}{2}
    \end{align*}
    Since the directional limits do not agree with each other, the limit does
    not exist.

    \subsection*{Example 3}
    \begin{align*}
        \lim_{(x,y)\to (0,0)} \frac{xy^2}{x^2+y^4}
    \end{align*}
    Fix $x=0$
    \begin{align*}
        \lim_{y\to 0} \frac{(0)y^2}{(0)^2+y^4} = \lim_{y\to 0} \frac{0}{y^4} = 0
    \end{align*}
    Fix $y=0$
    \begin{align*}
        \lim_{x\to 0} \frac{x(0)^2}{x^2+(0)^4} = \lim_{x\to 0} 0/x^2 = 0
    \end{align*}
    Fix $y=mx$
    \begin{align*}
        \lim_{x\to 0} \frac{x(mx)^2}{x^2+m^4x^4}
        = \lim_{x\to 0} \frac{m^2x^3}{x^2(1+m^4x^2)}
        = \lim_{x\to 0} \frac{m^2x}{1+m^4x^2} = \frac{0}{1} = 0
    \end{align*}
    Fix $x=y^2$
    \begin{align*}
        \lim_{y\to 0} \frac{y^2y^2}{(y^2)^2+y^4} = \lim_{y\to 0} \frac{y^4}{2y^4} = \frac{1}{2}
    \end{align*}
    Even though every directional limit from linear directions is zero, since
    there is one path that doesn't agree with the other limits, the limit
    doesn't exist.

    \subsection*{Multi Variable Squeeze Theorem}
    Given $f(x,y) \le g(x,y) \le h(x,y)$ and that the limit of $f(x,y)$ and
    $h(x,y)$ approaches the same finite value, the limit of $g(x,y)$ must
    approach the same finite value.
    \[
        \Big(\lim_{(x,y)\to(a,b)}f(x,y)=L\ \text{ and } \lim_{(x,y)\to(a,b)}h(x,y)=L\Big)
        \Rightarrow \lim_{(x,y)\to(a,b)}g(x,y)=L
    \]
    If $0\le |g(x,y)-L|\le h(x,y)$ and $h(x,y)$ approaches zero, then
    $g(x,y)=L$.

    \subsection*{Example 4}
    \[
        f(x,y) = \frac{xy^2}{x^2+y^2} 
    \]
    \begin{align*}
        0 \le |f(x,y)| &\le \Big|\frac{xy^2}{x^2+y^2}\Big| \\
        0 \le |f(x,y)| &\le \frac{|x|y^2}{x^2+y^2} \\
        0 \le |f(x,y)| &\le \frac{|x|(x^2+y^2)}{x^2+y^2} \\
        0 \le |f(x,y)| &\le |x|
    \end{align*}
    Thus,
    \begin{align*}
        0 \le \lim_{(x,y)\to (0,0)} |f(x,y)| &\le \lim_{x\to 0} |x| \\
        0 \le \lim_{(x,y)\to (0,0)} |f(x,y)| &\le 0
    \end{align*}
    Therefore by squeeze theorem,
    \[
        \lim_{(x,y)\to (0,0)} f(x,y) = 0
    \]

\end{document}
