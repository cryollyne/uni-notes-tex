%%%%%%%%%%%%%%%%%%%%%%%%%%%%% Define Article %%%%%%%%%%%%%%%%%%%%%%%%%%%%%%%%%%
\documentclass{article}
%%%%%%%%%%%%%%%%%%%%%%%%%%%%%%%%%%%%%%%%%%%%%%%%%%%%%%%%%%%%%%%%%%%%%%%%%%%%%%%

%%%%%%%%%%%%%%%%%%%%%%%%%%%%% Using Packages %%%%%%%%%%%%%%%%%%%%%%%%%%%%%%%%%%
\usepackage{geometry}
\usepackage{graphicx}
\usepackage{amssymb}
\usepackage{amsmath}
\usepackage{amsthm}
\usepackage{empheq}
\usepackage{mdframed}
\usepackage{booktabs}
\usepackage{lipsum}
\usepackage{graphicx}
\usepackage{color}
\usepackage{psfrag}
\usepackage{pgfplots}
\usepackage{bm}
%%%%%%%%%%%%%%%%%%%%%%%%%%%%%%%%%%%%%%%%%%%%%%%%%%%%%%%%%%%%%%%%%%%%%%%%%%%%%%%

% Other Settings

%%%%%%%%%%%%%%%%%%%%%%%%%% Page Setting %%%%%%%%%%%%%%%%%%%%%%%%%%%%%%%%%%%%%%%
\geometry{a4paper}

%%%%%%%%%%%%%%%%%%%%%%%%%% Define some useful colors %%%%%%%%%%%%%%%%%%%%%%%%%%
\definecolor{ocre}{RGB}{243,102,25}
\definecolor{mygray}{RGB}{243,243,244}
\definecolor{deepGreen}{RGB}{26,111,0}
\definecolor{shallowGreen}{RGB}{235,255,255}
\definecolor{deepBlue}{RGB}{61,124,222}
\definecolor{shallowBlue}{RGB}{235,249,255}
%%%%%%%%%%%%%%%%%%%%%%%%%%%%%%%%%%%%%%%%%%%%%%%%%%%%%%%%%%%%%%%%%%%%%%%%%%%%%%%

%%%%%%%%%%%%%%%%%%%%%%%%%% Define an orangebox command %%%%%%%%%%%%%%%%%%%%%%%%
\newcommand\orangebox[1]{\fcolorbox{ocre}{mygray}{\hspace{1em}#1\hspace{1em}}}
%%%%%%%%%%%%%%%%%%%%%%%%%%%%%%%%%%%%%%%%%%%%%%%%%%%%%%%%%%%%%%%%%%%%%%%%%%%%%%%

%%%%%%%%%%%%%%%%%%%%%%%%%%%% English Environments %%%%%%%%%%%%%%%%%%%%%%%%%%%%%
\newtheoremstyle{mytheoremstyle}{3pt}{3pt}{\normalfont}{0cm}{\rmfamily\bfseries}{}{1em}{{\color{black}\thmname{#1}~\thmnumber{#2}}\thmnote{\,--\,#3}}
\newtheoremstyle{myproblemstyle}{3pt}{3pt}{\normalfont}{0cm}{\rmfamily\bfseries}{}{1em}{{\color{black}\thmname{#1}~\thmnumber{#2}}\thmnote{\,--\,#3}}
\theoremstyle{mytheoremstyle}
\newmdtheoremenv[linewidth=1pt,backgroundcolor=shallowGreen,linecolor=deepGreen,leftmargin=0pt,innerleftmargin=20pt,innerrightmargin=20pt,]{theorem}{Theorem}[section]
\theoremstyle{mytheoremstyle}
\newmdtheoremenv[linewidth=1pt,backgroundcolor=shallowBlue,linecolor=deepBlue,leftmargin=0pt,innerleftmargin=20pt,innerrightmargin=20pt,]{definition}{Definition}[section]
\theoremstyle{myproblemstyle}
\newmdtheoremenv[linecolor=black,leftmargin=0pt,innerleftmargin=10pt,innerrightmargin=10pt,]{problem}{Problem}[section]
%%%%%%%%%%%%%%%%%%%%%%%%%%%%%%%%%%%%%%%%%%%%%%%%%%%%%%%%%%%%%%%%%%%%%%%%%%%%%%%

%%%%%%%%%%%%%%%%%%%%%%%%%%%%%%% Plotting Settings %%%%%%%%%%%%%%%%%%%%%%%%%%%%%
\usepgfplotslibrary{colorbrewer}
\pgfplotsset{width=8cm,compat=1.9}
%%%%%%%%%%%%%%%%%%%%%%%%%%%%%%%%%%%%%%%%%%%%%%%%%%%%%%%%%%%%%%%%%%%%%%%%%%%%%%%

%%%%%%%%%%%%%%%%%%%%%%%%%%%%%%% Title & Author %%%%%%%%%%%%%%%%%%%%%%%%%%%%%%%%
\title{Multivariate Integration}
\author{Patrick Chen}
\date{April 2, 2025}
%%%%%%%%%%%%%%%%%%%%%%%%%%%%%%%%%%%%%%%%%%%%%%%%%%%%%%%%%%%%%%%%%%%%%%%%%%%%%%%

\begin{document}
    \maketitle
    \section*{Riemann Sum}
    For a function $f: \mathbb{R} \mapsto \mathbb{R}$ of one independent
    variable,  the area under a function  can be expressed as a Riemann sum.
    \[
        \sum_{i=1}^{n} f(x_i)\Delta x
    \]
    For multiple independent variables, $f: \mathbb{R}^2 \mapsto \mathbb{R}$ can
    be expressed as a multidimensional Riemann sum. The volume under a function
    of two independent variables in the domain $a\le x\le b$ and $c\le y\le d$
    can be expressed as follows.
    \begin{align*}
        A &= \sum_{i=1}^{n} \sum_{j=1}^{m} f(x_i,y_j) \Delta x \Delta y \\
        \Delta x &= \frac{b-a}{m} \\
        \Delta y &= \frac{d-c}{n}
    \end{align*}
    By convention, $+x$ is described as right, $-x$ is described as left, $+y$
    is described as upper and $-y$ is described as lower. Interval notation can
    be extended into multiple dimensions by defining $\mathbf{r}\in[a,b]\times[c,d]$ as
    $x\in[a,b]$, $y\in[c,d]$.

    \section*{Definite Integrals}
    \[
        \int_R f(x,y) dA = \iint_R f(x,y)\ dx\ dy = \lim_{m,n\to
        \infty}\sum_{i=1}^{m} \sum_{j=1}^{n} f(x_1,y_1)\Delta x \Delta y
    \]

    Given that $f(x,y)$ and $g(x,y)$ are continuous function on disjoint
    rectangular regions $R,R_1$:
    \begin{itemize}
        \item $\int_R f(x,y)\ dA$ is the volume below the surface
        \item $\int_{R,R_1} f(x,y)\ dA = \int_R f(x,y)\ dA + \int_{R_1} f(x,y)\ dA$
        \item $\int_R f(x,y) + g(x,y)\ dA = \int_R f(x,y)\ dA + \int_R g(x,y)\ dA$
        \item $\int_R kf(x,y)\ dA = k \int_R f(x,y)\ dA$
        \item $f(x,y) \le g(x,y)$ for all $(x,y)\in R$ implies that
            $\int_R f(x,y)\ dA \le \int_R g(x,y)\ dA$
    \end{itemize}

    \subsection*{Fubini's Theorem}
    If $f(x,y)$ is continuous on a finite rectangle $R=\{a\le x\le y,\ c\le y\le d\}$
    then the integral commutes.
    \[
        \int_R f(x,y)\ dA = \int_a^b \int_c^d f(x,y)\ dy\ dx = \int_c^d \int_a^b f(x,y)\ dx\ dy
    \]
    When evaluating an integral, integrate from the inside out. Since the outer
    variable is constant when integrating along the inside variable's domain,
    the inner integral can be evaluated like a integral of one dimension.

    \subsection*{Separable Integrals}
    An integral over a rectangular region $R$ is a separable integral if the
    integrand has the  form $f(x)g(y)$.
    \[
        \int_{a}^{b} \int_{c}^{d} f(x)g(y) \ dy \ dx
        = \int_{a}^{b} f(x)\cdot \Big(\int_{c}^{d} g(y) \ dy\Big) \ dx
        = \Big(\int_{a}^{b} f(x) \ dx\Big)\Big(\int_{c}^{d} g(y) \ dy\Big)
    \]

    \subsection*{Example}
    Approximate the volume under the surface $z=xy^2$ on the rectangular region
    given by $x\in [1,5]$, $y\in[2,8]$ using two sub-intervals in the $x$
    direction and three in the $y$ direction. Use upper left endpoints.

    \begin{align*}
        m = 2 &\Rightarrow \Delta x = \frac{5-1}{2} = 2 \\
        n = 3 &\Rightarrow \Delta y = \frac{8-2}{3} = 2
    \end{align*}
    \begin{gather*}
        x_1 = 1,\quad x_2 = 3 \\
        y_1 = 4,\quad y_2 = 6,\quad y_3 = 8
    \end{gather*}
    \begin{align*}
        A &= \sum_{i=1}^{n} \sum_{j=1}^{m} f(x_i,y_j) \Delta x \Delta y \\
        A &= \sum_{i=1}^{n} \sum_{j=1}^{m} f(x_i,y_j) 2\cdot 2 \\
        A &= 4 (
            f(1,4) + f(1,6) + f(1,8)
            +f(3,4) + f(3,6) + f(3,8)
        ) \\
        A &= 4 (
            1(4^2) + 1(6)^2 + 1(8^2)
            + 3(4^2) + 3(6)^2 + 3(8^2)
        ) \\
        A &= 1856
    \end{align*}

    \subsection*{Example 2}
    \begin{align*}
        \int x\cos(xy) \ dA,&\quad R=[0,\pi]\times [1,2] \\
        \int_{0}^{\pi} \int_{1}^{2} x\cos(xy) \ dy \ dx
        &= \int_{0}^{\pi} \Big[\sin(xy)\Big]_{y=1}^2 \ dx \\
        &= \int_{0}^{\pi} \sin(2x) - \sin(x) \ dx \\
        &= \Big[ -\frac{1}{2} \cos(2x) + \cos(x)\Big]_{x=0}^\pi \\
        &= -2
    \end{align*}

    \subsection*{Example 3}
    \begin{align*}
        \int_{[-2,2]\times[0,1]} x(1-y^2)\ dA
        &= \int_{-2}^2 \int_0^1 x(1-y^2)\ dy\ dx \\
        &= \int_{-2}^2 x\int_0^1 (1-y^2)\ dy\ dx \\
        &= \int_{-2}^2 x \Big[y - \frac{1}{3} y^3\Big]_{y=0}^1 dx \\
        &= \int_{-2}^2 x (\frac{2}{3}) dx \\
        &= (\frac{2}{3}) \int_{-2}^2 x dx \\
        &= (\frac{2}{3}) (0)
    \end{align*}
    Another way to evaluate it
    \begin{align*}
        \int_{[-2,2]\times[0,1]} x(1-y^2)\ dA
        &= \int_{-2}^2 \int_0^1 x(1-y^2)\ dy\ dx \\
        &= \Big(\int_{-2}^2 x\ dx\Big) \Big(\int_0^1 (1-y^2)\ dy\Big)\\
        &= (0) \int_0^1 (1-y^2)\ dy\\
        &= 0
    \end{align*}

    \section*{Non rectangular domains}
    \subsection*{Type 1 Regions}
    Type 1 regions can be cut into continuous strips in the $y$ directions. If
    the upper bound is given by $h(x)$ and the lower bound is given by $g(x)$,
    then the integral can be found by integrating over the $y$ axis then the $x$
    axis.
    \[
        A = \int_{a}^{b} \int_{g(x)}^{h(x)} f(x,y) \ dy \ dx
    \]
    \subsection*{Type 2 Regions}
    Type 2 regions are regions that can be cut into continuous strips in the $x$
    direction. The integral can be evaluated similarly to type 1 regions
    \[
        A = \int_{c}^{d} \int_{p(y)}^{q(y)} f(x,y) \ dx \ dy
    \]

    \subsection*{Other Regions}
    If a region is nether type 1 nor type 2, it can be broken up into components
    that are either type 1 or type 2. To find the overall integral, evaluate the
    integrals of the subregions and sum the result.
    \[
        \int_{R,R_1} f(x,y) dA = \int_R f(x,y) dA + \int_{R_1} f(x,y) dA
    \]

    \subsection*{Example 1}
    Let $D=\{(x,y)\ |\ x^2+y^2\le4, y\ge 0\}$ be a semicircle that of radius 2
    on the $+x,+y$ and $-x,+y$ quadrant. Evaluate $\int_D x^2y \ dA$
    \begin{align*}
        x^2 + y^2 &= 4 \\
        y_{\text{min}} &= 0 \\
        y_{\text{max}} &= \sqrt{4-x^2}
    \end{align*}
    \begin{align*}
        \int_D x^2y\ dA 
        &= \int_{-2}^{2} \int_{0}^{\sqrt{4-x^2}} x^2y \ dy \ dx \\
        &= \int_{-2}^{2} \Big[ \frac{1}{2} x^2y^2\Big]_{y=0}^{\sqrt{4-x^2}} \ dy \ dx \\
        &= \int_{-2}^{2} \frac{1}{2} x^2\Big(\sqrt{4-x^2}\Big)^2 \ dy \ dx \\
        &= \frac{1}{2} \int_{-2}^{2} x^2(4-x^2) \ dx \\
        &= \frac{1}{2} \int_{-2}^{2} 4x^2 - x^4 \ dx \\
        &= \Big[ \frac{2x^3}{3} - \frac{x^5}{10} \Big]_{x=-2}^{2} \\
        &= \frac{64}{15}
    \end{align*}

    \subsection*{Example 2}
    Evaluate the integral of $f(x,y)=x^2-y$ on the region enclosed by $y=x^2+3$
    and $y=4x^2$.
    \begin{align*}
        x^2+3 = 4x^2 \\
        3x^2 = 3 \\
        x^2 = 1 \\
        x = \pm 1
    \end{align*}
    Thus the endpoints on the $x$ axis is $-1$ and $1$.
    \begin{align*}
        \int_D f(x,y)\ dA 
        &= \int_{-1}^{1} \int_{4x^2}^{x^2+3} x^2-y \ dy \ dx \\
        &= \int_{-1}^{1} \Big[x^2y - \frac{y^2}{2} \Big]_{y=4x^2}^{x^2+3} \ dx \\
        &= \int_{-1}^{1} \Big(x^2(x^2+3) - \frac{(x^2+3)^2}{2}\Big) - \Big(x^2(4x^2) - \frac{(4x^2)^2}{2}\Big) \ dx \\
        &= \int_{-1}^{1} (x^4+3x^2 - \frac{x^4}{2} - \frac{6x^2}{2} - \frac{9}{2}) - (4x^4-8x^4) \ dx \\
        &= \int_{-1}^{1} \frac{x^4}{2} -\frac{9}{2} + 4x^4 \ dx \\
        &= \int_{-1}^{1} \frac{9x^4}{2} -\frac{9}{2} \ dx \\
        &= \frac{9}{2} \Big[ \frac{x^5}{5} - x\Big]_{x=-1}^{1} \\
        &= -\frac{36}{5}
    \end{align*}

    \subsection*{Area of a Shape}
    The area can of a shape $R$ be computed by integrating $1$ over the domain.
    \[
        A_R = \int_R 1\ dA
    \]

    \subsection*{Average Height of a 3D function}
    The average height is the integral of the function divided by the area
    \[
        Z_{\text{avg}} = \frac{\int_D f(x,y)\ dA}{\int_D 1\ dA}
    \]

    \subsection*{Volume Between Two Surfaces}
    The volume between two surfaces can be computed in the same way area between
    two functions can be computed
    \begin{align*}
        A &= \int_R f(x,y)-g(x,y)\ dA \\
        \text{where} f(x,y) &\ge g(x,y)\ \forall (x,y)\in R
    \end{align*}

\end{document}
