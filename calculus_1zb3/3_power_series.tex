%%%%%%%%%%%%%%%%%%%%%%%%%%%%% Define Article %%%%%%%%%%%%%%%%%%%%%%%%%%%%%%%%%%
\documentclass{article}
%%%%%%%%%%%%%%%%%%%%%%%%%%%%%%%%%%%%%%%%%%%%%%%%%%%%%%%%%%%%%%%%%%%%%%%%%%%%%%%

%%%%%%%%%%%%%%%%%%%%%%%%%%%%% Using Packages %%%%%%%%%%%%%%%%%%%%%%%%%%%%%%%%%%
\usepackage{geometry}
\usepackage{graphicx}
\usepackage{amssymb}
\usepackage{amsmath}
\usepackage{amsthm}
\usepackage{empheq}
\usepackage{mdframed}
\usepackage{booktabs}
\usepackage{lipsum}
\usepackage{graphicx}
\usepackage{color}
\usepackage{psfrag}
\usepackage{pgfplots}
\usepackage{bm}
%%%%%%%%%%%%%%%%%%%%%%%%%%%%%%%%%%%%%%%%%%%%%%%%%%%%%%%%%%%%%%%%%%%%%%%%%%%%%%%

% Other Settings

%%%%%%%%%%%%%%%%%%%%%%%%%% Page Setting %%%%%%%%%%%%%%%%%%%%%%%%%%%%%%%%%%%%%%%
\geometry{a4paper}

%%%%%%%%%%%%%%%%%%%%%%%%%% Define some useful colors %%%%%%%%%%%%%%%%%%%%%%%%%%
\definecolor{ocre}{RGB}{243,102,25}
\definecolor{mygray}{RGB}{243,243,244}
\definecolor{deepGreen}{RGB}{26,111,0}
\definecolor{shallowGreen}{RGB}{235,255,255}
\definecolor{deepBlue}{RGB}{61,124,222}
\definecolor{shallowBlue}{RGB}{235,249,255}
%%%%%%%%%%%%%%%%%%%%%%%%%%%%%%%%%%%%%%%%%%%%%%%%%%%%%%%%%%%%%%%%%%%%%%%%%%%%%%%

%%%%%%%%%%%%%%%%%%%%%%%%%% Define an orangebox command %%%%%%%%%%%%%%%%%%%%%%%%
\newcommand\orangebox[1]{\fcolorbox{ocre}{mygray}{\hspace{1em}#1\hspace{1em}}}
%%%%%%%%%%%%%%%%%%%%%%%%%%%%%%%%%%%%%%%%%%%%%%%%%%%%%%%%%%%%%%%%%%%%%%%%%%%%%%%

%%%%%%%%%%%%%%%%%%%%%%%%%%%% English Environments %%%%%%%%%%%%%%%%%%%%%%%%%%%%%
\newtheoremstyle{mytheoremstyle}{3pt}{3pt}{\normalfont}{0cm}{\rmfamily\bfseries}{}{1em}{{\color{black}\thmname{#1}~\thmnumber{#2}}\thmnote{\,--\,#3}}
\newtheoremstyle{myproblemstyle}{3pt}{3pt}{\normalfont}{0cm}{\rmfamily\bfseries}{}{1em}{{\color{black}\thmname{#1}~\thmnumber{#2}}\thmnote{\,--\,#3}}
\theoremstyle{mytheoremstyle}
\newmdtheoremenv[linewidth=1pt,backgroundcolor=shallowGreen,linecolor=deepGreen,leftmargin=0pt,innerleftmargin=20pt,innerrightmargin=20pt,]{theorem}{Theorem}[section]
\theoremstyle{mytheoremstyle}
\newmdtheoremenv[linewidth=1pt,backgroundcolor=shallowBlue,linecolor=deepBlue,leftmargin=0pt,innerleftmargin=20pt,innerrightmargin=20pt,]{definition}{Definition}[section]
\theoremstyle{myproblemstyle}
\newmdtheoremenv[linecolor=black,leftmargin=0pt,innerleftmargin=10pt,innerrightmargin=10pt,]{problem}{Problem}[section]
%%%%%%%%%%%%%%%%%%%%%%%%%%%%%%%%%%%%%%%%%%%%%%%%%%%%%%%%%%%%%%%%%%%%%%%%%%%%%%%

%%%%%%%%%%%%%%%%%%%%%%%%%%%%%%% Plotting Settings %%%%%%%%%%%%%%%%%%%%%%%%%%%%%
\usepgfplotslibrary{colorbrewer}
\pgfplotsset{width=8cm,compat=1.9}
%%%%%%%%%%%%%%%%%%%%%%%%%%%%%%%%%%%%%%%%%%%%%%%%%%%%%%%%%%%%%%%%%%%%%%%%%%%%%%%

%%%%%%%%%%%%%%%%%%%%%%%%%%%%%%% Title & Author %%%%%%%%%%%%%%%%%%%%%%%%%%%%%%%%
\title{Power Series}
\author{Patrick Chen}
\date{Jan 29, 2025}
%%%%%%%%%%%%%%%%%%%%%%%%%%%%%%%%%%%%%%%%%%%%%%%%%%%%%%%%%%%%%%%%%%%%%%%%%%%%%%%

\begin{document}
    \maketitle
    A power series is a sum of powers. The output of a power series is a
    function.
    \[
        \sum_{n=0}^{\infty} c_n (x-a)^n
    \]
    The domain of the function is the set of all $x\in \mathbb{R}$ where the
    series converges. The power series is said to be centered on $a$. When $a=0$, the
    series is a 0-centered power series. The interval of convergence is centered
    on $a$. The difference between the center and where it diverges is called
    the radius of convergence $R$.

    \subsection*{Example}
    \begin{align*}
        \sum_{n=0}^{\infty} \frac{3^nx^n}{n+1}
    \end{align*}
    Using ratio test:
    \begin{align*}
        \lim_{n\to \infty} \Big|\frac{a_{n+1}}{a_n}\Big|
        &= \lim_{n\to \infty}
        \bigg|\frac{\frac{3^{n+1}x^{n+1}}{n+2}} {\frac{3^nx^n}{n+1}}\bigg| \\
        &= \lim_{n\to \infty} \Big|\frac{3x(n+1)} {n+2}\Big| \\
        &= 3\cdot|x|
    \end{align*}
    This series will absolutely converge when $3\cdot|x| < 1$ and diverge when
    $3\cdot|x|>1$.

    \begin{align*}
        x &= \frac{1}{3} \\
        \sum_{n=0}^{\infty} \frac{3^n (\frac{1}{3})^n}{n+1}
        &= \sum_{n=0}^{\infty} \frac{1}{n+1} \\
        &= \sum_{n=1}^{\infty} \frac{1}{n}
    \end{align*}
    This will diverge because the $p=1$.

    \begin{align*}
        x &= -\frac{1}{3} \\
        \sum_{n=0}^{\infty} \frac{3^n (-\frac{1}{3})^n}{n+1}
        &= \sum_{n=0}^{\infty} \frac{(-1)^n}{n+1} \\
    \end{align*}
    Using the alternating series test, this will converge. Thus the interval of
    convergence is $x\in[-\frac{1}{3,} \frac{1}{3})$

    \subsection*{Example 2}
    Find the center and radius of convergence for the following series
    \begin{align*}
        \sum_{n=1}^{\infty} \frac{(x-2)^n}{5^nn^2}
    \end{align*}
    \begin{align*}
        \lim_{n\to \infty} \Big| \frac{a_{n+1}}{a_n} \Big|
        &= \lim_{n\to \infty} \bigg|\frac{\frac{(x-2)^{n+1}}{5^{n+1}(n+1)^2}}{\frac{(x-2)^n}{5^nn^2}}\bigg| \\
        &= \lim_{n\to \infty} \Big|\frac{(x-2)n^2}{5(n+1)^2}\Big| \\
        &= \frac{|x-2|}{5}
    \end{align*}
    \begin{align*}
        \frac{|x-2|}{5} < 1 \\
        |x-2| < 5
    \end{align*}
    \[
        a = 2,\ R = 5
    \]

    \subsection*{Example 3}
    \begin{align*}
        \sum_{n=0}^{\infty} n!x^n
    \end{align*}
    \begin{align*}
        \lim_{n\to \infty} \Big|\frac{a_{n+1}}{a_n}\Big|
        &= \lim_{n\to \infty} \Big|\frac{(n+1)!x^{n+1}}{n!x^n}\Big| \\
        &= \lim_{n\to \infty} |x|(n+1) \\
        &= \infty
    \end{align*}
    This power series diverges for all non-zero values $x$

    \subsection*{Example 4}
    \begin{align*}
        \sum_{n=1}^{\infty} \frac{x^n}{n!}
    \end{align*}
    \begin{align*}
        \lim_{n\to \infty} \Big|\frac{a_{n+1}}{a_n}\Big|
        &= \lim_{n\to \infty} \Big|\frac{\frac{x^{n+1}}{(n+1)!}}{\frac{x^n}{n!}}\Big| \\
        &= \lim_{n\to \infty} \frac{|x|}{n+1} \\
        &= 0
    \end{align*}
    This power series converges for all finite values $x$

    \section*{Functions as Power Series}
    Functions can be represented as power series. For example,
    $f(x)=\frac{1}{1-x}$ is represented by the following power series.
    \begin{align*}
        \sum_{n=0}^{\infty} x^n,\ -1 < x < 1
    \end{align*}
    Using geometric series formula:
    \begin{align*}
        \sum_{n=0}^{\infty} x^n
        &= \sum_{n=1}^{\infty} x^{n-1} \\
        &= \frac{1}{1-x}
    \end{align*}

    \subsection*{Example}
    \begin{align*}
        \frac{4x}{2-x} = 4x \cdot (\frac{1}{2-x}) \\
        = \frac{4x}{2} \frac{1}{1-(\frac{x}{2})} \\
        = 2x \Big(\sum_{n=0}^{\infty} (\frac{x}{2})^n\Big) \\
        = 2x \Big(\sum_{n=0}^{\infty} (\frac{x}{2})^n\Big)
    \end{align*}

    \subsection*{Example 2}
    Find the power series of $f(x) = \frac{1}{(1-x)^2}$
    \begin{align*}
        f(x) &= \frac{d}{dx} (\frac{1}{1-x}) \\
        &= \frac{d}{dx} \Big(\sum_{n=0}^{\infty} x^n\Big) \\
        &= \sum_{n=0}^{\infty} \Big(\frac{d}{dx} x^n\Big) \\
        &= 0 + \sum_{n=1}^{\infty} (nx^{n-1}) \\
        &= \sum_{n=0}^{\infty} (n+1)x^{n} \\
    \end{align*}
    The radius of convergence is the same because by the ratio test, $\frac{n+2}{n+1}$
    will approach $1$ and not affect the outcome when multiplied.

    \subsection*{Example 3}
    Find the power series of $f(x) = \ln(1-x)$.
    \begin{align*}
        f(x) &= \ln(1-x) \\
        f'(x) &= -\frac{1}{1-x} \\
        f'(x) &= -\Big(\sum_{n=0}^{\infty} x^n\Big) \\
        \int f'(x) dx &= -\int \Big(\sum_{n=0}^{\infty} x^n\Big) dx \\
        f(x) &= -\sum_{n=0}^{\infty} \Big(\int x^n dx\Big) \\
        f(x) &= -\sum_{n=0}^{\infty} \Big(\frac{x^{n+1}}{n+1}\Big) + c
    \end{align*}
    Finding the constant of integration:
    \begin{align*}
        f(0) &= -\sum_{n=0}^{\infty} (\frac{0^{n+1}}{n+1}) + c \\
        \ln(1-0) &= -\sum_{n=0}^{\infty} (0) + c \\
        0 &= c
    \end{align*}
    Thus,
    \begin{align*}
        f(x) = -\sum_{n=0}^{\infty} (\frac{x^{n+1}}{n+1}) \\
        f(x) = -\sum_{n=1}^{\infty} (\frac{x^{n}}{n})
    \end{align*}

    \section*{Taylor and MacLaurin Series}
    Suppose $f(x)$ has a zero-centered power series. $f(0)$ is the constant term
    of the power series. $f'(0)$ is the $x$ term of the power series.
    $\frac{f''(0)}{2}$ is the $x^2$ term of the power series. In general,
    $\frac{1}{n!} [\frac{d^n}{dx^n}f](0)$ is the $x^n$ term of the power series.
    \begin{align*}
        f(x) &= c_0 + c_1 x + c_2 x^2 + c_3 x^3 + \dots \\
        f'(x) &= c_1 + 2 c_2 x + 3 c_3 x^2 + 4 c_4 x^3 \dots \\
        f'''(x) &= 2 c_2 + 6 c_3 x + 12 c_4 x^2 + 20 c_5 x^3\dots
    \end{align*}
    \begin{align*}
        f(x) &= \sum_{n=0}^{\infty} \frac{f^{(n)}(0)}{n!} x^n \\
        \text{where } f^{(n)}(x) &= \frac{d^n}{dx^n} f(x)
    \end{align*}

    The Taylor series is an extension of the MacLaurin Series where the series
    may not be centered at zero. When finding a Taylor series, it may be better
    to not simplify the expressions of the derivative to help when finding a
    pattern.
    \begin{align*}
        \sum_{n=0}^{\infty} \frac{f^{(n)}(a)}{n!} (x-a)^n
    \end{align*}

    \subsection*{Common Taylor Series}
    \begin{center}
    \renewcommand{\arraystretch}{2}
    \begin{tabular}[c]{c|c|c}
        $f(x)$ & power series & domain \\
        \hline
        $\frac{1}{1-x}$
        & $\sum_{n=0}^\infty x^n$
        & $-1 < x < 1$ \\

        $e^x$
        & $\sum_{n=0}^\infty \frac{x^n}{n!}$
        & $x\in\mathbb{R}$ \\

        $\cos(x)$
        & $\sum_{n=0}^\infty \frac{(-1)^nx^{2n}}{2n}!$
        & $x\in\mathbb{R}$ \\

        $\sin(x)$
        & $\sum_{n=0}^\infty \frac{(-1)^nx^{2n+1}}{2n+1}!$
        & $x\in\mathbb{R}$
    \end{tabular}
    \end{center}

    \subsection*{Taylor Polynomial}
    A $m^{th}$ order Taylor polynomial $T_m(x)$ is the partial sum of the Taylor
    series.
    \[
        T_m(x) = \sum_{n=0}^{m} \frac{f^{(n)}(a)}{n!} (x-a)^n
    \]

    \subsection*{Taylor Remainder Theorem}
    The absolute difference between a function and its $m^{th}$ Taylor
    polynomial is $R_m(x)$. It is always less than the next term of the Taylor
    series with the $(m+1)^{th}$ derivative of the function replaced with the
    supremum of the $(m+1)^{th}$ derivative of the function.
    \[
        |f(x) - T_m(x)| = R_m(x) \le \frac{M|x-a|^{m+1}}{(m+1)!}, \quad
        M \ge sup|f^{(m+1)}(x)|
    \]

    \subsection*{Example}
    Find the 0-centered power series for $f(x)=e^x$
    \begin{align*}
        f(0) &= e^0 = 1 \\
        f'(0) &= e^0 = 1 \\
        f'''(0) &= e^0 = 1 \\
        f^{(n)}(0) &= e^0 = 1
    \end{align*}
    \begin{align*}
        f(x) &= \sum_{n=0}^{\infty} \frac{f^{(n)}(0)}{n!} x^n \\
             &= \sum_{n=1}^{\infty} \frac{1}{n!} x^n \\
             &= \sum_{n=1}^{\infty} \frac{x^n}{n!}
    \end{align*}

    \subsection*{Example 2}
    Find the Taylor series for $f(x) = \frac{1}{x^2}$ centered at $a=2$
    \begin{align*}
        f(x) = \sum_{n=0}^{\infty} \frac{f^{(n)}(2)}{n!} (x-2)^n
    \end{align*}
    \begin{center}
    \renewcommand{\arraystretch}{2}
    \begin{tabular}[c]{c|c|c}
        n & $f^{(n)}(x)$ & $f^{(n)}(2)$ \\
        \hline
        0 & $x^{-2}$ & $\frac{1}{2^2}$ \\
        1 & $(-2)x^{-3}$ & $\frac{-2}{2^3}$ \\
        2 & $(-3)(-2)x^{-4}$ & $\frac{(-3)(-2)}{2^4}$ \\
        3 & $(-4)(-3)(-2)x^{-5}$ & $\frac{(-4)(-3)(-2)}{2^4}$ \\
        n & $\frac{(-1)^n(n+1)!}{x^{n+2}}$ & $\frac{(-1)^n(n+1)!}{2^{n+2}}$ \\
    \end{tabular}
    \end{center}
    \[
        f^{(n)}(2) = \frac{(-1)^n (n+1)!}{2^{n+2}}
    \]

    \begin{align*}
        \frac{1}{x^2} &= \sum_{n=0}^{\infty} \frac{\frac{(-1)^n (n+1)!}{2^{n+2}}}{n!} (x-2)^n \\
        &= \sum_{n=0}^{\infty} \frac{(-1)^n(n+1)}{2^{n+2}} (x-2)^n
    \end{align*}

    \subsection*{Example 3}
    Find the Taylor series for $f(x)=\ln(x)$ centered at $a=5$.
    \begin{align*}
        \sum_{n=0}^{\infty} \frac{f^{(n)}(5)}{n!} (x-5)^n
    \end{align*}

    \begin{center}
    \renewcommand{\arraystretch}{2}
    \begin{tabular}[c]{c|c|c}
        n & $f^{(n)}(x)$ & $f^{(n)}(5)$ \\
        \hline
        0 & $\ln(x)$ & $\ln(5)$ \\
        1 & $\frac{1}{x}$ & $\frac{1}{5}$ \\
        2 & $\frac{-1}{x^2}$ & $\frac{-1}{5^2}$ \\
        3 & $\frac{(-1)(-2)}{x^3}$ & $\frac{(-1)(-2)}{5^3}$ \\
        4 & $\frac{(-1)(-2)(-3)}{x^4}$ & $\frac{(-1)(-2)(-3)}{5^4}$ \\
        n & $\frac{(-1)^{n-1}(n-1)!}{x^n}$ & $\frac{(-1)^{n-1}(n-1)!}{5^n}$ \\
    \end{tabular}
    \end{center}

    \begin{align*}
        f(x) &= \frac{\ln(5)}{0!}(x-5)^0 + \sum_{n=1}^{\infty} \frac{\frac{(-1)^{n-1}(n-1)!}{5^n}}{n!} (x-5)^n \\
             &= \ln(5) + \sum_{n=1}^{\infty} \frac{(-1)^{n-1}}{n5^n} (x-5)^n
    \end{align*}

    \subsection*{Example 4}
    \begin{align*}
        \sum_{n=0}^{\infty} \frac{1}{2^n n!}
        &= \sum_{n=0}^{\infty} \frac{(\frac{1}{2})^n}{n!} \\
        &= \sum_{n=0}^{\infty} \frac{x^n}{n!},\ x = \frac{1}{2} \\
        &= e^x,\ x = \frac{1}{2} \\
        &= e^{\frac{1}{2}} \\
        &= \sqrt{e}
    \end{align*}
    \subsection*{Example 5}
    \begin{align*}
        \sum_{n=0}^{\infty} \frac{(-1)^n\pi^{2n}}{(9^n)(2n)!}
        &= \sum_{n=0}^{\infty} \frac{(-1)^n\frac{\pi^{2n}}{3^{2n}}}{(2n)!} \\
        &= \sum_{n=0}^{\infty} \frac{(-1)^n(\frac{\pi}{3})^{2n}}{(2n)!} \\
        &= \sum_{n=0}^{\infty} \frac{(-1)^nx^{2n}}{(2n)!},\ x=\frac{\pi}{3} \\
        &= \cos(x),\ x=\frac{\pi}{3} \\
        &= \cos(\frac{\pi}{3}) \\
        &= \frac{1}{2}
    \end{align*}
    \subsection*{Example 6}
    \begin{align*}
        \sum_{n=0}^{\infty} \frac{(-1)^n\pi^{2n+2}}{4^n(2n+1)!}
        &= 2\pi \sum_{n=0}^{\infty} \frac{(-1)^n\pi^{2n+1}}{(2^{2n+1})(2n+1)!} \\
        &= 2\pi \sum_{n=0}^{\infty} \frac{(-1)^n(\frac{\pi}{2})^{2n+1}}{(2n+1)!} \\
        &= 2\pi \sum_{n=0}^{\infty} \frac{(-1)^nx^{2n+1}}{(2n+1)!},\ x= \frac{\pi}{2} \\
        &= 2\pi \sin(x),\ x= \frac{\pi}{2} \\
        &= 2\pi \sin(\frac{\pi}{2}) \\
        &= 2\pi
    \end{align*}

    \subsection*{Example 7}
    Find $T_4(x)$ about $x=0$ for $f(x)=\cos(x)$ and estimate the upper bound on
    the error using the Taylor remainder estimate on the interval $[-2,2]$.
    \begin{align*}
        T_4(x) &= \sum_{n=0}^{4} \frac{f^{(n)}(0)}{n!} x^n \\
               &= f(0) + f'(0)x + \frac{f''(0)}{2} x^2 + \frac{f^{(3)}(0)}{6} x^3 + \frac{f^{(4)}(0)}{24} x^4 \\
               &= 1 - \frac{x^2}{2} + \frac{x^4}{24}
    \end{align*}
    \begin{align*}
        |\cos(x) - T_4(x)| &\le \frac{M|x-a|^{m+1}}{(m+1)!} \\
        \frac{M|x-a|^{m+1}}{(m+1)!} &= \frac{M|x|^{5}}{5} \\
        \forall x\in[-2,2] &,\ |x| \le 2 \\
        |\cos(x) - T_4(x)| &\le M \frac{2^5}{5!} \\
        |\cos(x) - T_4(x)| &\le M \frac{4}{15}
    \end{align*}
    \begin{align*}
        M &\ge sup_{[-2,2]}|f^{(m+1)}(x)| \\
        sup_{[-2,2]}|f^{(m+1)}(x)|
          &= sup_{[-2,2]}|f^{(5)}\cos(x)| \\
          &= sup_{[-2,2]}|-\sin(x)| \\
          &= 1 \\
        \therefore M &= 1
    \end{align*}
    \begin{align*}
        |\cos(x) - T_4(x)| &\le \frac{4}{15}
    \end{align*}

    \section*{Binomial Series}
    \begin{align*}
        (1+x)^m &=
        \binom{m}{0} +
        \binom{m}{1} x +
        \binom{m}{2} x^2 + \dots +
        \binom{m}{m} x^m \\
        \binom{n}{k} &= \frac{n!}{k!(n-k)!}
    \end{align*}
    \begin{align*}
        (1+x)^m &= \sum_{n=0}^{m} \binom{m}{n} x^n \\
                &= 1 + \sum_{n=1}^{m} \frac{k(k-1)(k-1)(k-3)\dots(k-n+1)}{n!}x^n
    \end{align*}
    Since for any term greater than $m$ will have a zero factor,
    \[
        (1+x)^k = \sum_{n=1}^{\infty} \binom{k}{n} x^n
        = 1+\sum_{n=1}^{\infty} \frac{k(k-1)\dots(k-n+1)}{n!}
    \]

    \subsection*{Example}
    Use binomial series to find the zero-centered power series for
    $f(x)=\frac{1}{\sqrt{4+x}}$.

    \begin{align*}
        \frac{1}{\sqrt{4+x}}
        &= (4+x)^{-\frac{1}{2}} \\
        &= \frac{1}{2} (1+\frac{x}{4})^{-\frac{1}{2}} \\
        &= \frac{1}{2} (1 + u)^{-\frac{1}{2}},\ u = \frac{x}{4} \\
        &= \frac{1}{2} \Big(1+\sum_{n=1}^{\infty} \binom{-\frac{1}{2}}{n} u^n\Big) \\
        &= \frac{1}{2} \Big(1+\sum_{n=1}^{\infty} \binom{-\frac{1}{2}}{n} \frac{x^n}{4^n}\Big) \\
        &= \frac{1}{2} \Big(1 + \sum_{n=1}^{\infty} \frac{(-\frac{1}{2})(-\frac{3}{2})\dots(-\frac{1}{2}-n+1)}{n!} \frac{x^n}{4^n}\Big) \\
        &= \frac{1}{2} + \frac{1}{2}\sum_{n=1}^{\infty} (-1)^n \frac{(1)(3)(5)\dots(2n-1)}{n!2^n} \frac{x^n}{2^{2n}} \\
        &= \frac{1}{2} + \sum_{n=1}^{\infty}
        \frac{(-1)^n\ (1)(3)(5)\dots(2n-1)}{n!\ 2^{3n+1}} x^n
    \end{align*}
\end{document}
