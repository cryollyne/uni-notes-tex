%%%%%%%%%%%%%%%%%%%%%%%%%%%%% Define Article %%%%%%%%%%%%%%%%%%%%%%%%%%%%%%%%%%
\documentclass{article}
%%%%%%%%%%%%%%%%%%%%%%%%%%%%%%%%%%%%%%%%%%%%%%%%%%%%%%%%%%%%%%%%%%%%%%%%%%%%%%%

%%%%%%%%%%%%%%%%%%%%%%%%%%%%% Using Packages %%%%%%%%%%%%%%%%%%%%%%%%%%%%%%%%%%
\usepackage{geometry}
\usepackage{graphicx}
\usepackage{amssymb}
\usepackage{amsmath}
\usepackage{amsthm}
\usepackage{empheq}
\usepackage{mdframed}
\usepackage{booktabs}
\usepackage{lipsum}
\usepackage{graphicx}
\usepackage{color}
\usepackage{psfrag}
\usepackage{pgfplots}
\usepackage{bm}
%%%%%%%%%%%%%%%%%%%%%%%%%%%%%%%%%%%%%%%%%%%%%%%%%%%%%%%%%%%%%%%%%%%%%%%%%%%%%%%

% Other Settings

%%%%%%%%%%%%%%%%%%%%%%%%%% Page Setting %%%%%%%%%%%%%%%%%%%%%%%%%%%%%%%%%%%%%%%
\geometry{a4paper}

%%%%%%%%%%%%%%%%%%%%%%%%%% Define some useful colors %%%%%%%%%%%%%%%%%%%%%%%%%%
\definecolor{ocre}{RGB}{243,102,25}
\definecolor{mygray}{RGB}{243,243,244}
\definecolor{deepGreen}{RGB}{26,111,0}
\definecolor{shallowGreen}{RGB}{235,255,255}
\definecolor{deepBlue}{RGB}{61,124,222}
\definecolor{shallowBlue}{RGB}{235,249,255}
%%%%%%%%%%%%%%%%%%%%%%%%%%%%%%%%%%%%%%%%%%%%%%%%%%%%%%%%%%%%%%%%%%%%%%%%%%%%%%%

%%%%%%%%%%%%%%%%%%%%%%%%%% Define an orangebox command %%%%%%%%%%%%%%%%%%%%%%%%
\newcommand\orangebox[1]{\fcolorbox{ocre}{mygray}{\hspace{1em}#1\hspace{1em}}}
%%%%%%%%%%%%%%%%%%%%%%%%%%%%%%%%%%%%%%%%%%%%%%%%%%%%%%%%%%%%%%%%%%%%%%%%%%%%%%%

%%%%%%%%%%%%%%%%%%%%%%%%%%%% English Environments %%%%%%%%%%%%%%%%%%%%%%%%%%%%%
\newtheoremstyle{mytheoremstyle}{3pt}{3pt}{\normalfont}{0cm}{\rmfamily\bfseries}{}{1em}{{\color{black}\thmname{#1}~\thmnumber{#2}}\thmnote{\,--\,#3}}
\newtheoremstyle{myproblemstyle}{3pt}{3pt}{\normalfont}{0cm}{\rmfamily\bfseries}{}{1em}{{\color{black}\thmname{#1}~\thmnumber{#2}}\thmnote{\,--\,#3}}
\theoremstyle{mytheoremstyle}
\newmdtheoremenv[linewidth=1pt,backgroundcolor=shallowGreen,linecolor=deepGreen,leftmargin=0pt,innerleftmargin=20pt,innerrightmargin=20pt,]{theorem}{Theorem}[section]
\theoremstyle{mytheoremstyle}
\newmdtheoremenv[linewidth=1pt,backgroundcolor=shallowBlue,linecolor=deepBlue,leftmargin=0pt,innerleftmargin=20pt,innerrightmargin=20pt,]{definition}{Definition}[section]
\theoremstyle{myproblemstyle}
\newmdtheoremenv[linecolor=black,leftmargin=0pt,innerleftmargin=10pt,innerrightmargin=10pt,]{problem}{Problem}[section]
%%%%%%%%%%%%%%%%%%%%%%%%%%%%%%%%%%%%%%%%%%%%%%%%%%%%%%%%%%%%%%%%%%%%%%%%%%%%%%%

%%%%%%%%%%%%%%%%%%%%%%%%%%%%%%% Plotting Settings %%%%%%%%%%%%%%%%%%%%%%%%%%%%%
\usepgfplotslibrary{colorbrewer}
\pgfplotsset{width=8cm,compat=1.9}
%%%%%%%%%%%%%%%%%%%%%%%%%%%%%%%%%%%%%%%%%%%%%%%%%%%%%%%%%%%%%%%%%%%%%%%%%%%%%%%

%%%%%%%%%%%%%%%%%%%%%%%%%%%%%%% Title & Author %%%%%%%%%%%%%%%%%%%%%%%%%%%%%%%%
\title{Parametric Function}
\author{Patrick Chen}
\date{March 5, 2025}
%%%%%%%%%%%%%%%%%%%%%%%%%%%%%%%%%%%%%%%%%%%%%%%%%%%%%%%%%%%%%%%%%%%%%%%%%%%%%%%

\begin{document}
    \maketitle
    A parametric function are functions describing the position along some axis
    for a given parameter. Typically, these functions are continuous and the
    parameter is named $t$. A 2D parametric function can be plotted by varying
    the value of $t$ and plotting the $x$ and $y$ positions. The direction of
    increasing $t$ along the curve is called the orientation of the
    parametrization. This is typically indicated by arrows.
    \[
        x = f(t) \quad y = g(t)
    \]
    Any function $f: \mathbb{R} \mapsto \mathbb{R}$ can be trivially
    parameterized as $x=t$ and $y=f(t)$

    \section*{Isolation}
    A plot for a parametric function may be plotted more easily by isolating and
    substituting $t$.
    \subsection*{Example}
    \[
        x = \ln(t) \quad y=\frac{t^4+1}{2t^2}
    \]
    \begin{align*}
        x   &= \ln(t) \\
        e^x &= t \\
        y   &=\frac{t^4+1}{2t^2} \\
        &=\frac{(e^x)^4+1}{2(e^x)^2} \\
        &=\frac{e^{4x}+1}{2e^{2x}} \\
        &=\frac{e^{2x} + e^{-2x}}{2} \\
        &=\cosh(2x)
    \end{align*}

    \section*{Parametric Differentiation}
    Parametric functions can be differentiated by dividing the $y$-derivative with
    respect to $t$ by the $x$-derivative with respect to $t$. Let $y = h(x)$.
    \begin{align*}
        \frac{dy}{dt} &= \frac{d}{dt} h(x)  \\
        \frac{dy}{dt} &= \Big(\frac{d}{dx} h(x)\Big) \frac{dx}{dt} \\
        \frac{d}{dx} h(x) &= \frac{\Big(\frac{dy}{dt}\Big)}{\Big(\frac{dx}{dt}\Big)} \\
        \frac{dy}{dx} &= \frac{\Big(\frac{dy}{dt}\Big)}{\Big(\frac{dx}{dt}\Big)}
    \end{align*}

    \begin{itemize}
        \item if $\frac{dy}{dt}$ is zero and $\frac{dx}{dt}$ is not zero, it is
            a horizontal tangent.
        \item if $\frac{dx}{dt}$ is zero and $\frac{dy}{dt}$ is not zero, it is
            a vertical tangent.
    \end{itemize}

    \subsection*{Higher Derivatives}
    \begin{align*}
        \frac{d}{dt}(\frac{dy}{dx}) &= \frac{d}{dx}(\frac{dy}{dx}) \cdot\frac{dx}{dt} \\
        &= \frac{d^2y}{dx^2} \frac{dx}{dt} \\
    \end{align*}
    Thus
    \begin{align*}
        \frac{d^2y}{dx^2} = \frac{\frac{d}{dt} (\frac{dy}{dx})}{\frac{dx}{dt}} \\
        \frac{d^ny}{dx^n} = \frac{\frac{d}{dt} \frac{d^{n-1} y}{dx^{n-1}}}{\frac{dx}{dt}}
    \end{align*}

    \section*{Parametric Integration}
    Parametric functions can be integrated as follows.
    \[
        \int_{a}^{b} f(x) \ dx = \int_{t_0}^{t_1} y(t) \frac{dx}{dt} \ dt
    \]

    \subsection*{Example}
    Integrate the following parametric function on the interval $x\in[0,\pi]$
    \[
        x(t) = t-\sin(t),\quad y(t)=1-\cos(t)
    \]
    \begin{align*}
        x=0 &\Rightarrow t=0 \\
        x=\pi &\Rightarrow t=\pi
    \end{align*}

    \begin{align*}
        \frac{d}{dt} x(t) = 1-\cos (t)
    \end{align*}
    \begin{align*}
        \int_{0}^{\pi} y(t)(\frac{dx}{dt}) \ dt
        &= \int_{0}^{\pi} (1-\cos t)(1-\cos t) \ dx \\
        &= \int_{0}^{\pi} 1 - 2\cos t + \cos^2 t \ dx \\
        &= \int_{0}^{\pi} 1 - 2\cos t + \frac{1}{2} (1 + \cos 2t) \ dx \\
        &= \Big[\frac{3}{2} t - 2\sin t +\frac{1}{4} \cos 2t\Big]_0^\pi \\
        &= \frac{3}{2} \pi
    \end{align*}

    \subsection*{Arclength}
    \begin{align*}
        \int_c ds = \int_{t_0}^{t_1} \sqrt{\Big(\frac{dx}{dt}\Big)^1 + \Big(\frac{dy}{dt}\Big)^2} \ dt
    \end{align*}

    \subsection*{Surface Area of Revolution}
    about the x-axis
    \begin{align*}
        \int_I 2\pi r ds = \int_{t_0}^{t_1} 2\pi y(t) \sqrt{\Big(\frac{dx}{dt}\Big)^2 + \Big(\frac{dy}{dt}\Big)^2} \ dt
    \end{align*}
    about the y-axis
    \begin{align*}
        \int_I 2\pi r ds = \int_{t_0}^{t_1} 2\pi x(t) \sqrt{\Big(\frac{dx}{dt}\Big)^2 + \Big(\frac{dy}{dt}\Big)^2} \ dt
    \end{align*}

\end{document}
