%%%%%%%%%%%%%%%%%%%%%%%%%%%%% Define Article %%%%%%%%%%%%%%%%%%%%%%%%%%%%%%%%%%
\documentclass{article}
%%%%%%%%%%%%%%%%%%%%%%%%%%%%%%%%%%%%%%%%%%%%%%%%%%%%%%%%%%%%%%%%%%%%%%%%%%%%%%%

%%%%%%%%%%%%%%%%%%%%%%%%%%%%% Using Packages %%%%%%%%%%%%%%%%%%%%%%%%%%%%%%%%%%
\usepackage{geometry}
\usepackage{graphicx}
\usepackage{amssymb}
\usepackage{amsmath}
\usepackage{amsthm}
\usepackage{empheq}
\usepackage{mdframed}
\usepackage{booktabs}
\usepackage{lipsum}
\usepackage{graphicx}
\usepackage{color}
\usepackage{psfrag}
\usepackage{pgfplots}
\usepackage{bm}
%%%%%%%%%%%%%%%%%%%%%%%%%%%%%%%%%%%%%%%%%%%%%%%%%%%%%%%%%%%%%%%%%%%%%%%%%%%%%%%

% Other Settings

%%%%%%%%%%%%%%%%%%%%%%%%%% Page Setting %%%%%%%%%%%%%%%%%%%%%%%%%%%%%%%%%%%%%%%
\geometry{a4paper}

%%%%%%%%%%%%%%%%%%%%%%%%%% Define some useful colors %%%%%%%%%%%%%%%%%%%%%%%%%%
\definecolor{ocre}{RGB}{243,102,25}
\definecolor{mygray}{RGB}{243,243,244}
\definecolor{deepGreen}{RGB}{26,111,0}
\definecolor{shallowGreen}{RGB}{235,255,255}
\definecolor{deepBlue}{RGB}{61,124,222}
\definecolor{shallowBlue}{RGB}{235,249,255}
%%%%%%%%%%%%%%%%%%%%%%%%%%%%%%%%%%%%%%%%%%%%%%%%%%%%%%%%%%%%%%%%%%%%%%%%%%%%%%%

%%%%%%%%%%%%%%%%%%%%%%%%%% Define an orangebox command %%%%%%%%%%%%%%%%%%%%%%%%
\newcommand\orangebox[1]{\fcolorbox{ocre}{mygray}{\hspace{1em}#1\hspace{1em}}}
%%%%%%%%%%%%%%%%%%%%%%%%%%%%%%%%%%%%%%%%%%%%%%%%%%%%%%%%%%%%%%%%%%%%%%%%%%%%%%%

%%%%%%%%%%%%%%%%%%%%%%%%%%%% English Environments %%%%%%%%%%%%%%%%%%%%%%%%%%%%%
\newtheoremstyle{mytheoremstyle}{3pt}{3pt}{\normalfont}{0cm}{\rmfamily\bfseries}{}{1em}{{\color{black}\thmname{#1}~\thmnumber{#2}}\thmnote{\,--\,#3}}
\newtheoremstyle{myproblemstyle}{3pt}{3pt}{\normalfont}{0cm}{\rmfamily\bfseries}{}{1em}{{\color{black}\thmname{#1}~\thmnumber{#2}}\thmnote{\,--\,#3}}
\theoremstyle{mytheoremstyle}
\newmdtheoremenv[linewidth=1pt,backgroundcolor=shallowGreen,linecolor=deepGreen,leftmargin=0pt,innerleftmargin=20pt,innerrightmargin=20pt,]{theorem}{Theorem}[section]
\theoremstyle{mytheoremstyle}
\newmdtheoremenv[linewidth=1pt,backgroundcolor=shallowBlue,linecolor=deepBlue,leftmargin=0pt,innerleftmargin=20pt,innerrightmargin=20pt,]{definition}{Definition}[section]
\theoremstyle{myproblemstyle}
\newmdtheoremenv[linecolor=black,leftmargin=0pt,innerleftmargin=10pt,innerrightmargin=10pt,]{problem}{Problem}[section]
%%%%%%%%%%%%%%%%%%%%%%%%%%%%%%%%%%%%%%%%%%%%%%%%%%%%%%%%%%%%%%%%%%%%%%%%%%%%%%%

%%%%%%%%%%%%%%%%%%%%%%%%%%%%%%% Plotting Settings %%%%%%%%%%%%%%%%%%%%%%%%%%%%%
\usepgfplotslibrary{colorbrewer}
\pgfplotsset{width=8cm,compat=1.9}
%%%%%%%%%%%%%%%%%%%%%%%%%%%%%%%%%%%%%%%%%%%%%%%%%%%%%%%%%%%%%%%%%%%%%%%%%%%%%%%

%%%%%%%%%%%%%%%%%%%%%%%%%%%%%%% Title & Author %%%%%%%%%%%%%%%%%%%%%%%%%%%%%%%%
\title{Improper Integrals}
\author{Patrick Chen}
\date{Jan 8, 2025}
%%%%%%%%%%%%%%%%%%%%%%%%%%%%%%%%%%%%%%%%%%%%%%%%%%%%%%%%%%%%%%%%%%%%%%%%%%%%%%%

\begin{document}
    \maketitle
    \subsection*{Integrals to Infinity}
    When an integral has an infinity in its bounds, it is equal to the limit as
    the bound goes to infinity. If $f(x)$ is continuous on $[a,\infty)$, and
    $g(x)$ is continuous on $(-\infty,b]$, then:
    \begin{align*}
        \int_{a}^{\infty} f(x) \ dx &= \lim_{n\to \infty} \int_{a}^{n} f(x) \ dx \\
        \int_{-\infty}^{b} g(x) \ dx &= \lim_{n\to -\infty} \int_{n}^{b} g(x) \ dx
    \end{align*}
    If the limit exists, then the integral is said to be convergent. Otherwise,
    the integral is divergent. If both bounds are infinity, then it can be
    broken up into two integrals. If $f(x)$ is continuous on $(-\infty,\infty)$,
    the integral can be defined as:
    \begin{align*}
        \int_{-\infty}^{\infty} f(x) \ dx
        = \int_{-\infty}^{a} f(x) \ dx + \int_{a}^{\infty} f(x) \ dx
    \end{align*}
    If both of the integrals are convergent, then the original integral is
    convergent. If any of the two integrals are divergent, then the original
    integral is divergent.

    \subsection*{Example 1}
    \begin{align*}
        \int_{2}^{\infty} \cos(x) \ dx \\
        &= \lim_{n\to \infty} \int_{2}^{n} \cos(x) \ dx \\
        &= \lim_{n\to \infty} \sin(x) \ \Big|_{2}^{n} \\
        &= \lim_{n\to \infty} \sin(n) - \sin(2)
    \end{align*}
    Since $\sin(x)$ does not converge, the integral is divergent.

    \subsection*{Example 2}
    \begin{align*}
        \int_{2}^{\infty} \frac{1}{x} \ dx
        &= \lim_{n\to \infty} \int_{2}^{n} \frac{1}{x} \ dx \\
        &= \lim_{n\to \infty} \ln|x| \ \Big|_{2}^{n} \\
        &= \ln(\infty) - \ln(2) \\
        &= \infty
    \end{align*}
    Since the limit is infinity, the integral is divergent.

    \subsection*{Example 3}
    \begin{align*}
        \int_{-\infty}^{\infty} \frac{1}{1+x^2} \ dx
        &= \lim_{b\to \infty} \int_{0}^{b} \frac{1}{1+x^2} \ dx
         + \lim_{a\to -\infty} \int_{a}^{0} \frac{1}{1+x^2} \ dx \\
        &= \Big(\lim_{b\to \infty} \tan^{-1}(x) \ \Big|_{0}^{b}\Big)
         + \Big(\lim_{a\to -\infty} \tan^{-1}(x) \ \Big|_{a}^{0}\Big) \\
        &= \Big(\lim_{b\to \infty} \tan^{-1}(b) - \tan^{-1}(0)\Big)
         + \Big(\lim_{a\to -\infty} \tan^{-1}(0) - \tan^{-1}(a)\Big) \\
        &= \Big(\frac{\pi}{2} - 0\Big) + \Big(0 - (-\frac{\pi}{2})\Big) \\
        &= \pi
    \end{align*}

    \subsection*{Type 1 p-integrals}
    \begin{align*}
        \int_{1}^{\infty} \frac{1}{x^p} \ dx
    \end{align*}
    This integral converges for all $p>1$ and diverge for all $p\le1$
    \begin{align*}
        \int_{1}^{\infty} \frac{1}{x^p} \ dx
        &= \int_{1}^{\infty} x^{-p} \ dx \\
        &= \Big[ \frac{1}{-p+1} x^{-p+1} \Big]_{1}^{\infty} \\
        &= \frac{1}{1-p} \bigg(\big(\lim_{b\to \infty} b^{p-1}\big) - 1\bigg)
    \end{align*}
    For $p>1$, the limit results in $\infty^\text{negative}$ and will converge
    to zero. If $p\le1$, the limit will result in $\infty^\text{positive}$ and
    will diverge.
    For improper Type 1 integrals, only the behaviour at infinity or negative
    infinity determines the convergence.

    \section*{Integrals That Contain Infinity}
    If $f(x)$ is continuous on (a,b] then 
    \begin{align*}
        \int_{a}^{b} f(x) \ dx = \lim_{c\to a^+} \int_{c}^{b} f(x) \ dx
    \end{align*}
    If $f(x)$ is continuous on [a,b), then
    \begin{align*}
        \int_{a}^{b} f(x) \ dx = \lim_{c\to b^-} \int_{a}^{c} f(x) \ dx
    \end{align*}
    If there is a discontinuity or a infinity value in function on the domain
    being integrated, then the integral needs to be split on the discontinuity.
    \begin{align*}
        \int_{a}^{d} f(x) \ dx
        &= \lim_{b\to x_{dis}^-} \int_{a}^{b} f(x) \ dx
        + \lim_{c\to x_{dis}^+} \int_{c}^{d} f(x) \ dx \\
        \text{where } x_{dis} &= \text{ the position of the discontinuity}
    \end{align*}
    If either of the component integrals diverges, the original is said to be
    divergent. 

    \subsection*{Example}
    \begin{align*}
        \int_{0}^{1} \frac{1}{x^3} \ dx
        &= \lim_{x\to 0^+} \int_{a}^{1} x^{-3} \ dx \\
        &= \lim_{x\to 0^+} \frac{x^{-2}}{2} \ \Big|_{a}^{1} \\
        &= \lim_{a\to 0^+} -\frac{1}{2} + \frac{1}{2a^2} \\
        &= \infty
    \end{align*}
    This integral diverges

    \subsection*{Example 2}
    \begin{align*}
        \int_{0}^{1} \frac{1}{\sqrt{x}} \ dx
        &= \lim_{x\to 0^+} \int_{a}^{1} x^{-\frac{1}{2}} \ dx \\
        &= \lim_{x\to 0^+} 2\sqrt{x} \ \Big|_{a}^{1} \\
        &= \lim_{a\to 0^+} 2\sqrt{1} - 2\sqrt{a} \\
        &= 2\sqrt{1} - 2\sqrt{0} \\
        &= 2
    \end{align*}
    This integral converges

    \subsection*{Type 2 p-integrals}
    type 2 p-integrals are of the following.
    \begin{align*}
        \int_{0}^{a} \frac{1}{x^p} \ dx
    \end{align*}
    These integrals converge for all $p<1$ and diverges to infinity for all
    $p\ge 1$.

    \subsection*{Example 3}
    \begin{align*}
        \int_{0}^{\infty} \frac{e^{-\sqrt{x}}}{\sqrt{x}} \ dx
        = \lim_{a\to 0^-} \int_{0}^{1} \frac{e^{-\sqrt{x}}}{\sqrt{x}} \ dx
        + \lim_{b\to \infty} \int_{1}^{b} \frac{e^{-\sqrt{x}}}{\sqrt{x}} \ dx \\
    \end{align*}
    \begin{align*}
        &\int \frac{e^{-\sqrt{x}}}{\sqrt{x}} \ dx \\
        u &= \sqrt{x} \\
        du &= \frac{1}{2\sqrt{x}} dx \\
        \int \frac{e^{-\sqrt{x}}}{\sqrt{x}} \ dx
        &= 2\int e^{-u} \ du \\
        &= -2e^{-u} + c \\
        &= -2e^{-\sqrt{x}} + c
    \end{align*}
    \begin{align*}
        I &= \lim_{a\to 0^-} \Big[-2e^{-\sqrt{x}} \ \Big]_{a}^{1}
        + \lim_{b\to \infty} \Big[-2e^{-\sqrt{x}} \ \Big]_{1}^{b} \\
          &= (-2e^{-1} + 2e^0) + (-2e^{-\infty} + 2e^{-1}) \\
          &= (-2e^{-1} + 2) + (0 + 2e^{-1}) \\
          &= 2
    \end{align*}

    \section*{Convergence}
    If $f(x) \ge g(x) > 0$ and $\int_{a}^{\infty} g(x) \ dx$ diverges then
    $\int_{0}^{\infty} f(x) \ dx$ also diverges.

    \subsection*{Example}
    \[
        \int_{5}^{\infty} \frac{2+cos(x)}{x} \ dx \\
    \]
    \begin{align*}
        \cos(x) &\in [-1,1] \\
        2+\cos(x) &\in [1,3] \\
        \frac{2+\cos(x)}{x} &\in [\frac{1}{x}, \frac{3}{x}] \\
    \end{align*}
    \begin{gather*}
        \int_{5}^{\infty} \frac{1}{x} \ dx
        \le \int_{5}^{\infty} \frac{2+cos(x)}{x} \ dx
        \le \int_{5}^{\infty} \frac{3}{x} \ dx \\
        \infty \le \int_{5}^{\infty} \frac{2+cos(x)}{x} \ dx \le \infty \\
        \therefore \int_{5}^{\infty} \frac{2+cos(x)}{x} \ dx \rightarrow \infty
    \end{gather*}

\end{document}
