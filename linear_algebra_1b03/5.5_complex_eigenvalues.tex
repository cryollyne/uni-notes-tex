%%%%%%%%%%%%%%%%%%%%%%%%%%%%% Define Article %%%%%%%%%%%%%%%%%%%%%%%%%%%%%%%%%%
\documentclass{article}
%%%%%%%%%%%%%%%%%%%%%%%%%%%%%%%%%%%%%%%%%%%%%%%%%%%%%%%%%%%%%%%%%%%%%%%%%%%%%%%

%%%%%%%%%%%%%%%%%%%%%%%%%%%%% Using Packages %%%%%%%%%%%%%%%%%%%%%%%%%%%%%%%%%%
\usepackage{geometry}
\usepackage{graphicx}
\usepackage{amssymb}
\usepackage{amsmath}
\usepackage{amsthm}
\usepackage{empheq}
\usepackage{mdframed}
\usepackage{booktabs}
\usepackage{lipsum}
\usepackage{graphicx}
\usepackage{color}
\usepackage{psfrag}
\usepackage{pgfplots}
\usepackage{bm}
%%%%%%%%%%%%%%%%%%%%%%%%%%%%%%%%%%%%%%%%%%%%%%%%%%%%%%%%%%%%%%%%%%%%%%%%%%%%%%%

% Other Settings

%%%%%%%%%%%%%%%%%%%%%%%%%% Page Setting %%%%%%%%%%%%%%%%%%%%%%%%%%%%%%%%%%%%%%%
\geometry{a4paper}

%%%%%%%%%%%%%%%%%%%%%%%%%% Define some useful colors %%%%%%%%%%%%%%%%%%%%%%%%%%
\definecolor{ocre}{RGB}{243,102,25}
\definecolor{mygray}{RGB}{243,243,244}
\definecolor{deepGreen}{RGB}{26,111,0}
\definecolor{shallowGreen}{RGB}{235,255,255}
\definecolor{deepBlue}{RGB}{61,124,222}
\definecolor{shallowBlue}{RGB}{235,249,255}
%%%%%%%%%%%%%%%%%%%%%%%%%%%%%%%%%%%%%%%%%%%%%%%%%%%%%%%%%%%%%%%%%%%%%%%%%%%%%%%

%%%%%%%%%%%%%%%%%%%%%%%%%% Define an orangebox command %%%%%%%%%%%%%%%%%%%%%%%%
\newcommand\orangebox[1]{\fcolorbox{ocre}{mygray}{\hspace{1em}#1\hspace{1em}}}
%%%%%%%%%%%%%%%%%%%%%%%%%%%%%%%%%%%%%%%%%%%%%%%%%%%%%%%%%%%%%%%%%%%%%%%%%%%%%%%

%%%%%%%%%%%%%%%%%%%%%%%%%%%% English Environments %%%%%%%%%%%%%%%%%%%%%%%%%%%%%
\newtheoremstyle{mytheoremstyle}{3pt}{3pt}{\normalfont}{0cm}{\rmfamily\bfseries}{}{1em}{{\color{black}\thmname{#1}~\thmnumber{#2}}\thmnote{\,--\,#3}}
\newtheoremstyle{myproblemstyle}{3pt}{3pt}{\normalfont}{0cm}{\rmfamily\bfseries}{}{1em}{{\color{black}\thmname{#1}~\thmnumber{#2}}\thmnote{\,--\,#3}}
\theoremstyle{mytheoremstyle}
\newmdtheoremenv[linewidth=1pt,backgroundcolor=shallowGreen,linecolor=deepGreen,leftmargin=0pt,innerleftmargin=20pt,innerrightmargin=20pt,]{theorem}{Theorem}[section]
\theoremstyle{mytheoremstyle}
\newmdtheoremenv[linewidth=1pt,backgroundcolor=shallowBlue,linecolor=deepBlue,leftmargin=0pt,innerleftmargin=20pt,innerrightmargin=20pt,]{definition}{Definition}[section]
\theoremstyle{myproblemstyle}
\newmdtheoremenv[linecolor=black,leftmargin=0pt,innerleftmargin=10pt,innerrightmargin=10pt,]{problem}{Problem}[section]
%%%%%%%%%%%%%%%%%%%%%%%%%%%%%%%%%%%%%%%%%%%%%%%%%%%%%%%%%%%%%%%%%%%%%%%%%%%%%%%

%%%%%%%%%%%%%%%%%%%%%%%%%%%%%%% Plotting Settings %%%%%%%%%%%%%%%%%%%%%%%%%%%%%
\usepgfplotslibrary{colorbrewer}
\pgfplotsset{width=8cm,compat=1.9}
%%%%%%%%%%%%%%%%%%%%%%%%%%%%%%%%%%%%%%%%%%%%%%%%%%%%%%%%%%%%%%%%%%%%%%%%%%%%%%%

%%%%%%%%%%%%%%%%%%%%%%%%%%%%%%% Title & Author %%%%%%%%%%%%%%%%%%%%%%%%%%%%%%%%
\title{Complex Eigenvalues}
\author{Patrick Chen}
\date{Nov 19, 2024}
%%%%%%%%%%%%%%%%%%%%%%%%%%%%%%%%%%%%%%%%%%%%%%%%%%%%%%%%%%%%%%%%%%%%%%%%%%%%%%%

\begin{document}
    \maketitle
    \section*{Complex Numbers}
    The imaginary unit $i$ is a number defined to be a number such that
    $i^2=-1$. Complex numbers take the form $z=a+bi$.
    \begin{itemize}
        \item Addition: $(a+bi) + (c+di) = (a+c) + (b+d)i$
        \item Multiplication: $(a+bi) \cdot (c+di) = (ac-bd) + (ad+bc)i$
        \item Conjugate: $\overline{a+bi} = a-bi$
        \item Magnitude: $||z|| = \sqrt{z\bar{z}} = \sqrt{a^2+b^2}$
        \item Division: $\frac{1}{z} = \frac{1}{z} \frac{\bar{z}}{\bar{z}} = \frac{\bar{z}}{a^2+b^2}$
    \end{itemize}

    Complex addition and multiplication are both associative and commutative.
    Complex numbers are useful because every non-constant complex polynomial of
    degree $n$ has $n$ (possibly repeated) root when considering complex
    polynomials.
    \begin{align*}
        r &= \sqrt{a^2+b^2} \\
        a &= r\cos(\theta) \\
        b &= r\sin(\theta) \\
        z &= r(\cos(\theta) + i\sin(\theta)) = re^{i\theta}
    \end{align*}
    Complex Numbers can also be represented as angle and magnitude in what is
    called polar coordinates. The angle is measured from the x-axis,
    counter-clockwise.

    \section*{Complex Eigenvalues}
    $\mathbb{C}^n$ is a complex vector space where addition and scalar
    multiplication is done component-wise. For a complex $n\times n$ matrix $A$,
    we say $\lambda\in \mathbb{C}$ is a eigenvalue of $A$ if for some non-zero
    $u\in \mathbb{C}^n$, $Au=\lambda u$. Linear systems, matrices, determinants,
    eigenvalues, and diagonalization all remain unchanged when extending from
    $\mathbb{R}^n$ to $\mathbb{C}^n$.

    \subsection*{General Diagonalization for Complex Matrices}
    A matrix $A$ is only diagonalizable when the algebraic multiplicity equals
    the geometric multiplicity. Every symmetric real matrix has only real valued
    eigenvalues and is diagonalizable. If the matrix has $n$ different
    eigenvalues, then $A$ is diagonalizable.

    \subsection*{General Notion of an eigenvalue}
    Suppose that $T: V \mapsto V$ is a linear transformation defined on a vector
    space $V$. If $T(v)=\lambda v$ for some non-zero $v\in V$, we say that
    $\lambda$ is a eigenvalue for $T$ and $v$ is a eigenvector for $\lambda$.

    \subsection*{Example 1}
    \begin{align*}
        A = \begin{bmatrix}
            0 & -1 \\
            1 & 0
        \end{bmatrix} \\
        det(A-\lambda I) = \lambda^2 + 1 \\
        \lambda = \pm i
    \end{align*}

    \begin{align*}
        \lambda &= i \\
        A-\lambda I &= \begin{bmatrix}
            -i & -1 \\
            1 & -i
        \end{bmatrix} \\
        rref(A-\lambda I) &= \begin{bmatrix}
            1 & -i \\
            0 & 0
        \end{bmatrix} \\
        z_1 - iz_2 &= 0 \\
        z_1 &= iz_2 \\
        z &= t \begin{bmatrix}
            i \\ 1
        \end{bmatrix}
    \end{align*}

    \begin{align*}
        \lambda &= -i \\
        A-\lambda I &= \begin{bmatrix}
            i & -1 \\
            1 & i
        \end{bmatrix} \\
        rref(A-\lambda I) &= \begin{bmatrix}
            1 & i \\
            0 & 0
        \end{bmatrix} \\
        z_1 + iz_2 &= 0 \\
        z_1 &= -iz_2 \\
        z &= t \begin{bmatrix}
            -i \\ 1
        \end{bmatrix}
    \end{align*}

    \begin{align*}
        P &= \begin{bmatrix}
            i & -i \\
            1 & 1
        \end{bmatrix} \\
        P^{-1} &= \frac{1}{2i} \begin{bmatrix}
            1 & i \\
            -1 & i
        \end{bmatrix} \\
        D &= \begin{bmatrix}
            i & 0 \\
            0 & -i
        \end{bmatrix} \\
        A &= 
        \begin{bmatrix}
            i & -i \\
            1 & 1
        \end{bmatrix}
        \begin{bmatrix}
            i & 0 \\
            0 & -i
        \end{bmatrix}
        \frac{1}{2i} \begin{bmatrix}
            1 & i \\
            -1 & i
        \end{bmatrix} \\
        &= \frac{1}{2i}
        \begin{bmatrix}
            i & -i \\
            1 & 1
        \end{bmatrix}
        \begin{bmatrix}
            i & 0 \\
            0 & -i
        \end{bmatrix}
        \begin{bmatrix}
            1 & i \\
            -1 & i
        \end{bmatrix} \\
    \end{align*}

    \subsection*{Example 2}
    \begin{align*}
        A &= \begin{bmatrix}
            1 & -1 & 2 \\
            0 & 2 & 1 \\
            0 & 0 & 3 \\
        \end{bmatrix} \\
        \lambda = 1,2,3
    \end{align*}
    Since there are three distinct eigenvalues and the matrix $A$ is $3\times3$,
    the matrix is diagonalizable.

    \subsection*{Example 3}
    Suppose that there is a transformation $T$ from infinitely differentiable
    function ($C^\infty$) to infinitely different functions. Find the
    eigenvalues for $T$.
    \begin{align*}
        T: C^\infty \mapsto C^\infty \\
        T(f) = \frac{df}{dx}
    \end{align*}

    \begin{align*}
        \frac{df}{dx} = \lambda f \\
        f = ke^{\lambda x}
    \end{align*}
    Every $\lambda\in\mathbb{C}$ is a eigenvalue

    \subsection*{Example 4}
    Suppose that there is a linear transformation $T$ from polynomials to
    polynomials. Find the eigenvalues for $T$.
    \begin{align*}
        T: P \mapsto P \\
        T(p) = xp
    \end{align*}

    \begin{align*}
        T(p) = \lambda p
        p &= a_0 + a_1 x + \dots a_nx^n \\
        T(p) &= a_0 x + a_1x^2 + \dots a_nx^{n+1}
    \end{align*}
    If $\lambda\ne0$, then $a_0=0$, and all the other coefficients are zero by
    induction. If $\lambda=0$, then $p=0$, thus there are no eigenvalues.

\end{document}
