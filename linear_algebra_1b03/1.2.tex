%%%%%%%%%%%%%%%%%%%%%%%%%%%%% Define Article %%%%%%%%%%%%%%%%%%%%%%%%%%%%%%%%%%
\documentclass{article}
%%%%%%%%%%%%%%%%%%%%%%%%%%%%%%%%%%%%%%%%%%%%%%%%%%%%%%%%%%%%%%%%%%%%%%%%%%%%%%%

%%%%%%%%%%%%%%%%%%%%%%%%%%%%% Using Packages %%%%%%%%%%%%%%%%%%%%%%%%%%%%%%%%%%
\usepackage{geometry}
\usepackage{graphicx}
\usepackage{amssymb}
\usepackage{amsmath}
\usepackage{amsthm}
\usepackage{empheq}
\usepackage{mdframed}
\usepackage{booktabs}
\usepackage{lipsum}
\usepackage{graphicx}
\usepackage{color}
\usepackage{psfrag}
\usepackage{pgfplots}
\usepackage{bm}
%%%%%%%%%%%%%%%%%%%%%%%%%%%%%%%%%%%%%%%%%%%%%%%%%%%%%%%%%%%%%%%%%%%%%%%%%%%%%%%

% Other Settings

%%%%%%%%%%%%%%%%%%%%%%%%%% Page Setting %%%%%%%%%%%%%%%%%%%%%%%%%%%%%%%%%%%%%%%
\geometry{a4paper}

%%%%%%%%%%%%%%%%%%%%%%%%%% Define some useful colors %%%%%%%%%%%%%%%%%%%%%%%%%%
\definecolor{ocre}{RGB}{243,102,25}
\definecolor{mygray}{RGB}{243,243,244}
\definecolor{deepGreen}{RGB}{26,111,0}
\definecolor{shallowGreen}{RGB}{235,255,255}
\definecolor{deepBlue}{RGB}{61,124,222}
\definecolor{shallowBlue}{RGB}{235,249,255}
%%%%%%%%%%%%%%%%%%%%%%%%%%%%%%%%%%%%%%%%%%%%%%%%%%%%%%%%%%%%%%%%%%%%%%%%%%%%%%%

%%%%%%%%%%%%%%%%%%%%%%%%%% Define an orangebox command %%%%%%%%%%%%%%%%%%%%%%%%
\newcommand\orangebox[1]{\fcolorbox{ocre}{mygray}{\hspace{1em}#1\hspace{1em}}}
%%%%%%%%%%%%%%%%%%%%%%%%%%%%%%%%%%%%%%%%%%%%%%%%%%%%%%%%%%%%%%%%%%%%%%%%%%%%%%%

%%%%%%%%%%%%%%%%%%%%%%%%%%%% English Environments %%%%%%%%%%%%%%%%%%%%%%%%%%%%%
\newtheoremstyle{mytheoremstyle}{3pt}{3pt}{\normalfont}{0cm}{\rmfamily\bfseries}{}{1em}{{\color{black}\thmname{#1}~\thmnumber{#2}}\thmnote{\,--\,#3}}
\newtheoremstyle{myproblemstyle}{3pt}{3pt}{\normalfont}{0cm}{\rmfamily\bfseries}{}{1em}{{\color{black}\thmname{#1}~\thmnumber{#2}}\thmnote{\,--\,#3}}
\theoremstyle{mytheoremstyle}
\newmdtheoremenv[linewidth=1pt,backgroundcolor=shallowGreen,linecolor=deepGreen,leftmargin=0pt,innerleftmargin=20pt,innerrightmargin=20pt,]{theorem}{Theorem}[section]
\theoremstyle{mytheoremstyle}
\newmdtheoremenv[linewidth=1pt,backgroundcolor=shallowBlue,linecolor=deepBlue,leftmargin=0pt,innerleftmargin=20pt,innerrightmargin=20pt,]{definition}{Definition}[section]
\theoremstyle{myproblemstyle}
\newmdtheoremenv[linecolor=black,leftmargin=0pt,innerleftmargin=10pt,innerrightmargin=10pt,]{problem}{Problem}[section]
%%%%%%%%%%%%%%%%%%%%%%%%%%%%%%%%%%%%%%%%%%%%%%%%%%%%%%%%%%%%%%%%%%%%%%%%%%%%%%%

%%%%%%%%%%%%%%%%%%%%%%%%%%%%%%% Plotting Settings %%%%%%%%%%%%%%%%%%%%%%%%%%%%%
\usepgfplotslibrary{colorbrewer}
\pgfplotsset{width=8cm,compat=1.9}
%%%%%%%%%%%%%%%%%%%%%%%%%%%%%%%%%%%%%%%%%%%%%%%%%%%%%%%%%%%%%%%%%%%%%%%%%%%%%%%

%%%%%%%%%%%%%%%%%%%%%%%%%%%%%%% Title & Author %%%%%%%%%%%%%%%%%%%%%%%%%%%%%%%%
\title{Row Reduction}
\author{Patrick Chen}
\date{Sept 5, 2024}
%%%%%%%%%%%%%%%%%%%%%%%%%%%%%%%%%%%%%%%%%%%%%%%%%%%%%%%%%%%%%%%%%%%%%%%%%%%%%%%

\begin{document}
    \maketitle

    Finding solutions to systems of linear equations:
    \begin{align*}
        \begin{matrix}
               & & 2y &+& 3z &=& -4 \\
            2x &+& 6y &+& 2z &=& 4 \\
            3x & &    &+& z  &=& 3
        \end{matrix}
    \end{align*}

    An augmented matrix representing the system of linear equations:
    \begin{align*}
        \begin{bmatrix}
            0 & 2 & 2 & -4 \\
            2 & 6 & 2 & 4 \\
            3 & 0 & 1 & 3 \\
        \end{bmatrix}
    \end{align*}

    The coefficient matrix representing the system of linear equations:
    \begin{align*}
        \begin{bmatrix}
            0 & 2 & 2  \\
            2 & 6 & 2  \\
            3 & 0 & 1  \\
        \end{bmatrix}
    \end{align*}

    The row reduction algorithm is as follows:
    \begin{enumerate}
        \item multiply the first row by the reciprical of the first element.
        This will make the first row have the form:
        \begin{align*}
            \begin{bmatrix}
                1 & a_{12} & a_{13} & \ldots & a_{1m} & b_1
            \end{bmatrix}
        \end{align*}

        \item subtract multiples of the first row from every row below such that
        the first element of those rows is zero. This should make the matrix
        take the following form:
        \begin{align*}
            \begin{bmatrix}
                1 & a_{12} & a_{13} &        & a_{1m} & b_1 \\
                0 & a_{22} & a_{23} & \ldots & a_{2m} & b_2 \\
                0 & a_{32} & a_{33} &        & a_{3m} & b_3 \\
                  & \vdots &        & \ddots &        &     \\
                0 & a_{n2} & a_{n3} &        & a_{nm} & b_n
            \end{bmatrix}
        \end{align*}

        \item repeat for the second element, then the third, etc. afterward the
        matrix should take the form:
        \begin{align*}
            \begin{bmatrix}
                1 & a_{12} & a_{13} &        & a_{1m} & b_1 \\
                0 & 1      & a_{23} & \ldots & a_{2m} & b_2 \\
                0 & 0      & 1      &        & a_{3m} & b_3 \\
                  & \vdots &        & \ddots &        &     \\
                0 & 0      & 0      &        & 1      & b_n
            \end{bmatrix}
        \end{align*}

        \item subtract multiples of the second row from each row above such the
            second column results in zero. Repeat for third row and third
            column, then fourth, etc. This should result in the matrix taking
            the form:
        \begin{align*}
            \begin{bmatrix}
                1 & 0      & 0      &        & 0      & b_1 \\
                0 & 1      & 0      & \ldots & 0      & b_2 \\
                0 & 0      & 1      &        & 0      & b_3 \\
                  & \vdots &        & \ddots &        &     \\
                0 & 0      & 0      &        & 1      & b_n
            \end{bmatrix}
        \end{align*}
    \end{enumerate}

    Row Reduction example
    \begin{align*}
        \begin{bmatrix}
            0 & 2 & 2 & -4 \\
            2 & 6 & 2 & 4 \\
            3 & 0 & 1 & 3 \\
        \end{bmatrix}
        &=\begin{bmatrix}
            2 & 6 & 2 & 4 \\
            0 & 2 & 2 & -4 \\
            3 & 0 & 1 & 3 \\
        \end{bmatrix} \text{swap row (1), (2)} \\
        &=\begin{bmatrix}
            1 & 3 & 1 & 2 \\
            0 & 2 & 2 & -4 \\
            3 & 0 & 1 & 3 \\
        \end{bmatrix} \text{divide row (1) by 2} \\
        &=\begin{bmatrix}
            1 & 3 & 1 & 2 \\
            0 & 2 & 2 & -4 \\
            0 &-9 &-2 &-3 \\
        \end{bmatrix}  \text{subtract 3 $\times$ row (1) from row (3)}\\
        &=\begin{bmatrix}
            1 & 3 & 1 & 2 \\
            0 & 1 & 1 & -2 \\
            0 &-9 &-2 &-3 \\
        \end{bmatrix} \text{divide row (2) by 2}\\
        &=\begin{bmatrix}
            1 & 3 & 1 & 2 \\
            0 & 1 & 1 & -2 \\
            0 & 0 & 7 & -21 \\
        \end{bmatrix} \text{add 9 $\times$ row (2) to row (3)}\\
        &=\begin{bmatrix}
            1 & 3 & 1 & 2 \\
            0 & 1 & 1 & -2 \\
            0 & 0 & 1 & -3 \\
        \end{bmatrix} \text{divide row (3) by 7}\\
        &=\begin{bmatrix}
            1 & 0 &-2 & 8 \\
            0 & 1 & 1 & -2 \\
            0 & 0 & 1 & -3 \\
        \end{bmatrix} \text{subtract 3 $\times$ row (2) from row (1)}\\
        &=\begin{bmatrix}
            1 & 0 & 0 & 2 \\
            0 & 1 & 0 & 1 \\
            0 & 0 & 1 & -3 \\
        \end{bmatrix} \text{add 2 $\times$ row (3) to row (1),
                            subtract row (3) from row (2)}\\
    \end{align*}

    \pagebreak

\end{document}
