%%%%%%%%%%%%%%%%%%%%%%%%%%%%% Define Article %%%%%%%%%%%%%%%%%%%%%%%%%%%%%%%%%%
\documentclass{article}
%%%%%%%%%%%%%%%%%%%%%%%%%%%%%%%%%%%%%%%%%%%%%%%%%%%%%%%%%%%%%%%%%%%%%%%%%%%%%%%

%%%%%%%%%%%%%%%%%%%%%%%%%%%%% Using Packages %%%%%%%%%%%%%%%%%%%%%%%%%%%%%%%%%%
\usepackage{geometry}
\usepackage{graphicx}
\usepackage{amssymb}
\usepackage{amsmath}
\usepackage{amsthm}
\usepackage{empheq}
\usepackage{mdframed}
\usepackage{booktabs}
\usepackage{lipsum}
\usepackage{graphicx}
\usepackage{color}
\usepackage{psfrag}
\usepackage{pgfplots}
\usepackage{bm}
%%%%%%%%%%%%%%%%%%%%%%%%%%%%%%%%%%%%%%%%%%%%%%%%%%%%%%%%%%%%%%%%%%%%%%%%%%%%%%%

% Other Settings

%%%%%%%%%%%%%%%%%%%%%%%%%% Page Setting %%%%%%%%%%%%%%%%%%%%%%%%%%%%%%%%%%%%%%%
\geometry{a4paper}

%%%%%%%%%%%%%%%%%%%%%%%%%% Define some useful colors %%%%%%%%%%%%%%%%%%%%%%%%%%
\definecolor{ocre}{RGB}{243,102,25}
\definecolor{mygray}{RGB}{243,243,244}
\definecolor{deepGreen}{RGB}{26,111,0}
\definecolor{shallowGreen}{RGB}{235,255,255}
\definecolor{deepBlue}{RGB}{61,124,222}
\definecolor{shallowBlue}{RGB}{235,249,255}
%%%%%%%%%%%%%%%%%%%%%%%%%%%%%%%%%%%%%%%%%%%%%%%%%%%%%%%%%%%%%%%%%%%%%%%%%%%%%%%

%%%%%%%%%%%%%%%%%%%%%%%%%% Define an orangebox command %%%%%%%%%%%%%%%%%%%%%%%%
\newcommand\orangebox[1]{\fcolorbox{ocre}{mygray}{\hspace{1em}#1\hspace{1em}}}
%%%%%%%%%%%%%%%%%%%%%%%%%%%%%%%%%%%%%%%%%%%%%%%%%%%%%%%%%%%%%%%%%%%%%%%%%%%%%%%

%%%%%%%%%%%%%%%%%%%%%%%%%%%% English Environments %%%%%%%%%%%%%%%%%%%%%%%%%%%%%
\newtheoremstyle{mytheoremstyle}{3pt}{3pt}{\normalfont}{0cm}{\rmfamily\bfseries}{}{1em}{{\color{black}\thmname{#1}~\thmnumber{#2}}\thmnote{\,--\,#3}}
\newtheoremstyle{myproblemstyle}{3pt}{3pt}{\normalfont}{0cm}{\rmfamily\bfseries}{}{1em}{{\color{black}\thmname{#1}~\thmnumber{#2}}\thmnote{\,--\,#3}}
\theoremstyle{mytheoremstyle}
\newmdtheoremenv[linewidth=1pt,backgroundcolor=shallowGreen,linecolor=deepGreen,leftmargin=0pt,innerleftmargin=20pt,innerrightmargin=20pt,]{theorem}{Theorem}[section]
\theoremstyle{mytheoremstyle}
\newmdtheoremenv[linewidth=1pt,backgroundcolor=shallowBlue,linecolor=deepBlue,leftmargin=0pt,innerleftmargin=20pt,innerrightmargin=20pt,]{definition}{Definition}[section]
\theoremstyle{myproblemstyle}
\newmdtheoremenv[linecolor=black,leftmargin=0pt,innerleftmargin=10pt,innerrightmargin=10pt,]{problem}{Problem}[section]
%%%%%%%%%%%%%%%%%%%%%%%%%%%%%%%%%%%%%%%%%%%%%%%%%%%%%%%%%%%%%%%%%%%%%%%%%%%%%%%

%%%%%%%%%%%%%%%%%%%%%%%%%%%%%%% Plotting Settings %%%%%%%%%%%%%%%%%%%%%%%%%%%%%
\usepgfplotslibrary{colorbrewer}
\pgfplotsset{width=8cm,compat=1.9}
%%%%%%%%%%%%%%%%%%%%%%%%%%%%%%%%%%%%%%%%%%%%%%%%%%%%%%%%%%%%%%%%%%%%%%%%%%%%%%%

%%%%%%%%%%%%%%%%%%%%%%%%%%%%%%% Title & Author %%%%%%%%%%%%%%%%%%%%%%%%%%%%%%%%
\title{Determinants}
\author{Patrick Chen}
\date{Oct 10, 2024}
%%%%%%%%%%%%%%%%%%%%%%%%%%%%%%%%%%%%%%%%%%%%%%%%%%%%%%%%%%%%%%%%%%%%%%%%%%%%%%%

\begin{document}
    \maketitle
    A $n\times n$ matrix $A$ will map the unit vectors in $\mathbb{R}^n$ to
    columns of $A$, which is also in $\mathbb{R}^n$. The n-dimensional
    equivalent to volume of a shape, when mapped through the matrix, will be
    scaled by $det\ A$. If $det\ A = 0$, then the matrix $A$ is not invertible.

    \subsection*{2x2 Matrix}
    \begin{align*}
        A = \begin{bmatrix}
            a & b \\
            c & d
        \end{bmatrix} \\
        \begin{bmatrix}
            a & b \\
            0 & -\frac{cb}{a} + d
        \end{bmatrix} \text{ ($-\frac{c}{a}$ R1 + R2)}
    \end{align*}
    A is invertible iff $a\ne 0$ and $-\frac{cb}{a} +d\ne 0$.
    The determinant of $A$ is $ad-bc$ then
    \begin{align*}
        A^{-1} = \frac{1}{det(A)} \begin{bmatrix}
            d & -b \\
            -c & a
        \end{bmatrix}
    \end{align*}

    The area of a parallelogram formed by $\begin{bmatrix} a \\ c \end{bmatrix}$
    and $\begin{bmatrix} b \\ d \end{bmatrix}$ is equal to $|det\ A|$

    If a edge of the parallelogram is not on the x-axis, it can be rotated onto
    the x-axis with a rotation matrix. Rotation matrices preserves the
    determinant so the determinant is unchanged after being rotated.

    \subsection*{3x3 Matrix}
    The absolute value of the determinant of a $3\times 3$ matrix is the volume
    of a parallelepiped formed by the three columns in $\mathbb{R}^3$. A
    parallelepiped with $v_1,v_2$ in the xy-plane forming a parallelogram, and
    $v_3=[x_3,y_3,z_3]^T$ will have a volume $z_3*(\text{area of
    parallelogram})$. Any parallelepiped can be rotated to have a plane laying
    in the xy-plane.

    \begin{align*}
        det \begin{bmatrix}
            a & b & x_3 \\
            c & d & y_3 \\
            0 & 0 & z_3
        \end{bmatrix} = z_3 \cdot det \begin{bmatrix}
            a & b \\
            c & d
        \end{bmatrix}
    \end{align*}

    \subsection*{General Matrix}
    The determinant can be generalized to higher dimensions by rotating one
    "cell" into a (n-1) volume without a $x_n$ component, calculating the
    determinant for the first (n-1) vectors, then multiplying by the last
    vector's $x_n$ component.

    \section*{General Determinant Formula}
    Suppose the matrix $A$ is a $n\times n$ matrix and $A_{ij}$ is the matrix
    $A$ with the ith row and the jth column removed. $A_{ij}$ will be a size
    $(n-1)\times (n-1)$ matrix. The $(i,j)$-cofactor of $A$ is the determinant
    of $A_{ij}$ multiplied by one or negative one depending on if the sum of the
    row index and column index if even.
    \begin{align*}
        C_{ij} = (-1)^{i+j} \det(A_{ij})
    \end{align*}
    The determinant is defined as
    \begin{align*}
        \det(A) &= a_{i1}C_{i1} +
                  a_{i2}C_{i2} +
                  a_{i3}C_{i3} +
                  \dots +
                  a_{in}C_{in} \\
                &= \sum_{j=0}^n a_{ji}C_{ji}
    \end{align*}
    The determinant can also be found by summing across the columns instead of
    the rows.

    \begin{align*}
        \det(A) &= a_{1j}C_{1j} +
                   a_{2j}C_{2j} +
                   a_{3j}C_{3j} +
                   \dots +
                   a_{nj}C_{nj} \\
                &= \sum_{i=0}^n a_{ji}C_{ji}
    \end{align*}
    The determinant is well defined when summing across any row or column.

    \subsection*{Example}
    Computing the determinant of a matrix across the first row
    \begin{align*}
        A &= \begin{bmatrix}
            1 & -1 & 2 \\
            3 &  1 & 4 \\
            2 & 0 & 1
        \end{bmatrix} \\
        \det(A) &= (1)(1)\det\bigg(\begin{bmatrix}
            1 & 4 \\
            0 & 1
        \end{bmatrix}\bigg) + (-1)(-1)\det\bigg(\begin{bmatrix}
            3 & 4 \\
            2 & 1
        \end{bmatrix}\bigg) + (1)(2)\det\bigg(\begin{bmatrix}
            3 & 1 \\
            2 & 0
        \end{bmatrix}\bigg) \\
        &= 1-5+2(-2)  \\
        &= -8
    \end{align*}

    Computing the determinant of the same matrix across the second column
    \begin{align*}
        A &= \begin{bmatrix}
            1 & -1 & 2 \\
            3 &  1 & 4 \\
            2 & 0 & 1
        \end{bmatrix} \\
        \det(A) &= (-1)(-1)\det\bigg(\begin{bmatrix}
            3 & 4 \\
            2 & 1
        \end{bmatrix}\bigg) + (1)(1)\det\bigg(\begin{bmatrix}
            1 & 2 \\
            2 & 1
        \end{bmatrix}\bigg) + (-1)(0)\det\bigg(\begin{bmatrix}
            1 & 2 \\
            3 & 4
        \end{bmatrix}\bigg) \\
        &= (-5) + (-3) + 0 \\
        &= -8
    \end{align*}

    \subsection*{Example 2}
    \begin{align*}
        A = \begin{bmatrix}
            1 & 2 & -5 & 6 \\
            0 & 3 & 4 & 7 \\
            0 & 0 & 1 & -2 \\
            0 & 0 & 0 & 3
        \end{bmatrix}
    \end{align*}
    Finding the determinant across the fourth row
    \begin{align*}
        \det(A) &= 0 + 0 + 0 + (1)(3)\det(\begin{bmatrix}
            1 & 2 & -5 \\
            0 & 3 & 4 \\
            0 & 0 & 1
        \end{bmatrix}) \\
        &= 3(0 + 0 + 1(\det(\begin{bmatrix}
            1 & 2 \\
            0 & 3
        \end{bmatrix}))) \\
        &= 3(1(3)) \\
        &= 9
    \end{align*}

\end{document}
