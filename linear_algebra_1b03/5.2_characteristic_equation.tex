%%%%%%%%%%%%%%%%%%%%%%%%%%%%% Define Article %%%%%%%%%%%%%%%%%%%%%%%%%%%%%%%%%%
\documentclass{article}
%%%%%%%%%%%%%%%%%%%%%%%%%%%%%%%%%%%%%%%%%%%%%%%%%%%%%%%%%%%%%%%%%%%%%%%%%%%%%%%

%%%%%%%%%%%%%%%%%%%%%%%%%%%%% Using Packages %%%%%%%%%%%%%%%%%%%%%%%%%%%%%%%%%%
\usepackage{geometry}
\usepackage{graphicx}
\usepackage{amssymb}
\usepackage{amsmath}
\usepackage{amsthm}
\usepackage{empheq}
\usepackage{mdframed}
\usepackage{booktabs}
\usepackage{lipsum}
\usepackage{graphicx}
\usepackage{color}
\usepackage{psfrag}
\usepackage{pgfplots}
\usepackage{bm}
%%%%%%%%%%%%%%%%%%%%%%%%%%%%%%%%%%%%%%%%%%%%%%%%%%%%%%%%%%%%%%%%%%%%%%%%%%%%%%%

% Other Settings

%%%%%%%%%%%%%%%%%%%%%%%%%% Page Setting %%%%%%%%%%%%%%%%%%%%%%%%%%%%%%%%%%%%%%%
\geometry{a4paper}

%%%%%%%%%%%%%%%%%%%%%%%%%% Define some useful colors %%%%%%%%%%%%%%%%%%%%%%%%%%
\definecolor{ocre}{RGB}{243,102,25}
\definecolor{mygray}{RGB}{243,243,244}
\definecolor{deepGreen}{RGB}{26,111,0}
\definecolor{shallowGreen}{RGB}{235,255,255}
\definecolor{deepBlue}{RGB}{61,124,222}
\definecolor{shallowBlue}{RGB}{235,249,255}
%%%%%%%%%%%%%%%%%%%%%%%%%%%%%%%%%%%%%%%%%%%%%%%%%%%%%%%%%%%%%%%%%%%%%%%%%%%%%%%

%%%%%%%%%%%%%%%%%%%%%%%%%% Define an orangebox command %%%%%%%%%%%%%%%%%%%%%%%%
\newcommand\orangebox[1]{\fcolorbox{ocre}{mygray}{\hspace{1em}#1\hspace{1em}}}
%%%%%%%%%%%%%%%%%%%%%%%%%%%%%%%%%%%%%%%%%%%%%%%%%%%%%%%%%%%%%%%%%%%%%%%%%%%%%%%

%%%%%%%%%%%%%%%%%%%%%%%%%%%% English Environments %%%%%%%%%%%%%%%%%%%%%%%%%%%%%
\newtheoremstyle{mytheoremstyle}{3pt}{3pt}{\normalfont}{0cm}{\rmfamily\bfseries}{}{1em}{{\color{black}\thmname{#1}~\thmnumber{#2}}\thmnote{\,--\,#3}}
\newtheoremstyle{myproblemstyle}{3pt}{3pt}{\normalfont}{0cm}{\rmfamily\bfseries}{}{1em}{{\color{black}\thmname{#1}~\thmnumber{#2}}\thmnote{\,--\,#3}}
\theoremstyle{mytheoremstyle}
\newmdtheoremenv[linewidth=1pt,backgroundcolor=shallowGreen,linecolor=deepGreen,leftmargin=0pt,innerleftmargin=20pt,innerrightmargin=20pt,]{theorem}{Theorem}[section]
\theoremstyle{mytheoremstyle}
\newmdtheoremenv[linewidth=1pt,backgroundcolor=shallowBlue,linecolor=deepBlue,leftmargin=0pt,innerleftmargin=20pt,innerrightmargin=20pt,]{definition}{Definition}[section]
\theoremstyle{myproblemstyle}
\newmdtheoremenv[linecolor=black,leftmargin=0pt,innerleftmargin=10pt,innerrightmargin=10pt,]{problem}{Problem}[section]
%%%%%%%%%%%%%%%%%%%%%%%%%%%%%%%%%%%%%%%%%%%%%%%%%%%%%%%%%%%%%%%%%%%%%%%%%%%%%%%

%%%%%%%%%%%%%%%%%%%%%%%%%%%%%%% Plotting Settings %%%%%%%%%%%%%%%%%%%%%%%%%%%%%
\usepgfplotslibrary{colorbrewer}
\pgfplotsset{width=8cm,compat=1.9}
%%%%%%%%%%%%%%%%%%%%%%%%%%%%%%%%%%%%%%%%%%%%%%%%%%%%%%%%%%%%%%%%%%%%%%%%%%%%%%%

%%%%%%%%%%%%%%%%%%%%%%%%%%%%%%% Title & Author %%%%%%%%%%%%%%%%%%%%%%%%%%%%%%%%
\title{Characteristic Equation}
\author{Patrick Chen}
\date{Nov 18, 2024}
%%%%%%%%%%%%%%%%%%%%%%%%%%%%%%%%%%%%%%%%%%%%%%%%%%%%%%%%%%%%%%%%%%%%%%%%%%%%%%%

\begin{document}
    \maketitle

    \subsection*{Finding eigenvalues}
    If $\lambda$ is an eigenvalue for $A$, then $A-\lambda I=0$ has a
    non-trivial solution and is not invertible. This is equivalent to finding
    when the determinant of $A-\lambda I$ is equal to zero. In a upper and lower
    triangular matrix, the eigenvalues are the values along the diagonal.
    \begin{align*}
        \det(A-\lambda I) = 0
    \end{align*}

    This determinant will give a polynomial of degree $n$. This is called the
    characteristic polynomial.
    \begin{itemize}
        \item The eigenvalues of $A$ are solutions of the characteristic
            equation of $A$.
        \item The algebraic multiplicity of an eigenvalue $\lambda$ is degree of
            the $\lambda$ root in the characteristic equation.
        \item The geometric multiplicity of $\lambda$ is the dimension of the
            $\lambda$-eigenspace.
        \item The geometric multiplicity is always less than or equal to the
            algebraic multiplicity.
    \end{itemize}

    \subsection*{Example 1}
    \begin{align*}
        A &= \begin{bmatrix}
            1 & 1 \\
            0 & 1
        \end{bmatrix} \\
        det(A-\lambda I) &= det\begin{bmatrix}
            1-\lambda & 1 \\
            0 & 1-\lambda
        \end{bmatrix} \\
        &= (1-\lambda)^2
    \end{align*}
    $\lambda=1$ is the only eigenvalue and has a algebraic multiplicity of 2.
    \begin{align*}
        \begin{bmatrix}
            0 & 1 \\
            0 & 0
        \end{bmatrix}u &= \begin{bmatrix}
            0 \\ 0
        \end{bmatrix} \\
        u &= t \begin{bmatrix}
            1 \\ 0
        \end{bmatrix}
    \end{align*}
    The dimension of the 1-eigenspace has a dimension of 1, thus a geometric
    multiplicity of 1.

    \subsection*{Example 2}
    Find the eigenvalues of the following matrix.
    \begin{align*}
        A &= \begin{bmatrix}
            1 & -7 & 3 & 4 \\
            0 & 2 & 1 & -1 \\
            0 & 0 & 6 & 3 \\
            0 & 0 & 0 & -1 \\
        \end{bmatrix}
    \end{align*}
    \begin{align*}
        A-\lambda I &= \begin{bmatrix}
            1-\lambda & -7 & 3 & 4 \\
            0 & 2-\lambda & 1 & -1 \\
            0 & 0 & 6-\lambda & 3 \\
            0 & 0 & 0 & -1-\lambda \\
        \end{bmatrix} \\
        det(A-\lambda I) &= (1-\lambda)(2-\lambda)(6-\lambda)(-1-\lambda) = 0 \\
        \lambda &= 1,2,6,-1
    \end{align*}
\end{document}
