%%%%%%%%%%%%%%%%%%%%%%%%%%%%% Define Article %%%%%%%%%%%%%%%%%%%%%%%%%%%%%%%%%%
\documentclass{article}
%%%%%%%%%%%%%%%%%%%%%%%%%%%%%%%%%%%%%%%%%%%%%%%%%%%%%%%%%%%%%%%%%%%%%%%%%%%%%%%

%%%%%%%%%%%%%%%%%%%%%%%%%%%%% Using Packages %%%%%%%%%%%%%%%%%%%%%%%%%%%%%%%%%%
\usepackage{geometry}
\usepackage{graphicx}
\usepackage{amssymb}
\usepackage{amsmath}
\usepackage{amsthm}
\usepackage{empheq}
\usepackage{mdframed}
\usepackage{booktabs}
\usepackage{lipsum}
\usepackage{graphicx}
\usepackage{color}
\usepackage{psfrag}
\usepackage{pgfplots}
\usepackage{bm}
%%%%%%%%%%%%%%%%%%%%%%%%%%%%%%%%%%%%%%%%%%%%%%%%%%%%%%%%%%%%%%%%%%%%%%%%%%%%%%%

% Other Settings

%%%%%%%%%%%%%%%%%%%%%%%%%% Page Setting %%%%%%%%%%%%%%%%%%%%%%%%%%%%%%%%%%%%%%%
\geometry{a4paper}

%%%%%%%%%%%%%%%%%%%%%%%%%% Define some useful colors %%%%%%%%%%%%%%%%%%%%%%%%%%
\definecolor{ocre}{RGB}{243,102,25}
\definecolor{mygray}{RGB}{243,243,244}
\definecolor{deepGreen}{RGB}{26,111,0}
\definecolor{shallowGreen}{RGB}{235,255,255}
\definecolor{deepBlue}{RGB}{61,124,222}
\definecolor{shallowBlue}{RGB}{235,249,255}
%%%%%%%%%%%%%%%%%%%%%%%%%%%%%%%%%%%%%%%%%%%%%%%%%%%%%%%%%%%%%%%%%%%%%%%%%%%%%%%

%%%%%%%%%%%%%%%%%%%%%%%%%% Define an orangebox command %%%%%%%%%%%%%%%%%%%%%%%%
\newcommand\orangebox[1]{\fcolorbox{ocre}{mygray}{\hspace{1em}#1\hspace{1em}}}
%%%%%%%%%%%%%%%%%%%%%%%%%%%%%%%%%%%%%%%%%%%%%%%%%%%%%%%%%%%%%%%%%%%%%%%%%%%%%%%

%%%%%%%%%%%%%%%%%%%%%%%%%%%% English Environments %%%%%%%%%%%%%%%%%%%%%%%%%%%%%
\newtheoremstyle{mytheoremstyle}{3pt}{3pt}{\normalfont}{0cm}{\rmfamily\bfseries}{}{1em}{{\color{black}\thmname{#1}~\thmnumber{#2}}\thmnote{\,--\,#3}}
\newtheoremstyle{myproblemstyle}{3pt}{3pt}{\normalfont}{0cm}{\rmfamily\bfseries}{}{1em}{{\color{black}\thmname{#1}~\thmnumber{#2}}\thmnote{\,--\,#3}}
\theoremstyle{mytheoremstyle}
\newmdtheoremenv[linewidth=1pt,backgroundcolor=shallowGreen,linecolor=deepGreen,leftmargin=0pt,innerleftmargin=20pt,innerrightmargin=20pt,]{theorem}{Theorem}[section]
\theoremstyle{mytheoremstyle}
\newmdtheoremenv[linewidth=1pt,backgroundcolor=shallowBlue,linecolor=deepBlue,leftmargin=0pt,innerleftmargin=20pt,innerrightmargin=20pt,]{definition}{Definition}[section]
\theoremstyle{myproblemstyle}
\newmdtheoremenv[linecolor=black,leftmargin=0pt,innerleftmargin=10pt,innerrightmargin=10pt,]{problem}{Problem}[section]
%%%%%%%%%%%%%%%%%%%%%%%%%%%%%%%%%%%%%%%%%%%%%%%%%%%%%%%%%%%%%%%%%%%%%%%%%%%%%%%

%%%%%%%%%%%%%%%%%%%%%%%%%%%%%%% Plotting Settings %%%%%%%%%%%%%%%%%%%%%%%%%%%%%
\usepgfplotslibrary{colorbrewer}
\pgfplotsset{width=8cm,compat=1.9}
%%%%%%%%%%%%%%%%%%%%%%%%%%%%%%%%%%%%%%%%%%%%%%%%%%%%%%%%%%%%%%%%%%%%%%%%%%%%%%%

%%%%%%%%%%%%%%%%%%%%%%%%%%%%%%% Title & Author %%%%%%%%%%%%%%%%%%%%%%%%%%%%%%%%
\title{Orthogonal Sets}
\author{Patrick Chen}
\date{Dec 2, 2024}
%%%%%%%%%%%%%%%%%%%%%%%%%%%%%%%%%%%%%%%%%%%%%%%%%%%%%%%%%%%%%%%%%%%%%%%%%%%%%%%

\begin{document}
    \maketitle
    \section*{Orthogonal Sets}
    A orthogonal set if a set where every vector is orthogonal to each other. A
    orthonormal set is a orthogonal set where all the vectors have length 1.

    \subsection*{Projection}
    The projection of a vector $v$ onto a vector $u$, written $proj_u(v)$ is a
    scalar multiple of $u$ and $v-proj_u(v)$ is orthogonal to $u$. If $W$ is a
    subspace, the projection of a vector $y$ onto $W$ is written as $proj_W(y)$.
    \begin{align*}
        proj_u (v) &= \frac{v\cdot u}{||u||^2} u \\
        u\cdot (v-proj_u(v)) &= u\cdot \Big(v - \frac{v\cdot u}{||u||^2} u\Big) \\
        &= u\cdot v - \frac{v\cdot u}{||u||^2} (u\cdot u) \\
        &= u\cdot v - v\cdot u \\
        &= \vec{0}
    \end{align*}

    Suppose that $W$ is a subspace of $\mathbb{R}^n$ with an orthogonal basis
    $v_1,\dots,v_k$. Then any $y\in \mathbb{R}^n$ can be written as $\hat{y}+z$
    where $\hat{y} \in W$ and $z\in W^\perp$.
    $\hat{y}=proj_{v1}(y)+\dots+proj_{vk}(y)$ It is very important that the basis is
    orthogonal or else this will not work.

    \subsection*{Gram-Schmidt process}
    Every subspace of $\mathbb{R}^n$ has an orthonormal basis. If
    $v_1,\dots,v_i$ are linearly independent and $W_i = span\{v_1,\dots,v_k\}$.
    A basis $B=\{u_1,\dots,u_k\}$ can be computed:
    \begin{align*}
        u_1 &= v_1 \\
        u_2 &= v_2 - proj_{W1}(v_2) \\
        u_3 &= v_3 - proj_{W2}(v_3) \\
        & \vdots \\
        u_k &= v_k - proj_{W(k-1)}(v_k)
    \end{align*}

    \subsection*{Example 1}
    Orthogonalize the basis and project $y = [2\ 1\ 3]^T$ onto $V$.
    \begin{align*}
        B = \{u, v\} &= \Bigg\{
        \begin{bmatrix}
            1 \\ 1 \\ 1
        \end{bmatrix},
        \begin{bmatrix}
            1 \\ 0 \\ 1
        \end{bmatrix}
        \Bigg\} \\
        proj_u(v) &= \frac{2}{3} \begin{bmatrix}
            1 \\ 1 \\ 1
        \end{bmatrix} \\
        v-proj_u(v) &= \begin{bmatrix}
            \frac{1}{3} \\ -\frac{2}{3} \\ \frac{1}{3}
        \end{bmatrix} \\
        V = \{v_1, v_2\} &= \Bigg\{
        \begin{bmatrix}
            1 \\ 1 \\ 1
        \end{bmatrix},
        \frac{1}{3} \begin{bmatrix}
            1 \\ -2 \\ 1
        \end{bmatrix}
        \Bigg\} \\
        proj_V(y) &= proj_{v1}(y) + proj_{v2}(y) \\
        &= \frac{6}{3} v_1 + \frac{1}{\frac{6}{9}} v_2 \\
        &= 2 v_1 + \frac{3}{2} v_2 \\
        &= 2 \begin{bmatrix}
            1 \\ 1 \\ 1
        \end{bmatrix}
        + \frac{3}{2} \frac{1}{3} \begin{bmatrix}
            1 \\ -2 \\ 1
        \end{bmatrix} \\
        &= 2 \begin{bmatrix}
            1 \\ 1 \\ 1
        \end{bmatrix}
        + \frac{1}{2} \begin{bmatrix}
            1 \\ -2 \\ 1
        \end{bmatrix} \\
        &= \begin{bmatrix}
            \frac{5}{2} \\\ 1 \\ \frac{5}{2}
        \end{bmatrix} \\
        z &= y-proj_V(y) \\
        &= \begin{bmatrix}
            2 \\ 1 \\ 3
        \end{bmatrix} - \begin{bmatrix}
            \frac{5}{2} \\\ 1 \\ \frac{5}{2}
        \end{bmatrix} \\
        &= \begin{bmatrix}
            -\frac{1}{2} \\ 0 \\ \frac{1}{2}
        \end{bmatrix} \\
        z\cdot v_1 &= 0 \\
        z\cdot v_2 &= 0
    \end{align*}

    \subsection*{Example 2}
    Apply the Gram-Schmidt process to the following basis
    \begin{align*}
        U = \Bigg\{
            \begin{bmatrix}
                1 \\ 1 \\ 1
            \end{bmatrix},
            \begin{bmatrix}
                -1 \\ 0 \\ 1
            \end{bmatrix},
            \begin{bmatrix}
                2 \\ 2 \\ 3
            \end{bmatrix}
        \Bigg\}
    \end{align*}
    $u_1$ and $u_2$ are already orthogonal, therefore $v_2 = u_2-proj_{v1}(u_2) = u_2$
    \begin{align*}
        v_3 &= u_3 - proj_{v1}(u_3) - proj_{v2}(u_3) \\
            &=
            \begin{bmatrix}
                2 \\ 2 \\ 3
            \end{bmatrix} - \frac{7}{3} \begin{bmatrix}
                1 \\ 1 \\ 1
            \end{bmatrix} - \frac{1}{2} \begin{bmatrix}
                -1 \\ 0 \\ 1
            \end{bmatrix} \\
            &= \renewcommand{\arraystretch}{1.4}\begin{bmatrix}
                \frac{1}{6} \\ -\frac{1}{3} \\ \frac{1}{6}
            \end{bmatrix}
    \end{align*}


\end{document}
