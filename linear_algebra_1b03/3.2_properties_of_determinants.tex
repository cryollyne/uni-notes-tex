%%%%%%%%%%%%%%%%%%%%%%%%%%%%% Define Article %%%%%%%%%%%%%%%%%%%%%%%%%%%%%%%%%%
\documentclass{article}
%%%%%%%%%%%%%%%%%%%%%%%%%%%%%%%%%%%%%%%%%%%%%%%%%%%%%%%%%%%%%%%%%%%%%%%%%%%%%%%

%%%%%%%%%%%%%%%%%%%%%%%%%%%%% Using Packages %%%%%%%%%%%%%%%%%%%%%%%%%%%%%%%%%%
\usepackage{geometry}
\usepackage{graphicx}
\usepackage{amssymb}
\usepackage{amsmath}
\usepackage{amsthm}
\usepackage{empheq}
\usepackage{mdframed}
\usepackage{booktabs}
\usepackage{lipsum}
\usepackage{graphicx}
\usepackage{color}
\usepackage{psfrag}
\usepackage{pgfplots}
\usepackage{bm}
%%%%%%%%%%%%%%%%%%%%%%%%%%%%%%%%%%%%%%%%%%%%%%%%%%%%%%%%%%%%%%%%%%%%%%%%%%%%%%%

% Other Settings

%%%%%%%%%%%%%%%%%%%%%%%%%% Page Setting %%%%%%%%%%%%%%%%%%%%%%%%%%%%%%%%%%%%%%%
\geometry{a4paper}

%%%%%%%%%%%%%%%%%%%%%%%%%% Define some useful colors %%%%%%%%%%%%%%%%%%%%%%%%%%
\definecolor{ocre}{RGB}{243,102,25}
\definecolor{mygray}{RGB}{243,243,244}
\definecolor{deepGreen}{RGB}{26,111,0}
\definecolor{shallowGreen}{RGB}{235,255,255}
\definecolor{deepBlue}{RGB}{61,124,222}
\definecolor{shallowBlue}{RGB}{235,249,255}
%%%%%%%%%%%%%%%%%%%%%%%%%%%%%%%%%%%%%%%%%%%%%%%%%%%%%%%%%%%%%%%%%%%%%%%%%%%%%%%

%%%%%%%%%%%%%%%%%%%%%%%%%% Define an orangebox command %%%%%%%%%%%%%%%%%%%%%%%%
\newcommand\orangebox[1]{\fcolorbox{ocre}{mygray}{\hspace{1em}#1\hspace{1em}}}
%%%%%%%%%%%%%%%%%%%%%%%%%%%%%%%%%%%%%%%%%%%%%%%%%%%%%%%%%%%%%%%%%%%%%%%%%%%%%%%

%%%%%%%%%%%%%%%%%%%%%%%%%%%% English Environments %%%%%%%%%%%%%%%%%%%%%%%%%%%%%
\newtheoremstyle{mytheoremstyle}{3pt}{3pt}{\normalfont}{0cm}{\rmfamily\bfseries}{}{1em}{{\color{black}\thmname{#1}~\thmnumber{#2}}\thmnote{\,--\,#3}}
\newtheoremstyle{myproblemstyle}{3pt}{3pt}{\normalfont}{0cm}{\rmfamily\bfseries}{}{1em}{{\color{black}\thmname{#1}~\thmnumber{#2}}\thmnote{\,--\,#3}}
\theoremstyle{mytheoremstyle}
\newmdtheoremenv[linewidth=1pt,backgroundcolor=shallowGreen,linecolor=deepGreen,leftmargin=0pt,innerleftmargin=20pt,innerrightmargin=20pt,]{theorem}{Theorem}[section]
\theoremstyle{mytheoremstyle}
\newmdtheoremenv[linewidth=1pt,backgroundcolor=shallowBlue,linecolor=deepBlue,leftmargin=0pt,innerleftmargin=20pt,innerrightmargin=20pt,]{definition}{Definition}[section]
\theoremstyle{myproblemstyle}
\newmdtheoremenv[linecolor=black,leftmargin=0pt,innerleftmargin=10pt,innerrightmargin=10pt,]{problem}{Problem}[section]
%%%%%%%%%%%%%%%%%%%%%%%%%%%%%%%%%%%%%%%%%%%%%%%%%%%%%%%%%%%%%%%%%%%%%%%%%%%%%%%

%%%%%%%%%%%%%%%%%%%%%%%%%%%%%%% Plotting Settings %%%%%%%%%%%%%%%%%%%%%%%%%%%%%
\usepgfplotslibrary{colorbrewer}
\pgfplotsset{width=8cm,compat=1.9}
%%%%%%%%%%%%%%%%%%%%%%%%%%%%%%%%%%%%%%%%%%%%%%%%%%%%%%%%%%%%%%%%%%%%%%%%%%%%%%%

%%%%%%%%%%%%%%%%%%%%%%%%%%%%%%% Title & Author %%%%%%%%%%%%%%%%%%%%%%%%%%%%%%%%
\title{Properties of Determinants}
\author{Patrick Chen}
\date{Oct 21, 2024}
%%%%%%%%%%%%%%%%%%%%%%%%%%%%%%%%%%%%%%%%%%%%%%%%%%%%%%%%%%%%%%%%%%%%%%%%%%%%%%%

\newcommand\adj{\text{adj}}

\begin{document}
    \maketitle
    \section*{Properties of Determinants}
    For $n\times n$ matrices $A$ and $B$ and scalars $k$:
    \begin{align*}
        \det(AB) &= \det(A)\det(B) \\
        \det(A)  &= \det(A^T) \\
        \det(kA) &= k^n\det(A) \\
        A^{-1} \text{exists }&\text{iif} \det(A)=0
    \end{align*}

    If the matrix $A$ is invertible, then there is some combination of elementary matrices
    that when left multiplies with $A$, will give the identity.
    \begin{align*}
        \det(E_k\dots E_2E_1A) = \det(I) \\
        \det(E_k)\dots\det(E_2)\det(E_1)\det(A) = 1
    \end{align*}
    Since the determinant of elementary matrices are all non-zero, the
    determinant of $A$ cannot be zero, thus $A$ is invertible iff $\det A\ne0$.

    \section*{Efficient Computations of Determinants}
    Upper triangular matrices are matrices where the only non-zero values are
    either on or above the main diagonal. Everything under the main diagonals is
    zero. Lower triangular matrices are the opposite. The determinant of a upper
    triangular or lower triangular matrix is the product of the diagonals. This
    is evident from applying the cofactor definition of the determinant with the
    rows with all zeros except for the main diagonal.

    Multiplying by a elementary matrix will modify it depending on what type of
    elementary matrix it is. Since the transpose of a matrix has the same
    determinant, it is possible to preform row operations on the transpose of
    the matrix. This corresponds with column operations on the original matrix.
    \begin{enumerate}
        \item If the row operation is interchanging two rows, then the
            determinant is negated.
            \begin{align*}
                \det(EA)=-\det(A)
            \end{align*}
        \item If the row operation is multiplying a row by $k$, then the
            determinant will be multiplied by k.
            \begin{align*}
                \det(EA)=k\det(A)
            \end{align*}
        \item If the row operation is adding or subtracting a multiple of
            another row, then the determinant will be unchanged.
            \begin{align*}
                \det(EA)=\det(A)
            \end{align*}
    \end{enumerate}

    \subsection*{Example}
    \begin{align*}
        &\det\begin{bmatrix}
            1 & -1 & 2 \\
            3 & 1 & 4 \\
            2 & 0 & 1
        \end{bmatrix} \\
        =&\det\begin{bmatrix}
            4 & 0 & 6 \\
            3 & 1 & 4 \\
            2 & 0 & 1
        \end{bmatrix} \\
        =&\det\begin{bmatrix}
            4 & 6 \\
            2 & 1
        \end{bmatrix} \\
        = -8
    \end{align*}

    \subsection*{Example 2}
    \begin{align*}
        \det \begin{bmatrix}
            2 & 1 & 4 \\
            1 & 0 & -1 \\
            3 & 2 & 0
        \end{bmatrix} \\
        = -\det \begin{bmatrix}
            1 & 0 & -1 \\
            2 & 1 & 4 \\
            3 & 2 & 0
        \end{bmatrix} \\
        = -\det \begin{bmatrix}
            1 & 0 & -1 \\
            0 & 1 & 6 \\
            0 & 2 & 3
        \end{bmatrix} \\
        = -\det \begin{bmatrix}
            1 & 0 & -1 \\
            0 & 1 & 6 \\
            0 & 0 & -9
        \end{bmatrix} \\
        = 9
    \end{align*}

    \section*{Linearity of Determinants}
    The determinant of a set matrix where one column is the input to the
    transformation and all other columns are fixed is a linear transformation.
    \begin{align*}
        T: \mathbb{R}^n \mapsto& \mathbb{R}\\
        T(x) =& \det(v_1,\dots,v_{j-1},x,v_{j+1},\dots,v_n) \\
        \text{where }& v_1\dots v_{j-1} \text{ is constant,} \\
                     & v_{j+1}\dots v_n \text{ is constant}
    \end{align*}

    Proof of linearity:
    \begin{align*}
        T(x+y)
        &= \det(v_1,\dots,v_{j-1},x+y,v_{j+1},\dots,v_n) \\
        &= \sum_{i=0}^n a_{ji} (-1)^{i+j} \det(A_{ij}) \\
        &= \sum_{i=0}^n (x_i+y_i) (-1)^{i+j} \det(A_{ij}) \\
        &= \sum_{i=0}^n \bigg(x_i(-1)^{i+j} \det(A_{ij})
            +y_i(-1)^{i+j} \det(A_{ij})\bigg) \\
        &= \bigg(\sum_{i=0}^n x_i(-1)^{i+j} \det(A_{ij})\bigg)
            +\bigg(\sum_{i=0}^ny_i(-1)^{i+j} \det(A_{ij})\bigg) \\
        &= \det(v_1,\dots,v_{j-1},x,v_{j+1},\dots,v_n)
            +\det(v_1,\dots,v_{j-1},y,v_{j+1},\dots,v_n) \\
        &= T(x)+T(y)
    \end{align*}

    \begin{align*}
        T(\lambda x) = & \det(v_1,\dots,v_{j-1},\lambda x,v_{j+1},\dots,v_n) \\
        =& \sum_{i=0}^n \lambda a_{ji} (-1)^{i+j} \det(A_{ij}) \\
        =& \lambda \sum_{i=0}^n a_{ji} (-1)^{i+j} \det(A_{ij}) \\
        =& \lambda \det(v_1,\dots,v_{j-1},x,v_{j+1},\dots,v_n) \\
        =& \lambda T(x)
    \end{align*}

    \subsection*{Example 3}
    \begin{align*}
        \det \begin{bmatrix}
            1 & 4 & 1 \\
            2 & 2 & 0 \\
            0 & 3 & 1
        \end{bmatrix} &=
        \det \begin{bmatrix}
            1 & 4 & 1 \\
            0 & 2 & 0 \\
            0 & 3 & 1
        \end{bmatrix} +
        \det \begin{bmatrix}
            0 & 4 & 1 \\
            2 & 2 & 0 \\
            0 & 3 & 1
        \end{bmatrix} \\
        &= \det \begin{bmatrix}
            2 & 0 \\
            3 & 1
        \end{bmatrix} + (-2) \det \begin{bmatrix}
            4 & 1 \\
            3 & 1
        \end{bmatrix} \\
        & = 2-2 \\
        &= 0
    \end{align*}

    \subsection*{Example 4}
    \begin{align*}
        \det \begin{bmatrix}
            a & d & g \\
            b & e & h \\
            c & f & i
        \end{bmatrix} = 
        \det \begin{bmatrix}
            a & d & g \\
            0 & e & h \\
            0 & f & i
        \end{bmatrix} + 
        \det \begin{bmatrix}
            0 & d & g \\
            b & e & h \\
            0 & f & i
        \end{bmatrix} +
        \det \begin{bmatrix}
            0 & d & g \\
            0 & e & h \\
            c & f & i
        \end{bmatrix} \\
        =\det \begin{bmatrix}
            a & d & g \\
            0 & 0 & h \\
            0 & 0 & i
        \end{bmatrix} + 
        \det \begin{bmatrix}
            0 & d & g \\
            b & 0 & h \\
            0 & 0 & i
        \end{bmatrix} +
        \det \begin{bmatrix}
            0 & d & g \\
            0 & 0 & h \\
            c & 0 & i
        \end{bmatrix} \\ +
        \det \begin{bmatrix}
            a & 0 & g \\
            0 & e & h \\
            0 & 0 & i
        \end{bmatrix} + 
        \det \begin{bmatrix}
            0 & 0 & g \\
            b & e & h \\
            0 & 0 & i
        \end{bmatrix} +
        \det \begin{bmatrix}
            0 & 0 & g \\
            0 & e & h \\
            c & 0 & i
        \end{bmatrix} \\ +
        \det \begin{bmatrix}
            a & 0 & g \\
            0 & 0 & h \\
            0 & f & i
        \end{bmatrix} + 
        \det \begin{bmatrix}
            0 & 0 & g \\
            b & 0 & h \\
            0 & f & i
        \end{bmatrix} +
        \det \begin{bmatrix}
            0 & 0 & g \\
            0 & 0 & h \\
            c & f & i
        \end{bmatrix}
    \end{align*}

    After repeating this, in general, all columns will have only one non-zero
    value. If a entire row in a matrix is zero, then the determinant will be
    zero. Evaluating the determinant, the first column will have $n$ choices for
    a non-zero determinant. The column will have $n-1$ choices for a non-zero
    determinant. In general the sequence will be $[n, n-1, n-2,\dots, 3, 2, 1]$.
    Since recursively splitting matrices with linearity will give every possible
    combination of columns, it will contain every matrix with non-zero
    determinants (as well as a lot of determinant zero matrices). This means
    that we need to sum up $n!$ values to find the determinant.

\end{document}
