%%%%%%%%%%%%%%%%%%%%%%%%%%%%% Define Article %%%%%%%%%%%%%%%%%%%%%%%%%%%%%%%%%%
\documentclass{article}
%%%%%%%%%%%%%%%%%%%%%%%%%%%%%%%%%%%%%%%%%%%%%%%%%%%%%%%%%%%%%%%%%%%%%%%%%%%%%%%

%%%%%%%%%%%%%%%%%%%%%%%%%%%%% Using Packages %%%%%%%%%%%%%%%%%%%%%%%%%%%%%%%%%%
\usepackage{geometry}
\usepackage{graphicx}
\usepackage{amssymb}
\usepackage{amsmath}
\usepackage{amsthm}
\usepackage{empheq}
\usepackage{mdframed}
\usepackage{booktabs}
\usepackage{lipsum}
\usepackage{graphicx}
\usepackage{color}
\usepackage{psfrag}
\usepackage{pgfplots}
\usepackage{bm}
%%%%%%%%%%%%%%%%%%%%%%%%%%%%%%%%%%%%%%%%%%%%%%%%%%%%%%%%%%%%%%%%%%%%%%%%%%%%%%%

% Other Settings

%%%%%%%%%%%%%%%%%%%%%%%%%% Page Setting %%%%%%%%%%%%%%%%%%%%%%%%%%%%%%%%%%%%%%%
\geometry{a4paper}

%%%%%%%%%%%%%%%%%%%%%%%%%% Define some useful colors %%%%%%%%%%%%%%%%%%%%%%%%%%
\definecolor{ocre}{RGB}{243,102,25}
\definecolor{mygray}{RGB}{243,243,244}
\definecolor{deepGreen}{RGB}{26,111,0}
\definecolor{shallowGreen}{RGB}{235,255,255}
\definecolor{deepBlue}{RGB}{61,124,222}
\definecolor{shallowBlue}{RGB}{235,249,255}
%%%%%%%%%%%%%%%%%%%%%%%%%%%%%%%%%%%%%%%%%%%%%%%%%%%%%%%%%%%%%%%%%%%%%%%%%%%%%%%

%%%%%%%%%%%%%%%%%%%%%%%%%% Define an orangebox command %%%%%%%%%%%%%%%%%%%%%%%%
\newcommand\orangebox[1]{\fcolorbox{ocre}{mygray}{\hspace{1em}#1\hspace{1em}}}
%%%%%%%%%%%%%%%%%%%%%%%%%%%%%%%%%%%%%%%%%%%%%%%%%%%%%%%%%%%%%%%%%%%%%%%%%%%%%%%

%%%%%%%%%%%%%%%%%%%%%%%%%%%% English Environments %%%%%%%%%%%%%%%%%%%%%%%%%%%%%
\newtheoremstyle{mytheoremstyle}{3pt}{3pt}{\normalfont}{0cm}{\rmfamily\bfseries}{}{1em}{{\color{black}\thmname{#1}~\thmnumber{#2}}\thmnote{\,--\,#3}}
\newtheoremstyle{myproblemstyle}{3pt}{3pt}{\normalfont}{0cm}{\rmfamily\bfseries}{}{1em}{{\color{black}\thmname{#1}~\thmnumber{#2}}\thmnote{\,--\,#3}}
\theoremstyle{mytheoremstyle}
\newmdtheoremenv[linewidth=1pt,backgroundcolor=shallowGreen,linecolor=deepGreen,leftmargin=0pt,innerleftmargin=20pt,innerrightmargin=20pt,]{theorem}{Theorem}[section]
\theoremstyle{mytheoremstyle}
\newmdtheoremenv[linewidth=1pt,backgroundcolor=shallowBlue,linecolor=deepBlue,leftmargin=0pt,innerleftmargin=20pt,innerrightmargin=20pt,]{definition}{Definition}[section]
\theoremstyle{myproblemstyle}
\newmdtheoremenv[linecolor=black,leftmargin=0pt,innerleftmargin=10pt,innerrightmargin=10pt,]{problem}{Problem}[section]
%%%%%%%%%%%%%%%%%%%%%%%%%%%%%%%%%%%%%%%%%%%%%%%%%%%%%%%%%%%%%%%%%%%%%%%%%%%%%%%

%%%%%%%%%%%%%%%%%%%%%%%%%%%%%%% Plotting Settings %%%%%%%%%%%%%%%%%%%%%%%%%%%%%
\usepgfplotslibrary{colorbrewer}
\pgfplotsset{width=8cm,compat=1.9}
%%%%%%%%%%%%%%%%%%%%%%%%%%%%%%%%%%%%%%%%%%%%%%%%%%%%%%%%%%%%%%%%%%%%%%%%%%%%%%%

%%%%%%%%%%%%%%%%%%%%%%%%%%%%%%% Title & Author %%%%%%%%%%%%%%%%%%%%%%%%%%%%%%%%
\title{Linear Independence and Basis}
\author{Patrick Chen}
\date{Nov 4, 2024}
%%%%%%%%%%%%%%%%%%%%%%%%%%%%%%%%%%%%%%%%%%%%%%%%%%%%%%%%%%%%%%%%%%%%%%%%%%%%%%%

\begin{document}
    \maketitle
    \section*{Linear Independence}
    If $V$ is a vector space, we say that $v_1,\dots,v_n\in V$ are linearly
    independent if the only solution to $c_1v_1+\dots+c_nv_n=0$ has $c_1,\dots
    c_n = 0$. For infinite dimensions, a set of vectors $B$ in $V$ is linearly
    independent if every finite subset of $B$ is linearly independent.

    The set $e_1,\dots,e_n$ is maximally linearly independent in $\mathbb{R}^n$
    because there is not anything that can be added to the set without causing
    it to become linearly dependent. $\mathbb{R}^n$ cannot be spanned by fewer
    than $n$ vectors.

    % A linear transformation with finite dimensions form $V$ to $V$ is onto if it
    % is one-to-one and one-to-one if it is onto. This is not true for linear
    % transformations with infinite dimensions.

    \section*{Basis for a Vector Space}
    A subset $B$ is said to be a basis for a vector space $V$ if $B$  is
    linearly independent and $B$ spans $V$. Every vector space has a basis. In
    $\mathbb{R}^n$, a $n\times n$ matrix $A$ is invertible if the columns of $A$
    form a basis.

    \subsection*{Spanning Set Theorem}
    if $V$ is a vector space and $H = span\{v_1,\dots,v_n\}$ then there is a
    subset $\{v_1,\dots,v_n\}$ which is a basis for $H$.

    To find this basis, we must filter the vectors that are not in span of all
    previous vectors. Suppose that after this process, the vectors are not
    linearly independent. This implies that there is some linear combination
    vectors that, when added, will equal zero and is not the trivial solution.
    By rearranging the equation, it is possible to show that there is some
    vector that is a linear combination of the other vectors and thus in the
    span of the other vectors. This implies that one of the vectors is a linear
    combination of the previous vectors. This contradicts with the algorithm
    used to produce these vectors so thus, the assumption that the vectors are
    linearly dependent is false. Below is an example of this.

    \begin{align*}
        c_1v_1 + c_3v_3 + c_5v_5 + c_8v_8 = 0 \\
        c_8 = -\frac{c_1}{c_8} v_1 -\frac{c_2}{c_8} v_3 -\frac{c_5}{c_8} v_5
    \end{align*}

    To finding the basis for the columnspace of matrix $A$, we compute the
    reduced row echelon form of $A$ and take the pivot columns of $A$.

    \subsection*{Rowspace}
    If $A$ is a $n\times m$ matrix, then the rowspace $row(A)$ is a subset of
    $\mathbb{R}^m$. Elementary row operations do not change the rowspace.
    Proving this for swapping and constant multiple is trivial.
    \begin{align*}
        r_i \leftarrow r_i + cr_j \\
        span\{r_1,\dots,r_i+cr_j,\dots,r_n\} \\
        (r_i+cr_j) \in span\{r_1,\dots,r_i+cr_j,\dots,r_n\} \\
        (r_i+cr_j) - cr_j \in span\{r_1,\dots,r_i+cr_j,\dots,r_n\} \\
        r_i \in span\{r_1,\dots,r_i+cr_j,\dots,r_n\}
    \end{align*}

    Since $r_i$ is still in the span after adding a multiple of a row, $r_i$ is
    still in the span and this row operation doesn't change the span of the
    rows. The basis for the rowspace of $A$ is the basis for the columnspace of
    $A^T$.

    \subsection*{Example 1}
    Prove that $B=\{1,x,x^2,x^3\dots\}$ is linearly independent.
    \begin{align*}
        c_0 + c_1 x + x_2 x^2 + \dots + c_n x^n = 0 \text{ for all values } x
    \end{align*}
    This means that for every $x$, this polynomial is equal to zero. Every
    polynomial of degree $n>0$ has at most $n$ roots. A degree $0$ polynomial
    can have infinite roots if it is the constant zero function. Since this
    equality must hold for every $x$, which there are infinite values of, the
    polynomial must be the zero polynomial with $c_0\dots c_n=0$.

    \subsection*{Example 2}
    The set $\{\sin x, \cos x\}$ is a subset of functions from $\mathbb{R}$ to
    $\mathbb{R}$. Prove that this set of functions is linearly independent.

    \begin{align*}
        c \sin x + d \cos x &= 0 \text{ for all values } x \\
        \sin x &= -\frac{d}{c} \cos x \\
    \end{align*}
    Since this is not true for all values $x$, sin and cos are not scalar
    multiples of each other and thus not linearly dependent.
\end{document}
