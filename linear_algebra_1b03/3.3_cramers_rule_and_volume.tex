%%%%%%%%%%%%%%%%%%%%%%%%%%%%% Define Article %%%%%%%%%%%%%%%%%%%%%%%%%%%%%%%%%%
\documentclass{article}
%%%%%%%%%%%%%%%%%%%%%%%%%%%%%%%%%%%%%%%%%%%%%%%%%%%%%%%%%%%%%%%%%%%%%%%%%%%%%%%

%%%%%%%%%%%%%%%%%%%%%%%%%%%%% Using Packages %%%%%%%%%%%%%%%%%%%%%%%%%%%%%%%%%%
\usepackage{geometry}
\usepackage{graphicx}
\usepackage{amssymb}
\usepackage{amsmath}
\usepackage{amsthm}
\usepackage{empheq}
\usepackage{mdframed}
\usepackage{booktabs}
\usepackage{lipsum}
\usepackage{graphicx}
\usepackage{color}
\usepackage{psfrag}
\usepackage{pgfplots}
\usepackage{bm}
%%%%%%%%%%%%%%%%%%%%%%%%%%%%%%%%%%%%%%%%%%%%%%%%%%%%%%%%%%%%%%%%%%%%%%%%%%%%%%%

% Other Settings

%%%%%%%%%%%%%%%%%%%%%%%%%% Page Setting %%%%%%%%%%%%%%%%%%%%%%%%%%%%%%%%%%%%%%%
\geometry{a4paper}

%%%%%%%%%%%%%%%%%%%%%%%%%% Define some useful colors %%%%%%%%%%%%%%%%%%%%%%%%%%
\definecolor{ocre}{RGB}{243,102,25}
\definecolor{mygray}{RGB}{243,243,244}
\definecolor{deepGreen}{RGB}{26,111,0}
\definecolor{shallowGreen}{RGB}{235,255,255}
\definecolor{deepBlue}{RGB}{61,124,222}
\definecolor{shallowBlue}{RGB}{235,249,255}
%%%%%%%%%%%%%%%%%%%%%%%%%%%%%%%%%%%%%%%%%%%%%%%%%%%%%%%%%%%%%%%%%%%%%%%%%%%%%%%

%%%%%%%%%%%%%%%%%%%%%%%%%% Define an orangebox command %%%%%%%%%%%%%%%%%%%%%%%%
\newcommand\orangebox[1]{\fcolorbox{ocre}{mygray}{\hspace{1em}#1\hspace{1em}}}
%%%%%%%%%%%%%%%%%%%%%%%%%%%%%%%%%%%%%%%%%%%%%%%%%%%%%%%%%%%%%%%%%%%%%%%%%%%%%%%

%%%%%%%%%%%%%%%%%%%%%%%%%%%% English Environments %%%%%%%%%%%%%%%%%%%%%%%%%%%%%
\newtheoremstyle{mytheoremstyle}{3pt}{3pt}{\normalfont}{0cm}{\rmfamily\bfseries}{}{1em}{{\color{black}\thmname{#1}~\thmnumber{#2}}\thmnote{\,--\,#3}}
\newtheoremstyle{myproblemstyle}{3pt}{3pt}{\normalfont}{0cm}{\rmfamily\bfseries}{}{1em}{{\color{black}\thmname{#1}~\thmnumber{#2}}\thmnote{\,--\,#3}}
\theoremstyle{mytheoremstyle}
\newmdtheoremenv[linewidth=1pt,backgroundcolor=shallowGreen,linecolor=deepGreen,leftmargin=0pt,innerleftmargin=20pt,innerrightmargin=20pt,]{theorem}{Theorem}[section]
\theoremstyle{mytheoremstyle}
\newmdtheoremenv[linewidth=1pt,backgroundcolor=shallowBlue,linecolor=deepBlue,leftmargin=0pt,innerleftmargin=20pt,innerrightmargin=20pt,]{definition}{Definition}[section]
\theoremstyle{myproblemstyle}
\newmdtheoremenv[linecolor=black,leftmargin=0pt,innerleftmargin=10pt,innerrightmargin=10pt,]{problem}{Problem}[section]
%%%%%%%%%%%%%%%%%%%%%%%%%%%%%%%%%%%%%%%%%%%%%%%%%%%%%%%%%%%%%%%%%%%%%%%%%%%%%%%

%%%%%%%%%%%%%%%%%%%%%%%%%%%%%%% Plotting Settings %%%%%%%%%%%%%%%%%%%%%%%%%%%%%
\usepgfplotslibrary{colorbrewer}
\pgfplotsset{width=8cm,compat=1.9}
%%%%%%%%%%%%%%%%%%%%%%%%%%%%%%%%%%%%%%%%%%%%%%%%%%%%%%%%%%%%%%%%%%%%%%%%%%%%%%%

%%%%%%%%%%%%%%%%%%%%%%%%%%%%%%% Title & Author %%%%%%%%%%%%%%%%%%%%%%%%%%%%%%%%
\title{Cramer's Rule and Volume}
\author{Patrick Chen}
\date{Oct 28, 2024}
%%%%%%%%%%%%%%%%%%%%%%%%%%%%%%%%%%%%%%%%%%%%%%%%%%%%%%%%%%%%%%%%%%%%%%%%%%%%%%%

\newcommand\adj{\text{adj}}

\begin{document}
    \maketitle
    \section*{Cramer's Rule}
    For an $n\times n$ matrix $A$ and any vector $b\in \mathbb{R}^n$, we define
    $\det(A)_i(b)$ as the determinant of the matrix $A$ with the ith column
    replaced by $b$. If $A$ is invertible, then the solution of $Ax=b$ is given
    by the following formula
    \begin{align*}
        x_i = \frac{\det(A)_i(b)}{\det(A)}
    \end{align*}

    We can use Cramer's rule to get a explicit formula for the inverse
    \begin{align*}
        \adj(A) = (C_{ij})^T \\
        A^{-1} = \frac{\adj(A)}{\det(A)}
    \end{align*}

    \section*{Example 1}
    \begin{align*}
        A = \begin{bmatrix}
            1  & 0 & 1 \\
            0  & 1 & 0 \\
            -1 & 1 & 1
        \end{bmatrix} && b=\begin{bmatrix}
            2 \\
            1 \\
            2
        \end{bmatrix}
    \end{align*}
    \begin{align*}
        \det(A) = 2 \\
        \det \begin{bmatrix}
            2  & 0 & 1 \\
            1  & 1 & 0 \\
            2 & 1 & 1
        \end{bmatrix} = 1 \\
        \det \begin{bmatrix}
            1  & 2 & 1 \\
            0  & 1 & 0 \\
            -1 & 2 & 1
        \end{bmatrix} = 2 \\
        \det \begin{bmatrix}
            1  & 0 & 2 \\
            0  & 1 & 1 \\
            -1 & 1 & 2
        \end{bmatrix} = 3
    \end{align*}
    \begin{align*}
        x_1 = 1/2 \\
        x_2 = 1 \\
        x_3 = 3/2
    \end{align*}

    \subsection*{Example 2}
    Compute $A^{-1}$ using Cramer's rule
    \begin{align*}
        A = \begin{bmatrix}
            1  & 0 & 1 \\
            0  & 1 & 0 \\
            -1 & 1 & 1
        \end{bmatrix} \\
        C_{11} = 1 \\
        C_{12} = 0 \\
        C_{13} = 1 \\
        C_{21} = 1 \\
        C_{22} = 2 \\
        C_{23} = -1 \\
        C_{31} = -1 \\
        C_{32} = 2 \\
        C_{33} = 1 \\
        \adj(A) = \begin{bmatrix}
            1 & 0 & 1 \\
            1 & 2 & -1 \\
            -1 & 0 & 1
        \end{bmatrix}^T \\
        = \begin{bmatrix}
            1 & 1 & -1 \\
            0 & 2 & 0 \\
            1 & -1 & 1
        \end{bmatrix} \\
        A^{-1} = \frac{\adj{A}}{\det{A}} \\
        = \frac{1}{2} \begin{bmatrix}
            1 & 1 & -1 \\
            0 & 2 & 0 \\
            1 & -1 & 1
        \end{bmatrix}
    \end{align*}

    \section*{Relation with area}

    Suppose that $S$ is a region in $\mathbb{R}^2$ with finite area and $T:
    \mathbb{R}^2 \mapsto \mathbb{R}^2$ is a linear transformation with a
    standard matrix $A$. The area of $T(S)$ is equal to $|\det A|$ times the
    area of $S$. The same is true for volumes in $\mathbb{R}^3$ and general
    n-volumes in $\mathbb{R}^n$.

\end{document}
