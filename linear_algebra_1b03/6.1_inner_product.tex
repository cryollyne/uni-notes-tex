%%%%%%%%%%%%%%%%%%%%%%%%%%%%% Define Article %%%%%%%%%%%%%%%%%%%%%%%%%%%%%%%%%%
\documentclass{article}
%%%%%%%%%%%%%%%%%%%%%%%%%%%%%%%%%%%%%%%%%%%%%%%%%%%%%%%%%%%%%%%%%%%%%%%%%%%%%%%

%%%%%%%%%%%%%%%%%%%%%%%%%%%%% Using Packages %%%%%%%%%%%%%%%%%%%%%%%%%%%%%%%%%%
\usepackage{geometry}
\usepackage{graphicx}
\usepackage{amssymb}
\usepackage{amsmath}
\usepackage{amsthm}
\usepackage{empheq}
\usepackage{mdframed}
\usepackage{booktabs}
\usepackage{lipsum}
\usepackage{graphicx}
\usepackage{color}
\usepackage{psfrag}
\usepackage{pgfplots}
\usepackage{bm}
%%%%%%%%%%%%%%%%%%%%%%%%%%%%%%%%%%%%%%%%%%%%%%%%%%%%%%%%%%%%%%%%%%%%%%%%%%%%%%%

% Other Settings

%%%%%%%%%%%%%%%%%%%%%%%%%% Page Setting %%%%%%%%%%%%%%%%%%%%%%%%%%%%%%%%%%%%%%%
\geometry{a4paper}

%%%%%%%%%%%%%%%%%%%%%%%%%% Define some useful colors %%%%%%%%%%%%%%%%%%%%%%%%%%
\definecolor{ocre}{RGB}{243,102,25}
\definecolor{mygray}{RGB}{243,243,244}
\definecolor{deepGreen}{RGB}{26,111,0}
\definecolor{shallowGreen}{RGB}{235,255,255}
\definecolor{deepBlue}{RGB}{61,124,222}
\definecolor{shallowBlue}{RGB}{235,249,255}
%%%%%%%%%%%%%%%%%%%%%%%%%%%%%%%%%%%%%%%%%%%%%%%%%%%%%%%%%%%%%%%%%%%%%%%%%%%%%%%

%%%%%%%%%%%%%%%%%%%%%%%%%% Define an orangebox command %%%%%%%%%%%%%%%%%%%%%%%%
\newcommand\orangebox[1]{\fcolorbox{ocre}{mygray}{\hspace{1em}#1\hspace{1em}}}
%%%%%%%%%%%%%%%%%%%%%%%%%%%%%%%%%%%%%%%%%%%%%%%%%%%%%%%%%%%%%%%%%%%%%%%%%%%%%%%

%%%%%%%%%%%%%%%%%%%%%%%%%%%% English Environments %%%%%%%%%%%%%%%%%%%%%%%%%%%%%
\newtheoremstyle{mytheoremstyle}{3pt}{3pt}{\normalfont}{0cm}{\rmfamily\bfseries}{}{1em}{{\color{black}\thmname{#1}~\thmnumber{#2}}\thmnote{\,--\,#3}}
\newtheoremstyle{myproblemstyle}{3pt}{3pt}{\normalfont}{0cm}{\rmfamily\bfseries}{}{1em}{{\color{black}\thmname{#1}~\thmnumber{#2}}\thmnote{\,--\,#3}}
\theoremstyle{mytheoremstyle}
\newmdtheoremenv[linewidth=1pt,backgroundcolor=shallowGreen,linecolor=deepGreen,leftmargin=0pt,innerleftmargin=20pt,innerrightmargin=20pt,]{theorem}{Theorem}[section]
\theoremstyle{mytheoremstyle}
\newmdtheoremenv[linewidth=1pt,backgroundcolor=shallowBlue,linecolor=deepBlue,leftmargin=0pt,innerleftmargin=20pt,innerrightmargin=20pt,]{definition}{Definition}[section]
\theoremstyle{myproblemstyle}
\newmdtheoremenv[linecolor=black,leftmargin=0pt,innerleftmargin=10pt,innerrightmargin=10pt,]{problem}{Problem}[section]
%%%%%%%%%%%%%%%%%%%%%%%%%%%%%%%%%%%%%%%%%%%%%%%%%%%%%%%%%%%%%%%%%%%%%%%%%%%%%%%

%%%%%%%%%%%%%%%%%%%%%%%%%%%%%%% Plotting Settings %%%%%%%%%%%%%%%%%%%%%%%%%%%%%
\usepgfplotslibrary{colorbrewer}
\pgfplotsset{width=8cm,compat=1.9}
%%%%%%%%%%%%%%%%%%%%%%%%%%%%%%%%%%%%%%%%%%%%%%%%%%%%%%%%%%%%%%%%%%%%%%%%%%%%%%%

%%%%%%%%%%%%%%%%%%%%%%%%%%%%%%% Title & Author %%%%%%%%%%%%%%%%%%%%%%%%%%%%%%%%
\title{Inner Product}
\author{Patrick Chen}
\date{Nov 28, 2024}
%%%%%%%%%%%%%%%%%%%%%%%%%%%%%%%%%%%%%%%%%%%%%%%%%%%%%%%%%%%%%%%%%%%%%%%%%%%%%%%

\begin{document}
    \maketitle
    \section*{Dot product}
    The dot product $u\cdot v$ is a special case of inner product $\langle
    u,v\rangle$ on $\mathbb{R}^n$.
    \begin{align*}
        u\cdot v = u^T v = \sum_n u_nv_n
    \end{align*}
    The dot product satisfies the following properties:
    \begin{enumerate}
        \item $u\cdot v = v\cdot u$
        \item $u\cdot (v+w) = u\cdot v + u\cdot w$
        \item $(cu)\cdot v = c(u\cdot v)$
        \item $u\cdot u \ge 0$
        \item $u\cdot u = 0$ iff $u=0$
    \end{enumerate}
    Properties (4) and(5) are very important for dot products.

    \section*{Length and Angle}
    Let $u,v\in\mathbb{R}^n$.
    \begin{itemize}
        \item $||u||$ represents the magnitude of $u$
        \item $r$ represents the distance between $u$ and $v$
        \item $\theta$ represents the angle between $u$ and $v$
    \end{itemize}
    \begin{align*}
        ||u|| &= \sqrt{u\cdot u} = \sqrt{u_1^2 +u_2^2 + \dots + u_n^2} \\
        r &= ||u-v|| \\
        \cos(\theta)  &= \frac{u\cdot v}{||u||\ ||v||} \\
    \end{align*}

    \subsection*{Orthogonality}
    Using cosine law
    \begin{align*}
        ||u||^2 + ||v||^2 - 2||u||||v||\cos(\theta)
        &= ||u-v||^2 \\
        &= (u-v)\cdot(u-v) \\
        &= (u\cdot u) - 2(u \cdot v) + (v\cdot v) \\
        ||u||^2 + ||v||^2 - 2||u||||v||\cos(\theta)
        &= ||u||^2  + ||v||^2 - 2(u\cdot v) \\
        -2u\cdot v &= -2 ||u|| ||v|| \cos(\theta) \\
        \frac{u\cdot v}{||u|| ||v||}&=  \cos(\theta)
    \end{align*}
    If $\theta = \frac{\pi}{2}$, then $\cos\theta$ is zero. Two vectors are orthogonal
    to each other if their dot products are zero.

    \subsection*{Set of orthogonal vectors}
    A set of vectors $\{v_1,\dots,v_n\}$ are orthogonal if all vectors in the
    set are non-zero and all pairs of vectors have a dot product of zero. Often
    the $e_1,\dots,e_n$ basis vectors are used as orthogonal vectors. A set of
    orthogonal vectors are always linearly independent.

    \subsection*{Proof of linear independence}
    Suppose there is a orthogonal set $\{v_1,\dots,v_k\}$ and $c_1v_1+\dots+c_kv_k = 0$
    \begin{align*}
        (c_1v_1+\dots+c_kv_k) \cdot v_1 &= 0 \cdot v_1 \\
        c_1(v_1\cdot v_1)+\dots+c_k(v_k\cdot v_1) &= 0 \\
        c_1||v_1||^2+0+\dots+0 &= 0 \\
        c_1||v_1||^2 &= 0 \\
        c_1 &= 0
    \end{align*}
    By using this same process with each of the vectors in the orthogonal set,
    all of the coefficients must be equal to zero and thus, orthogonal sets are
    linearly independent.

    \subsection*{Orthogonal Basis}
    The coordinates in an orthogonal basis with dot products. If $w\in
    span\{v_1,\dots,v_k\}$, then:
    \begin{align*}
        w &= c_1v_1 + \dots + c_kv_k \\
        w\cdot v_i &= c_1 (v_1\cdot v_i) + \dots + c_i(v_i\cdot v_i) + \dots + c_k(v_k\cdot v_i)\\
        w\cdot v_i &= c_1 (0) + \dots + c_i||v_i||^2 + \dots + c_k(0)\\
        w\cdot v_i &= c_i ||v_i||^2 \\
        \frac{w\cdot v_i}{||v_i||^2} &= c_i
    \end{align*}
    Often orthonormal basis are used. Since vectors in an orthonormal basis are
    both mutually orthogonal and unit length, $||v_i||^2 = 1$ and thus, the
    division is necessary.
    \begin{align*}
        c_i = w\cdot v_i
    \end{align*}

    \subsection*{Example 1}
    Find $[w]_B$
    \begin{align*}
        B = \Bigg\{
            \begin{bmatrix}
                1 \\ 0 \\ 1
            \end{bmatrix},
            \begin{bmatrix}
                -1 \\ 1 \\ 1
            \end{bmatrix},
            \begin{bmatrix}
                1 \\ 2 \\ -1
            \end{bmatrix}
       \Bigg\},\quad w = \begin{bmatrix}
        2 \\ 3 \\ 1
       \end{bmatrix}
    \end{align*}
    \begin{align*}
       ||v_1|| &= 2 \\
       ||v_2|| &= 3 \\
       ||v_3|| &= 6 \\
       \frac{w\cdot v_1}{||v_1||^2} &= \frac{3}{2} \\
       \frac{w\cdot v_2}{||v_2||^2} &= \frac{2}{3} \\
       \frac{w\cdot v_3}{||v_3||^2} &= \frac{7}{6} \\
       w &= \frac{3}{2} v_1 + \frac{2}{3} v_2 + \frac{7}{6} v_3 \\
       [w]_B &= \renewcommand{\arraystretch}{1.4}\begin{bmatrix}
            \frac{3}{2} \\ \frac{2}{3} \\ \frac{7}{6}
       \end{bmatrix}
    \end{align*}

    \section*{Orthogonal complement}
    Suppose that $W$ is a subspace of $\mathbb{R}^n$. The orthogonal complement
    of $W$, written $W^\perp$ is defined as all vectors that are orthogonal to
    $W$.
    \begin{align*}
        W^\perp &= \{ u\in\mathbb{R}^n : u\cdot w = 0 \text{ for all } w \in W\} \\
        W\cap W^\perp &= \{0\}
    \end{align*}

    The orthogonal complex is a subspace and as such is closed under addition
    and scalar multiplication. For any $w\in W$ and $u,v\in W^\perp$:
    \begin{align*}
        w \cdot (u+v) &= w\cdot u + w\cdot v \\
        &= 0 + 0 \\
        &= 0 \\
        cw &= w\cdot cu \\
        &= c(w\cdot u) \\
        &= c(0) \\
        &= 0
    \end{align*}
    The intersection of $W$ and $W^\perp$ is $\{\vec{0}\}$.
    \begin{align*}
        w &\in W \\
        w &\in W^\perp \\
        w \cdot w &= 0 \\
        \therefore w &= \vec{0}
    \end{align*}
    The vector space $W$ is contained inside the orthogonal computed of the
    orthogonal complex of $W$ If $w\in W$ and $u\in W^\perp$, $w\cdot u = 0$
    therefore $w\in (W^\perp)^\perp$.
    \begin{align*}
        W \subseteq (W^\perp)^\perp
    \end{align*}

\end{document}
