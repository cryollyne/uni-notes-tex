%%%%%%%%%%%%%%%%%%%%%%%%%%%%% Define Article %%%%%%%%%%%%%%%%%%%%%%%%%%%%%%%%%%
\documentclass{article}
%%%%%%%%%%%%%%%%%%%%%%%%%%%%%%%%%%%%%%%%%%%%%%%%%%%%%%%%%%%%%%%%%%%%%%%%%%%%%%%

%%%%%%%%%%%%%%%%%%%%%%%%%%%%% Using Packages %%%%%%%%%%%%%%%%%%%%%%%%%%%%%%%%%%
\usepackage{geometry}
\usepackage{graphicx}
\usepackage{amssymb}
\usepackage{amsmath}
\usepackage{amsthm}
\usepackage{empheq}
\usepackage{mdframed}
\usepackage{booktabs}
\usepackage{lipsum}
\usepackage{graphicx}
\usepackage{color}
\usepackage{psfrag}
\usepackage{pgfplots}
\usepackage{bm}
%%%%%%%%%%%%%%%%%%%%%%%%%%%%%%%%%%%%%%%%%%%%%%%%%%%%%%%%%%%%%%%%%%%%%%%%%%%%%%%

% Other Settings

%%%%%%%%%%%%%%%%%%%%%%%%%% Page Setting %%%%%%%%%%%%%%%%%%%%%%%%%%%%%%%%%%%%%%%
\geometry{a4paper}

%%%%%%%%%%%%%%%%%%%%%%%%%% Define some useful colors %%%%%%%%%%%%%%%%%%%%%%%%%%
\definecolor{ocre}{RGB}{243,102,25}
\definecolor{mygray}{RGB}{243,243,244}
\definecolor{deepGreen}{RGB}{26,111,0}
\definecolor{shallowGreen}{RGB}{235,255,255}
\definecolor{deepBlue}{RGB}{61,124,222}
\definecolor{shallowBlue}{RGB}{235,249,255}
%%%%%%%%%%%%%%%%%%%%%%%%%%%%%%%%%%%%%%%%%%%%%%%%%%%%%%%%%%%%%%%%%%%%%%%%%%%%%%%

%%%%%%%%%%%%%%%%%%%%%%%%%% Define an orangebox command %%%%%%%%%%%%%%%%%%%%%%%%
\newcommand\orangebox[1]{\fcolorbox{ocre}{mygray}{\hspace{1em}#1\hspace{1em}}}
%%%%%%%%%%%%%%%%%%%%%%%%%%%%%%%%%%%%%%%%%%%%%%%%%%%%%%%%%%%%%%%%%%%%%%%%%%%%%%%

%%%%%%%%%%%%%%%%%%%%%%%%%%%% English Environments %%%%%%%%%%%%%%%%%%%%%%%%%%%%%
\newtheoremstyle{mytheoremstyle}{3pt}{3pt}{\normalfont}{0cm}{\rmfamily\bfseries}{}{1em}{{\color{black}\thmname{#1}~\thmnumber{#2}}\thmnote{\,--\,#3}}
\newtheoremstyle{myproblemstyle}{3pt}{3pt}{\normalfont}{0cm}{\rmfamily\bfseries}{}{1em}{{\color{black}\thmname{#1}~\thmnumber{#2}}\thmnote{\,--\,#3}}
\theoremstyle{mytheoremstyle}
\newmdtheoremenv[linewidth=1pt,backgroundcolor=shallowGreen,linecolor=deepGreen,leftmargin=0pt,innerleftmargin=20pt,innerrightmargin=20pt,]{theorem}{Theorem}[section]
\theoremstyle{mytheoremstyle}
\newmdtheoremenv[linewidth=1pt,backgroundcolor=shallowBlue,linecolor=deepBlue,leftmargin=0pt,innerleftmargin=20pt,innerrightmargin=20pt,]{definition}{Definition}[section]
\theoremstyle{myproblemstyle}
\newmdtheoremenv[linecolor=black,leftmargin=0pt,innerleftmargin=10pt,innerrightmargin=10pt,]{problem}{Problem}[section]
%%%%%%%%%%%%%%%%%%%%%%%%%%%%%%%%%%%%%%%%%%%%%%%%%%%%%%%%%%%%%%%%%%%%%%%%%%%%%%%

%%%%%%%%%%%%%%%%%%%%%%%%%%%%%%% Plotting Settings %%%%%%%%%%%%%%%%%%%%%%%%%%%%%
\usepgfplotslibrary{colorbrewer}
\pgfplotsset{width=8cm,compat=1.9}
%%%%%%%%%%%%%%%%%%%%%%%%%%%%%%%%%%%%%%%%%%%%%%%%%%%%%%%%%%%%%%%%%%%%%%%%%%%%%%%

%%%%%%%%%%%%%%%%%%%%%%%%%%%%%%% Title & Author %%%%%%%%%%%%%%%%%%%%%%%%%%%%%%%%
\title{Subspaces}
\author{Patrick Chen}
\date{Oct 31, 2024}
%%%%%%%%%%%%%%%%%%%%%%%%%%%%%%%%%%%%%%%%%%%%%%%%%%%%%%%%%%%%%%%%%%%%%%%%%%%%%%%

\begin{document}
    \maketitle
    The dimension of a vector space is the amount of free variables that exist.

    \section*{Nullspace}
    For a $n\times m$ matrix $A$, the set of solutions for $Ax=0$ is a subspace
    of $\mathbb{R}^m$ called the nullspace of $A$, written as $Nul(A)$. The
    nullspace is a subspace, because it is closed under addition and scalar
    multiplication.
    \begin{align*}
        Ax &= 0 \\
        Ay &= 0 \\
        A(x + y) &= 0 + 0 \\
        Ax + Ay &= 0
    \end{align*}
    \begin{align*}
        Ax &= 0 \\
        A(cx) &= cAx \\
              &= c0 \\
              &= 0
    \end{align*}

    \section*{Span as a Subspace}
    The span of $v_1\dots v_k\in \mathbb{R}^n$ is a subspace. Zero is in the
    span of this set because all the vectors can have coefficients zero.
    \begin{align*}
        (c_1v_1 +\dots+c_kv_k) + (d_1v_1 +\dots+d_kv_k)
        &= (c_1+d_1)v_1 +\dots+(c_1+d_1)v_k \\
        c(d_1v_1+\dots d_kv_k) &= (cd_1)+\dots+(cd_k)v_k
    \end{align*}

    \section*{Columnspace}
    For a $n\times m$ matrix $A$, the column space of $A$ is a subspace of
    $\mathbb{R}^n$ equal to the span of all the columns of $A$. Column space is
    written as $Col(A)$.

    \section*{Transformations Between Vector Spaces}
    Suppose that $V$ and $W$ are vector spaces and $T: V \mapsto W$ is a
    function from $V$ to $W$ The transformation $T$ is called a linear
    transformation from $V$ to $W$ if for all $u,v\in V$ and $c\in \mathbb{R}$,
    \begin{align*}
        T(u+v) &= T(u) + T(v) \\
        T(cu) &= cT(u)
    \end{align*}

    \section*{Kernel and Range}
    Suppose that $T: V \mapsto W$ is a linear transformation. The kernel ($ker$)
    of a linear transformation is a generalization of the nullspace of a matrix for
    linear transformations that are not matrix transformations. Likewise, the
    range ($rng$) is the generalization for the column space. When working with
    matrices, the nullspace is the same as the kernel and the column space is
    the same as the range.
    \begin{align*}
        ker(T) &= \{ v\in V\ |\ T(v) = 0 \} \\
        rng(T) &= \{T(v)\ |\ v \in V \}
    \end{align*}

    The kernel and range of a linear transformation is a subspace.
    \begin{align*}
        0           &\in ker(T) \\
        u,v         &\in ker(T) \\
        T(u) + T(v) &= 0 + 0 \\
                    &= 0 \in ker(T) \\
        cT(u)       &= c\cdot 0 \\
                    &= 0 \in ker(T)
    \end{align*}
    \begin{align*}
        0           &\in rng(T) \\
        T(u),T(v)   &\in rng(T) \\
        T(u) + T(v) &= T(u+v) \in rng(T) \\
        cT(u) &= T(cu) \in rng(T)
    \end{align*}

    \subsection*{Example 1}
    Find the null space of the following matrix.
    \begin{align*}
        A = \begin{bmatrix}
            2 & 0 & 1 & 3 & 0 & 4 \\
            1 & 2 & 1 & 0 & 1 & 5 \\
            0 & 0 & 1 & 0 & 1 & 1
        \end{bmatrix} \\
        rref(A) = \begin{bmatrix}
            1 & 0 & 0 & \frac{3}{2} & -\frac{1}{2} & \frac{3}{2} \\
            0 & 1 & 0 & -\frac{3}{4} & \frac{1}{4} & \frac{5}{4} \\
            0 & 0 & 1 & 0 & 1 & 1
        \end{bmatrix} \\
        \begin{matrix}
            1 x_1 &+ 0 x_2 &+ 0 x_3 &+ \frac{3}{2}  x_4 &- \frac{1}{2}  x_5 &+ \frac{3}{2} x_6 & =0 \\
            0 x_1 &+ 1 x_2 &+ 0 x_3 &- \frac{3}{4}  x_4 &+ \frac{1}{4}  x_5 &+ \frac{5}{4} x_6 & =0 \\
            0 x_1 &+ 0 x_2 &+ 1 x_3 &+ 0            x_4 &+ 1            x_5 &+ 1           x_6 & =0
        \end{matrix} \\
        x = \begin{bmatrix}
            -\frac{3}{2} x_4 + \frac{1}{2} x_5 - \frac{3}{2} x_6 \\
            \frac{3}{4} x_4 -\frac{1}{4} x_5 - \frac{5}{4} x_6 \\
            -x_5 -x_6 \\
            x_4 \\
            x_5 \\
            x_6
        \end{bmatrix} \\
        = \begin{bmatrix}
            -\frac{3}{2} \\
            \frac{3}{4} \\
            0 \\
            1 \\
            0 \\
            0
        \end{bmatrix} x_4
        + \begin{bmatrix}
            \frac{1}{2} \\
            -\frac{1}{4} \\
            -1 \\
            0 \\
            1 \\
            0
        \end{bmatrix} x_5
        + \begin{bmatrix}
            -\frac{3}{2} \\
            -\frac{5}{4} \\
            -1 \\
            0 \\
            0 \\
            1
        \end{bmatrix} x_6
    \end{align*}

    \subsection*{Example 2}
    Suppose that $V$ is the vector space of all real-valued polynomials.
    \begin{align*}
        T: V \mapsto V \\
        T(p) = xp
    \end{align*}
    \begin{align*}
        p,q&\in V \\
        T(p+q) &= x(p+q) \\
        &= xp + xq \\
        &= T(p) + T(q) \\
        T(cp) &= x(cp) \\
        &= cxp \\
        &= cT(p)
    \end{align*}

    \begin{align*}
        ker(T) &= \{0\} \\
        rng(T) &= \text{all polynomials with a zero constant term}
    \end{align*}

    \begin{align*}
        D: V \mapsto V \\
        D(p) = \frac{dp}{dx}
    \end{align*}
    \begin{align*}
        D(p+q) &= \frac{d(p+q)}{dx} \\
        &= \frac{dp}{dx} + \frac{dq}{dx} \\
        &= D(p) + D(q) \\
        D(cp) &= \frac{d(cp)}{dx} \\
        &= c\frac{dp}{dx} \\
        &= cD(p)
    \end{align*}

    \begin{align*}
        ker(D) &= \text{constants} \\
        rng(D) &= V
    \end{align*}

\end{document}
