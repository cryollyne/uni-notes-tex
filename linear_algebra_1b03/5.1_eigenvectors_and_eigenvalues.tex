%%%%%%%%%%%%%%%%%%%%%%%%%%%%% Define Article %%%%%%%%%%%%%%%%%%%%%%%%%%%%%%%%%%
\documentclass{article}
%%%%%%%%%%%%%%%%%%%%%%%%%%%%%%%%%%%%%%%%%%%%%%%%%%%%%%%%%%%%%%%%%%%%%%%%%%%%%%%

%%%%%%%%%%%%%%%%%%%%%%%%%%%%% Using Packages %%%%%%%%%%%%%%%%%%%%%%%%%%%%%%%%%%
\usepackage{geometry}
\usepackage{graphicx}
\usepackage{amssymb}
\usepackage{amsmath}
\usepackage{amsthm}
\usepackage{empheq}
\usepackage{mdframed}
\usepackage{booktabs}
\usepackage{lipsum}
\usepackage{graphicx}
\usepackage{color}
\usepackage{psfrag}
\usepackage{pgfplots}
\usepackage{bm}
%%%%%%%%%%%%%%%%%%%%%%%%%%%%%%%%%%%%%%%%%%%%%%%%%%%%%%%%%%%%%%%%%%%%%%%%%%%%%%%

% Other Settings

%%%%%%%%%%%%%%%%%%%%%%%%%% Page Setting %%%%%%%%%%%%%%%%%%%%%%%%%%%%%%%%%%%%%%%
\geometry{a4paper}

%%%%%%%%%%%%%%%%%%%%%%%%%% Define some useful colors %%%%%%%%%%%%%%%%%%%%%%%%%%
\definecolor{ocre}{RGB}{243,102,25}
\definecolor{mygray}{RGB}{243,243,244}
\definecolor{deepGreen}{RGB}{26,111,0}
\definecolor{shallowGreen}{RGB}{235,255,255}
\definecolor{deepBlue}{RGB}{61,124,222}
\definecolor{shallowBlue}{RGB}{235,249,255}
%%%%%%%%%%%%%%%%%%%%%%%%%%%%%%%%%%%%%%%%%%%%%%%%%%%%%%%%%%%%%%%%%%%%%%%%%%%%%%%

%%%%%%%%%%%%%%%%%%%%%%%%%% Define an orangebox command %%%%%%%%%%%%%%%%%%%%%%%%
\newcommand\orangebox[1]{\fcolorbox{ocre}{mygray}{\hspace{1em}#1\hspace{1em}}}
%%%%%%%%%%%%%%%%%%%%%%%%%%%%%%%%%%%%%%%%%%%%%%%%%%%%%%%%%%%%%%%%%%%%%%%%%%%%%%%

%%%%%%%%%%%%%%%%%%%%%%%%%%%% English Environments %%%%%%%%%%%%%%%%%%%%%%%%%%%%%
\newtheoremstyle{mytheoremstyle}{3pt}{3pt}{\normalfont}{0cm}{\rmfamily\bfseries}{}{1em}{{\color{black}\thmname{#1}~\thmnumber{#2}}\thmnote{\,--\,#3}}
\newtheoremstyle{myproblemstyle}{3pt}{3pt}{\normalfont}{0cm}{\rmfamily\bfseries}{}{1em}{{\color{black}\thmname{#1}~\thmnumber{#2}}\thmnote{\,--\,#3}}
\theoremstyle{mytheoremstyle}
\newmdtheoremenv[linewidth=1pt,backgroundcolor=shallowGreen,linecolor=deepGreen,leftmargin=0pt,innerleftmargin=20pt,innerrightmargin=20pt,]{theorem}{Theorem}[section]
\theoremstyle{mytheoremstyle}
\newmdtheoremenv[linewidth=1pt,backgroundcolor=shallowBlue,linecolor=deepBlue,leftmargin=0pt,innerleftmargin=20pt,innerrightmargin=20pt,]{definition}{Definition}[section]
\theoremstyle{myproblemstyle}
\newmdtheoremenv[linecolor=black,leftmargin=0pt,innerleftmargin=10pt,innerrightmargin=10pt,]{problem}{Problem}[section]
%%%%%%%%%%%%%%%%%%%%%%%%%%%%%%%%%%%%%%%%%%%%%%%%%%%%%%%%%%%%%%%%%%%%%%%%%%%%%%%

%%%%%%%%%%%%%%%%%%%%%%%%%%%%%%% Plotting Settings %%%%%%%%%%%%%%%%%%%%%%%%%%%%%
\usepgfplotslibrary{colorbrewer}
\pgfplotsset{width=8cm,compat=1.9}
%%%%%%%%%%%%%%%%%%%%%%%%%%%%%%%%%%%%%%%%%%%%%%%%%%%%%%%%%%%%%%%%%%%%%%%%%%%%%%%

%%%%%%%%%%%%%%%%%%%%%%%%%%%%%%% Title & Author %%%%%%%%%%%%%%%%%%%%%%%%%%%%%%%%
\title{Eigenvectors and Eigenvalues}
\author{Patrick Chen}
\date{Nov 14, 2024}
%%%%%%%%%%%%%%%%%%%%%%%%%%%%%%%%%%%%%%%%%%%%%%%%%%%%%%%%%%%%%%%%%%%%%%%%%%%%%%%

\begin{document}
    \maketitle
    For a $n\times n$ matrix $A$ and a non-zero vector $u\in \mathbb{R}^n$ where
    $Au=\lambda u$, then $\lambda$ is a eigenvalue and $u$ is a eigenvector. The
    $\lambda$-eigenspace of $A$ is the set of all eigenvectors with eigenvalues
    of $\lambda$. The eigenspace is a subspace of $\mathbb{R}^n$ equal to the
    nullspace of $A-\lambda I$. This means that if a eigenvector is scaled, it
    will still be an eigenvector.

    \begin{align*}
        Au = \lambda u \\
        Au - \lambda u = 0 \\
        (A-\lambda I)u = 0
    \end{align*}
    Since $u\ne \vec{0}$, the determinant of $A-\lambda I$ must be zero.

    \subsection*{Example 1}
    Check if $6$ is a eigenvalue of the following matrix.
    \begin{align*}
        A &= \begin{bmatrix}
            1 & 3 & 2 \\
            2 & 1 & 3 \\
            3 & 2 & 1
        \end{bmatrix} \\
        Au &= 6u \\
        Au-6u &= 0 \\
        (A-6I)u &= 0 \\
        A-6I &= \begin{bmatrix}
            -5 & 3 & 2 \\
            2 & -5 & 3 \\
            3 & 2 & -5
        \end{bmatrix} \\
        rref(A-6I) &= \begin{bmatrix}
            1 & 0 & -1 \\
            0 & 1 & -1 \\
            0 & 0 & 0
        \end{bmatrix} \\
        u &= \begin{bmatrix}
            x_3 \\ x_3 \\ x_3
        \end{bmatrix} \\
        u &= x_3 \begin{bmatrix}
            1 \\ 1 \\ 1
        \end{bmatrix}
    \end{align*}

    \subsection*{Example 2}
    $\lambda=2$ is a eigenvalue
    \begin{align*}
        A &= \begin{bmatrix}
            1 & 1 & 0 \\
            1 & 1 & 0 \\
            0 & 0 & 2
        \end{bmatrix} \\
        A-2I &= \begin{bmatrix}
            -1 & 1 & 0 \\
            1 & -1 & 0 \\
            0 & 0 & 0
        \end{bmatrix} \\
        rref(A-2I) &= \begin{bmatrix}
            1 & -1 & 0 \\
            0 & 0 & 0 \\
            0 & 0 & 0
        \end{bmatrix} \\
        u &= x_1 \begin{bmatrix}
            1 \\ 1 \\ 0
        \end{bmatrix} + x_2 \begin{bmatrix}
            0 \\ 0 \\ 1
        \end{bmatrix}
    \end{align*}

\end{document}
