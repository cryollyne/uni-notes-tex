%%%%%%%%%%%%%%%%%%%%%%%%%%%%% Define Article %%%%%%%%%%%%%%%%%%%%%%%%%%%%%%%%%%
\documentclass{article}
%%%%%%%%%%%%%%%%%%%%%%%%%%%%%%%%%%%%%%%%%%%%%%%%%%%%%%%%%%%%%%%%%%%%%%%%%%%%%%%

%%%%%%%%%%%%%%%%%%%%%%%%%%%%% Using Packages %%%%%%%%%%%%%%%%%%%%%%%%%%%%%%%%%%
\usepackage{geometry}
\usepackage{graphicx}
\usepackage{amssymb}
\usepackage{amsmath}
\usepackage{amsthm}
\usepackage{empheq}
\usepackage{mdframed}
\usepackage{booktabs}
\usepackage{lipsum}
\usepackage{graphicx}
\usepackage{color}
\usepackage{psfrag}
\usepackage{pgfplots}
\usepackage{bm}
%%%%%%%%%%%%%%%%%%%%%%%%%%%%%%%%%%%%%%%%%%%%%%%%%%%%%%%%%%%%%%%%%%%%%%%%%%%%%%%

% Other Settings

%%%%%%%%%%%%%%%%%%%%%%%%%% Page Setting %%%%%%%%%%%%%%%%%%%%%%%%%%%%%%%%%%%%%%%
\geometry{a4paper}

%%%%%%%%%%%%%%%%%%%%%%%%%% Define some useful colors %%%%%%%%%%%%%%%%%%%%%%%%%%
\definecolor{ocre}{RGB}{243,102,25}
\definecolor{mygray}{RGB}{243,243,244}
\definecolor{deepGreen}{RGB}{26,111,0}
\definecolor{shallowGreen}{RGB}{235,255,255}
\definecolor{deepBlue}{RGB}{61,124,222}
\definecolor{shallowBlue}{RGB}{235,249,255}
%%%%%%%%%%%%%%%%%%%%%%%%%%%%%%%%%%%%%%%%%%%%%%%%%%%%%%%%%%%%%%%%%%%%%%%%%%%%%%%

%%%%%%%%%%%%%%%%%%%%%%%%%% Define an orangebox command %%%%%%%%%%%%%%%%%%%%%%%%
\newcommand\orangebox[1]{\fcolorbox{ocre}{mygray}{\hspace{1em}#1\hspace{1em}}}
%%%%%%%%%%%%%%%%%%%%%%%%%%%%%%%%%%%%%%%%%%%%%%%%%%%%%%%%%%%%%%%%%%%%%%%%%%%%%%%

%%%%%%%%%%%%%%%%%%%%%%%%%%%% English Environments %%%%%%%%%%%%%%%%%%%%%%%%%%%%%
\newtheoremstyle{mytheoremstyle}{3pt}{3pt}{\normalfont}{0cm}{\rmfamily\bfseries}{}{1em}{{\color{black}\thmname{#1}~\thmnumber{#2}}\thmnote{\,--\,#3}}
\newtheoremstyle{myproblemstyle}{3pt}{3pt}{\normalfont}{0cm}{\rmfamily\bfseries}{}{1em}{{\color{black}\thmname{#1}~\thmnumber{#2}}\thmnote{\,--\,#3}}
\theoremstyle{mytheoremstyle}
\newmdtheoremenv[linewidth=1pt,backgroundcolor=shallowGreen,linecolor=deepGreen,leftmargin=0pt,innerleftmargin=20pt,innerrightmargin=20pt,]{theorem}{Theorem}[section]
\theoremstyle{mytheoremstyle}
\newmdtheoremenv[linewidth=1pt,backgroundcolor=shallowBlue,linecolor=deepBlue,leftmargin=0pt,innerleftmargin=20pt,innerrightmargin=20pt,]{definition}{Definition}[section]
\theoremstyle{myproblemstyle}
\newmdtheoremenv[linecolor=black,leftmargin=0pt,innerleftmargin=10pt,innerrightmargin=10pt,]{problem}{Problem}[section]
%%%%%%%%%%%%%%%%%%%%%%%%%%%%%%%%%%%%%%%%%%%%%%%%%%%%%%%%%%%%%%%%%%%%%%%%%%%%%%%

%%%%%%%%%%%%%%%%%%%%%%%%%%%%%%% Plotting Settings %%%%%%%%%%%%%%%%%%%%%%%%%%%%%
\usepgfplotslibrary{colorbrewer}
\pgfplotsset{width=8cm,compat=1.9}
%%%%%%%%%%%%%%%%%%%%%%%%%%%%%%%%%%%%%%%%%%%%%%%%%%%%%%%%%%%%%%%%%%%%%%%%%%%%%%%

%%%%%%%%%%%%%%%%%%%%%%%%%%%%%%% Title & Author %%%%%%%%%%%%%%%%%%%%%%%%%%%%%%%%
\title{Vectors}
\author{Patrick Chen}
\date{Sept 10, 2024}
%%%%%%%%%%%%%%%%%%%%%%%%%%%%%%%%%%%%%%%%%%%%%%%%%%%%%%%%%%%%%%%%%%%%%%%%%%%%%%%

\begin{document}
    \maketitle
    \subsection*{Vector Addition}
    A $n\times 1$ matrix is called a column vector. The set of all vectors with
    n entries is called $\mathbb{R}^n$. Vectors in $\mathbb{R}^n$ can be added
    by adding component wise. Geometrically, two vectors added together is the
    resultant of putting the vectors tip-to-tail.
    \begin{equation*}
        \begin{bmatrix}
            a_1 \\ a_2 \\ \vdots \\ a_n
        \end{bmatrix} +
        \begin{bmatrix}
            b_1 \\ b_2 \\ \vdots \\ b_n
        \end{bmatrix} =
        \begin{bmatrix}
            a_1 + b_1 \\ a_2 + b_2 \\ \vdots \\ a_n + b_n
        \end{bmatrix}
    \end{equation*}

    \begin{itemize}
        \item Vectors in $\mathbb{R}^2$ can be thought of as a point in 2D
        \item Vectors in $\mathbb{R}^3$ can be thought of as a point in 3D
        \item Vectors in higher dimensions cannot be visualized, but can still
            store useful information
    \end{itemize}

    \subsection*{Scalar Vector Multiplication}
    Multiplying a vector by a scalar will multiply every entry in the vector by
    the scalar.
    \begin{equation*}
        \lambda
        \begin{bmatrix}
            a_1 \\ a_2 \\ \vdots \\ a_n
        \end{bmatrix} = 
        \begin{bmatrix}
            \lambda a_1 \\ \lambda a_2 \\ \vdots \\ \lambda a_n
        \end{bmatrix}
    \end{equation*}

    \subsection*{Linear Combinations}
    When scalar multiples of vectors are added, it is called a linear
    combination of vectors:
    \begin{equation*}
        \lambda_1 v_1 + \lambda_2 v_2 + \dots + \lambda_n v_n
    \end{equation*}

    When any two vectors form a plane, any point on that plane can be expressed
    as a linear combination of those vectors. The span of some list of vectors
    is the set of all linear combination of those vectors. We can check of some
    vector $b$ is in the span of some list of vectors $v_1,v_2,\dots,v_n$ by
    checking if the augmented matrix $\begin{bmatrix} v_1 & v_2 & \dots & v_n &
    b \end{bmatrix}$ is consistent.
    \end{document}
