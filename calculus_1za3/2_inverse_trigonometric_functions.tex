%%%%%%%%%%%%%%%%%%%%%%%%%%%%% Define Article %%%%%%%%%%%%%%%%%%%%%%%%%%%%%%%%%%
\documentclass{article}
%%%%%%%%%%%%%%%%%%%%%%%%%%%%%%%%%%%%%%%%%%%%%%%%%%%%%%%%%%%%%%%%%%%%%%%%%%%%%%%

%%%%%%%%%%%%%%%%%%%%%%%%%%%%% Using Packages %%%%%%%%%%%%%%%%%%%%%%%%%%%%%%%%%%
\usepackage{geometry}
\usepackage{graphicx}
\usepackage{amssymb}
\usepackage{amsmath}
\usepackage{amsthm}
\usepackage{empheq}
\usepackage{mdframed}
\usepackage{booktabs}
\usepackage{lipsum}
\usepackage{graphicx}
\usepackage{color}
\usepackage{psfrag}
\usepackage{pgfplots}
\usepackage{bm}
%%%%%%%%%%%%%%%%%%%%%%%%%%%%%%%%%%%%%%%%%%%%%%%%%%%%%%%%%%%%%%%%%%%%%%%%%%%%%%%

% Other Settings

%%%%%%%%%%%%%%%%%%%%%%%%%% Page Setting %%%%%%%%%%%%%%%%%%%%%%%%%%%%%%%%%%%%%%%
\geometry{a4paper}

%%%%%%%%%%%%%%%%%%%%%%%%%% Define some useful colors %%%%%%%%%%%%%%%%%%%%%%%%%%
\definecolor{ocre}{RGB}{243,102,25}
\definecolor{mygray}{RGB}{243,243,244}
\definecolor{deepGreen}{RGB}{26,111,0}
\definecolor{shallowGreen}{RGB}{235,255,255}
\definecolor{deepBlue}{RGB}{61,124,222}
\definecolor{shallowBlue}{RGB}{235,249,255}
%%%%%%%%%%%%%%%%%%%%%%%%%%%%%%%%%%%%%%%%%%%%%%%%%%%%%%%%%%%%%%%%%%%%%%%%%%%%%%%

%%%%%%%%%%%%%%%%%%%%%%%%%% Define an orangebox command %%%%%%%%%%%%%%%%%%%%%%%%
\newcommand\orangebox[1]{\fcolorbox{ocre}{mygray}{\hspace{1em}#1\hspace{1em}}}
%%%%%%%%%%%%%%%%%%%%%%%%%%%%%%%%%%%%%%%%%%%%%%%%%%%%%%%%%%%%%%%%%%%%%%%%%%%%%%%

%%%%%%%%%%%%%%%%%%%%%%%%%%%% English Environments %%%%%%%%%%%%%%%%%%%%%%%%%%%%%
\newtheoremstyle{mytheoremstyle}{3pt}{3pt}{\normalfont}{0cm}{\rmfamily\bfseries}{}{1em}{{\color{black}\thmname{#1}~\thmnumber{#2}}\thmnote{\,--\,#3}}
\newtheoremstyle{myproblemstyle}{3pt}{3pt}{\normalfont}{0cm}{\rmfamily\bfseries}{}{1em}{{\color{black}\thmname{#1}~\thmnumber{#2}}\thmnote{\,--\,#3}}
\theoremstyle{mytheoremstyle}
\newmdtheoremenv[linewidth=1pt,backgroundcolor=shallowGreen,linecolor=deepGreen,leftmargin=0pt,innerleftmargin=20pt,innerrightmargin=20pt,]{theorem}{Theorem}[section]
\theoremstyle{mytheoremstyle}
\newmdtheoremenv[linewidth=1pt,backgroundcolor=shallowBlue,linecolor=deepBlue,leftmargin=0pt,innerleftmargin=20pt,innerrightmargin=20pt,]{definition}{Definition}[section]
\theoremstyle{myproblemstyle}
\newmdtheoremenv[linecolor=black,leftmargin=0pt,innerleftmargin=10pt,innerrightmargin=10pt,]{problem}{Problem}[section]
%%%%%%%%%%%%%%%%%%%%%%%%%%%%%%%%%%%%%%%%%%%%%%%%%%%%%%%%%%%%%%%%%%%%%%%%%%%%%%%

%%%%%%%%%%%%%%%%%%%%%%%%%%%%%%% Plotting Settings %%%%%%%%%%%%%%%%%%%%%%%%%%%%%
\usepgfplotslibrary{colorbrewer}
\pgfplotsset{width=8cm,compat=1.9}
%%%%%%%%%%%%%%%%%%%%%%%%%%%%%%%%%%%%%%%%%%%%%%%%%%%%%%%%%%%%%%%%%%%%%%%%%%%%%%%

%%%%%%%%%%%%%%%%%%%%%%%%%%%%%%% Title & Author %%%%%%%%%%%%%%%%%%%%%%%%%%%%%%%%
\title{Inverse Trigonometric Functions}
\author{Patrick Chen}
\date{Sept 9, 2024}
%%%%%%%%%%%%%%%%%%%%%%%%%%%%%%%%%%%%%%%%%%%%%%%%%%%%%%%%%%%%%%%%%%%%%%%%%%%%%%%

\begin{document}
    \maketitle
    \section*{Domain and Range}
    The domain is the set of all points that results in a valid result when
    mapped through a function. For example, in the function $f(x)= \frac{1}{x}$,
    the input $0$ does not produce a valid result. The range is the set of all
    possible outputs a function can produce.

    Tips:
    \begin{enumerate}
        \item If there is a fraction, the denominator shouldn't be zero.

        \item If there is a square root (or any even root), the input of the
            square root should be greater than or equal to zero.

        \item In exponentials, the range is strictly greater than zero

        \item If there is a logarithm, the input must be strictly greater than
            zero.
    \end{enumerate}

    \subsection*{Example}
    \begin{align*}
        f(x)&=\tan(x) \\
            &= \frac{\sin(x)}{\cos(x)} \\
        \therefore \cos(x)&\ne 0 \\
        \Rightarrow x &\ne 2k\pi + \frac{\pi}{2}, 2k\pi + \frac{3\pi}{2} \\
                      & \text{where } k \in \mathbb{Z}
    \end{align*}

    \section*{Interval notation}
    A square bracket means that the interval is inclusive while a round bracket
    means that a interval is exclusive. Round and square brackets can be mixed
    to represent a interval that is closed on one side but open on another.

    \begin{align*}
        x\in [a,b] \Leftrightarrow a \le x \le b \\
        x\in (a,b) \Leftrightarrow a < x < b
    \end{align*}

    \section*{Inverse Trigonometric Functions}
    The sine function is not one-to-one; however, if we restrict the domain of
    sine to $[ -\frac{\pi}{2}, \frac{\pi}{2} ]$, it becomes a one-to-one
    function with the range $[-1,1]$. The inverse of the sine function is
    $sin^{-1}(y) = x$ or $\arcsin(y)= x$

    For cosine, a similar thing can be done by restricting the domain to
    $[0,\pi]$. $\tan(x)$ is also restricted to $[ -\frac{\pi}{2},
    \frac{\pi}{2}]$

    \begin{equation*}
        \begin{matrix}
            f(x) & \text{Domain} & \text{Range} \\
            \sin & [ -\frac{\pi}{2}, \frac{\pi}{2} ] & [-1,1] \\
            \cos & [ 0, \pi ] & [-1,1] \\
            \tan & [ -\frac{\pi}{2}, \frac{\pi}{2} ] & (-\infty, \infty)
        \end{matrix}
    \end{equation*}

\end{document}
