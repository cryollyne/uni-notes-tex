%%%%%%%%%%%%%%%%%%%%%%%%%%%%% Define Article %%%%%%%%%%%%%%%%%%%%%%%%%%%%%%%%%%
\documentclass{article}
%%%%%%%%%%%%%%%%%%%%%%%%%%%%%%%%%%%%%%%%%%%%%%%%%%%%%%%%%%%%%%%%%%%%%%%%%%%%%%%

%%%%%%%%%%%%%%%%%%%%%%%%%%%%% Using Packages %%%%%%%%%%%%%%%%%%%%%%%%%%%%%%%%%%
\usepackage{geometry}
\usepackage{graphicx}
\usepackage{amssymb}
\usepackage{amsmath}
\usepackage{amsthm}
\usepackage{empheq}
\usepackage{mdframed}
\usepackage{booktabs}
\usepackage{lipsum}
\usepackage{graphicx}
\usepackage{color}
\usepackage{psfrag}
\usepackage{pgfplots}
\usepackage{bm}
%%%%%%%%%%%%%%%%%%%%%%%%%%%%%%%%%%%%%%%%%%%%%%%%%%%%%%%%%%%%%%%%%%%%%%%%%%%%%%%

% Other Settings

%%%%%%%%%%%%%%%%%%%%%%%%%% Page Setting %%%%%%%%%%%%%%%%%%%%%%%%%%%%%%%%%%%%%%%
\geometry{a4paper}

%%%%%%%%%%%%%%%%%%%%%%%%%% Define some useful colors %%%%%%%%%%%%%%%%%%%%%%%%%%
\definecolor{ocre}{RGB}{243,102,25}
\definecolor{mygray}{RGB}{243,243,244}
\definecolor{deepGreen}{RGB}{26,111,0}
\definecolor{shallowGreen}{RGB}{235,255,255}
\definecolor{deepBlue}{RGB}{61,124,222}
\definecolor{shallowBlue}{RGB}{235,249,255}
%%%%%%%%%%%%%%%%%%%%%%%%%%%%%%%%%%%%%%%%%%%%%%%%%%%%%%%%%%%%%%%%%%%%%%%%%%%%%%%

%%%%%%%%%%%%%%%%%%%%%%%%%% Define an orangebox command %%%%%%%%%%%%%%%%%%%%%%%%
\newcommand\orangebox[1]{\fcolorbox{ocre}{mygray}{\hspace{1em}#1\hspace{1em}}}
%%%%%%%%%%%%%%%%%%%%%%%%%%%%%%%%%%%%%%%%%%%%%%%%%%%%%%%%%%%%%%%%%%%%%%%%%%%%%%%

%%%%%%%%%%%%%%%%%%%%%%%%%%%% English Environments %%%%%%%%%%%%%%%%%%%%%%%%%%%%%
\newtheoremstyle{mytheoremstyle}{3pt}{3pt}{\normalfont}{0cm}{\rmfamily\bfseries}{}{1em}{{\color{black}\thmname{#1}~\thmnumber{#2}}\thmnote{\,--\,#3}}
\newtheoremstyle{myproblemstyle}{3pt}{3pt}{\normalfont}{0cm}{\rmfamily\bfseries}{}{1em}{{\color{black}\thmname{#1}~\thmnumber{#2}}\thmnote{\,--\,#3}}
\theoremstyle{mytheoremstyle}
\newmdtheoremenv[linewidth=1pt,backgroundcolor=shallowGreen,linecolor=deepGreen,leftmargin=0pt,innerleftmargin=20pt,innerrightmargin=20pt,]{theorem}{Theorem}[section]
\theoremstyle{mytheoremstyle}
\newmdtheoremenv[linewidth=1pt,backgroundcolor=shallowBlue,linecolor=deepBlue,leftmargin=0pt,innerleftmargin=20pt,innerrightmargin=20pt,]{definition}{Definition}[section]
\theoremstyle{myproblemstyle}
\newmdtheoremenv[linecolor=black,leftmargin=0pt,innerleftmargin=10pt,innerrightmargin=10pt,]{problem}{Problem}[section]
%%%%%%%%%%%%%%%%%%%%%%%%%%%%%%%%%%%%%%%%%%%%%%%%%%%%%%%%%%%%%%%%%%%%%%%%%%%%%%%

%%%%%%%%%%%%%%%%%%%%%%%%%%%%%%% Plotting Settings %%%%%%%%%%%%%%%%%%%%%%%%%%%%%
\usepgfplotslibrary{colorbrewer}
\pgfplotsset{width=8cm,compat=1.9}
%%%%%%%%%%%%%%%%%%%%%%%%%%%%%%%%%%%%%%%%%%%%%%%%%%%%%%%%%%%%%%%%%%%%%%%%%%%%%%%

%%%%%%%%%%%%%%%%%%%%%%%%%%%%%%% Title & Author %%%%%%%%%%%%%%%%%%%%%%%%%%%%%%%%
\title{Arc Length}
\author{Patrick Chen}
\date{Dec 2, 2024}
%%%%%%%%%%%%%%%%%%%%%%%%%%%%%%%%%%%%%%%%%%%%%%%%%%%%%%%%%%%%%%%%%%%%%%%%%%%%%%%

\begin{document}
    \maketitle
    The length of a curve can be approximated by sampling it and connecting the
    points. The length is the limit as the samples go to infinity.
    \begin{align*}
        L_{line} &= \sqrt{(P_{ix} - P_{(i-1)x})^2 + (P_{iy} - P_{(i-1)y})^2} \\
        L_{line} &= \sqrt{(x_i-x_{i-1})^2 + (f(x_i) - f(x_{i-1}))^2} \\
        L &= \lim_{n\to \infty} \sum_{i=1}^{n} \sqrt{(x_i-x_{i-1})^2 + (f(x_i) - f(x_{i-1}))^2} \\
        &= \lim_{n\to \infty} \sum_{i=1}^{n} \sqrt{(\Delta x^2 (1+\frac{(f(x_i)-f(x_{i-1}))}{\Delta x}))} \\
        &= \lim_{n\to \infty} \sum_{i=1}^{n} \sqrt{\Delta x^2 (1 + f'(x_i)^2)} \\
        &= \lim_{n\to \infty} \sum_{i=1}^{n} \Delta x\sqrt{ (1 + f'(x_i)^2)} \\
    \end{align*}
    From this, the arc length of a curve is:
    \begin{align*}
        L = \int_{a}^{b} \sqrt{1+f'(x)^2} \ dx
    \end{align*}

    \subsection*{Example 1}
    Find the arc length of $x^{\frac{3}{2}}$ on the interval $[0,28]$.
    \begin{align*}
        \int_{0}^{28} \sqrt{1+ (\frac{3}{2} x^{\frac{1}{2}})^2} \ dx
        &= \int_{0}^{28} \sqrt{1+ \frac{9}{4} x} \ dx \\
        u &= 1+ \frac{9}{4} x \\
        du &= \frac{9}{4} dx \\
        I &= \frac{4}{9} \int_{1}^{64} u^{1/2} \ du \\
          &= \frac{8}{27} u^{\frac{3}{2}} \ \Big|_{1}^{64} \\
          &= \frac{8}{27} (64)^{\frac{3}{2}} - \frac{8}{27} (1)^{\frac{3}{2}} \\
          &= \frac{8}{27} 512 - \frac{8}{27} \\
          &= \frac{488}{27}
    \end{align*}

    \subsection*{Example 2}
    Find the arc length of $x^2-\frac{1}{8} \ln x$ on the interval $[1, e]$.
    \begin{align*}
        \int_{1}^{e} \sqrt{1+(2x-\frac{1}{8x})^2} \ dx
        &= \int_{1}^{e} \sqrt{1 + 4x^2 -\frac{1}{2} + \frac{1}{64x^2}} \ dx \\
        &= \int_{1}^{e} \sqrt{\frac{(1+16x^2)^2}{64x^2}} \ dx \\
        &= \int_{1}^{e} \frac{1+16x^2}{8x} \ dx \\
        &= \int_{1}^{e} \frac{1}{8x} + 2x \ dx \\
        &= \frac{1}{8} \ln x + x^2 \ \Big|_{1}^{e} \\
        &= (\frac{1}{8} \ln e + e^2) -(\frac{1}{8} \ln 1 + 1^2) \\
        &= (\frac{1}{8} + e^2) - 1 \\
        &= e^2 - \frac{7}{8}
    \end{align*}

    \subsection*{Example 3}
    Find the arc length of $g(x)=\int_{\pi/4}^{x} \sqrt{sec^8(t) -1} \ dt$ on
    the interval $[ \frac{\pi}{4,} \frac{\pi}{3} ]$.
    \begin{align*}
        \int_{\frac{\pi}{4}}^{\frac{\pi}{3}} \sqrt{1+(\sqrt{\sec^8(t) -1})^2} \ dx
        &= \int_{\frac{\pi}{4}}^{\frac{\pi}{3}} \sqrt{1+\sec^8(t) -1} \ dx \\
        &= \int_{\frac{\pi}{4}}^{\frac{\pi}{3}} \sec^4(t)\ dx \\
        &= \int_{\frac{\pi}{4}}^{\frac{\pi}{3}} \sec^2 x \sec^2 x\ dx \\
        &= \int_{\frac{\pi}{4}}^{\frac{\pi}{3}} (tan^2 x + 1) \sec^2 x\ dx \\
        u &= \tan x \\
        du &= \sec^2 x \\
        &= \int_{1}^{\sqrt{3}} u^2 + 1 \ du \\
        &= \frac{1}{3} u^3 + u \ \Big|_{1}^{\sqrt{3}} \\
        &= (\frac{1}{3} (\sqrt{3})^3 + \sqrt{3}) - (\frac{1}{3} (1)^3 + 1) \\
        &= (\sqrt{3} + \sqrt{3}) - \frac{4}{3} \\
        &= 2\sqrt{3} - \frac{4}{3}
    \end{align*}



\end{document}
