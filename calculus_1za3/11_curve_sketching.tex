%%%%%%%%%%%%%%%%%%%%%%%%%%%%% Define Article %%%%%%%%%%%%%%%%%%%%%%%%%%%%%%%%%%
\documentclass{article}
%%%%%%%%%%%%%%%%%%%%%%%%%%%%%%%%%%%%%%%%%%%%%%%%%%%%%%%%%%%%%%%%%%%%%%%%%%%%%%%

%%%%%%%%%%%%%%%%%%%%%%%%%%%%% Using Packages %%%%%%%%%%%%%%%%%%%%%%%%%%%%%%%%%%
\usepackage{geometry}
\usepackage{graphicx}
\usepackage{amssymb}
\usepackage{amsmath}
\usepackage{amsthm}
\usepackage{empheq}
\usepackage{mdframed}
\usepackage{booktabs}
\usepackage{lipsum}
\usepackage{graphicx}
\usepackage{color}
\usepackage{psfrag}
\usepackage{pgfplots}
\usepackage{bm}
%%%%%%%%%%%%%%%%%%%%%%%%%%%%%%%%%%%%%%%%%%%%%%%%%%%%%%%%%%%%%%%%%%%%%%%%%%%%%%%

% Other Settings

%%%%%%%%%%%%%%%%%%%%%%%%%% Page Setting %%%%%%%%%%%%%%%%%%%%%%%%%%%%%%%%%%%%%%%
\geometry{a4paper}

%%%%%%%%%%%%%%%%%%%%%%%%%% Define some useful colors %%%%%%%%%%%%%%%%%%%%%%%%%%
\definecolor{ocre}{RGB}{243,102,25}
\definecolor{mygray}{RGB}{243,243,244}
\definecolor{deepGreen}{RGB}{26,111,0}
\definecolor{shallowGreen}{RGB}{235,255,255}
\definecolor{deepBlue}{RGB}{61,124,222}
\definecolor{shallowBlue}{RGB}{235,249,255}
%%%%%%%%%%%%%%%%%%%%%%%%%%%%%%%%%%%%%%%%%%%%%%%%%%%%%%%%%%%%%%%%%%%%%%%%%%%%%%%

%%%%%%%%%%%%%%%%%%%%%%%%%% Define an orangebox command %%%%%%%%%%%%%%%%%%%%%%%%
\newcommand\orangebox[1]{\fcolorbox{ocre}{mygray}{\hspace{1em}#1\hspace{1em}}}
%%%%%%%%%%%%%%%%%%%%%%%%%%%%%%%%%%%%%%%%%%%%%%%%%%%%%%%%%%%%%%%%%%%%%%%%%%%%%%%

%%%%%%%%%%%%%%%%%%%%%%%%%%%% English Environments %%%%%%%%%%%%%%%%%%%%%%%%%%%%%
\newtheoremstyle{mytheoremstyle}{3pt}{3pt}{\normalfont}{0cm}{\rmfamily\bfseries}{}{1em}{{\color{black}\thmname{#1}~\thmnumber{#2}}\thmnote{\,--\,#3}}
\newtheoremstyle{myproblemstyle}{3pt}{3pt}{\normalfont}{0cm}{\rmfamily\bfseries}{}{1em}{{\color{black}\thmname{#1}~\thmnumber{#2}}\thmnote{\,--\,#3}}
\theoremstyle{mytheoremstyle}
\newmdtheoremenv[linewidth=1pt,backgroundcolor=shallowGreen,linecolor=deepGreen,leftmargin=0pt,innerleftmargin=20pt,innerrightmargin=20pt,]{theorem}{Theorem}[section]
\theoremstyle{mytheoremstyle}
\newmdtheoremenv[linewidth=1pt,backgroundcolor=shallowBlue,linecolor=deepBlue,leftmargin=0pt,innerleftmargin=20pt,innerrightmargin=20pt,]{definition}{Definition}[section]
\theoremstyle{myproblemstyle}
\newmdtheoremenv[linecolor=black,leftmargin=0pt,innerleftmargin=10pt,innerrightmargin=10pt,]{problem}{Problem}[section]
%%%%%%%%%%%%%%%%%%%%%%%%%%%%%%%%%%%%%%%%%%%%%%%%%%%%%%%%%%%%%%%%%%%%%%%%%%%%%%%

%%%%%%%%%%%%%%%%%%%%%%%%%%%%%%% Plotting Settings %%%%%%%%%%%%%%%%%%%%%%%%%%%%%
\usepgfplotslibrary{colorbrewer}
\pgfplotsset{width=8cm,compat=1.9}
%%%%%%%%%%%%%%%%%%%%%%%%%%%%%%%%%%%%%%%%%%%%%%%%%%%%%%%%%%%%%%%%%%%%%%%%%%%%%%%

%%%%%%%%%%%%%%%%%%%%%%%%%%%%%%% Title & Author %%%%%%%%%%%%%%%%%%%%%%%%%%%%%%%%
\title{Curve sketching}
\author{Patrick Chen}
\date{Oct 23, 2024}
%%%%%%%%%%%%%%%%%%%%%%%%%%%%%%%%%%%%%%%%%%%%%%%%%%%%%%%%%%%%%%%%%%%%%%%%%%%%%%%

\newcommand\lH{\stackrel{H}{=}}

\begin{document}
    \maketitle
    \section*{Example}
    \begin{align*}
        xe^{\frac{1}{x}}
    \end{align*}
    \begin{enumerate}
        \item Check the domain of the function.
            \begin{align*}
                y_1=x && D=\mathbb{R} \\
                y_2= \frac{1}{x} && D=\{x|x\in \mathbb{R}, x\ne 0\} \\
                y_3= e^x && D=\mathbb{R} \\
                y_4= e^{\frac{1}{x}} && D=\{x|x\in\mathbb{R}, x\ne 0\} \\
                y  = xe^{\frac{1}{x}} && D=\{x|x\in\mathbb{R}, x\ne 0\}
            \end{align*}

        \item Find the $x$ and $y$ intercepts.
            \begin{align*}
                f(0)
            \end{align*}
            Since $x\ne 0$, there is no y-intercept.
            \begin{align*}
                xe^{\frac{1}{x}} = 0
            \end{align*}
            Since $x\ne 0$ and $e^x$ is never zero, there is no x-intercept

        \item Symmetry
            \begin{itemize}
                \item Even functions are function that are symmetric about the
                    y-axis. $f(-x)=f(x)$

                \item Odd functions are functions that are an symmetric about
                    the origin. It can also be thought of as being the same
                    graph when rotated half a revolution.
                    $f(-x)=-f(x)$
            \end{itemize}

            \begin{align*}
                f(-x) = -xe^{-\frac{1}{x}} \\
                f(-x) \ne f(x) \\
                f(-x) \ne -f(x) \\
            \end{align*}
            $f(x)$ is neither even nor odd.

        \item Periodic
            \begin{itemize}
                \item A function is periodic if $f(x+p)=f(x)$ for all $x$ in the
                    domain, where $p$ is a positive constant. The smallest value
                    $p$ satisfying this is called the period.
            \end{itemize}

            \begin{align*}
                f(x+p) = (x+p)e^{\frac{1}{x+p}}
            \end{align*}
            This function is not periodic because no such $p$ exists.

        \item Find the asymptotes
            \begin{itemize}
                \item Horizontal asymptotes occur when the limit of a function
                    as the input approaches infinite remains finite.  If the
                    limit approaches infinity or negative infinity, then there
                    is no horizontal asymptotes.
                    \begin{align*}
                        \lim_{x\to \infty} f(x&) = L \\
                        \text{or} \\
                        \lim_{x\to -\infty} f(x&) = L
                    \end{align*}

                \item Vertical asymptotes are where the limit as the function
                    approaches some value is infinity or negative infinity
                    \begin{align*}
                        \lim_{x\to a^+} f(x) &= \pm\infty \\
                        \text{or} \\
                        \lim_{x\to a^-} f(x) &= \pm\infty
                    \end{align*}
            \end{itemize}

            Horizontal asymptotes of $xe^{\frac{1}{x}}$
            \begin{align*}
                &\lim_{x\to \infty} xe^{\frac{1}{x}} \\
                =& \infty\cdot e^{\frac{1}{\infty}} \\
                =& \infty\cdot e^{0} \\
                =& \infty\cdot 1 \\
                =& \infty
            \end{align*}
            \begin{align*}
                &\lim_{x\to -\infty} xe^{\frac{1}{x}} \\
                =& -\infty\cdot e^{\frac{1}{-\infty}} \\
                =& -\infty\cdot e^{0} \\
                =& -\infty\cdot 1 \\
                =& -\infty
            \end{align*}
            For vertical asymptotes of $xe^{\frac{1}{x}}$, we can check the restrictions
            in the domain of the function.
            \begin{align*}
                \lim_{x\to 0^+} xe^{\frac{1}{x}}
                &= \lim_{x\to 0^+} \frac{e^{\frac{1}{x}}}{\frac{1}{x}} \\
                &\lH\lim_{x\to 0^+} \frac{-\frac{1}{x^2}e^{\frac{1}{x}}}{-\frac{1}{x^2}} \\
                &= \lim_{x\to 0^+} e^{\frac{1}{x}} = e^{\frac{1}{0}} = e^\infty = \infty
            \end{align*}
            \begin{align*}
                \lim_{x\to 0^-} xe^{\frac{1}{x}}
                &= \lim_{x\to 0^-} \frac{e^{\frac{1}{x}}}{\frac{1}{x}} \\
                &\lH\lim_{x\to 0^-} \frac{-\frac{1}{x^2}e^{\frac{1}{x}}}{-\frac{1}{x^2}} \\
                &= \lim_{x\to 0^-} e^{\frac{1}{x}} = e^{\frac{1}{-0}} = e^{-\infty} = 0
            \end{align*}

        \item Intervals of increase or decrease and local maximums and minimums
            \begin{align*}
                f'(x)&=  e^{\frac{1}{x}} + x(-\frac{1}{x^2} e^{\frac{1}{x}}) \\
                     &=  e^{-\frac{1}{x}} (1-\frac{1}{x}) \\
                0    &=  e^{-\frac{1}{x}} (1-\frac{1}{x}) \\
                x &= 1
            \end{align*}

            \begin{center}
                \begin{tabular}[c]{l|l|l|l|l|l|l|l}
                    \hline
                    & $x<0$ & $x=0$ & $0<x<1$ & $x=1$ & $1<x$\\
                    \hline
                    && $m=\infty$ && local min &\\
                    f' & + & DNE & - & 0 & + \\
                    \hline
                \end{tabular}
            \end{center}

        \item Concavity and inflection points
            \begin{align*}
                f''(x) = \frac{1}{x^3} e^{\frac{1}{x}}
            \end{align*}

            \begin{center}
                \begin{tabular}[c]{l|l|l|l|l|l|l|l}
                    \hline
                    & $x<0$ & $x=0$ & $0<x<1$ & $x=1$ & $1<x$\\
                    \hline
                    && $m=\infty$ && local min & \\
                    f'' & - & DNE & + &  & + \\
                    \hline
                \end{tabular}
            \end{center}
    \end{enumerate}

    \begin{tikzpicture}
        \begin{axis} [width=\textwidth, height=\textwidth, xmin=-6, xmax=6, ymin=-6, ymax=6]
            \addplot [
                domain=-6:-0.001,
                samples=70,
                color=blue,
                ]
                {x*e^(1/x)};
            \addplot [
                domain=0.001:6,
                samples=70,
                color=blue,
                ]
                {min(x*e^(1/x), 6)};
        \end{axis}
    \end{tikzpicture}
\end{document}
