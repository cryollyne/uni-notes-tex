%%%%%%%%%%%%%%%%%%%%%%%%%%%%% Define Article %%%%%%%%%%%%%%%%%%%%%%%%%%%%%%%%%%
\documentclass{article}
%%%%%%%%%%%%%%%%%%%%%%%%%%%%%%%%%%%%%%%%%%%%%%%%%%%%%%%%%%%%%%%%%%%%%%%%%%%%%%%

%%%%%%%%%%%%%%%%%%%%%%%%%%%%% Using Packages %%%%%%%%%%%%%%%%%%%%%%%%%%%%%%%%%%
\usepackage{geometry}
\usepackage{graphicx}
\usepackage{amssymb}
\usepackage{amsmath}
\usepackage{amsthm}
\usepackage{empheq}
\usepackage{mdframed}
\usepackage{booktabs}
\usepackage{lipsum}
\usepackage{graphicx}
\usepackage{color}
\usepackage{psfrag}
\usepackage{pgfplots}
\usepackage{bm}
%%%%%%%%%%%%%%%%%%%%%%%%%%%%%%%%%%%%%%%%%%%%%%%%%%%%%%%%%%%%%%%%%%%%%%%%%%%%%%%

% Other Settings

%%%%%%%%%%%%%%%%%%%%%%%%%% Page Setting %%%%%%%%%%%%%%%%%%%%%%%%%%%%%%%%%%%%%%%
\geometry{a4paper}

%%%%%%%%%%%%%%%%%%%%%%%%%% Define some useful colors %%%%%%%%%%%%%%%%%%%%%%%%%%
\definecolor{ocre}{RGB}{243,102,25}
\definecolor{mygray}{RGB}{243,243,244}
\definecolor{deepGreen}{RGB}{26,111,0}
\definecolor{shallowGreen}{RGB}{235,255,255}
\definecolor{deepBlue}{RGB}{61,124,222}
\definecolor{shallowBlue}{RGB}{235,249,255}
%%%%%%%%%%%%%%%%%%%%%%%%%%%%%%%%%%%%%%%%%%%%%%%%%%%%%%%%%%%%%%%%%%%%%%%%%%%%%%%

%%%%%%%%%%%%%%%%%%%%%%%%%% Define an orangebox command %%%%%%%%%%%%%%%%%%%%%%%%
\newcommand\orangebox[1]{\fcolorbox{ocre}{mygray}{\hspace{1em}#1\hspace{1em}}}
%%%%%%%%%%%%%%%%%%%%%%%%%%%%%%%%%%%%%%%%%%%%%%%%%%%%%%%%%%%%%%%%%%%%%%%%%%%%%%%

%%%%%%%%%%%%%%%%%%%%%%%%%%%% English Environments %%%%%%%%%%%%%%%%%%%%%%%%%%%%%
\newtheoremstyle{mytheoremstyle}{3pt}{3pt}{\normalfont}{0cm}{\rmfamily\bfseries}{}{1em}{{\color{black}\thmname{#1}~\thmnumber{#2}}\thmnote{\,--\,#3}}
\newtheoremstyle{myproblemstyle}{3pt}{3pt}{\normalfont}{0cm}{\rmfamily\bfseries}{}{1em}{{\color{black}\thmname{#1}~\thmnumber{#2}}\thmnote{\,--\,#3}}
\theoremstyle{mytheoremstyle}
\newmdtheoremenv[linewidth=1pt,backgroundcolor=shallowGreen,linecolor=deepGreen,leftmargin=0pt,innerleftmargin=20pt,innerrightmargin=20pt,]{theorem}{Theorem}[section]
\theoremstyle{mytheoremstyle}
\newmdtheoremenv[linewidth=1pt,backgroundcolor=shallowBlue,linecolor=deepBlue,leftmargin=0pt,innerleftmargin=20pt,innerrightmargin=20pt,]{definition}{Definition}[section]
\theoremstyle{myproblemstyle}
\newmdtheoremenv[linecolor=black,leftmargin=0pt,innerleftmargin=10pt,innerrightmargin=10pt,]{problem}{Problem}[section]
%%%%%%%%%%%%%%%%%%%%%%%%%%%%%%%%%%%%%%%%%%%%%%%%%%%%%%%%%%%%%%%%%%%%%%%%%%%%%%%

%%%%%%%%%%%%%%%%%%%%%%%%%%%%%%% Plotting Settings %%%%%%%%%%%%%%%%%%%%%%%%%%%%%
\usepgfplotslibrary{colorbrewer}
\pgfplotsset{width=8cm,compat=1.9}
%%%%%%%%%%%%%%%%%%%%%%%%%%%%%%%%%%%%%%%%%%%%%%%%%%%%%%%%%%%%%%%%%%%%%%%%%%%%%%%

%%%%%%%%%%%%%%%%%%%%%%%%%%%%%%% Title & Author %%%%%%%%%%%%%%%%%%%%%%%%%%%%%%%%
\title{Integrals}
\author{Patrick Chen}
\date{Oct 28, 2024}
%%%%%%%%%%%%%%%%%%%%%%%%%%%%%%%%%%%%%%%%%%%%%%%%%%%%%%%%%%%%%%%%%%%%%%%%%%%%%%%

\begin{document}
    \maketitle
    \section*{Antiderivatives}
    A function $F$ is called an antiderivative or indefinite integral of $f$ on
    an interval $I$ if $F'(x)=f(x)$ for all $x$ in $I$. Since adding a constant
    doesn't change the derivative of a function, the addition of a constant is
    often required for integration.

    \begin{align*}
        F'(x) = f(x) \\
        F(x) = \int f(x) \ dx
    \end{align*}

    \section*{Table of Antiderivatives}
    \begin{center}
        \renewcommand{\arraystretch}{2}
        \begin{tabular}[c]{l|l}
            \hline
            function & antiderivative \\
            \hline
            $x^n$ when $n\ne -1$ & $\frac{1}{n+1} x^{n+1} + c$ \\
            \hline
            $\frac{1}{x}$ & $\ln |x| + c$ \\
            \hline
            $e^x$ & $e^x + c$ \\
            \hline
            $\cos x$ & $\sin x + c$ \\
            \hline
            $\sin x$ & $-\cos x + c$ \\
            \hline
            $\sec^2 x$ & $\tan x + c$ \\
            \hline
            $\sec x \tan x$ & $\sec + c$ \\
            \hline
            $\frac{1}{\sqrt{1-x^2}}$ & $\sin^{-1} x + c$ \\
            \hline
            $\frac{1}{\sqrt{1+x^2}}$ & $\tan^{-1} x + c$ \\
            \hline
            $a^x$ & $\frac{a^x}{\ln a} + c$ \\
            \hline
        \end{tabular}
    \end{center}

    \section*{Linearity of Integral}
    Just like derivatives, integrals are linear.
    \begin{align*}
        \int f(x)+g(x) \ dx &= \int f(x) \ dx + \int g(x) \ dx \\
        \int cf(x) \ dx &= c \int f(x) \ dx
    \end{align*}

    \subsection*{Example 1}
    \begin{align*}
        g'(x) = 4 \sin x + \frac{2x^5-\sqrt{x}}{x} && g(0)=2
    \end{align*}
    \begin{align*}
        g(x)&=\int 4 \sin x + \frac{2x^5-\sqrt{x}}{x} \ dx \\
        &= \int 4 \sin x \ dx + \int \frac{2x^5}{x} \ dx - \int \frac{\sqrt{x}}{x} \ dx \\
        &= 4\int \sin x \ dx + 2\int x^4 \ dx - \int x^{-\frac{1}{2}} \ dx \\
        &= 4(-\cos x) + 2(\frac{x^5}{5}) - 2\sqrt{x} + c \\
        &= -4\cos x + \frac{2}{5} x^5 -2\sqrt{x} + c
    \end{align*}
    \begin{align*}
        g(0) &= -4 \cos x + \frac{2}{5} x^5 - 2\sqrt{x} + c \\
        2    &= -4 \cos 0 + \frac{2}{5} 0^5 - 2\sqrt{0} + c \\
        2    &= -4 + c \\
        6    &= c \\
        g(x) &= -4\cos x + \frac{2}{5} x^5 -2\sqrt{x} + 6
    \end{align*}

    \subsection*{Example 2}
    \begin{align*}
        f''(x) &= 12x^2 + 6x - 4 \\
        f(0)   &= 4 \\
        f(1)   &= 1
    \end{align*}
    \begin{align*}
        f'(x) &= \int 12x^2 + 6x - 4 \ dx \\
        f'(x) &= 4x^3 + 3x^2 - 4x + c_1 \\
        f(x)  &= \int 4x^3 + 3x^2 - 4x + c_1 \ dx \\
        f(x)  &= x^4 + x^3 -2x^2 + c_1x + c_2
    \end{align*}
    \begin{align*}
        f(0) &= 0 + 0 - 0 + 0 + c_2 \\
        4    &= c_2 \\
        f(1) &= 1 + 1 - 2(1) + c_1 + 4 \\
        1    &= c_1 + 4 \\
        -3   &=  c_1 \\
        f(x) &= x^4 + x^3 -x^2 - 3x + 4
    \end{align*}


    \section*{Sigma Notation}
    Sigma notation is a compact way to write a sum of many numbers.
    \begin{align*}
        \sum_{i=m}^n a_{i} = a_{m} + a_{m+1} + a_{m+2} + \dots + a_{n-1} + a_{n}
    \end{align*}

    The index of the sum can be changes.
    \begin{align*}
        \sum_{i=m}^{n} a_i = \sum_{i=m+k}^{n+k} a_{i-k} &&
        \sum_{i=m}^{n} a_i = \sum_{i=m-k}^{n-k} a_{i+k}
    \end{align*}

    Summations are linear.
    \begin{align*}
        \sum_{i=1}^{n} ca_i &= c\sum_{i=1}^{n} a_i \\
        \sum_{i=m}^{n} (a_i + b_i) &= \sum_{i=m}^{n} a_i + \sum_{i=m}^{n} b_i \\
    \end{align*}

    There are a closed form solutions for many summations
    \begin{align*}
        \sum_{i=1}^{n} c &= cn \\
        \sum_{i=1}^{n} i &= \frac{n(n+1)}{2} \\
        \sum_{i=1}^{n} i^2 &= \frac{n(n+1)(2n+1)}{6} \\
        \sum_{i=1}^{n} i^3 &= \bigg(\frac{n(n+1)}{2}\bigg)^2
    \end{align*}

    \subsection*{Example 3}
    \begin{align*}
        f(x) &= x^2 \\
        \lim_{n\to \infty} \sum_{i=1}^{n} \frac{1}{n} (\frac{i}{n})^2
        &= \lim_{n\to \infty} \frac{1}{n} \sum_{i=1}^{n} \frac{i^2}{n^2} \\
        &= \lim_{n\to \infty} \frac{1}{n^3} \sum_{i=i}^{n} i^2 \\
        &= \lim_{n\to \infty} \frac{1}{n^3} \frac{n(n+1)(2n+1)}{6} \\
        &= \lim_{n\to \infty} \frac{(n+1)(2n+1)}{6n^2} \\
        &= \lim_{n\to \infty} \frac{2n^2 + 3n + 1}{6n^2} \\
        &= \lim_{n\to \infty} \frac{n^2 (2+ 3n^{-1} + n^{-2})}{6n^2} \\
        &= \lim_{n\to \infty} \frac{2+ 3n^{-1} + n^{-2}}{6} \\
        &= \frac{2}{6} \\
        &= \frac{1}{3}
    \end{align*}

    \section*{Estimating Area}
    The area under a curve can be estimated with rectangles. The area in the
    interval $[0,1]$ can be estimated by the following formula. $L_n$ is the
    area with the rectangles having a height determined by the left most
    part of the rectangle and $R_n$ has height determined by the right most
    part. The height of the rectangle doesn't have to be the left most or
    right most point, it can be any point in between the bounds of the
    rectangle. $M_n$ is the sum using the mid point of each rectangle.
    \begin{align*}
        L_n &= \sum_{i=1}^{n} \frac{1}{n}\cdot f(\frac{i}{n}) \\
        R_n &= \sum_{i=1}^{n} \frac{1}{n}\cdot f(\frac{i-1}{n}) \\
        M_n &= \sum_{i=1}^{n} \frac{1}{n}\cdot f(\frac{2i+1}{2})
    \end{align*}

    The precise area under a function is the limit as the rectangle width
    approaches zero.
    \begin{align*}
        \lim_{n\to \infty} R_n = \lim_{n\to \infty} L_n
    \end{align*}

    \begin{align*}
        R_n = \lim_{n\to \infty} \frac{1}{n} \sum_{i=0}^{n} f(\frac{i}{n})
    \end{align*}

    In general, if $f$ is continuous and positive on a interval $[a,b]$,
    then the area under the graph of $f(x)$ on the interval is given by the
    following formulas.
    \begin{align*}
        \Delta x &= \frac{b-a}{n} \\
        A = R_n &= \lim_{n\to \infty} \sum_{i=1}^{n} \Delta x f(a+i\Delta x) \\
        A = L_n &= \lim_{n\to \infty} \sum_{i=1}^{n} \Delta x f(a+(i-1)\Delta x) \\
        A = M_n &= \lim_{n\to \infty} \sum_{i=1}^{n} \Delta x f(a+x_i^*)
    \end{align*}

    \section*{Definite Integral}
    The definite integral on the interval $[a,b]$ is the signed area under a
    graph. If $f(x)$ is negative, then the area is considered negative. In
    other words, the integral on a interval is the area of every region
    where $f(x)$ is above the x-axis minus the area of every region where
    $f(x)$ is below the x-axis.
    \begin{align*}
        \int_{a}^{b} f(x) \ dx = \lim_{n\to \infty} \sum_{i=1}^{n} f(x_i)\Delta x
    \end{align*}

    \section*{Example 4}
    \begin{align*}
        \int_{0}^{3} (x-1) \ dx
        &= \lim_{n\to \infty} \sum_{i=1}^{n} \Delta x f(x_i) \\
        &= \lim_{n\to \infty} \sum_{i=1}^{n} \frac{3}{n} f(0+\frac{3i}{n}) \\
        &= \lim_{n\to \infty} \sum_{i=1}^{n} \frac{3}{n} (\frac{3i}{n} - 1) \\
        &= \lim_{n\to \infty} \frac{3}{n}
            \Big(\frac{3}{n} \sum_{i=1}^{n} i -
        \sum_{i=1}^{n} 1\Big) \\
        &= \lim_{n\to \infty} \frac{3}{n} (\frac{3}{n} \frac{n(n+1)}{2} - n) \\
        &= \lim_{n\to \infty} \frac{3}{n} ( \frac{3n+3}{2} - \frac{2n}{2} ) \\
        &= \lim_{n\to \infty} \frac{3}{n} ( \frac{n+3}{2}) \\
        &= \lim_{n\to \infty} \frac{3n+9}{2n} \\
        &= \frac{3}{2}
    \end{align*}

    \section*{Properties of Definite Integrals}
    Definite integrals are linear
    \begin{align*}
        \int_{a}^{b} f(x) + g(x) \ dx &= \int_{a}^{b} f(x) \ dx + \int_{a}^{b} g(x) \ dx \\
        \int_{a}^{b} cf(x) \ dx &= c \int_{a}^{b} f(x) \ dx
    \end{align*}
    The integral of a constant is just the constant times the length of the
    bounds of the integral.
    \begin{align*}
        \int_{a}^{b} c \ dx = c(b-a)
    \end{align*}
    Inverting the bounds of a integral will negate the result. This implies
    that integrals with both bounds equal will result in zero.
    \begin{align*}
        \int_{a}^{b} f(x) \ dx &= -\int_{b}^{a} f(x) \ dx \\
        \int_{a}^{a} f(x) \ dx &= 0
    \end{align*}
    Integrals can be combined if they share a common bound with the same
    function.
    \begin{align*}
        \int_{a}^{c} f(x) \ dx + \int_{c}^{b} f(x) \ dx = \int_{a}^{b} f(x) \ dx
    \end{align*}

    \section*{Fundamental Theorem of Calculus}
    If $f$ is continuous on $[a,b]$ and $F(x)$ is the antiderivative of $f$,
    then the area under the function $f$ on the interval $[a,b]$ is equal to the
    antiderivative evaluated on $b$ minus the antiderivative evaluated at $a$.
    \begin{align*}
        \int_{a}^{b} f(x) \ dx = F(b) - F(a)
    \end{align*}

    If $f(t)$ is continuous on $[a,b]$ and $g$ is continuous on $[a,b]$ and
    differentiable on $(a,b)$, then the derivative with respect to $x$ of the
    definite integral with respect to $t$ of $f(t)$ from $a$ to $x$ is the
    function $f$.
    \begin{align*}
        g(x) &= \int_{a}^{x} f(t) \ dt \\
        g'(x) &= f(t)
    \end{align*}

    \subsection*{Example 5}
    \begin{align*}
        \int_{3}^{6} \frac{1}{x} \ dx
        &= \ln(|x|) \ \Big|_3^6 \\
        &= \ln(|6|)-\ln(|3|) \\
        &= (\ln 2 + \ln 3) - \ln 3 \\
        &= \ln 2
    \end{align*}

    \subsection*{Example 6}
    \begin{align*}
        \int_{1}^{3} e^{-2x} \ 
        &= -1/2 e^{-2x} \ \Big|_{1}^{3} \\
        &= (-\frac{1}{2} e^{-6}) - (-\frac{1}{2} e^{-2}) \\
        &= -\frac{1}{2} e^{-6} + \frac{1}{2} e^{-2}
    \end{align*}

    \subsection*{Example 7}
    Find $g'(x)$.
    \begin{align*}
        g(x) &= \int_{2}^{x^3} \sin(\ln t) \ dt \\
        h(x) &= \int_{2}^{x} \sin(\ln t) \ dt \\
        g(x) &= h(x^3) \\
        g'(x)&= 3x^2h'(x^3) \\
        g'(x)&= 3x^2\sin(\ln x^3)
    \end{align*}

    \section*{Area between curves}
    If $f(x)\ge g(x)$, then
    \begin{align*}
        \int_{a}^{b} f(x) \ dx - \int_{a}^{b} g(x) \ dx = \int_{a}^{b} f(x)-g(x) \ dx
    \end{align*}
    If $f(x)$ is not greater or equal to $g(x)$, then
    \begin{align*}
        \int_{a}^{b} g(x) \ dx - \int_{a}^{b} f(x) \ dx = \int_{a}^{b} g(x)-f(x) \ dx
    \end{align*}
    If $f$ and $g$ intersect, then split the integral at so the bounds are at
    the intersection point, then apply the previous rules.

    \section*{Symmetry}
    If an even function is integrated on symmetric bounds, the result will be
    twice the are under the curve from zero to the first bound.
    \begin{align*}
        \int_{-a}^{a} f(x) \ dx = 2 \int_{0}^{a} f(x) \ dx
    \end{align*}
    If an odd function is integrated on symmetric bounds, the result will be
    zero.
    \begin{align*}
        \int_{-a}^{a} f(x) \ dx = 0
    \end{align*}

\end{document}
