%%%%%%%%%%%%%%%%%%%%%%%%%%%%% Define Article %%%%%%%%%%%%%%%%%%%%%%%%%%%%%%%%%%
\documentclass{article}
%%%%%%%%%%%%%%%%%%%%%%%%%%%%%%%%%%%%%%%%%%%%%%%%%%%%%%%%%%%%%%%%%%%%%%%%%%%%%%%

%%%%%%%%%%%%%%%%%%%%%%%%%%%%% Using Packages %%%%%%%%%%%%%%%%%%%%%%%%%%%%%%%%%%
\usepackage{geometry}
\usepackage{graphicx}
\usepackage{amssymb}
\usepackage{amsmath}
\usepackage{amsthm}
\usepackage{empheq}
\usepackage{mdframed}
\usepackage{booktabs}
\usepackage{lipsum}
\usepackage{graphicx}
\usepackage{color}
\usepackage{psfrag}
\usepackage{pgfplots}
\usepackage{bm}
%%%%%%%%%%%%%%%%%%%%%%%%%%%%%%%%%%%%%%%%%%%%%%%%%%%%%%%%%%%%%%%%%%%%%%%%%%%%%%%

% Other Settings

%%%%%%%%%%%%%%%%%%%%%%%%%% Page Setting %%%%%%%%%%%%%%%%%%%%%%%%%%%%%%%%%%%%%%%
\geometry{a4paper}

%%%%%%%%%%%%%%%%%%%%%%%%%% Define some useful colors %%%%%%%%%%%%%%%%%%%%%%%%%%
\definecolor{ocre}{RGB}{243,102,25}
\definecolor{mygray}{RGB}{243,243,244}
\definecolor{deepGreen}{RGB}{26,111,0}
\definecolor{shallowGreen}{RGB}{235,255,255}
\definecolor{deepBlue}{RGB}{61,124,222}
\definecolor{shallowBlue}{RGB}{235,249,255}
%%%%%%%%%%%%%%%%%%%%%%%%%%%%%%%%%%%%%%%%%%%%%%%%%%%%%%%%%%%%%%%%%%%%%%%%%%%%%%%

%%%%%%%%%%%%%%%%%%%%%%%%%% Define an orangebox command %%%%%%%%%%%%%%%%%%%%%%%%
\newcommand\orangebox[1]{\fcolorbox{ocre}{mygray}{\hspace{1em}#1\hspace{1em}}}
%%%%%%%%%%%%%%%%%%%%%%%%%%%%%%%%%%%%%%%%%%%%%%%%%%%%%%%%%%%%%%%%%%%%%%%%%%%%%%%

%%%%%%%%%%%%%%%%%%%%%%%%%%%% English Environments %%%%%%%%%%%%%%%%%%%%%%%%%%%%%
\newtheoremstyle{mytheoremstyle}{3pt}{3pt}{\normalfont}{0cm}{\rmfamily\bfseries}{}{1em}{{\color{black}\thmname{#1}~\thmnumber{#2}}\thmnote{\,--\,#3}}
\newtheoremstyle{myproblemstyle}{3pt}{3pt}{\normalfont}{0cm}{\rmfamily\bfseries}{}{1em}{{\color{black}\thmname{#1}~\thmnumber{#2}}\thmnote{\,--\,#3}}
\theoremstyle{mytheoremstyle}
\newmdtheoremenv[linewidth=1pt,backgroundcolor=shallowGreen,linecolor=deepGreen,leftmargin=0pt,innerleftmargin=20pt,innerrightmargin=20pt,]{theorem}{Theorem}[section]
\theoremstyle{mytheoremstyle}
\newmdtheoremenv[linewidth=1pt,backgroundcolor=shallowBlue,linecolor=deepBlue,leftmargin=0pt,innerleftmargin=20pt,innerrightmargin=20pt,]{definition}{Definition}[section]
\theoremstyle{myproblemstyle}
\newmdtheoremenv[linecolor=black,leftmargin=0pt,innerleftmargin=10pt,innerrightmargin=10pt,]{problem}{Problem}[section]
%%%%%%%%%%%%%%%%%%%%%%%%%%%%%%%%%%%%%%%%%%%%%%%%%%%%%%%%%%%%%%%%%%%%%%%%%%%%%%%

%%%%%%%%%%%%%%%%%%%%%%%%%%%%%%% Plotting Settings %%%%%%%%%%%%%%%%%%%%%%%%%%%%%
\usepgfplotslibrary{colorbrewer}
\pgfplotsset{width=8cm,compat=1.9}
%%%%%%%%%%%%%%%%%%%%%%%%%%%%%%%%%%%%%%%%%%%%%%%%%%%%%%%%%%%%%%%%%%%%%%%%%%%%%%%

%%%%%%%%%%%%%%%%%%%%%%%%%%%%%%% Title & Author %%%%%%%%%%%%%%%%%%%%%%%%%%%%%%%%
\title{Continuity}
\author{Patrick Chen}
\date {Sept 11, 2024}
%%%%%%%%%%%%%%%%%%%%%%%%%%%%%%%%%%%%%%%%%%%%%%%%%%%%%%%%%%%%%%%%%%%%%%%%%%%%%%%

\begin{document}
    \maketitle

    \section*{Limits}
    \begin{align*}
        \lim_{x\to a^-} f(x) = l
    \end{align*}
    This means that we can make f(x) as close to $l$ as we want, provided that
    we choose a $x$ close enough to $a$ to the left. If both left limit and
    right limit exists and are the same, we say that the function has a limit.
    \begin{align*}
        \lim_{x\to a^-}f(x) = \lim_{x\to a^+}f(x) = \lim_{x\to a}f(x)
    \end{align*}

    \section*{Continuity}
    Consider
    \begin{align*}
    f(x) =
        \begin{cases}
            x^2 & x\ne 2 \\
            0   & x=2
        \end{cases} &&
    g(x) = 
        \begin{cases}
            x^2 & x\ne 2 \\
            undefined & otherwise
        \end{cases} &&
    h(x) = x^2
    \end{align*}
    When $f(x)$, $g(x)$, and $h(x)$ is approached from either the left or the
    right, it is equal to 4.
    \begin{align*}
        \lim_{x\to2^-} f(x) = 4 \\
        \lim_{x\to2^+} f(x) = 4
    \end{align*}

    For a function to be continuous at a point, the limit of the function at the
    point must exist and be equal to the function evaluated at the point.
    \begin{align*}
        \lim_{x\to a}f(x) = f(a)
    \end{align*}

    \subsection*{Directional Continuity}
    A function if continuous from the right at a point $x$ if the right limit is
    equal to the function evaluated at $x$. Likewise with left continuity.
    \begin{align*}
        \lim_{x\to a^+} f(x) = f(a) && \text{left continuous} \\
        \lim_{x\to a^-} f(x) = f(a) && \text{right continuous}
    \end{align*}

    \subsection*{Continuity Rules}
    If two functions are continuous at $a$ and $c$ is a constant, the following
    is also continuous
    \begin{enumerate}
        \item $f(a)+g(a)$
        \item $f(a)-g(a)$
        \item $cf(a)$
        \item $f(a)g(a)$
        \item $ \frac{f(a)}{g(a)} $ if $g(a)\ne0$
        \item $[f\circ g](a)$
    \end{enumerate}

    The following types of functions are continuous at every number in their
    domains:
    \begin{itemize}
        \item polynomial
        \item trigonometric functions
        \item exponential functions
        \item rational functions (when denominator is non-zero)
        \item root functions
        \item inverse trigonometric functions
        \item logarithmic functions
    \end{itemize}

    \subsection*{Intermediate Value Theorem}
    \begin{theorem}[Intermediate Value  Theorem]
        Suppose that f is continuous on the closed interval $[a,b]$ and let $N$
        be any number between $f(a)$ and $f(b)$, where $f(a)\ne (b)$. Then there
        exist a number $c$ in $(a,b)$ such that $f(c) = n$.
    \end{theorem}
    In other words, if a continuous function is evaluated at two points, any
    point in between the outputs at those points are also outputs of the
    function. Note that IVT requires a closed interval, $(a,b]$ and $[a, b)$ is
    not enough to satisfy the IVT. \\
    Consider the following function. It is continuous on the interval $[-3,0)$
    and $f(-3)= -\frac{1}{3}$ and $f(0) = 3$ but there is no $x$ such that
    $f(x)=2$.
    \begin{align*}
        f(x) = \begin{cases}
            3, &\text{ if } x = 0\\
            \frac{1}{x} &\text{ if } x<0 
        \end{cases}
    \end{align*}

    \subsection*{Example}
    \begin{align*}
        \ln(1+\cos(x))
    \end{align*}
    $\ln$ is continuous when the input is in its domain (positive numbers) and
    $1+\cos(x)$ is continuous for all numbers.
    \begin{align*}
        1+\cos(x) > 0
    \end{align*}
    Since $\cos(x)$ ranges from $-1$ to $1$, $1+\cos(x)$ has a range of $[0,2]$.
    This is problematic because $0$ is in the range of $1+\cos(x)$ but not in
    the domain of $\ln(x)$, thus there is a discontinuity in $\ln(1+\cos(x))$.

    \subsection*{Example 2}
    let $f(x)=e^{-x}-x$. Show that the function has at least one root on the
    interval $[0,1]$.
    \begin{align*}
        f(0) &= e^{-0}-0 = 1 \\
        f(1) &= e^{-1}-1 = \frac{1}{e} - 1 \approx -0.63
    \end{align*}
    By the intermediate value theorem, $f(1) < 0 < f(0)$, thus there exist
    some input $N$ of the function such that $f(N) = 0$

    \subsection*{Example 3}
    Show that $4x^3 -6x^2 = 2 -3x$ has at lease one root.
    \begin{align*}
        4x^3 -6x^2 = 2 -3x \\
        4x^3 -6x^2 +3x -2 = 0 \\
        \text{let } f(x) = 4x^3 -6x^2 +3x -2
    \end{align*}
    \begin{align*}
        f(0) &= 4(0)^3 -6(0)^2 +3(0) -2 \\
        f(0) &= -2 \\
        f(2) &= 4(2)^3 -6(2)^2 +3(2) -2 \\
        f(2) &= 32 -24 +6 -2 \\
        f(2) &= 12
    \end{align*}
    By the intermediate value theorem, there exist some value $N$ in the
    interval $[0,2]$ there $f(x)=0$.

\end{document}
