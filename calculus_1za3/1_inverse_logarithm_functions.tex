%%%%%%%%%%%%%%%%%%%%%%%%%%%%% Define Article %%%%%%%%%%%%%%%%%%%%%%%%%%%%%%%%%%
\documentclass{article}
%%%%%%%%%%%%%%%%%%%%%%%%%%%%%%%%%%%%%%%%%%%%%%%%%%%%%%%%%%%%%%%%%%%%%%%%%%%%%%%

%%%%%%%%%%%%%%%%%%%%%%%%%%%%% Using Packages %%%%%%%%%%%%%%%%%%%%%%%%%%%%%%%%%%
\usepackage{geometry}
\usepackage{graphicx}
\usepackage{amssymb}
\usepackage{amsmath}
\usepackage{amsthm}
\usepackage{empheq}
\usepackage{mdframed}
\usepackage{booktabs}
\usepackage{lipsum}
\usepackage{graphicx}
\usepackage{color}
\usepackage{psfrag}
\usepackage{pgfplots}
\usepackage{bm}
%%%%%%%%%%%%%%%%%%%%%%%%%%%%%%%%%%%%%%%%%%%%%%%%%%%%%%%%%%%%%%%%%%%%%%%%%%%%%%%

% Other Settings

%%%%%%%%%%%%%%%%%%%%%%%%%% Page Setting %%%%%%%%%%%%%%%%%%%%%%%%%%%%%%%%%%%%%%%
\geometry{a4paper}

%%%%%%%%%%%%%%%%%%%%%%%%%% Define some useful colors %%%%%%%%%%%%%%%%%%%%%%%%%%
\definecolor{ocre}{RGB}{243,102,25}
\definecolor{mygray}{RGB}{243,243,244}
\definecolor{deepGreen}{RGB}{26,111,0}
\definecolor{shallowGreen}{RGB}{235,255,255}
\definecolor{deepBlue}{RGB}{61,124,222}
\definecolor{shallowBlue}{RGB}{235,249,255}
%%%%%%%%%%%%%%%%%%%%%%%%%%%%%%%%%%%%%%%%%%%%%%%%%%%%%%%%%%%%%%%%%%%%%%%%%%%%%%%

%%%%%%%%%%%%%%%%%%%%%%%%%% Define an orangebox command %%%%%%%%%%%%%%%%%%%%%%%%
\newcommand\orangebox[1]{\fcolorbox{ocre}{mygray}{\hspace{1em}#1\hspace{1em}}}
%%%%%%%%%%%%%%%%%%%%%%%%%%%%%%%%%%%%%%%%%%%%%%%%%%%%%%%%%%%%%%%%%%%%%%%%%%%%%%%

%%%%%%%%%%%%%%%%%%%%%%%%%%%% English Environments %%%%%%%%%%%%%%%%%%%%%%%%%%%%%
\newtheoremstyle{mytheoremstyle}{3pt}{3pt}{\normalfont}{0cm}{\rmfamily\bfseries}{}{1em}{{\color{black}\thmname{#1}~\thmnumber{#2}}\thmnote{\,--\,#3}}
\newtheoremstyle{myproblemstyle}{3pt}{3pt}{\normalfont}{0cm}{\rmfamily\bfseries}{}{1em}{{\color{black}\thmname{#1}~\thmnumber{#2}}\thmnote{\,--\,#3}}
\theoremstyle{mytheoremstyle}
\newmdtheoremenv[linewidth=1pt,backgroundcolor=shallowGreen,linecolor=deepGreen,leftmargin=0pt,innerleftmargin=20pt,innerrightmargin=20pt,]{theorem}{Theorem}[section]
\theoremstyle{mytheoremstyle}
\newmdtheoremenv[linewidth=1pt,backgroundcolor=shallowBlue,linecolor=deepBlue,leftmargin=0pt,innerleftmargin=20pt,innerrightmargin=20pt,]{definition}{Definition}[section]
\theoremstyle{myproblemstyle}
\newmdtheoremenv[linecolor=black,leftmargin=0pt,innerleftmargin=10pt,innerrightmargin=10pt,]{problem}{Problem}[section]
%%%%%%%%%%%%%%%%%%%%%%%%%%%%%%%%%%%%%%%%%%%%%%%%%%%%%%%%%%%%%%%%%%%%%%%%%%%%%%%

%%%%%%%%%%%%%%%%%%%%%%%%%%%%%%% Plotting Settings %%%%%%%%%%%%%%%%%%%%%%%%%%%%%
\usepgfplotslibrary{colorbrewer}
\pgfplotsset{width=8cm,compat=1.9}
%%%%%%%%%%%%%%%%%%%%%%%%%%%%%%%%%%%%%%%%%%%%%%%%%%%%%%%%%%%%%%%%%%%%%%%%%%%%%%%

%%%%%%%%%%%%%%%%%%%%%%%%%%%%%%% Title & Author %%%%%%%%%%%%%%%%%%%%%%%%%%%%%%%%
\title{Inverse and Logarithms}
\author{Patrick Chen}
\date{Sept 5, 2024}
%%%%%%%%%%%%%%%%%%%%%%%%%%%%%%%%%%%%%%%%%%%%%%%%%%%%%%%%%%%%%%%%%%%%%%%%%%%%%%%

\begin{document}
    \maketitle
    \section*{Inverse Functions}
    A function $f$ is called one-to-one function if it never takes on the same
    value twice. If for every horizontal lines, there are no horizontal lines
    that intersect the graph more than once, then the function satisfies the
    horizontal line test. If a function satisfies the horizontal line test, it
    is one-to-one.


    Let $f$ be a one-to-one function with domain A and range B. Its inverse
    $f^-1$ has domain of B and range A (inverse functions have inverted domain
    and ranges).
    \begin{align*}
        f(x) = y &\Leftrightarrow f^{-1}(y) = x \\
        f: A \mapsto B &\Leftrightarrow f^{-1}: B \mapsto A \\
    \end{align*}

    \begin{theorem}
        Cancellation equations:
        \begin{align*}
            f^{-1}(f(x)) = x \text{ for every x in A} \\
            f(f^{-1}(y)) = y \text{ for every y in B}
        \end{align*}
    \end{theorem}

    Example:
    let $f(x)=x^3$. Find $f^{-1}(x)$


    \begin{align*}
        f^{-1}(x)=\sqrt[3]{x} \\
        f^{-1}(f(x))=x \\
        \sqrt[3]{x^3}=x \\
        x=x
    \end{align*}
    \begin{align*}
        f(f^{-1}(y))=y \\
        (\sqrt[3]{y})^3=y \\
        y=y
    \end{align*}

    If you have the graph of $f$, $f^{-1}$ is the function $f$ reflected upon
    the x=y line. This is equivalent to swapping $x$ and $y$ for every point on
    the graph. The inverse of a function can be found algebraically by following
    these steps:

    \begin{enumerate}
        \item write $y=f(x)$
        \item solve this equation for x in terms of y
        \item interchange x and y
    \end{enumerate}

    \section*{Logarithmic Functions}

    If $b>0$ and $b\ne 1$, the exponential function $f(x)=b^x$ is either
    strictly increasing or decreasing and therefore it is one-to-one and has an
    inverse function $f^{-1} = \log_{b}\ y$.

    \begin{align*}
        b^x=y &\Leftrightarrow log_by=x \\
        \log_b(b^x)&=x \text{ for every } x \in \mathbb{R} \\
        b^{\log_by}&=y \text{ for every } y > 0
    \end{align*}

    Laws of Logarithms
    \begin{enumerate}
        \item $\log_b(xy)=\log_bx+\log_by$

        \item $\log_b(\frac{x}{y})=\log_bx-\log_by$

        \item $\log_b(x^r)=r\log_b(x)$
    \end{enumerate}

    \subsection*{Natural Logarithms}
    The natural logarithm is a special logarithm with a base of $e$.
    \begin{align*}
        \ln(x)&=\log_e(x) \\
        \text{where } e &\approx 2.718
    \end{align*}

    \begin{align*}
        e^x=y &\Leftrightarrow \ln y=x \\
        \ln(e^x)&=x \text{ for every } x \in \mathbb{R} \\
        e^{\ln y}&=y \text{ for every } y > 0
    \end{align*}

    Log change of base formula
    \begin{align*}
        \frac{\log_b a}{\log_b c} = \log_c a
    \end{align*}

\end{document}
