%%%%%%%%%%%%%%%%%%%%%%%%%%%%% Define Article %%%%%%%%%%%%%%%%%%%%%%%%%%%%%%%%%%
\documentclass{article}
%%%%%%%%%%%%%%%%%%%%%%%%%%%%%%%%%%%%%%%%%%%%%%%%%%%%%%%%%%%%%%%%%%%%%%%%%%%%%%%

%%%%%%%%%%%%%%%%%%%%%%%%%%%%% Using Packages %%%%%%%%%%%%%%%%%%%%%%%%%%%%%%%%%%
\usepackage{geometry}
\usepackage{graphicx}
\usepackage{amssymb}
\usepackage{amsmath}
\usepackage{amsthm}
\usepackage{empheq}
\usepackage{mdframed}
\usepackage{booktabs}
\usepackage{lipsum}
\usepackage{graphicx}
\usepackage{color}
\usepackage{psfrag}
\usepackage{pgfplots}
\usepackage{bm}
%%%%%%%%%%%%%%%%%%%%%%%%%%%%%%%%%%%%%%%%%%%%%%%%%%%%%%%%%%%%%%%%%%%%%%%%%%%%%%%

% Other Settings

%%%%%%%%%%%%%%%%%%%%%%%%%% Page Setting %%%%%%%%%%%%%%%%%%%%%%%%%%%%%%%%%%%%%%%
\geometry{a4paper}

%%%%%%%%%%%%%%%%%%%%%%%%%% Define some useful colors %%%%%%%%%%%%%%%%%%%%%%%%%%
\definecolor{ocre}{RGB}{243,102,25}
\definecolor{mygray}{RGB}{243,243,244}
\definecolor{deepGreen}{RGB}{26,111,0}
\definecolor{shallowGreen}{RGB}{235,255,255}
\definecolor{deepBlue}{RGB}{61,124,222}
\definecolor{shallowBlue}{RGB}{235,249,255}
%%%%%%%%%%%%%%%%%%%%%%%%%%%%%%%%%%%%%%%%%%%%%%%%%%%%%%%%%%%%%%%%%%%%%%%%%%%%%%%

%%%%%%%%%%%%%%%%%%%%%%%%%% Define an orangebox command %%%%%%%%%%%%%%%%%%%%%%%%
\newcommand\orangebox[1]{\fcolorbox{ocre}{mygray}{\hspace{1em}#1\hspace{1em}}}
%%%%%%%%%%%%%%%%%%%%%%%%%%%%%%%%%%%%%%%%%%%%%%%%%%%%%%%%%%%%%%%%%%%%%%%%%%%%%%%

%%%%%%%%%%%%%%%%%%%%%%%%%%%% English Environments %%%%%%%%%%%%%%%%%%%%%%%%%%%%%
\newtheoremstyle{mytheoremstyle}{3pt}{3pt}{\normalfont}{0cm}{\rmfamily\bfseries}{}{1em}{{\color{black}\thmname{#1}~\thmnumber{#2}}\thmnote{\,--\,#3}}
\newtheoremstyle{myproblemstyle}{3pt}{3pt}{\normalfont}{0cm}{\rmfamily\bfseries}{}{1em}{{\color{black}\thmname{#1}~\thmnumber{#2}}\thmnote{\,--\,#3}}
\theoremstyle{mytheoremstyle}
\newmdtheoremenv[linewidth=1pt,backgroundcolor=shallowGreen,linecolor=deepGreen,leftmargin=0pt,innerleftmargin=20pt,innerrightmargin=20pt,]{theorem}{Theorem}[section]
\theoremstyle{mytheoremstyle}
\newmdtheoremenv[linewidth=1pt,backgroundcolor=shallowBlue,linecolor=deepBlue,leftmargin=0pt,innerleftmargin=20pt,innerrightmargin=20pt,]{definition}{Definition}[section]
\theoremstyle{myproblemstyle}
\newmdtheoremenv[linecolor=black,leftmargin=0pt,innerleftmargin=10pt,innerrightmargin=10pt,]{problem}{Problem}[section]
%%%%%%%%%%%%%%%%%%%%%%%%%%%%%%%%%%%%%%%%%%%%%%%%%%%%%%%%%%%%%%%%%%%%%%%%%%%%%%%

%%%%%%%%%%%%%%%%%%%%%%%%%%%%%%% Plotting Settings %%%%%%%%%%%%%%%%%%%%%%%%%%%%%
\usepgfplotslibrary{colorbrewer}
\pgfplotsset{width=8cm,compat=1.9}
%%%%%%%%%%%%%%%%%%%%%%%%%%%%%%%%%%%%%%%%%%%%%%%%%%%%%%%%%%%%%%%%%%%%%%%%%%%%%%%

%%%%%%%%%%%%%%%%%%%%%%%%%%%%%%% Title & Author %%%%%%%%%%%%%%%%%%%%%%%%%%%%%%%%
\title{Evaluating Integrals}
\author{Patrick Chen}
\date{Nov 7, 2024}
%%%%%%%%%%%%%%%%%%%%%%%%%%%%%%%%%%%%%%%%%%%%%%%%%%%%%%%%%%%%%%%%%%%%%%%%%%%%%%%

\begin{document}
    \maketitle
    The rules of derivatives can be inverted into rules of integrals.

    \section*{Substitution Rule}
    The substitution rule (also called u-substitution) of the integral is the
    reverse of the chain rule for derivative. The function $f$ must be
    continuous on the range of $g(x)=u$.
    \begin{align*}
        (F(u(x)))' &= f(u(x))u'(x) \\
        F(u(x)) &= \int f(u(x))u'(x) \ dx
    \end{align*}

    When using substitution for definite integrals, the bounds needs to be
    changed to account for the change in integration variable. The bounds can be
    found by substituting the old bounds in for $x$ and solving for $u$.
    \begin{align*}
        \int_{a}^{b} f(u(x)) u'(x) \ dx
        = \int_{u(a)}^{u(b)} f(u) \ du
    \end{align*}

    \subsection*{Example 1}
    Evaluate $\int 2x\sqrt{1+x^2} \ dx$
    \begin{align*}
        u &= 1+x^2 \\
        \frac{du}{dx} &= 2x \\
        du &= 2x\ dx \\
        \int 2x\sqrt{1+x^2} \ dx &= \int \sqrt{u}\ 2x\ dx \\
        &= \int \sqrt{u} \ du \\
        &= \int u^{\frac{1}{2}} \ du \\
        &= \frac{2}{3} u^{\frac{3}{2}} + c \\
        &= \frac{2}{3} (1+x^2)^{\frac{3}{2}} + c
    \end{align*}

    \subsection*{Example 2}
    Evaluate $\int x^3\cos(x^4+2) \ dx$
    \begin{align*}
        u  &= x^4 + 2 \\
        du &= 4x^3 dx \\
        \int x^3\cos(x^4+2) \ dx
        &= \int \frac{4x^3}{4} \cos(u) \ dx \\
        &= \frac{1}{4} \int \cos(u) \ du \\
        &= \frac{1}{4} \sin(u) + c \\
        &= \frac{1}{4} \sin(x^4 + 2) + c
    \end{align*}

    \subsection*{Example 3}
    Evaluate $\int \tan x \ dx$
    \begin{align*}
        u &= \cos x \\
        du &= -\sin x dx \\
        \int \tan x \ dx
        &= \int \frac{\sin x}{\cos x} \ dx \\
        &= \int -\frac{1}{u} \ du \\
        &= -\ln |u| \\
        &= -\ln |\cos x|
    \end{align*}

    \subsection*{Example 4}
    Evaluate $\int_{0}^{1} \frac{1}{x+4} \ dx$.
    \begin{align*}
        u &= x+4 \\
        du &= dx \\
        \int_{0}^{1} \frac{1}{x+4} \ dx
        &= \int_{(0)+4}^{(1)+4} \frac{1}{u} \ du \\
        &= \int_{4}^{5} \frac{1}{u} \ du \\
        &= \ln|u| \ \Big|_{4}^{5} \\
        &= \ln|5| - \ln|4| \\
        &= \ln(\frac{5}{4})
    \end{align*}

    \section*{Integration by Parts}
    Integration by part is the reverse of product rule. Often the formula is
    rearranged to a more convenient form.
    \begin{align*}
        \frac{d}{dx} (f(x)g(x)) &=  f'(x)g(x) + f(x)g'(x) \\
                                &\Downarrow \int \\
        f(x)g(x) &= \int f'(x)g(x) \ dx + \int f(x)g'(x) \ dx \\
        \int f(x)g'(x) \ dx &= f(x)g(x)-\int f'(x)g(x) \ dx
    \end{align*}
    \begin{align*}
        f(x) = u &\Rightarrow f'(x)dx =du\\
        g(x) = v &\Rightarrow g'(x)dx = dv
    \end{align*}
    \begin{align*}
        \int u \ dv = uv - \int v \ du
    \end{align*}

    When using integration by parts to solve a integral, the values $u$ and $dv$
    need to be chosen and substituted into the previous formula. For definite
    integrals, its the same formula and the bounds do not change.
    \begin{align*}
        \int_{a}^{b} u \ dv = uv \ \Big|_{a}^{b} - \int_{a}^{b} v \ du
    \end{align*}

    \subsection*{Example 5}
    \begin{align*}
        \int x\sin x \ dx \\
    \end{align*}

    \begin{align*}
        x = u &\Rightarrow du = dx \\
        sinx dx = dv &\Rightarrow v = -cosx \\
        \int x\sin x \ dx\\
        \int u \ dv
        &= uv - \int v \ du \\
        &= x(-cosx) - \int (-cos) \ du \\
        &= -xcosx + sin x
    \end{align*}

    \subsection*{Example 6}
    \begin{align*}
        \int x^2 e^x \ dx \\
    \end{align*}
    \begin{align*}
        u = x^2 &\Rightarrow du = 2x dx \\
        dv = e^x &\Rightarrow v = e^x \\
        \int x^2 e^x \ dx
        &= x^2 e^x - \int e^x \cdot 2x \ dx \\
        &= x^2 e^x - 2\int e^x \cdot x \ dx \\
        u = x &\Rightarrow du = dx \\
        dv = e^x &\Rightarrow v = e^x \\
        &= x^2 e^x - 2(xe^x - \int e^x \ dx) \\
        &= x^2 e^x - 2(xe^x - e^x) \\
        &= x^2 e^x - 2xe^x + 2e^x
    \end{align*}

    \subsection*{Example 7}
    \begin{align*}
        \int_{0}^{1} \tan^{-1} x \ dx \\
    \end{align*}
    \begin{align*}
        u = \tan^{-1} x &\Rightarrow du = \frac{1}{x^2+1} dx \\
        dv = dx &\Rightarrow v = x \\
        \int_{0}^{1} \tan^{-1} x \ dx
        &= x \tan^{-1} x \ \Big|_{0}^{1} - \int_{0}^{1} x \frac{1}{x^2+1} \ dx \\
        u = x^2+1 \\
        du = 2x \\
        &= x \tan^{-1} x \ \Big|_{0}^{1} - \int_{1}^{5} \frac{1}{u} \ du \\
        &= x \tan^{-1} x \ \Big|_{0}^{1} - \ln|u| \ \Big|_{1}^{5} \\
        &= (1 \tan^{-1} 1 - 0 \tan^{-1} 0) - (\ln 5 - \ln 1) \\
        &= (1 \frac{\pi}{4} - 0) - (\ln5 - 0) \\
        &= 1 \frac{\pi}{4} - \ln5
    \end{align*}

    \subsection*{Example 8}
    \begin{align*}
        \int_{1}^{e} x^4(\ln x)^2 \ dx
    \end{align*}
    \begin{align*}
        u = (\ln x)^2 => du = (2 \ln x)/x \\
        dv = x^4 &\Rightarrow v = x^5/5 \\
        (\ln x)^2 \frac{x^5}{5} - \int_{1}^{e} \frac{x^5}{5} 2 (\ln x)/x \ dx \\
        (\ln x)^2 \frac{x^5}{5} - \int_{1}^{e} \frac{x^4}{5} 2 (\ln x) \ dx
    \end{align*}

    \subsection*{Example 9}
    \begin{align*}
        \int e^x \sin x \ dx
    \end{align*}
    \begin{align*}
        u = e^x &\Rightarrow du = e^x dx \\
        dv = sinx dx &\Rightarrow -\cos x \\
        I &= -e^x\cos x - \int -\cos x e^x \ dx \\
        &= -e^x\cos x + \int \cos x e^x \ dx \\
        u = e^x &\Rightarrow du = e^x \\
        dv = \cos x &\Rightarrow v = \sin x \\
        I &= -e^x\cos x + (e^x \sin x - \int \sin x e^x \ dx) \\
        I &= -e^x\cos x + (e^x \sin x - I) \\
        2I &= -e^x\cos x + e^x \sin x \\
        I &= \frac{e^x \sin x - e^x\cos x}{2}
    \end{align*}

    \subsection*{Example 10}
    \begin{align*}
        \int \cos(x) \ln (\sin x) \ dx
    \end{align*}
    \begin{align*}
        u = \sin x \\
        du = \cos x dx \\
        I &= \int \ln(u) \ du \\
        u_1 = \ln u &\Rightarrow du_1 = 1/u \\
        dv = du &\Rightarrow v = u \\
        I &= u\ln u - \int u (1/u) \ du \\
        &= u\ln u - \int 1 \ du \\
        &= u\ln u - u \\
        &= u(\ln u - 1) \\
        &= \sin x(\ln(\sin x) - 1)
    \end{align*}

    \section*{Powers of Sine and Cosine}
    If any of the powers are odd, then pick one of the following substitutions.
    \begin{align*}
        \int sin^m(x)cos^{2k+1}x \ dx &= \int sin^m (cos^2 x)^k cosx \ dx \\
                                      &= \int sin^m x (1-sin^2 x)^k cos x \ dx \\
        u &= sin x \\
        \int sin^{2k+1} x cos^n x \ dx &= \int (sin^2)^k cos^n sin \ dx \\
                                       &= \int (1-cos^2)^k cos^n sin \ dx \\
        u &= cos x
    \end{align*}

    If both of the powers are even, then use the half angle formulas
    \begin{align*}
        \sin^2 x = \frac{1}{2} (1-\cos 2x) \\
        \cos^2 x = \frac{1}{2} (1+\cos 2x)
    \end{align*}

    Similar substitutions will work for powers of tan and sec.
    \begin{align*}
        \tan^2(x)+1 = \sec^2(x)
    \end{align*}

    Product Identities
    \begin{align*}
        \sin A \cos B = \frac{1}{2} (\sin(A-B)+\sin(A+B)) \\
        \sin A \sin B = \frac{1}{2} (\cos(A-B)-\cos(A+B)) \\
        \cos A \cos B = \frac{1}{2} (\cos(A-B)+\cos(A+B))
    \end{align*}


    \subsection*{Example 11}
    \begin{align*}
        \int \sin^5 x \cos^2 x \ dx
    \end{align*}
    \begin{align*}
        u &= \cos x \\
        du &= -\sin x \\
        I &= \int (sin^2 x)^2 \cos^2 x \sin x \ dx \\
        I &= \int (1-\cos^2 x)^2 \cos^2 x \sin x \ dx \\
        I &= -\int (1-u^2)^2 u^2 \ du \\
        I &= -\int (1-2u^2+u^4) u^2 \ du \\
        I &= -\int  \ du
    \end{align*}

    \subsection*{Example 12}
    \begin{align*}
        \int sin^4 x \ dx
    \end{align*}
    \begin{align*}
        I &= \int (\sin^2 x)^2 \ dx \\
        &= \int (\frac{1}{2} (1-\cos 2x))^2 \ dx \\
        &= \int \frac{1}{4} (1 - 2\cos 2x + \cos^2 2x) \ dx \\
        &= \frac{1}{4} \int (1 - 2\cos 2x + \cos^2 2x) \ dx \\
        &= \frac{1}{4} \int (1 - 2\cos 2x + \frac{1}{2} (1+\cos 4x)) \ dx \\
        &= \frac{1}{4} \int (\frac{3}{2} - 2\cos 2x + \frac{1}{2} \cos 4x)) \ dx \\
        &= \frac{1}{4} (\frac{3}{2} x - 2 (\frac{1}{2} \sin 2x) + \frac{1}{2} (\frac{1}{4} \sin 4x)) \\
        &= \frac{1}{4} (\frac{3}{2} x - \sin 2x + \frac{1}{8} \sin 4x) \\
        &= \frac{3}{8} x - \frac{1}{4} \sin 2x + \frac{1}{32} \sin 4x
    \end{align*}

    \subsection*{Example 13}
    \begin{align*}
        \int tan^6 x sec^4 x \ dx
    \end{align*}
    \begin{align*}
        sec^2 x = 1 + tan^2 x
    \end{align*}
    \begin{align*}
        u = tan x \\
        du = sec^2 x \\
        I &= \int u^6 (1+u^2) \ du
    \end{align*}

    \subsection*{Example 14}
    \begin{align*}
        \int \sin 4x \cos 5x \ dx \\
        = \int 1/2 (\sin(-x) + sin(9x)) \ dx \\
        = \frac{1}{2} \int - \sin x + \sin 9x \ dx \\
        = \frac{1}{2} (\cos x - \frac{1}{9} \cos 9x)
    \end{align*}

    \subsection*{Example 15}
    \begin{align*}
        \int sec^2(x) \cos^5(tan(x)) \ dx \\
        u &= tan x \\
        du &= sec^2 x\ dx \\
        &=\int \cos^5(u) \ du \\
        &=\int (1-\sin^2(u))^2 \cos(u) \ du \\
        s &= \sin u \\
        ds &= \cos u\ du \\
        &=\int (1-s^2)^2 \ ds \\
        &=\int s^4 -2s^2 + 1 \ dx \\
        &= \frac{1}{5} s^5 - \frac{2}{3} s^3 + s + c \\
        &= \frac{1}{5} (\sin u)^5 - \frac{2}{3} (\sin u)^3 + \sin u + c \\
        &= \frac{1}{5} (\sin (\tan x))^5 - \frac{2}{3} (\sin (\tan x))^3 + \sin (\tan x) + c
    \end{align*}

\end{document}
